\chapter{Exceptional data with a short partition}\label{short-partition:ch}

\section{Realizability on the sphere}

In this final chapter, we will give a full solution of the Hurwitz existence problem for candidate data containing a partition of length $2$; as usual, we will assume that $\Sigma=\sphere{}$ and $n\ge 3$. This specific instance of the existence problem had already received some interest in the literature, leading to a few partial results.
\begin{itemize}
\item \cref{monodromy:th:sphere-[1 d-1]} addresses the cases where $\pi_n=[1,d-1]$; the proof was actually borrowed from \resultcite{proposition}{5.3}{edmonds}.
\item \textcite{pervova-existence-ii} dealt with the cases where $n=3$ and $\pi_3=[2,d-2]$.
\item \textcite{pakovich} solved the existence problem for $\len{\pi_n}=2$ and $\tSigma=\sphere{}$.
\end{itemize}
In particular, we will consistently exploit the results by \citeauthor{pakovich} as a base case for genus-reducing combinatorial moves. We will now state the relevant theorems, but we decide to omit the proofs: while the core ideas are very ingenious, filling in the details is quite tedious and time-consuming\footnote{The same could probably be said about the other proofs in this chapter, which cannot be omitted for obvious reasons.}. We refer the interested reader to \cite{pakovich}.

\begin{theorem}\label{short-partition:th:realizability-on-sphere-n-3}
Let $\DD=\datum{\sphere{}}{d}{\pi_1,\pi_2,[s,d-s]}$ be a candidate datum. Then $\DD$ is realizable unless it satisfies one of the following.
\begin{enumerate}[(1)]
\item $\DD=\datum{\sphere{}}{12}{[2,2,2,2,2,2],[1,1,1,3,3,3],[6,6]}$.
\item $\DD=\datum{\sphere{}}{2k}{[2,\ldots,2],[2,\ldots,2],[s,2k-s]}$ with $k\ge 2$, $s\neq k$.
\item $\DD=\datum{\sphere{}}{2k}{[2,\ldots,2],[1,2,\ldots,2,3],[k,k]}$ with $k\ge2$.
\item $\DD=\datum{\sphere{}}{4k+2}{[2,\ldots,2],[1,\ldots,1,k+1,k+2],[2k+1,2k+1]}$ with $k\ge 1$.
\item $\DD=\datum{\sphere{}}{4k}{[2,\ldots,2],[1,\ldots,1,k+1,k+1],[2k-1,2k+1]}$ with $k\ge2$.
\item\label{short-partition:th:realizability-on-sphere-n-3:it:6} $\DD=\datum{\sphere{}}{kh}{[h,\ldots,h],[1,\ldots,1,k+1],[lh,(k-l)h]}$ with $h\ge 2$, $k\ge 2$, $1\le l\le k-1$.
\end{enumerate}
\end{theorem}

\begin{theorem}\label{short-partition:th:realizability-on-sphere-n-ge-4}
Let $\DD=\datum{\sphere{}}{d}{\pi_1,\ldots,\pi_{n-1},[s,d-s]}$ be a candidate datum with $n\ge 4$. Then $\DD$ is realizable.
\end{theorem}

In the upcoming proofs, we will also make extensive use of the computational results from \cref{computational-results:ch}; although delegating work to the computer is never necessary (and in theory we could prove the same results by hand), doing so will save us a lot of effort in dealing with tricky corner-cases, and allow us to focus more on the reduction-oriented part of the proofs.

\section{Realizability on the torus for \texorpdfstring{$n=3$}{n=3}}

As anticipated in the title, this section deals with the cases where $n=3$ and $\tSigma=\surf{1}$. Due to the relatively large number of families of exceptional data listed in \cref{short-partition:th:realizability-on-sphere-n-3}, this is the instance of the problem which will require the heaviest casework.

\begin{theorem} \label{short-partition:th:realizability-on-torus-n-3}
Let $\DD=\datum{\surf{1}}{d}{\pi_1,\pi_2,[s,d-s]}$ be a candidate datum. Then $\DD$ is realizable unless it satisfies one of the following.
\begin{enumerate}[(1)]
\item $\DD=\datum{\surf{1}}{6}{[3,3],[3,3],[2,4]}$.
\item $\DD=\datum{\surf{1}}{8}{[2,2,2,2],[4,4],[3,5]}$.
\item $\DD=\datum{\surf{1}}{12}{[2,2,2,2,2,2],[3,3,3,3],[5,7]}$.
\item $\DD=\datum{\surf{1}}{16}{[2,2,2,2,2,2,2,2],[1,3,3,3,3,3],[8,8]}$.
\item $\DD=\datum{\surf{1}}{2k}{[2,\ldots,2],[2,\ldots,2,3,5],[k,k]}$ with $k\ge 5$.
\end{enumerate}
\end{theorem}
\begin{proof}
Thanks to \cref{monodromy:th:sphere-[1 d-1]} we can assume that $2\le s\le d-2$. If $d\le 16$, a computer-aided search (see the results in \cref{computational-results:sc:lists}) shows that the only exceptional cases are:
\begin{enumerate}[(1)]
\item $\DD=\datum{\surf{1}}{6}{[3, 3],[3, 3],[2, 4]}$;
\item $\DD=\datum{\surf{1}}{8}{[2, 2, 2, 2],[4, 4],[3, 5]}$;
\item $\DD=\datum{\surf{1}}{10}{[2, 2, 2, 2, 2],[2, 3, 5],[5, 5]}$;
\item $\DD=\datum{\surf{1}}{12}{[2, 2, 2, 2, 2, 2],[3, 3, 3, 3],[5, 7]}$;
\item $\DD=\datum{\surf{1}}{12}{[2, 2, 2, 2, 2, 2],[2, 2, 3, 5],[6, 6]}$;
\item $\DD=\datum{\surf{1}}{14}{[2, 2, 2, 2, 2, 2, 2],[2, 2, 2, 3, 5],[7, 7]}$;
\item $\DD=\datum{\surf{1}}{16}{[2, 2, 2, 2, 2, 2, 2, 2],[1, 3, 3, 3, 3, 3],[8, 8]}$;
\item $\DD=\datum{\surf{1}}{16}{[2, 2, 2, 2, 2, 2, 2, 2],[2, 2, 2, 2, 3, 5],[8, 8]}$.

\end{enumerate}
This is in agreement with the theorem statement, so we can assume that $d\ge 17$. We now analyze several cases.
\paragraph{Case 1.} Assume that:
\begin{assumptions}
\item $x\in\pi_1$ for some $x\ge 4$;
\item $2\in\pi_2$;
\item $\pi_2\neq[2,\ldots,2]$.
\end{assumptions}
\Cref{combinatorial-move:b:4 2} gives
\[
\DD\cmove\datum{\sphere{}}{d-2}{\pi_1\setminus[x]\cup[1,x-3],\pi_2\setminus[2],[s-1,d-s-1]},
\]
which is realizable by \cref{short-partition:th:realizability-on-sphere-n-3} unless one of the following holds\footnote{Since this proof is long enough as it is, we will refrain from explaining in detail why candidate data of a certain form are realizable by \cref{short-partition:th:realizability-on-sphere-n-3}; we leave the tedious yet elementary casework required to the motivated reader. In this situation, for instance, one could simply note that $\pi_1\setminus[x]\cup[1,x-3]$ and $\pi_2\setminus[2]$ are both different from $[2,\ldots,2]$, therefore the only exceptional data to consider among those listed in \cref{short-partition:th:realizability-on-sphere-n-3} are those belonging to family \ref{short-partition:th:realizability-on-sphere-n-3:it:6}.}.
\begin{itemize}
\item $\DD=\datum{\surf{1}}{kh+2}{[1,\ldots,1,k+4],[2,h,\ldots,h],[lh+1,(k-l)h+1]}$ with $k\ge 2$, $h\ge 3$, $1\le l\le k-1$. By applying \cref{combinatorial-move:b:4 2} we get
\[
\DD\cmove\datum{\sphere{}}{kh}{[1,\ldots,1,2,k],[h,\ldots,h],[lh,(k-l)h]},
\]
which is realizable by \cref{short-partition:th:realizability-on-sphere-n-3}.
\item $\DD=\datum{\surf{1}}{kh+2}{[1,\ldots,1,4,k+1],[2,h,\ldots,h],[lh+1,(k-l)h+1]}$ with $k\ge 2$, $h\ge 3$, $1\le l\le k-1$. By applying \cref{combinatorial-move:b:4 3} we get
\[
\DD\cmove\datum{\sphere{}}{kh}{[1,\ldots,1,2,k+1],[2,h-2,h,\ldots,h],[lh,(k-l)h]},
\]
which is realizable by \cref{short-partition:th:realizability-on-sphere-n-3}.
\end{itemize}

\paragraph{Case 2.} Assume that:
\begin{assumptions}
\item $x\in\pi_1$ for some $x\ge 4$;
\item $y\in\pi_2$ for some $y\ge 4$;
\item $2\not\in\pi_1$ and $2\not\in\pi_2$.
\end{assumptions}
\Cref{combinatorial-move:b:4 3} gives
\[
\DD\cmove\datum{\sphere{}}{d-2}{\pi_1\setminus[x]\cup[x-2],\pi_2\setminus[y]\cup[y-2],[s-1,d-s-1]},
\]
which is realizable by \cref{short-partition:th:realizability-on-sphere-n-3} unless
\[
\DD=\datum{\surf{1}}{kh+2}{[1,\ldots,1,k+3],[h,\ldots,h,h+2],[lh+1,(k-l)h+1]}
\]
for some $k\ge 2$, $h\ge 3$, $1\le l\le k-1$. If this is the case, applying \cref{combinatorial-move:b:4 3} yields
\[
\DD\cmove\datum{\sphere{}}{kh}{[1,\ldots,1,k+1],[h-2,h,\ldots,h,h+2],[lh,(k-l)h]},
\]
which is realizable by \cref{short-partition:th:realizability-on-sphere-n-3}.

\paragraph{Case 3.} Assume that:
\begin{assumptions}
\item $x\in\pi_1$ for some $x\ge 4$;
\item $\max(\pi_2)=3$;
\item $2\not\in\pi_2$.
\end{assumptions}
\Cref{combinatorial-move:b:4 3} gives
\[
\DD\cmove=\datum{\sphere{}}{d-2}{\pi_1\setminus[x]\cup[x-2],\pi_2\setminus[3]\cup[1],[s-1,d-s-1]},
\]
which is realizable by \cref{short-partition:th:realizability-on-sphere-n-3} unless
\[
\DD=\datum{\surf{1}}{2h+2}{[h,h+2],[1,\ldots,1,3,3],[h+1,h+1]}
\]
for some $h\ge 8$ (recall that we are assuming $d\ge 17$). If this is the case, applying \Cref{combinatorial-move:b:[1 1 3]} yields
\[
\DD\cmove\datum{\sphere{}}{2h+2}{[h,h+2],[1,\ldots,1,3],[h+1,h+1]},
\]
which is realizable by \cref{short-partition:th:realizability-on-sphere-n-3}.

\paragraph{Case 4.} Assume that:
\begin{assumptions}
\item $\max(\pi_1)=3$;
\item $\max(\pi_2)\le 3$;
\item $\pi_2\neq[2,\ldots,2]$.
\end{assumptions}
We analyze a few sub-cases.
\begin{itemize}
\item \textbf{Case 4.1:} $[1,1]\subs\pi_1$. \Cref{combinatorial-move:b:[1 1 3]} gives
\[
\DD\cmove\datum{\sphere{}}{d}{\pi_1\setminus[3]\cup[1,1,1],\pi_2,[s,d-s]},
\]
which is realizable by \cref{short-partition:th:realizability-on-sphere-n-3}.
\item \textbf{Case 4.2:} $[3,3]\subs\pi_1$ and $[2,2]\subs\pi_2$. If $s=2$ or $s=d-2$ then $\DD$ is realizable by \cref{dessins:th:special-case-[2 d-2]}. Otherwise, \cref{combinatorial-move:b:[3 3] [2 2]} gives
\[
\DD\cmove\datum{\sphere{}}{d-4}{\pi_1\setminus[3,3]\cup[1,1],\pi_2\setminus[2,2],[s-2,d-s-2]},
\]
which is realizable by \cref{short-partition:th:realizability-on-sphere-n-3}.
\item \textbf{Case 4.3:} $[2,2]\not\subs\pi_2$. The \RH{} formula immediately implies that $3\in\pi_2$. Assume by contradiction that $\DD$ is not realizable. If this is the case, $[1,1]\not\subs\pi_2$ by case 4.1, but then $[3,3]\subs\pi_2$. It follows (case 4.2) that $[2,2]\not\subs\pi_1$; moreover, case 4.1 also implies that $[1,1]\not\subs\pi_1$. In other words, both $\pi_1$ and $\pi_2$ can be written as $\rho\cup[3,\ldots,3]$, where $\rho\subs[1,2]$ (possibly different for $\pi_1$ and $\pi_2$). As a consequence, we have the inequalities
\begin{align*}
d\ge 3\len{\pi_1}-3,&&d\ge 3\len{\pi_2}-3,
\end{align*}
which contradict the \RH{} formula if $d\ge 13$.
\item \textbf{Case 4.4:} $[3,3]\not\subs\pi_1$. From the \RH{} formula it follows that $[3,3]\subs\pi_2$, but then $\DD$ is realizable by case 4.1.
\end{itemize}

\paragraph{Case 5.} Assume that:
\begin{assumptions}
\item $\max(\pi_1)\ge 4$;
\item $\pi_2=[2,\ldots,2]$.
\end{assumptions}
Let $x=\max(\pi_1)$; \cref{combinatorial-move:b:4 2} gives
\[
\DD\cmove\datum{\sphere{}}{d-2}{\pi_1\setminus[x]\cup[1,x-3],[2,\ldots,2],[s-1,d-s-1]},
\]
which is realizable by \cref{short-partition:th:realizability-on-sphere-n-3} unless one of the following holds.
\begin{itemize}
\item $\DD=\datum{\surf{1}}{2k+2}{[2,\ldots,2,3,5],[2,\ldots,2],[k+1,k+1]}$ with $k\ge 7$. This is in fact one of the exceptional data listed in the statement.
\item $\DD=\datum{\surf{1}}{2k+2}{[2,\ldots,2,6],[2,\ldots,2],[k+1,k+1]}$ with $k\ge 7$. By applying \cref{combinatorial-move:b:4 2} we get
\[
\DD\cmove\datum{\sphere{}}{2k}{[2,\ldots,2],[2,\ldots,2],[k,k]},
\]
which is realizable by \cref{short-partition:th:realizability-on-sphere-n-3}.
\item $\DD=\datum{\surf{1}}{4k+4}{[1,\ldots,1,k+2,k+4],[2,\ldots,2],[2k+2,2k+2]}$ with $k\ge 3$. By applying \cref{combinatorial-move:b:4 2} we get
\[
\DD\cmove\datum{\sphere{}}{4k+2}{[1,\ldots,1,2,k,k+2],[2,\ldots,2],[2k+1,2k+1]},
\]
which is realizable by \cref{short-partition:th:realizability-on-sphere-n-3}.
\item $\DD=\datum{\surf{1}}{4k+4}{[1,\ldots,1,k+1,k+5],[2,\ldots,2],[2k+2,2k+2]}$ with $k\ge 3$. By applying \cref{combinatorial-move:b:4 2} we get
\[
\DD\cmove\datum{\sphere{}}{4k+2}{[1,\ldots,1,2,k+1,k+1],[2,\ldots,2],[2k+1,2k+1]},
\]
which is realizable by \cref{short-partition:th:realizability-on-sphere-n-3}.
\item $\DD=\datum{\surf{1}}{4k+2}{[1,\ldots,1,k+1,k+4],[2,\ldots,2],[2k,2k+2]}$ with $k\ge 4$. By applying \cref{combinatorial-move:b:4 2} we get
\[
\DD\cmove\datum{\sphere{}}{4k}{[1,\ldots,1,2,k,k+1],[2,\ldots,2],[2k-1,2k+1]},
\]
which is realizable by \cref{short-partition:th:realizability-on-sphere-n-3}.
\item $\DD=\datum{\surf{1}}{2k+2}{[1,\ldots,1,k+4],[2,\ldots,2],[2l+1,2(k-l)+1]}$ with $k\ge 7$, $1\le l\le k-1$. By applying \cref{combinatorial-move:b:4 2} we get
\[
\DD\cmove\datum{\sphere{}}{2k}{[1,\ldots,1,2,k],[2,\ldots,2],[2l,2(k-l)]},
\]
which is realizable by \cref{short-partition:th:realizability-on-sphere-n-3}.
\end{itemize}

\paragraph{Case 6.} Assume that:
\begin{assumptions}
\item $\max(\pi_1)=3$;
\item $\pi_2=[2,\ldots,2]$.
\end{assumptions}
The \RH{} formula immediately implies that $[3,3,3,3]\subs\pi_1$. If $s=2$ or $s=d-2$ then $\DD$ is realizable by \cref{dessins:th:special-case-[2 d-2]}. Otherwise, \cref{combinatorial-move:b:[3 3] [2 2]} gives
\[
\DD\cmove\datum{\sphere{}}{d-4}{\pi_1\setminus[3,3]\cup[1,1],[2,\ldots,2],[s-2,d-s-2]},
\]
which is realizable by \cref{short-partition:th:realizability-on-sphere-n-3}, since we are assuming that $d\ge 17$.

The cases we have analyzed, up to swapping $\pi_1$ and $\pi_2$, cover all the candidate data of the form $\datum{\surf{1}}{d}{\pi_1,\pi_2,[s,d-s]}$. We have shown that every datum which is not listed in the statement is realizable, therefore the proof is complete.
\end{proof}

\section{Realizability on higher genus surfaces for \texorpdfstring{$n=3$}{n=3}}

It turns out that, for $n=3$ and $\len{\pi_3}=2$, there are no exceptional data on surfaces with genus $g\ge 2$.

\begin{theorem}\label{short-partition:th:realizability-on-higher-genus-n-3}
Let $\DD=\datum{\surf{g}}{d}{\pi_1,\pi_2,[s,d-s]}$ be a candidate datum with $g\ge 2$. Then $\DD$ is realizable.
\end{theorem}
\begin{proof}
Thanks to \cref{monodromy:th:sphere-[1 d-1]} we can assume that $2\le s\le d-2$. For $d\le 18$, a computer-aided search shows that there are no exceptional data (see the results in \cref{computational-results:sc:lists}). Therefore, we can further assume that $d\ge 19$. We proceed by induction on $g\ge 2$, analyzing several cases.

\paragraph{Case 1.} Assume that:
\begin{assumptions}
\item $x\in\pi_1$ for some $x\ge 4$;
\item $2\in\pi_2$.
\end{assumptions}
\Cref{combinatorial-move:b:4 2} gives
\[
\DD\cmove\datum{\surf{g-1}}{d-2}{\pi_1\setminus[x]\cup[1,x-3],\pi_2\setminus[2],[s-1,d-s-1]},
\]
which is realizable by \cref{short-partition:th:realizability-on-torus-n-3} if $g=2$, or by induction if $g\ge 3$.

\paragraph{Case 2.} Assume that:
\begin{assumptions}
\item $x\in\pi_1$ for some $x\ge 4$;
\item $y\in\pi_2$ for some $y\ge 3$;
\item $2\not\in\pi_1$ and $2\not\in\pi_2$.
\end{assumptions}
\Cref{combinatorial-move:b:4 3} gives
\[
\DD\cmove\datum{\surf{g-1}}{d-2}{\pi_1\setminus[x]\cup[x-2],\pi_2\setminus[y]\cup[y-2],[s-1,d-s-1]},
\]
which is realizable by \cref{short-partition:th:realizability-on-torus-n-3} if $g=2$, or by induction if $g\ge 3$.

\paragraph{Case 3.} Assume that:
\begin{assumptions}
\item $\max(\pi_1)=3$;
\item $\max(\pi_2)\le 3$.
\end{assumptions}
We analyze a few sub-cases.
\begin{itemize}
\item \textbf{Case 3.1:} $[1,1]\subs\pi_1$. \Cref{combinatorial-move:b:[1 1 3]} gives
\[
\DD\cmove\datum{\surf{g-1}}{d}{\pi_1\setminus[3]\cup[1,1,1],\pi_2,[s,d-s]},
\]
which is realizable by \cref{short-partition:th:realizability-on-torus-n-3} if $g=2$, or by induction if $g\ge 3$.
\item\textbf{Case 3.2:} $[3,3]\subs\pi_1$ and $[2,2]\subs\pi_2$. If $s=2$ or $s=d-2$ then $\DD$ is realizable by \cref{dessins:th:special-case-[2 d-2]}. Otherwise, \cref{combinatorial-move:b:[3 3] [2 2]} gives
\[
\DD\cmove\datum{\surf{g-1}}{d-4}{\pi_1\setminus[3,3]\cup[1,1],\pi_2\setminus[2,2],[s-2,d-s-2]},
\]
which is realizable by \cref{short-partition:th:realizability-on-torus-n-3} if $g=2$, or by induction if $g\ge 3$.
\item \textbf{Case 3.3:} $[2,2]\not\subs\pi_1$. The \RH{} formula immediately implies that $[3,3]\subs\pi_2$. Assume by contradiction that $\DD$ is not realizable. Then $[1,1]\not\subs\pi_1$ and $[1,1]\not\subs\pi_2$ (case 3.1), and moreover $[2,2]\not\subs\pi_1$ (case 3.2). In other words, both $\pi_1$ and $\pi_2$ can be written as $\rho\cup[3,\ldots,3]$, where $\rho\subs[1,2]$ (possibly different for $\pi_1$ and $\pi_2$). It is then easy to see that $\DD$ belongs to one of the families listed in \cref{dessins:th:special-families} and, therefore, is realizable.
\item \textbf{Case 3.4:} $[3,3]\not\subs\pi_1$. From the \RH{} formula it follows that $[3,3]\subs\pi_2$, but then $\DD$ is realizable by cases 3.2 and 3.3.
\end{itemize}
The cases we have analyzed, up to swapping $\pi_1$ and $\pi_2$, cover all the candidate data of the form $\datum{\surf{g}}{d}{\pi_1,\pi_2,[s,d-s]}$ with $g\ge 2$. We have shown that every datum is realizable, therefore the proof is complete.
\end{proof}

\section{Realizability for \texorpdfstring{$n\ge 4$}{n≥4}}

After solving the existence problem for $n=3$ and $\len{\pi_3}=2$, we turn to candidate data with $n\ge 4$ partitions. As we are about to see, exceptional data in this setting are very rare: there is an infinite family with $d=4$, which we already encountered in \cref{monodromy:th:sphere-d-equals-4}, and a single datum with $n=4$ and $d=8$.

\begin{theorem}\label{short-partition:th:realizability-on-higher-genus-n-ge-4}
Let $\DD=\datum{\surf{g}}{d}{\pi_1,\ldots,\pi_{n-1},[s,d-s]}$ be a candidate datum with $n\ge 4$. Then $\DD$ is realizable unless it satisfies one of the following.
\begin{enumerate}[(1)]
\item $\DD=\datum{\surf{2}}{8}{[2,2,2,2],[2,2,2,2],[2,2,2,2],[3,5]}$.
\item $\DD=\datum{\surf{n-3}}{4}{[2,2],\ldots,[2,2],[1,3]}$.
\end{enumerate}
\end{theorem}
\begin{proof}
If $d\le 16$, a computer-aided search (see the results in \cref{computational-results:sc:lists}) shows that the only exceptional cases are:
\begin{enumerate}[(1)]
\item $\DD=\datum{\surf{1}}{4}{[2, 2],[2, 2],[2, 2],[1, 3]}$;
\item $\DD=\datum{\surf{2}}{8}{[2, 2, 2, 2],[2, 2, 2, 2],[2, 2, 2, 2],[3, 5]}$.
\end{enumerate}
This is in agreement with the statement, so we can assume that $d\ge 17$. Moreover, every candidate datum with $g=0$ is realizable by \cref{short-partition:th:realizability-on-sphere-n-ge-4}; as a consequence, we only have to consider the cases where $g\ge 1$.

We will proceed by induction on $n$. We start with the base case $n=4$, which requires the heaviest casework. Fix a candidate datum $\DD=\datum{\surf{g}}{d}{\pi_1,\pi_2,\pi_3,[s,d-s]}$.
\begin{itemize}
\item Assume that the inequality $v(\pi_i)+v(\pi_j)<d$ holds for a pair of indices $1\le i<j\le 3$; up to reindexing, we can assume that $v(\pi_1)+v(\pi_2)<d$. \Cref{combinatorial-move:a:small-v} gives
\[
\DD\cmove\DD'=\datum{\surf{g}}{d}{\pi_1',\pi_3,[s,d-s]}
\]
for a suitable candidate datum $\DD'$, where $v(\pi_1')=v(\pi_1)+v(\pi_2)$. If $g\ge 2$, then $\DD'$ is realizable by \cref{short-partition:th:realizability-on-higher-genus-n-3}. If instead $g=1$, then $\DD'$ is realizable by \cref{short-partition:th:realizability-on-torus-n-3} unless
\[
\text{$\DD'=\datum{\surf{1}}{2k}{[2,\ldots,2],[2,\ldots,2,3,5],[k,k]}$ with $2k=d$.}
\]
If this is the case, then $\pi_4=[k,k]$ and $\{\pi_1',\pi_3\}=\{[2,\ldots,2],[2,\ldots,2,3,5]\}$. Some more casework is required to show that $\DD'$ can actually be chosen to be realizable; since $\DD\cmove\DD'$, this will imply that $\DD$ is realizable as well.
\begin{itemize}
\item If $\pi_1'=[2,\ldots,2]$, then $v(\pi_1')=k$ and $v(\pi_3)=k+2$. Assume without loss of generality that $v(\pi_1)\le k/2$. We have that
\[
k+2<1+k+2\le v(\pi_1)+v(\pi_3)\le \frac{k}{2}+k+2<d.
\]
Repeating the construction with $i=1$ and $j=3$ will yield a realizable $\DD'$.
\item If $\pi_3=[2,\ldots,2]$ and $v(\pi_1)\not\in\{2,k,k+1\}$ then $v(\pi_1)+v(\pi_3)<d$ and $v(\pi_1)+v(\pi_3)\not\in\{k,k+2\}$. Therefore, repeating the construction with $i=1$ and $j=3$ will yield a realizable $\DD'$.
\item If $\pi_3=[2,\ldots,2]$, $v(\pi_1)=2$ and $\pi_2\neq [2,\ldots,2]$, then repeating the construction with $i=1$ and $j=3$ will yield a realizable $\DD'$.
\item If $\pi_2=\pi_3=[2,\ldots,2]$ and $\pi_1\neq[1,\ldots,1,3]$, we follow a different approach. By applying \cref{combinatorial-move:a:large-v} to the partitions $\pi_2$ and $\pi_3$ we get
\[
\DD\cmove\datum{\sphere{}}{2k}{\pi_1,[k,k],[k,k]},
\]
which is realizable by \cref{short-partition:th:realizability-on-sphere-n-3}.
\item Finally, if $\pi_1=[1,\ldots,1,3]$ and $\pi_2=\pi_3=[2,\ldots,2]$, we have to work explicitly with permutations. Consider
\begin{align*}
\alpha_1=\cycle{1,3,5},&&\alpha_2=\cycle{1,2}\cycle{3,4}\cdots\cycle{2k-1,2k}.
\end{align*}
Clearly $[\alpha_1]=\pi_1$ and $[\alpha_2]=\pi_2$; moreover,
\[
\alpha_1\alpha_2=\cycle{1,2,3,4,5,6}\cycle{7,8}\cdots\cycle{2k-1,2k},
\]
so $[\alpha_1\alpha_2]=[2,\ldots,2,6]$. Note that
\[
v(\alpha_1)+v(\alpha_2)=2+k=v(\alpha_1\alpha_2),
\]
so \cref{monodromy:rm:combinatorial-move:a:small-v} gives
\[
\DD\cmove\datum{\surf{1}}{2k}{[2,\ldots,2,6],[2,\ldots,2],[k,k]},
\]
which is realizable by \cref{short-partition:th:realizability-on-torus-n-3}.
\end{itemize}

Up to swapping $\pi_1$ and $\pi_2$, this analysis covers all the possible cases.
\item Otherwise, the inequality $v(\pi_i)+v(\pi_j)\ge d$ holds for every $1\le i<j\le 3$. In particular, up to reindexing, we can assume that $v(\pi_3)\ge d/2$. Note that $v(\pi_1)+v(\pi_2)\ge d$ and $v(\pi_3)+v([s,d-s])\ge 1+d-2=d-1$, so \cref{combinatorial-move:a:large-v} gives
\[
\DD\cmove\DD'=\datum{\surf{g'}}{d}{\pi_1',\pi_3,[s,d-s]}
\]
for a suitable candidate datum $\DD'$ with $v(\pi_1')\ge d-2$. We can actually compute
\[
g'=\frac{1}{2}(v(\pi_1')+v(\pi_3)+v([s,d-s])-d+1\ge\frac{1}{2}\left(d-2+\frac{d}{2}+d-2\right)-d+1=\frac{d}{4}-1\ge 2.
\]
Therefore $\DD'$ is realizable by \cref{short-partition:th:realizability-on-higher-genus-n-3}, and $\DD$ is realizable as well.
\end{itemize}

We now turn to the case $n\ge 5$; we show by induction that every candidate datum $\DD=\datum{\surf{g}}{d}{\pi_1,\ldots,\pi_{n-1},[s,d-s]}$  different from $\datum{\surf{n-3}}{4}{[2,2],\ldots,[2,2],[1,3]}$ is realizable. The case $d=4$ is addressed by \cref{monodromy:th:sphere-d-equals-4}. If $n=5$ and $d=8$, the computational results in \cref{computational-results:sc:lists} imply that $\DD$ is realizable. Otherwise, the routine reduction argument relying on \cref{combinatorial-move:a:small-v,combinatorial-move:a:large-v} shows that $\DD$ is realizable. \qedhere
\end{proof}


\section{Exceptionality}

This section is devoted to showing that the candidate data listed in the statements of \cref{short-partition:th:realizability-on-sphere-n-3,short-partition:th:realizability-on-torus-n-3,short-partition:th:realizability-on-higher-genus-n-ge-4} are in fact exceptional. First of all, we address the special cases with small degree ($d\le 16$). The computational results in \cref{computational-results:sc:lists} show that the following candidate data are exceptional.
\begin{enumerate}[(1)]
\item $\DD=\datum{\sphere{}}{12}{[2,2,2,2,2,2],[1,1,1,3,3,3],[6,6]}$.
\item $\DD=\datum{\surf{1}}{6}{[3,3],[3,3],[2,4]}$.
\item $\DD=\datum{\surf{1}}{8}{[2,2,2,2],[4,4],[3,5]}$.
\item $\DD=\datum{\surf{1}}{12}{[2,2,2,2,2,2],[3,3,3,3],[5,7]}$.
\item $\DD=\datum{\surf{1}}{16}{[2,2,2,2,2,2,2,2],[1,3,3,3,3,3],[8,8]}$.
\item $\DD=\datum{\surf{2}}{8}{[2,2,2,2],[2,2,2,2],[2,2,2,2],[3,5]}$.
\end{enumerate}

We now turn to infinite families of exceptional data.

\begin{proposition}\label{short-partition:th:exceptional-d-4}
Let $n\ge 3$ be a positive integer. Then the candidate datum
\[
\DD=\datum{\surf{n-3}}{4}{[2,2],\ldots,[2,2],[1,3]}
\]
is exceptional.
\end{proposition}
\begin{proof}
In order to show the exceptionality of $\DD$, we will employ the monodromy approach. Let $\alpha_1,\ldots,\alpha_{n-1}\in\symgroup[4]$ be permutations matching $[2,2]$. It is easy to see that the three permutations matching $[2,2]$, together with the identity, form a subgroup
\[
\{\id,\cycle{1,2}\cycle{3,4},\cycle{1,3}\cycle{2,4},\cycle{1,4}\cycle{2,3}\}\subgroup\symgroup[4],
\]
which does not contain any $3$\=/cycles. As a consequence, the product $\alpha_1\cdots\alpha_{n-1}$ cannot match $[1,3]$; by \cref{hurwitz:th:monodromy-realizability-orientable}, this implies that $\DD$ is exceptional.
\end{proof}

For the other families of exceptional data, we will instead make use of \dessins{}.

\begin{proposition}\label{short-partition:th:exceptional-composite}
Let $h\ge 2$, $k\ge 2$, $1\le l\le k-1$ be integers. Then the candidate datum
\[
\DD=\datum{\sphere{}}{kh}{[h,\ldots,h],[1,\ldots,1,k+1],[lh,(k-l)h]}
\]
is exceptional.
\end{proposition}
\begin{proof}
Assume by contradiction that there is a \dessin{} $\Gamma\subs\sphere{}$ realizing $\DD$. Let $v$ be the white vertex of degree $k+1$. Since $\Gamma$ is connected, each of the $k$ black vertices must have an edge connecting it to $v$. Therefore there are exactly $k-1$ black vertices with one edge between them and $v$, and one black vertex with two edges between it and $v$; we call this special black vertex $u$. The two edges connecting $u$ and $v$ define the two complementary disks of $\Gamma$. As shown in the picture below, there are (excluding $u$) $a$ black vertices inside $D_1$ and $k-1-a$ black vertices inside $D_2$, for some integer $0\le a\le k-1$; of the $h-2$ white vertices of degree $1$ connected to $u$, $b$ lie inside $D_1$, while $h-2-b$ lie inside $D_2$, for some $0\le b\le h-2$. It is easy to compute the perimeters:
\begin{align*}
\frac{1}{2}\card{\partial D_1}=ah+b+1,&&\frac{1}{2}\card{\partial D_2}=(k-1-a)h+(h-2-b)+1.
\end{align*}
In particular, we see that $\card{\partial D_1}/2$ and $\card{\partial D_2}/2$ are not divisible by $h$, hence $\Gamma$ cannot be a \dessin{} realizing $\DD$.
\end{proof}

The following result will allow us to deal with candidate data of the form $\datum{\surf{g}}{d}{[2,\ldots,2],\pi_2,\pi_3}$.

\begin{lemma}\label{short-partition:th:lemma-[2 ... 2]}
Let $\DD=\datum{\surf{g}}{d}{[2,\ldots,2],\pi_2,\pi_3}$ be a combinatorial datum. Then $\DD$ is realizable if and only if there exists a graph $\Gamma$ embedded in $\surf{g}$ such that:
\begin{itemize}
\item $\pi_2=[k(v_1),\ldots,k(v_r)]$, where $v_1,\ldots,v_r$ are the vertices of $\Gamma$;
\item the complementary regions of $\Gamma$ are topological disks $D_1,\ldots,D_h$;
\item $\pi_3=[\card{\partial D_1},\ldots,\card{\partial D_h}]$.
\end{itemize}
\end{lemma}
\begin{proof}
The key idea is that vertices of degree $2$ do not contribute to the topology of a graph in any meaningful way.

If are given a \dessin{} $\Gamma'\subs\surf{g}$ realizing $\DD$, we can obtain a suitable graph $\Gamma$ simply by removing all the black vertices, and merging the two edges adjacent to each one of them into a single edge.

Conversely, if we are given a graph $\Gamma\subs\surf{g}$, we can recover a \dessin{} $\Gamma'$ by coloring the vertices of $\Gamma$ with the color white, and inserting a black vertex in the middle of every edge.
\end{proof}

This lemma suggests an approach by enumeration for showing the exceptionality of candidate data of the form $\datum{\surf{g}}{d}{[2,\ldots,2],\pi_2,\pi_3}$. In fact, if the number of fat graphs whose degrees are the entries of $\pi_2$ is small, we can check them one by one to see if the associated embedded graph satisfies the conditions of \cref{short-partition:th:lemma-[2 ... 2]}.

\begin{proposition}\label{short-partition:th:exceptional-sphere}
The following families of candidate data are exceptional.
\begin{enumerate}[(1)]
\item $\DD=\datum{\sphere{}}{2k}{[2,\ldots,2],[2,\ldots,2],[s,2k-s]}$ with $k\ge 2$, $s\neq k$.
\item $\DD=\datum{\sphere{}}{2k}{[2,\ldots,2],[1,2,\ldots,2,3],[k,k]}$ with $k\ge2$.
\item $\DD=\datum{\sphere{}}{4k+2}{[2,\ldots,2],[1,\ldots,1,k+1,k+2],[2k+1,2k+1]}$ with $k\ge 1$.
\item $\DD=\datum{\sphere{}}{4k}{[2,\ldots,2],[1,\ldots,1,k+1,k+1],[2k-1,2k+1]}$ with $k\ge2$.
\end{enumerate}
\end{proposition}
\begin{proof}
For each family listed in the statement, we follow the approach by enumeration presented above: we draw all the possible graphs embedded in $\sphere{}$ whose vertices have the entries of $\pi_2$ as degrees and whose complementary regions are two disks, and then we compute the perimeters of these disks, showing that they cannot be equal to the entries of $\pi_3$.
\begin{enumerate}[(1)]
\item There is only one graph embedded in $\sphere{}$ whose vertices have degrees $[2,\ldots,2]$.

By \cref{short-partition:th:lemma-[2 ... 2]}, it follows that $\DD$ is exceptional unless $s=k$.
\item A graph embedded in $\sphere{}$ whose vertices have degrees $[1,2,\ldots,2,3]$ can be represented by a diagram like the following, where $a$ denotes the number of blue edges.

The perimeters of the complementary disks are easily computed as
\begin{align*}
\card{\partial D_1}=k-a-1,&&\card{\partial D_2}=k+a+1,
\end{align*}
hence $\DD$ is exceptional by \cref{short-partition:th:lemma-[2 ... 2]}.
\item There are only three kinds of graphs embedded in $\sphere$ whose vertices have degrees $[1,\ldots,1,k+1,k+2]$.\todo{Probably a table?}
First case:
\begin{align*}
\card{\partial D_1}=4k-2a-2b,&&\card{\partial D_2}=2a+2b+2.
\end{align*}
Second case:
\begin{align*}
\card{\partial D_1}=2k-2b-1,&&\card{\partial D_2}=2k+2b+3.
\end{align*}
Third case:
\begin{align*}
\card{\partial D_1}=2k-2a+1,&&\card{\partial D_2}=2k+2a+1.
\end{align*}
\item A graph embedded in $\sphere$ whose vertices have degrees $[1,\ldots,1,k+1,k+2]$ can be represented by a diagram like the following, where $a$ denotes the number of blue edges connected to the vertex of degree $k+1$, and $b$ denotes the number of blue edges connected to the vertex of degree $k+2$.

The perimeters of the complementary disks are easily computed as
\begin{align*}
\card{\partial D_1}=k-a-1,&&\card{\partial D_2}=k+a+1,
\end{align*}
hence $\DD$ is exceptional by \cref{short-partition:th:lemma-[2 ... 2]}.
\end{enumerate}
\end{proof}

\begin{proposition}\label{short-partition:th:exceptional-torus}
Let $k\ge 5$ be an integer. Then the candidate datum
\[
\DD=\datum{\surf{1}}{2k}{[2,\ldots,2],[2,\ldots,2,3,5],[k,k]}
\]
is exceptional.
\end{proposition}
\begin{proof}
We start by enumerating all the abstract graphs whose vertices have degrees $[2,\ldots,2,3,5]$, without considering the embedding in $\surf{1}$. Once again, we observe that vertices of degree $2$ do not contribute to the topology of the graph, so we can reduce the number of cases to two.

\tikzset{external/force remake=false}
\def\myfirstradius{1.2}
\def\mysecondradius{1.8}
\def\drawtwovertices{
\path (0,0) coordinate (w1) pic{white vertex};
\path (\myradius,0) coordinate (w2) pic{white vertex};
}
\begin{center}
\setlength\tabcolsep{3em}
\tikzset{
quick/.style={simple edge=black edge}
}
\begin{tabular}{@{}cc@{}}
(1)&(2)\\
\tikzsetnextfilename{exceptionality-torus-first-graph}
\begin{tikzpicture}[baseline=0pt,graph picture]
\def\myradius{\myfirstradius}
\drawtwovertices
\path[quick] (w1) -- (w2);
\path[quick,every to/.style={distance=1.5*\myradius cm}] (w1) to[out=135,in=-135] (w1) (w2) to[out=120,in=30] (w2) to[out=-120,in=-30] (w2);
\begin{scope}[x={(\myradius,0)},y={(0,\myradius)}]
\node at (-0.5,0.5) {$a$};
\node at (0.4,0.2) {$b$};
\node at (1.7,0.7) {$c$};
\node at (1.7,-0.7) {$e$};
\end{scope}
\end{tikzpicture}&
\tikzsetnextfilename{exceptionality-torus-second-graph}
\begin{tikzpicture}[baseline=0pt,graph picture]
\def\myradius{\mysecondradius}
\drawtwovertices
\path[quick] (w1) -- (w2);
\path[quick] (w1) arc(180:0:{0.5*\myradius});
\path[quick] (w1) arc(-180:0:{0.5*\myradius});
\path[quick] (w2) to[out=45,in=-45,distance=\myradius cm] (w2);
\begin{scope}[x={(\myradius,0)},y={(0,\myradius)}]
\node at (1.5,0.3) {$a$};
\node at (0.3,0.6) {$b$};
\node at (0.5,0.1) {$c$};
\node at (0.3,-0.6) {$e$};
\end{scope}
\end{tikzpicture}
\end{tabular}
\end{center}

In the pictures above, the arcs are labeled with a letter representing their length (in terms of edges) or, in other words, the number of degree-$2$ vertices that lie on them plus $1$. In particular, $a+b+c+e=k$. We now analyze the two cases separately.
\NewDocumentEnvironment{dessintext}{}{\begin{longtable}{@{}p{0.5\linewidth}@{}p{0.5\linewidth}@{}}}{\end{longtable}}
\begin{enumerate}[(1)]
\tikzset{
contour edge settings/quick/.style 2 args={left=disk #1 boundary,right=disk #2 boundary,edge=black edge}
}
\item There is only one embedding of this graph in $\surf{1}$ whose complementary regions are two discs.
\begin{dessintext}
\makecell[c]{
\tikzsetnextfilename{exceptionality-torus-first-graph-1}
\begin{tikzpicture}[graph picture,every to/.style={distance=1.5*\myradius cm}]
\def\myradius{\myfirstradius}
\drawtwovertices
\begin{scope}[x={(\myradius,0)},y={(0,\myradius)}]
\node at (-0.6,0.6) {$a$};
\node at (0.5,0.3) {$b$};
\node at (1.6,0.7) {$c$};
\node at (1.6,-0.7) {$e$};
\path[use as bounding box] (-1,0) rectangle (2,0);
\end{scope}
\path[use as bounding box];
\drawtwovertices
\path[contour edge={quick={2}{2}}] (w1) -- (w2);
\path[contour edge={quick={1}{2}}] (w1) to[out=135,in=-135] (w1);
\path[contour edge={quick={2}{2}}] (w2) to[out=70,in=-30] (w2);
\path[contour edge={quick={2}{2},above}] (w2) to[out=-70,in=30] (w2);
\end{tikzpicture}}&$\begin{aligned}
&a,b,c,e\ge 1\\
&a+b+c+e=k\\
&\card{\partial D_1}=a<k
\end{aligned}$
\end{dessintext}
\item There are three topologically different embedding of this graph in $\surf{1}$ whose complementary regions are two disks.
\begin{dessintext}
\makecell[c]{
\tikzsetnextfilename{exceptionality-torus-second-graph-1}
\begin{tikzpicture}[graph picture]
\def\myradius{\mysecondradius}
\drawtwovertices
\begin{scope}[x={(\myradius,0)},y={(0,\myradius)}]
\node at (1.5,0.3) {$a$};
\node at (0.2,0.45) {$b$};
\node at (0.8,0.45) {$c$};
\node at (0.3,-0.6) {$e$};
\path[use as bounding box] (-0.1,0) rectangle (1.7,0);
\end{scope}
\path[use as bounding box];
\path[contour edge={above,quick={2}{2},left bcap=90}] (w1) to[out=60,in=-150,looseness=2.5] (w2);
\path[contour edge={quick={2}{2}}] (w1) to[out=-30,in=120,looseness=2.5] (w2);
\path[contour edge={quick={2}{2}}] (w1) arc(-180:0:{0.5*\myradius});
\path[contour edge={quick={2}{1}}] (w2) to[out=45,in=-45,distance=\myradius cm] (w2);
\end{tikzpicture}}&$\begin{aligned}
&a,b,c,e\ge 1\\
&a+b+c+e=k\\
&\card{\partial D_1}=a<k
\end{aligned}$\\
\makecell[c]{
\tikzsetnextfilename{exceptionality-torus-second-graph-2}
\begin{tikzpicture}[graph picture]
\def\myradius{\mysecondradius}
\drawtwovertices
\begin{scope}[x={(\myradius,0)},y={(0,\myradius)}]
\node at (1.5,0.3) {$a$};
\node at (0.3,0.65) {$b$};
\node at (0.5,0.15) {$c$};
\node at (0.3,-0.6) {$e$};
\path[use as bounding box] (-0.1,0) rectangle (1.7,0);
\end{scope}
\path[use as bounding box];
\path[contour edge={above,quick={1}{2},left bcap=90}] (w1) to (w2);
\path[contour edge={quick={2}{1}}] (w1) arc(180:0:{0.5*\myradius});
\path[contour edge={above,quick={2}{2}}] (w1) to[out=-90,in=-30,out looseness=2,in looseness=2.5] (w2);
\path[contour edge={quick={2}{2}}] (w2) to[out=30,in=-90,distance=1.5*\myradius cm] (w2);
\end{tikzpicture}
}&$\begin{aligned}
&a,b,c,e\ge 1\\
&a+b+c+e=k\\
&\card{\partial D_1}=b+c<k
\end{aligned}$\\
\makecell[c]{
\tikzsetnextfilename{exceptionality-torus-second-graph-3}
\begin{tikzpicture}[graph picture]
\def\myradius{\mysecondradius}
\drawtwovertices
\begin{scope}[x={(\myradius,0)},y={(0,\myradius)}]
\node at (1.5,0.3) {$a$};
\node at (0.2,0.45) {$b$};
\node at (0.8,0.45) {$c$};
\node at (0.3,-0.6) {$e$};
\path[use as bounding box] (-0.1,0) rectangle (1.7,0);
\end{scope}
\path[use as bounding box];
\path[contour edge={above,quick={2}{1},left bcap=90}] (w1) to[out=60,in=-150,looseness=2.5] (w2);
\path[contour edge={quick={1}{2}}] (w1) to[out=-30,in=120,looseness=2.5] (w2);
\path[contour edge={above,quick={2}{2}}] (w1) to[out=-90,in=-30,out looseness=2,in looseness=2.5] (w2);
\path[contour edge={quick={1}{2}}] (w2) to[out=30,in=-90,distance=1.5*\myradius cm] (w2);
\end{tikzpicture}
}&$\begin{aligned}
&a,b,c,e\ge 1\\
&a+b+c+e=k\\
&\card{\partial D_1}=a+b+c<k
\end{aligned}$
\end{dessintext}
\end{enumerate}

In all of the cases, we see that $\card{\partial D_1}\neq k$; if follows that there is no \dessin{} $\Gamma\subs\surf{1}$ such that $\DD(\Gamma)=\DD$.
\tikzset{external/force remake=false}
\end{proof}


\section{Final result}

By combining all the results in this chapter, we are finally able to compile the full list of exceptional data with a partition of length $2$.

\begin{solution-hurwitz*}
Let
\[
\DD=\datum{\surf{g}}{d}{\pi_1,\ldots,\pi_{n-1},[s,d-s]}
\]
be a candidate datum with $n\ge 3$. Then $\DD$ is exceptional if and only if one of the following holds.
\begin{enumerate}[(1)]
\item $\DD=\datum{\sphere{}}{12}{[2,2,2,2,2,2],[1,1,1,3,3,3],[6,6]}$.
\item $\DD=\datum{\sphere{}}{2k}{[2,\ldots,2],[2,\ldots,2],[s,2k-s]}$ with $k\ge 2$, $s\neq k$.
\item $\DD=\datum{\sphere{}}{2k}{[2,\ldots,2],[1,2,\ldots,2,3],[k,k]}$ with $k\ge2$.
\item $\DD=\datum{\sphere{}}{4k+2}{[2,\ldots,2],[1,\ldots,1,k+1,k+2],[2k+1,2k+1]}$ with $k\ge 1$.
\item $\DD=\datum{\sphere{}}{4k}{[2,\ldots,2],[1,\ldots,1,k+1,k+1],[2k-1,2k+1]}$ with $k\ge2$.
\item $\DD=\datum{\sphere{}}{kh}{[h,\ldots,h],[1,\ldots,1,k+1],[lh,(k-l)h]}$ with $h\ge 2$, $k\ge 2$, $1\le l\le k-1$.
\item $\DD=\datum{\surf{1}}{6}{[3,3],[3,3],[2,4]}$.
\item $\DD=\datum{\surf{1}}{8}{[2,2,2,2],[4,4],[3,5]}$.
\item $\DD=\datum{\surf{1}}{12}{[2,2,2,2,2,2],[3,3,3,3],[5,7]}$.
\item $\DD=\datum{\surf{1}}{16}{[2,2,2,2,2,2,2,2],[1,3,3,3,3,3],[8,8]}$.
\item $\DD=\datum{\surf{1}}{2k}{[2,\ldots,2],[2,\ldots,2,3,5],[k,k]}$ with $k\ge 5$.
\item $\DD=\datum{\surf{2}}{8}{[2,2,2,2],[2,2,2,2],[2,2,2,2],[3,5]}$.
\item $\DD=\datum{\surf{n-3}}{4}{[2,2],\ldots,[2,2],[1,3]}$.
\end{enumerate}
\end{solution-hurwitz*}