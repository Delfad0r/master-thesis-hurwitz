\section{Monodromia}


\subsection{Gruppo fondamentale e monodromia del rivestimento}
\tikzset{loop around point/.pic={
\draw[black edge] (0,0) to[in=180,out=45,out looseness=.1] (.75,.02) to[out=0,in=180] (.9,.1) to[out=0,in=90] (1,0) node[above] {$a_{#1}$} to[out=-90,in=0] (.9,-.1) to[out=180,in=0] (.75,-0.02) to[out=180,in=-45,in looseness=.1] (0,0);
\filldraw[surf boundary,fill=white] (.9,0) circle(.05);
},
pics/commutator/.style n args={3}{code={
\tikzmath{\i0=#1;\i1=#1+1;\i2=#1+2;\i3=#1+3;\i4=#1+4;}
\begin{scope}[every path/.style={line width=\edgelinewidth}]
\draw[#2,postaction={decorate,decoration={markings,mark=at position.5 with {\arrow[xshift=3.3pt]{Stealth[scale=.7]}}}}] (\i0) -- (\i1) node[midway,auto,colored label={#2},swap] {$b_{#3}$};
\draw[#2,postaction={decorate,decoration={markings,mark=at position.5 with {\arrow[xshift=3.3pt]{Stealth[scale=.7]}}}}] (\i3) -- (\i2) node[midway,auto,colored label={#2}] {$b_{#3}$};
\draw[#2,postaction={decorate,decoration={markings,mark=at position.5 with {\arrow[xshift=6.6pt]{Stealth[scale=.7] Stealth[scale=.7]}}}}] (\i1) -- (\i2) node[midway,auto,colored label={#2},swap] {$c_{#3}$};
\draw[#2,postaction={decorate,decoration={markings,mark=at position.5 with {\arrow[xshift=6.6pt]{Stealth[scale=.7] Stealth[scale=.7]}}}}] (\i4) -- (\i3) node[midway,auto,colored label={#2}] {$c_{#3}$};
\end{scope}
}}}
\begin{frame}
\begin{columns}[onlytextwidth]
\column{.65\textwidth}
Sia $\surf{g}$ la somma connessa di $g\ge 0$ tori. Se
\[
\holed{\surf{g}}=\surf{g}\setminus\{x_1,\ldots,x_n\},
\]
il gruppo fondamentale $\pi_1(\holed{\surf{g}},x_0)$ ammette la presentazione
\begin{multline*}
\langle \sidenotehighlight<1>{a_1}{\draw (mymark) to[out=60,in=180] (1,.55) node[anchor=west] {cammino chiuso intorno a $x_1$};},\ldots,a_n,b_1,\ldots,b_g,c_1,\ldots,c_g\mid\\
[b_1,c_1]\cdots[b_g,c_g]\cdot a_1\cdots a_n\rangle.
\end{multline*}
\column{.35\textwidth}
\begin{flushright}
\tikzsetnextfilename{slides-monodromy-fundamental-group}
\begin{tikzpicture}[x={(1.25,0)},y={(0,1.25)},every node/.style={scale=.7}]
\pgfsetlayers{main,graph vertex}
\begin{pgfonlayer}{graph vertex}
\foreach \i[evaluate=\i as \i using int(\i)] in {0,...,14} {
\fill ({(\i-4)*360/14}:1) coordinate (\i) circle(1pt);
}\end{pgfonlayer}
\fill[disk 1] (0) \foreach \i[evaluate=\i as \i using int(\i)] in {1,...,13} { -- (\i)} -- cycle;
%\node[below=4pt] at (0) {$x_0$};
\foreach \i/\j in {4/5,9/10} {\draw[surf boundary dashed] (\i) -- (\j);}
\pic{commutator={0}{purple}{1}};
\pic{commutator={5}{teal}{i}};
\pic{commutator={10}{green}{g}};
\pic[rotate=100,scale={1/.7}] at (0) {loop around point=1};
\pic[rotate=50,scale={1/.7}] at (0) {loop around point=n};
\foreach \a in {65,75,85} {\filldraw[surf boundary,fill=white] ($(0)+(\a:.9)$) circle (.02);}
\end{tikzpicture}
\end{flushright}
\end{columns}
\vspace{0.5em}
\begin{itemize}
\item Se $\map{f}{\tSigma}{\surf{g}}$ è un rivestimento \sidenote<1>{ramificato}{\draw (mymark) to[out=45,in=180] (1,.45) node[right] {in $\{x_1,\ldots,x_n\}$};}, $\pi_1(\holed{\surf{g}},x_0)$ agisce sulla fibra $f^{-1}(x_0)$ \textcolor{black!80}{(monodromia del rivestimento $\map{\holed{f}}{\holed{\tSigma}}{\holed{\surf{g}}}$)}.
\item Questa azione induce un morfismo di gruppi
\[
\map{\Mon}{\pi_1(\holed{\surf{g}},x_0)}{\op{\symgroup(f^{-1}(x_0))}\iso\symgroup[d]}.
\]
\item Le lunghezze dei cicli di $\Mon(a_i)$ corrispondono agli elementi di \sidenote<1>{$\pi(x_i)$.}{\draw[shorten <=3pt] (mymark) to[out=30,out looseness=2,in=-30] (.6,1.2) node[anchor=east,align=right] {scriviamo\\$[\Mon(a_i)]=\pi(x_i)$};} %Scriviamo []=
\end{itemize}
\end{frame}

\subsection{Criterio di realizzabilità}
\begin{frame}
\begin{mybox}
Un dato $\DD=\datum{\tSigma,\surf{g}}{d}{\pi_1,\ldots,\pi_n}$ è realizzabile se e solo se esistono permutazioni $\alpha_1,\ldots,\alpha_n,\allowbreak\beta_1,\ldots,\beta_g,\allowbreak\gamma_1,\ldots,\gamma_g\in\symgroup[d]$ tali che:
\begin{enumroman}
\item $[\alpha_i]=\pi_i$ per ogni $1\le i\le n$;
\item $[\beta_1,\gamma_1]\cdots[\beta_g,\gamma_g]\cdot\alpha_1\cdots\alpha_n=1$;
\item $\angled{\alpha_1,\ldots,\alpha_n,\beta_1,\ldots,\beta_g,\gamma_1,\ldots,\gamma_g}$ agisca \sidenote<1>{transitivamente}{\draw (mymark) to[out=-120,in=30] (-.7,-.5) node[anchor=east] {$\tSigma$ connessa};} su $\{1,\ldots,d\}$.
\end{enumroman}
\end{mybox}
\vspace{.5em}
\emph{Esempio.} Il dato $\DD=\datum{\sphere{},\sphere{}}{4}{[2,2],[2,2],[1,3]}$ è eccezionale. Se non lo fosse, esisterebbero $\alpha_1,\alpha_2,\alpha_3\in\symgroup[4]$ tali che:
\begin{enumroman}
\item $[\alpha_1]=[\alpha_2]=[2,2]$ e $[\alpha_3]=[1,3]$\sidenote{;}{\draw[shorten <=3pt] (mymark) to[out=0,in=180] (2,.2) node[anchor=west,align=left] {$\alpha_1=\cycle{\bullet,\bullet}\cycle{\bullet,\bullet}$\\$\alpha_2=\cycle{\bullet,\bullet}\cycle{\bullet,\bullet}$\\$\alpha_3=\cycle{\bullet,\bullet,\bullet}$};}
\item $\alpha_1\alpha_2\alpha_3=1$\sidenote{.}{\draw[shorten <=3pt] (mymark) to (1,0) node[anchor=west] {$\alpha_3=\alpha_2^{-1}\alpha_1^{-1}$};}
\end{enumroman}
Ma le permutazioni di tipo $[2,2]$ generano un sottogruppo di ordine $4$, che non contiene $\alpha_3$. %scrivere esplicitamente il sottogruppo
\end{frame}

\subsection{Conseguenze}
\begin{frame}<1-2>
\begin{mybox}
Ogni dato compatibile $\datum{\tSigma,\surf{g}}{d}{\pi_1,\ldots,\pi_n}$ con $g\ge 1$ è realizzabile.
\end{mybox}
\begin{itemize}
\item D'ora in poi considereremo solo dati della forma
\[
\datum{\tSigma,\alt<1>{\sphere}{
\begin{tikzpicture}[baseline=0pt]
\node[inner sep=0,anchor=base] (s) {$\sphere$};
\pgfinterruptboundingbox
\node[fit=(s),draw=violet,line width=.8pt,opacity=.75,cross out,inner sep=1pt]{};
\endpgfinterruptboundingbox
\end{tikzpicture}}}{d}{\pi_1,\ldots,\pi_n}. %omettiamo \sphere{}
\]
\end{itemize}
\begin{mybox}
Ogni dato compatibile $\datum{\tSigma}{d}{\pi_1,\ldots,\pi_{n-1},[d]}$ è realizzabile.
\end{mybox}
\begin{mybox}
Ogni dato compatibile $\datum{\tSigma}{d}{\pi_1,\ldots,\pi_{n-1},[1,d-1]}$ è realizzabile, eccezion fatta per:
\begin{enumarabic}
\item $\datum{\sphere{}}{2k}{[2,\ldots,2],[2,\ldots,2],[1,2k-1]}$ con $k\ge 2$;
\item $\datum{\surf{n-3}}{4}{[2,2],\ldots,[2,2],[1,3]}$.
\end{enumarabic}
\end{mybox}
\end{frame}