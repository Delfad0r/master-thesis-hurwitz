\documentclass[a4paper,10pt]{book}

\usepackage{mystyle-thesis}
\usepackage{mystyle-global}


\def\usefrontespizio{1}

%\tikzremake

%\includeonly{test}
\begin{document}

%\chapter{My}
\lipsum[1]
\begin{sideline}{Title}
\lipsum[1]
\end{sideline}
\begin{sideline}{Yo}
\begin{itemize}
\item \lipsum[1]
\item \lipsum[2]\sdlendhere
\end{itemize}
\end{sideline}
\begin{sideline}{Yo}
\lipsum[1]
\end{sideline}
\lipsum[2]

\ifnum\usefrontespizio=1
\begin{frontespizio}
\Margini{1.25in}{1.5in}{1.25in}{1.2in}
\Istituzione{Università di Pisa}
\Logo[4cm]{img/unipi-logo}
\Facolta{Scienze matematiche, fisiche e naturali}
\Corso[Laurea]{Matematica}
\Annoaccademico{2020--2021}
\Titoletto{Tesi di Laurea Magistrale}
\Titolo{Realizability of branching data \\ with a short partition}
\Candidato{Filippo Gianni Baroni}
\Relatore{Prof.~Carlo Petronio}
\end{frontespizio}
\cleardoublepage
\setcounter{page}{1}
\fi

\tableofcontents

%\documentclass[a4paper,10pt]{book}

%\usepackage{mystyle-thesis}
%\usepackage{mystyle-global}

%\begin{document}
\chapter*{Introduction}
\label{introduction:sc}
\addcontentsline{toc}{chapter}{\protect\numberline{}\nameref{introduction:sc}}
\markboth{Introduction}{Introduction}

The goal of this thesis is twofold. One one hand, we will try to provide a complete and coherent introduction to the Hurwitz existence problem, from the definition of branched covering to the presentation of the two main approaches which have historically been used to attack this problem -- namely monodromy and \dessins{}. On the other hand, we will develop \emph{genus-reducing combinatorial moves}, an original tool which will allow us to fully address a previously unsolved instance of the existence problem.

Finally, we will briefly touch on a computational approach to this problem. By compiling the full list of exceptional data of degree $d\le 29$, we extend the results obtained by \citeauthor{zheng}, who first proposed this method. Incidentally, we also provide additional evidence supporting the mysterious prime-degree conjecture.

\section*{Branched coverings}

A \emph{branched covering} between two (closed connected) topological surfaces $\tSigma$, $\Sigma$ is a continuous function $\map{f}{\tSigma}{\Sigma}$ which is locally modeled on the complex map $\xi\mapsto\xi^k$, where the positive integer $k$ is called \emph{local degree} and depends on the point $\wtilde{x}\in\tSigma$. The local degree is equal to $1$ for all but a finite number of points in $\tSigma$; the images of these points under $f$ form a discrete subset of $\Sigma$, whose elements are called \emph{branching points}. Let $x_1,\ldots,x_n\in\Sigma$ be the branching points, and define $\holed{\Sigma}=\Sigma\setminus\{x_1,\ldots,x_n\}$. The restriction of $f$ to $\holed{\tSigma}=f^{-1}(\holed{\Sigma})$ gives a standard covering map $\map{\holed{f}}{\holed{\tSigma}}{\holed{\Sigma}}$ of some positive degree $d$. It is not hard to see that, for every branching point $x_i\in\Sigma$, the local degrees of the preimages of $x_i$ form a \emph{partition} of $d$, which we denote by $\pi_i$. If we abstract away the topology of $f$, we are left with the \emph{branching datum}
\[
\DD(f)=\datum{\tSigma,\Sigma}{d}{\pi_1,\ldots,\pi_n}.
\]
The Hurwitz existence problem can be formulated as follows: given a \emph{combinatorial datum} $\DD=\datum{\tSigma,\Sigma}{d}{\pi_1,\ldots,\pi_n}$, is there any branched covering $\map{f}{\tSigma}{\Sigma}$ whose branching datum is equal to $\DD$?

There are a few easy necessary conditions for a combinatorial datum to be realized by some branched covering; one of them is called the \emph{Riemann-Hurwitz formula}, and states that, if $\DD$ is realizable, then
\[
d\chi(\tSigma)-\chi(\Sigma)=dn-\len{\pi_1}-\ldots-\len{\pi_n},
\]
where $\len{\pi_i}$ is the length of the partition $\pi_i$. A combinatorial datum which satisfies the Riemann-Hurwitz formula and the other conditions is called a \emph{candidate datum}. It turns out that there exist \emph{exceptional} (that is, non-realizable) candidate data, but finding other necessary conditions is remarkably hard. The difficulty of solving the Hurwitz existence problem lies in figuring out some kind of regularity in the apparently chaotic plethora of exceptional candidate data.

\section*{Monodromy}

Historically, the first fruitful approach to the existence problem was based on the monodromy action associated to the covering map $\map{\holed{f}}{\holed{\tSigma}}{\holed{\Sigma}}$. For instance, in the case of an orientable surface of genus $g$, elementary properties of covering spaces imply that a candidate datum $\datum{\tSigma,\surf{g}}{d}{\pi_1,\ldots,\pi_n}$ is realizable if and only if there exist permutations \let\ab\allowbreak $\alpha_1,\ldots,\ab\alpha_n,\ab\beta_1,\ldots,\ab\beta_g,\ab\gamma_1,\ldots,\ab\gamma_g\in\symgroup[d]$ such that:
\begin{enumroman}
\item $[\alpha_i]=\pi_i$ for each $1\le i\le n$;
\item $[\beta_1,\gamma_1]\cdots[\beta_g,\gamma_g]\cdot\alpha_1\cdots\alpha_n=1$;
\item the subgroup $\angled{\alpha_1,\ldots,\alpha_n,\beta_1,\ldots,\beta_g,\gamma_1,\ldots,\gamma_g}\subgroup\symgroup[d]$ acts transitively on $\{1,\ldots,d\}$.
\end{enumroman}
Here, $[\alpha]$ denotes the partition of $d$ given by the lengths of the (possibly trivial) cycles in the cycle-decomposition of the permutation $\alpha$. A similar criterion exists for non-orientable surfaces.

We see that the monodromy approach turns a topological problem into a group-theoretic one, which can be attacked by means of more algebraic tools. By using the criteria we have just mentioned, it is almost immediate to show that $\DD$ is realizable whenever $\chi(\Sigma)\le 0$. With the aid of a few more technical results about products in symmetric groups, we can also address the instances where $\Sigma=\RP[2]$, thereby reducing the Hurwitz existence problem to the setting $\Sigma=\sphere$. The monodromy approach also leads to some partial results for this case; however, a more promising tool is provided by \dessins{}, which were specifically designed to work on the sphere.

\section*{\texorpdfstring{\Dessins{}}{Dessins d'enfant}}

Assume for simplicity that $n=3$, and consider a branched covering $\map{f}{\surf{g}}{\sphere{}}$ with branching datum $\DD(f)=\datum{\surf{g},\sphere{}}{d}{\pi_1,\pi_2,\pi_3}$. Let $x$, $y$ and $z$ be the branching points corresponding, respectively, to $\pi_1$, $\pi_2$ and $\pi_3$. Consider a segment $e\subs\sphere{}$ connecting $x$ to $y$. It turns out that $\Gamma=f^{-1}(e)$ is a bipartite graph embedded in $\surf{g}$, having $f^{-1}(x)$ and $f^{-1}(y)$ as the sets of (say) black and white vertices respectively. The graph $\Gamma$ satisfies the following properties:
\begin{enumroman}
\item the degrees of the black vertices are the elements of $\pi_1$;
\item the degrees of the white vertices are the elements of $\pi_2$;
\item $\surf{g}\setminus\Gamma$ is a disjoint union of topological disks, and the perimeters of these disks are twice the elements of $\pi_3$.
\end{enumroman}

We say that a graph with these properties is a \emph{\dessin{}} realizing $\DD(f)$.
Perhaps unsurprisingly, a candidate datum $\DD$ is realizable if and only if there exists a \dessin{} realizing it. This gives yet another perspective on the Hurwitz existence problem, paving the way for an approach based on the combinatorial properties of graphs on surfaces. Together with some classical applications of \dessins{}, we present a novel set of ``combinatorial moves'', which operate on branching data with a short partition (that is, $\len{\pi_3}=2$). More specifically, these moves allows us to inductively reduce the existence problem to surfaces with lower genus. Building on results by \citeauthor{pakovich}, who classified the exceptional data of this form with $\tSigma=\sphere{}$, we are able to provide a complete list also for $g\ge 1$. Using another reduction technique, already well known and based on the monodromy approach, we also extend our result to the cases where $n\ge 4$, thereby fully solving the Hurwitz existence problem for candidate data with a short partition.



%\end{document}


\chapter{The Hurwitz existence problem}
\largevertices

\section{Branched coverings of surfaces}

According to standard terminology, a \emph{surface} is simply a topological $2$-manifold. We will, however, only be concerned with compact, connected surfaces without boundary. For the sake of conciseness, unless otherwise stated, we will always implicitly assume that the surfaces we mention have these properties.

Orientable and non-orientable surfaces are completely classified by the following structure theorem (see \resultcite{chapter}{12}{munkres}).
\begin{itemize}
\item An orientable surface is (homeomorphic to) a connected sum of $g\ge 0$ tori. We call such a connected sum a \emph{surface of genus $g$}, and we denote it by $\surf{g}$; by definition, we say that $\surf{0}$ is the $2$\=/sphere $\sphere{}$. The Euler characteristic of an orientable surface is given by $\chi(\surf{g})=2-2g$.
\item A non-orientable surface is (homeomorphic to) a connected sum of $g\ge 1$ real projective planes. We denote such a connected sum by $\nosurf{g}$. The Euler characteristic of a non-orientable surface is given by $\chi(\nosurf{g})=2-g$.
\end{itemize}


Loosely speaking, given two surfaces $\Sigma$ and $\tSigma$, a \emph{covering map} between them is a continuous function $\map{f}{\tSigma}{\Sigma}$ which is locally modeled on the identity function $\umap{\RR^2}{\RR^2}$. Sometimes, however, the notion of covering map can be too restrictive. Consider, for instance, the sphere $\sphere{}$: being simply connected, it does not admit any non-trivial coverings. However, every (non-constant) homolorphic function from the Riemann sphere to itself is \emph{almost} a covering map, in the sense that it is locally modeled on the identity $\umap{\CC}{\CC}$, except for a finite number of \emph{branching points}, where it looks like the map
\[
\Map{F_k}{\CC}{\CC}{\xi}{\xi^k}
\]
for some $k\ge 2$. In fact, it turns out that every (non-constant) holomorphic function between two Riemann surfaces has this remarkable property (see \resultcite{section}{3.2}{szamuely}). This motivates the following definition.

\begin{definition}
Let $\Sigma,\tSigma$ be two surfaces. A continuous function $\map{f}{\tSigma}{\Sigma}$ is a \emph{branched covering map} (or simply a \emph{branched covering}) if the following property holds: for every $x\in\Sigma$, $\wtilde{x}\in f^{-1}(x)$ there exist a positive integer $k$, open neighborhoods $U,\wtilde{U}$ of $x,\wtilde{x}$ respectively, and homeomorphisms $\map{\phi}{U}{\CC}$, $\map{\wtilde{\phi}}{\wtilde{U}}{\CC}$ such that $\phi(x)=0$, $\wtilde{\phi}(\wtilde{x})=0$, $f(\wtilde{U})=U$ and the diagram
\begin{diagram}
\wtilde{U}\rar{f}\dar{\wtilde{\phi}}&U\dar{\phi}\\
\CC\rar{F_k}&\CC
\end{diagram}
commutes ($F_k$ is the map defined above). We say that $\wtilde{U}$ is a \emph{trivializing neighborhood} of $\wtilde{x}$.
\end{definition}

More informally, we can say that a branched covering is a continuous function between surfaces which is locally modeled on the complex map $\xi\mapsto \xi^k$, where $k\ge 1$ depends on the point. Note that, for each point $\wtilde{x}\in\tSigma$, the integer $k$ is well-defined, independently of the charts $\phi$ and $\wtilde{\phi}$: using the notation from the definition, we have that $k$ is equal to the cardinality of $f^{-1}(y)\cap\wtilde{U}$, where $y$ is any point in $U\setminus\{x\}$. We call this integer the \emph{local degree} of $f$ at $\wtilde{x}$; to emphasize the dependence on $\wtilde{x}$, we will denote it by $k(\wtilde{x})$ (the function $f$ will always be clear from the context). 

\begin{center}
\tikzsetnextfilename{branched-covering-local-model}
\tdplotsetmaincoords{65}{-30}
\begin{tikzpicture}[tdplot_main_coords,x={(2,0,0)},y={(0,2,0)},z={(0,0,2)}]
\def\a{30};
\foreach \i in {1,2} {
\coordinate (y-\i) at ({sin(\a)*.7},{-cos(\a)*.7},{(\a/360+\i-2)*.7});
}
\coordinate (y) at ({sin(\a)*.7},{-cos(\a)*.7},-2.5);

\begin{scope}[shift={(0,0,-2.5)}]
\fill[disk 1] circle (1);
\foreach \i in {.1,.2,...,.9} { \draw[gray!80,line width=.1] circle (\i); }
\foreach \i in {0,...,29} {\draw[gray!80,line width=.1] (0,0) -- ({sin(\i*360/30)},{-cos(\i*360/30)},0); }
\draw circle (1);
\fill[black] circle (1pt) node[below left] {$x$};;

\fill[black] (y) circle (1pt) node[right=2pt] {$y$};
\end{scope}

\draw[violet,-{Latex[]}] (0,0,0) -- (0,0,-2.4) node[midway,left=2pt] {$f$};
\draw[violet,-{Latex[]}] (y-1) -- ([shift={(0,0,.1)}]y);

\newcounter{branchedcoveringcounter}
\path[decorate,decoration={show path construction,curveto code={
\filldraw[line width=.1pt,fill=disk 1,fill opacity=.9,draw=gray!80] (0,0,0) -- (\tikzinputsegmentfirst) -- (\tikzinputsegmentlast) -- cycle;
\draw (\tikzinputsegmentfirst) .. controls (\tikzinputsegmentsupporta) and (\tikzinputsegmentsupportb) .. (\tikzinputsegmentlast);
\foreach \i in {.1,.2,...,.9} {
	\draw[line width=.2,gray,opacity=.7] ($\i*(\tikzinputsegmentfirst)$) .. controls ($\i*(\tikzinputsegmentsupporta)$) and ($\i*(\tikzinputsegmentsupportb)$) .. ($\i*(\tikzinputsegmentlast)$);
}
\ifnumcomp{\the\value{branchedcoveringcounter}}{=}{0}{
	\draw[teal,line width=\edgelinewidth] (0,0,0) -- (\tikzinputsegmentfirst);
}{}
\ifnumcomp{\the\value{branchedcoveringcounter}}{=}{15}{
	\draw[violet] (y-2) -- (y-1);
}{}
\stepcounter{branchedcoveringcounter}
}}] plot[smooth,samples=61,variable=\t,domain=-360:360] ({sin(\t)},{-cos(\t)},{\t/360});
\draw[teal,line width=\edgelinewidth] (0,0,0) -- (0,-1,1);

\def\a{30};
\foreach \i in {1,2} {
\fill[black] (y-\i) circle (1pt) node[above right] {$\wtilde{y}_\i$};
}
\fill[black] (0,0,0) circle (1pt) node[below left] {$\wtilde{x}$};
\end{tikzpicture}
\end{center}

A point $\wtilde{x}\in\tSigma$ is called a \emph{branching point} if $k(\wtilde{x})>1$; in other words, if $f$ is \emph{not} a local homeomorphism in a neighborhood of $\wtilde{x}$. We also say that a point $x\in\Sigma$ is a \emph{branching point} if $f^{-1}(x)$ contains at least one branching point; usually, no ambiguity will arise as to which kind of branching point we are referring to.

It is not hard to see that branching points are quite rare. If $\wtilde{x}\in\tSigma$ is a branching point, then (using again the notation from the definition) every other point in $\wtilde{U}$ is not a branching point; by compactness, it follows that the set of branching points in $\tSigma$ is finite. As a consequence, the set of branching points in $\Sigma$ is finite as well.

Given a branched covering $\map{f}{\tSigma}{\Sigma}$, we denote by $\holed{\Sigma}$ the subspace of $\Sigma$ containing all the non-branching points; since the set of branching points is finite, $\holed{\Sigma}$ is a non-compact connected surface with finitely many punctures. We also set $\holed{\tSigma}=f^{-1}(\holed{\Sigma})$, and we denote by $\holed{f}$ the restriction of $f$ to $\holed{\tSigma}$. By construction, $\map{\holed{f}}{\holed{\tSigma}}{\holed{\Sigma}}$ is a covering map and, as such, has a well-defined degree $d$ (the number of preimages of an arbitrary point); we call this integer the \emph{degree} of the branched covering $f$. The following proposition shows that the notion of degree extends nicely to branching points.

\begin{proposition}\label{hurwitz:th:sum-of-preimage-degrees}
Let $\map{f}{\tSigma}{\Sigma}$ be a branched covering of degree $d$. Then for every point $x\in\Sigma$ we have that the set $f^{-1}(x)$ is finite and
\[
\sum_{\wtilde{x}\in f^{-1}(x)}k(\wtilde{x})=d.
\]
\end{proposition}
\begin{proof}
If $x$ is not a branching point, the conclusion follows immediately, since $x$ has exactly $d$ preimages, all of which have local degree equal to $1$. Assume now that $x$ is a branching point; it is clear from the definition that the set $f^{-1}(x)\subs\tSigma$ is discrete and, hence, finite. Let $f^{-1}(x)=\{\wtilde{x}_1,\ldots,\wtilde{x}_r\}$. Fix disjoint trivializing neighborhoods $\wtilde{U}_1,\ldots,\wtilde{U}_r$ of $\wtilde{x}_1,\ldots,\wtilde{x}_r$ respectively; a routine compactness argument shows that there exists an open neighborhood $U$ of $x$ such that $f^{-1}(U)\subs \wtilde{U}_1\cup\ldots\cup\wtilde{U}_r$. Fix a point $y\in U\setminus\{x\}$: it follows from the discussion above that $y$ is not a branching point and that $\card{f^{-1}(y)\cap\wtilde{U}_i}=k(\wtilde{x}_i)$ for every $1\le i\le r$. Since $\card{f^{-1}(y)}=d$, we immediately conclude that
\[
\sum_{i=1}^r k(\wtilde{x}_i)=\sum_{i=1}^r\card{f^{-1}(y)\cap\wtilde{U}_i}=\card{f^{-1}(y)}=d.\qedhere
\]
\end{proof}

\section{Branching data}

Let $\map{f}{\tSigma}{\Sigma}$ be a branched covering. For each point $x\in\Sigma$, as we have just seen in \cref{hurwitz:th:sum-of-preimage-degrees}, the sum $k(\wtilde{x}_1)+\ldots+k(\wtilde{x}_r)$ of the local degrees of $f$ at its preimages is equal to the degree $d$. Since there is no natural ordering on the set $f^{-1}(x)$, the appropriate combinatorial object for representing the collection $k(\wtilde{x}_1),\ldots,k(\wtilde{x}_r)$ is a \emph{partition}.

\begin{definition}
Let $d$ be a positive integer. A \emph{partition} of $d$ is an unordered finite multiset $\pi=[k_1,\ldots,k_r]$, where $k_i>0$ is an integer for every $1\le i\le r$ and $k_1+\ldots+k_r=d$.
\end{definition}

Given a positive integer $d$, we denote the set of all partitions of $d$ by $\Partitions{d}$. If $\pi=[k_1,\ldots,k_r]$ is a partition of $d$, we call the integer $r$ the \emph{length} (or \emph{size}, or \emph{cardinality}) of $\pi$, and we denote it by $\len{\pi}$. We also say that the \emph{sum} of $\pi$, denoted by $\sum\pi$, is $k_1+\ldots+k_r$ or, in other words, $d$. Finally, we introduce a new quantity, the \emph{branching number} $v(\pi)=d-\len{\pi}$, whose purpose will soon become apparent.

For every point $x\in\Sigma$, if $f^{-1}(x)=\{\wtilde{x}_1,\ldots,\wtilde{x}_r\}$, we can define the \emph{associated partition} $\pi(x)=[k(\wtilde{x}_1),\ldots,k(\wtilde{x}_r)]\in\Partitions{d}$. For all non-branching points, the associated partition will simply be $[1,\ldots,1]$. On the contrary, if $x$ is a branching point, then $\pi(x)\neq[1,\ldots,1]$, or, equivalently, $v(\pi(x))>0$.

\begin{definition}
Let $\map{f}{\tSigma}{\Sigma}$ be a branched covering of degree $d$. Let $x_1,\ldots,x_n\in\Sigma$ be the branching points of $f$. The \emph{branching datum} of $f$ is the tuple
\[
\DD(f)=\datum{\tSigma,\Sigma}{d}{\pi(x_1),\ldots,\pi(x_n)},
\]
well-defined up to a permutation of the partitions $\pi(x_1),\ldots,\pi(x_n)$.
\end{definition}

Branching data are a way to extract some combinatorial information from branched coverings. Even though the exact location of the branching points (both in $\Sigma$ and in $\tSigma$) is not encoded in the branching datum, this piece of information is completely irrelevant, since surfaces are \emph{homogeneous}\footnote{By \emph{homogeneous}, we mean that, for each $n\ge 1$, the group of homeomorphisms of a surface $\Sigma$ acts transitively on the set of $n$-uples of pairwise distinct points $(y_1,\ldots,y_n)$.}. However, the reader should not be induced to believe that the combinatorial information provided by the branching datum is enough to fully reconstruct the topology of $f$. In fact, it turns out that this is not the case: there can be many inequivalent branched coverings sharing the same branching datum. See \cite{sarti} for an in-depth discussion of this topic.

This entire thesis is devoted to the problem of determining what values can actually be attained by $\DD(f)$ as $f$ ranges over all the possible branched covering maps. We start with a very general definition.

\begin{definition}
A \emph{combinatorial datum} is a tuple
\[
\DD=\datum{\tSigma,\Sigma}{d}{\pi_1,\ldots,\pi_n},
\]
where $\Sigma,\tSigma$ are surfaces, $d$ and $n$ are positive integers, and $\pi_1,\ldots,\pi_n$ are partitions of $d$ different from $[1,\ldots,1]$.
\end{definition}

Again, a combinatorial datum is defined up to a permutation of the partitions $\pi_1,\ldots,\pi_n$. In other words, we will consider two combinatorial data equal if they have the same partitions, irrespective of the ordering.

Technically, we could also allow combinatorial data to have $n=0$ partitions. However, data of this kind would correspond to standard covering maps, without any branching points. Since the existence of covering maps between surfaces is completely understood  (even in the non-empty boundary case, as shown in \cite{massey-covering}), we will always assume that $n\ge 1$. For the same reason, when we say ``branched covering'', we will be implicitly excluding the trivial case of a standard covering map.

We say that a combinatorial datum $\DD$ is \emph{realizable} if $\DD=\DD(f)$ for some branched covering $\map{f}{\tSigma}{\Sigma}$, \emph{exceptional} otherwise. A first, naive guess could be that every combinatorial datum can be realized. However, it turns out that there are a few easy and general necessary conditions for a combinatorial datum to be associated to a branched covering. The first one, and arguably the most important, is known as the \emph{Riemann-Hurwitz formula}; we state it in the next proposition.

\begin{proposition}\label{hurwitz:th:riemann-hurwitz-formula}
Let $\map{f}{\tSigma}{\Sigma}$ be a branched covering, and let $\DD(f)=\datum{\tSigma,\Sigma}{d}{\pi_1,\ldots,\pi_n}$ be its branching datum. Then we have the equality
\[
d\chi(\Sigma)-\chi(\tSigma)=v(\pi_1)+\ldots+v(\pi_n).
\]
\end{proposition}
\begin{proof}
Using the same notation as above, consider the covering map $\map{\holed{f}}{\holed{\tSigma}}{\holed{\Sigma}}$. Since $\holed{f}$ has degree $d$, the Euler characteristics of $\holed{\Sigma}$ and $\holed{\tSigma}$ are related by the formula $d\chi(\holed{\Sigma})=\chi(\holed{\tSigma})$. Note that there are $n$ branching points in $\Sigma$, and the number of points in $\tSigma\setminus\holed{\tSigma}$ is $\len{\pi_1}+\ldots+\len{\pi_n}$; therefore
\begin{align*}
\chi(\holed{\Sigma})=\chi(\Sigma)-n,&&\chi(\holed{\tSigma})=\chi(\tSigma)-\len{\pi_1}-\ldots-\len{\pi_n}.
\end{align*}
As a consequence, we have that
\[
d\chi(\Sigma)-\chi(\tSigma)=d\chi(\holed{\Sigma})+dn-\chi(\holed{\tSigma})-\len{\pi_1}-\ldots-\len{\pi_n}=v(\pi_1)+\ldots+v(\pi_n).\qedhere
\]
\end{proof}

There are three more conditions that every branching datum must satisfy: two of them concern the orientability of $\Sigma$ and $\tSigma$, and the other one provides an additional constraint for the \emph{total branching number} $v(\pi_1)+\ldots+v(\pi_n)$; we group these four requirements in the following definition.

\begin{definition}\label{hurwitz:df:candidate-datum}
A combinatorial datum $\DD=\datum{\tSigma,\Sigma}{d}{\pi_1,\ldots,\pi_n}$ is a \emph{candidate datum} if it satisfies the following conditions:
\begin{enumroman}
\item\label{hurwitz:it:candidate-datum-def:1} $d\chi(\Sigma)-\chi(\tSigma)=v(\pi_1)+\ldots+v(\pi_n)$;
\item\label{hurwitz:it:candidate-datum-def:2} $v(\pi_1)+\ldots+v(\pi_n)$ is even;
\item\label{hurwitz:it:candidate-datum-def:3} if $\Sigma$ is orientable, then $\tSigma$ is orientable as well;
\item\label{hurwitz:it:candidate-datum-def:4} if $\Sigma$ is non-orientable and $\tSigma$ is orientable, then $d$ is even, and each $\pi_i$ can be written as $\pi'_i\cup\pi''_i$, where $\pi'_i$ and $\pi''_i$ are partitions of $d/2$.
\end{enumroman}
\end{definition}

\begin{remark}\label{hurwitz:rm:orientability-of-tsigma}
In \cref{hurwitz:th:candidate-datum-necessary-conditions}, we will show that conditions \ref{hurwitz:it:candidate-datum-def:1}--\ref{hurwitz:it:candidate-datum-def:4} are necessary for a combinatorial datum to be realizable (actually, the first one was already proved to be so in \cref{hurwitz:th:riemann-hurwitz-formula}). For now, let us focus on condition \ref{hurwitz:it:candidate-datum-def:3}, which is perhaps the most obvious one. If $\Sigma$ is orientable, then so is $\holed{\Sigma}$, since removing a finite number of points does not affect orientability. The non-compact surface $\holed{\tSigma}$, being a covering space of an orientable manifold, is itself orientable. Finally, this implies that $\tSigma$ is orientable as well.
\end{remark}

It would be natural to ask whether the necessary conditions we have enumerated are also sufficient for a candidate datum to be realizable. We will see in the following chapters that the answer is negative, and that, in general, determining the full list of exceptional data is remarkably difficult; we call this the \emph{Hurwitz existence problem}\footnote{The question originally posed by Hurwitz in \cite{hurwitz} was actually even harder: the task was counting \emph{how many} branched coverings (up to a suitable notion of isomorphism) realize a given candidate datum. We will, however, ignore this point of view, considering that the existence problem is hard enough as it is. The interested reader may find some relevant results on the \emph{Hurwitz enumeration problem} in \cite{hurwitz-counting-1,hurwitz-counting-2,hurwitz-counting-3,hurwitz-counting-4,hurwitz-counting-5}.}, and we state it in the following deliberately vague fashion.

\begin{hurwitz-existence-problem*}
Determine necessary and sufficient conditions for a candidate datum $\datum{\tSigma,\Sigma}{d}{\pi_1,\ldots,\pi_n}$ to be realizable.
\end{hurwitz-existence-problem*}

In the remainder of this thesis, we will try to address some instances of this problem. We will present a variety of techniques, which will provide us with a full solution for some classes of candidate data.


\section{Symmetric group and partitions}

The first approach we will present is based on the monodromy action. Before describing this action and how it relates to branched coverings, we will recall a few elementary facts about permutations and partitions.

Given a set $X$, we denote by $\symgroup(X)$ the set of bijective functions $\umap{X}{X}$. The set $\symgroup(X)$ is naturally endowed with a group structure, where the product is given by the composition operator $\circ$; we call this group the \emph{symmetric group} of $X$. Elements of $\symgroup(X)$ are also called \emph{permutations}.

We will sometimes need to work in a setting where the product in $\symgroup(X)$ is reversed. Given a group $G$ with product $\ast$, we define the \emph{opposite group} $\op{G}$, which has the same underlying set as $G$, but the product $\op{\ast}$ is defined as $g_1\op{\ast}g_2=g_2\ast g_1$.

If $d$ is a positive integer, we will employ the notation $\symgroup[d]$ as a shorthand for $\symgroup(\{1,\ldots,d\})$. Whenever $X$ is a finite (non-empty) set, we have an isomorphism $\symgroup(X)\iso\symgroup[d]$, where $d$ is the cardinality of $X$. We will therefore restrict out attentions to the groups $\symgroup[d]$, keeping in mind that everything we say also holds for symmetric groups of arbitrary finite sets.

Recall that every permutation $\alpha\in\symgroup[d]$ has a \emph{cycle decomposition} (that is, it can be written as a product of disjoint, possibly trivial cycles):
\[
\alpha=\cycle{x_{1,1},\ldots,x_{1,k_1}}\cycle{x_{2,1},\ldots,x_{2,k_2}}\cdots\cycle{x_{r,1},\ldots,x_{r,k_r}}.
\]
The natural action of (the subgroup generated by) $\alpha$ on the set $A=\{1,\ldots,d\}$ induces a decomposition $A=A_1\sqcup\ldots\sqcup A_r$ into orbits, where $A_i=\{x_{i,1},\ldots,x_{i,k_i}\}$. This decomposition, in turn, gives rise to a partition $\pi=[\card{A_1},\ldots,\card{A_r}]\in\Partitions{d}$. We say that $\alpha$ \emph{matches} the partition $\pi$, and we use the notation $[\alpha]$ to refer to the (unique) partition matched by $\alpha$. We also define the \emph{branching number} $v(\alpha)$ as the branching number of $[\alpha]$; note that $v(\alpha)=0$ if and only if $\alpha=\id\in\symgroup[d]$ is the trivial permutation.

It is well known that two permutations are conjugate if and only if they match the same partition. In other words, there is a natural bijection between conjugacy classes of $\symgroup[d]$ and the set of partitions $\Partitions{d}$.

We conclude this section with a very simple result relating the branching numbers of two permutations to the branching number of their product. Recall that every permutation has a well-defined \emph{sign}: \emph{even} permutations can only be written as a product of an even number of transpositions ($2$-cycles), while \emph{odd} permutations can only be written as a product of an odd number of transpositions. The \emph{alternating group} of order $d$ is the index-$2$ subgroup $\altgroup[d]\subgroup\symgroup[d]$ containing all the even permutations.

\begin{proposition}\label{hurwitz:th:branching-number-permutations-product}
Let $\alpha,\beta\in\symgroup[d]$ be permutations. Then $v(\alpha\beta)\equiv v(\alpha)+v(\beta)\pmod{2}$.
\end{proposition}
\begin{proof}
We show that $v(\alpha)$ is even if and only if $\alpha$ is even; the conclusion will then follow trivially. Fix a cycle decomposition
\[
\alpha=\cycle{x_{1,1},\ldots,x_{1,k_1}}\cycle{x_{2,1},\ldots,x_{2,k_2}}\cdots\cycle{x_{r,1},\ldots,x_{r,k_r}}.
\]
Since a $k$-cycle can be written as a product of $k-1$ transpositions, we have that $\alpha$ is the product of $d-r=v(\alpha)$ transpositions. This shows that $v(\alpha)$ is even if and only if $\alpha\in\altgroup[d]$, which concludes the proof.
\end{proof}


\section{Branched covering action of the fundamental group}

We now introduce the monodromy action in the general setting of topological spaces; we will assume that all the spaces we mention are locally path-connected and locally simply-connected. For a much more in-depth discussion of the monodromy action, and especially for a proof of \cref{hurwitz:th:covering-with-given-monodromy}, see \resultcite{section}{2.3}{szamuely} (but note that \citeauthor{szamuely}, unlike us, uses the right-to-left notation for path concatenation in the fundamental group).

Let $\map{f}{\wtilde{X}}{X}$ be a covering map between topological spaces, with $X$ path-connected. Fix a base-point $x_0\in X$, and let $\wtilde{x}_0\in\wtilde{X}$ be a point in $f^{-1}(x_0)$. Given a path $\map{\gamma}{[0,1]}{X}$ with $\gamma(0)=x_0$, denote by $\map{\lift{\gamma}{\wtilde{x}_0}}{[0,1]}{\wtilde{X}}$ the unique lift of $\gamma$ with respect to $f$ such that $\lift{\gamma}{\wtilde{x}_0}(0)=\wtilde{x}_0$. There is a very natural right action of the fundamental group $\pi_1(X,x_0)$ on the fiber $f^{-1}(x_0)$, defined by\footnote{Since the lifting operation is, in some sense, invariant up to homotopy, we will often use the same symbol to interchangeably represent a class in the fundamental group $\pi_1(X,x_0)$ and a representative of that class.}
\begin{align*}
\wtilde{x}\mon\gamma=\lift{\gamma}{\wtilde{x}}(1)&&\text{for $\wtilde{x}\in f^{-1}(x_0)$, $\gamma\in\pi_1(X,x_0)$.}
\end{align*}
We call this the \emph{monodromy action} of the covering map. It is easy to see that $\wtilde{X}$ is path-connected if and only if the monodromy action on $f^{-1}(x_0)$ is transitive.

There is a relation between monodromy and the fundamental group of $\wtilde{X}$. Fix a base-point $\wtilde{x}_0\in f^{-1}(x_0)\subs\wtilde{X}$. By elementary properties of covering spaces, we have that
\begin{align*}
f_*(\pi_1(\wtilde{X},\wtilde{x}_0))&=\{\gamma\in\pi_1(X,x_0):\lift{\gamma}{\wtilde{x}_0}=\wtilde{x}_0\}\\
&=\{\gamma\in\pi_1(X,x_0):\wtilde{x}_0\mon\gamma=\wtilde{x}_0\}\\
&=\stab{\pi_1(X,x_0)}{\wtilde{x}_0}.
\end{align*}

The monodromy action induces a group homomorphism $\map{\Mon}{\pi_1(X,x_0)}{\op{\symgroup(f^{-1}(x_0))}}$. If the covering map has a finite degree $d$, we will sometimes implicitly fix a bijection between $f^{-1}(x_0)$ and $\{1,\ldots,d\}$, and consider the homomorphism $\map{\Mon}{\pi_1(X,x_0)}{\op{\symgroup[d]}}$ instead; of course, this map is well-defined up to conjugation in $\op{\symgroup[d]}$. The following existence theorem shows that every group homomorphism of this kind is induced by the monodromy action of some covering space.

\begin{theorem}\label{hurwitz:th:covering-with-given-monodromy}
Let $X$ be a path-connected topological space, and $x_0\in X$ be a point. Let $d$ be a positive integer, and let $\map{\psi}{\pi_1(X,x_0)}{\op{\symgroup[d]}}$ be a group homomorphism. Then there exists a covering map $\map{f}{\wtilde{X}}{X}$ with $\Mon=\psi$ (up to conjugation).
\end{theorem}

After introducing monodromy for general topological spaces, we turn our attention to the case of branched coverings of surfaces. Let $\map{f}{\tSigma}{\Sigma}$ be a branched covering. As we have seen, by removing the branching points from $\Sigma$, we get an associated covering map $\map{\holed{f}}{\holed{\tSigma}}{\holed{\Sigma}}$. Fix a base-point $x_0\in\holed{\Sigma}$, and let $x\in\Sigma$ be any point. Then there is a path $\gamma\in\pi_1(\holed{\Sigma},x_0)$ which ``goes around $x$ once''; to be precise, we can construct $\gamma$ as follows. Pick a small neighborhood $U\subs\Sigma$ of $x$ which is homeomorphic to $\RR^2$ and does not contain any branching points other than (possibly) $x$; here we can define a loop $\map{\beta}{[0,1]}{U\setminus\{x\}}$ that goes around $x$ once (we do not care about the orientation). Finally, pick a path $\map{\alpha}{[0,1]}{\holed{\Sigma}}$ connecting $x_0$ to $\beta(0)=\beta(1)$, and define $\gamma=\alpha\cat\beta\cat\invpath{\alpha}$, where $\cat$ is the concatenation of paths and $\invpath{\alpha}$ denotes the inverse path $\invpath{\alpha}(t)=\alpha(1-t)$. Of course, the homotopy class of $\gamma\in\pi_1(\holed{\Sigma},x_0)$ depends on $\alpha$ and even $\beta$ (we could have chosen $\invpath{\beta}$ instead); we will say that any path constructed with the procedure we have described is a \emph{loop around $x$} (based at $x_0$).
\begin{center}
\tikzsetnextfilename{monodromy-loop-around-point}
\begin{tikzpicture}[x={(5,0)},y={(0,5)}]
\begin{scope}[every path/.style={line width=\edgelinewidth}]
\draw[teal,postaction={decorate,decoration={markings,mark=between positions .02 and .98 step .5cm with {\arrowreversed{Stealth[scale=.75]}}}}] (.75,-0.02) to[out=180,in=-45,in looseness=.1] node[pos=.5,below,colored label=teal] {$\alpha$} (0,0) to[in=180,out=45,out looseness=.1] node[pos=.5,above] {$\invpath{\alpha}$} (.75,.02);
\draw[purple,postaction={decorate,decoration={markings,mark=between positions .02 and .98 step .5cm with {\arrowreversed{Stealth[scale=.75]}}}}] (.75,.02) to[out=0,in=180] (.9,.1) to[out=0,in=90] (1,0) node[right=2pt,colored label={purple}] {$\beta$} to[out=-90,in=0] (.9,-.1) to[out=180,in=0] (.75,-0.02);
\end{scope}
\foreach \p/\l in {{(0,0)}/{x_0},{(.9,0)}/x} {\fill \p circle (1pt); \node[above] at \p {$\l$};}
\end{tikzpicture}
\end{center}

We are finally ready to prove the connection between branched coverings and monodromy of the associated covering maps.
\begin{proposition}\label{hurwitz:th:monodromy-permutation-matches-partition}
Let $\map{f}{\tSigma}{\Sigma}$ be a branched covering of degree $d$, and let $\map{\holed{f}}{\holed{\tSigma}}{\holed{\Sigma}}$ be the associated covering map. Fix a base-point $x_0\in\holed{\Sigma}$ and a point $y\in\Sigma$. Let $\gamma$ be any loop around $y$. Consider the monodromy homomorphism $\map{\Mon}{\pi_1(\holed{\Sigma},x_0)}{\op{\symgroup(f^{-1}(x_0))}}$. Then the permutation $\Mon(\gamma)$ matches $\pi(y)$.
\end{proposition}
\begin{proof}
Let the preimages $\wtilde{y}_1,\ldots,\wtilde{y}_r$ of $y$ have local degrees $k_1,\ldots,k_r$. Fix disjoint trivializing neighborhoods $\wtilde{U}_1,\ldots,\wtilde{U}_r$ of $\wtilde{y}_1,\ldots,\wtilde{y}_r$ respectively. Let $U\subs\Sigma$ be an open neighborhood of $x$ which is homeomorphic to $\RR^2$ and such that $f^{-1}(U)\subs \wtilde{U}_1\cup\ldots\cup\wtilde{U}_r$. Since $\gamma$ is a loop around $y$, we can write $\gamma=\alpha\cat\beta\cat\invpath{\alpha}$ as described above. Up to homotopy, we can assume that $\beta(t)\in U$ for every $t\in[0,1]$. Let $z=\alpha(1)=\beta(0)=\beta(1)\in U$; note that every preimage of $z$ lies exactly in one $\wtilde{U}_l$. Finally, let $\wtilde{x}_1,\ldots,\wtilde{x}_d$ be the preimages of $x_0$. We will show that $\wtilde{x}_i$ and $\wtilde{x}_j$ belong to the same orbit of $\Mon(\gamma)$ if and only if $\lift{\alpha}{\wtilde{x}_i}(1)$ and $\lift{\alpha}{\wtilde{x}_j}(1)$ lie in the same $\wtilde{U}_l$. Since $f^{-1}(z)\cap \wtilde{U}_l$ has cardinality $k_l$ and $\pi(y)=[k_1,\ldots,k_r]$, this will complete the proof.
\begin{twoimplications}
\rightimplication
Let $\wtilde{x}_i\in f^{-1}(x_0)$, $\wtilde{z}_1=\lift{\alpha}{\wtilde{x}_i}(1)$. Let $l$ be the unique index such that $\wtilde{z}_1\in\wtilde{U}_l$. For each $m\ge 1$, inductively define $\wtilde{z}_{m+1}=\lift{\beta}{\wtilde{z}_m}(1)$. Since the support of $\beta$ lies entirely in $U$, we have that the support of $\lift{\beta}{\wtilde{z}_m}$ lies entirely in $\wtilde{U}_l$ and, therefore, $\wtilde{z}_{m+1}\in\wtilde{U}_l$ as well. Since $\wtilde{U}_l$ is a trivializing neighborhood of $\wtilde{y}_l$, which has local degree $k_l$, it is easy to see that the sequence $\wtilde{z}_1,\wtilde{z}_2,\ldots$ is periodic of period $k_l$, and that $f^{-1}(z)\cap\wtilde{U}_l=\{\wtilde{z}_1,\ldots\wtilde{z}_{k_l}\}$. It is also clear that
\[
\Mon(\gamma)(\wtilde{x}_i)=\lift{\invpath{\alpha}}{\wtilde{z}_2}(1)
\]
and, by induction, that
\begin{align*}
\Mon(\gamma)^s(\wtilde{x}_i)=\lift{\invpath{\alpha}}{\wtilde{z}_{s+1}}(1)&&\text{for every $s\ge 1$.}
\end{align*}
This shows that, if $\wtilde{x}_j=\Mon(\gamma)^s(\wtilde{x}_i)$, then $\lift{\alpha}{\wtilde{x}_j}(1)=\wtilde{z}_{s+1}\in\wtilde{U}_l$.
\leftimplication
Conversely, assume that $\lift{\alpha}{\wtilde{x}_j}(1)\in\wtilde{U}_l$. This implies that $\lift{\alpha}{\wtilde{x}_j}(1)=\wtilde{z}_s$ for some $s\ge 1$. But then
\[
\wtilde{x}_j=\lift{\invpath{\alpha}}{\wtilde{z}_s}(1)=\Mon(\gamma)^{s-1}(\wtilde{x}_i),
\]
so $\wtilde{x}_i$ and $\wtilde{x}_j$ belong to the same orbit of $\Mon(\gamma)$.\qedhere
\end{twoimplications}
\end{proof}

\section{Monodromy and realizability}

From \Cref{hurwitz:th:monodromy-permutation-matches-partition}, we can derive a group-theoretic criterion for the realizability of a given combinatorial datum.

\begin{proposition}\label{hurwitz:th:monodromy-realizability-general}
Let $\Sigma$ be a surface, $d\ge 1$ be an integer, and $\pi_1,\ldots,\pi_n\in\Partitions{d}$ be partitions of $d$. Let $ x_1,\ldots x_n\in\Sigma$ be distinct points; define $\holed{\Sigma}=\Sigma\setminus\{x_1,\ldots,x_n\}$. Fix a base-point $x_0\in\holed{\Sigma}$ and loops $\gamma_1,\ldots,\gamma_n\in\pi_1(\holed{\Sigma},x_0)$ around $x_1,\ldots,x_n$ respectively. Then the following are equivalent to each other.
\begin{enumroman}
\item There exist a surface $\tSigma$ and a branched covering $\map{f}{\tSigma}{\Sigma}$ with branching datum $\DD(f)=\datum{\tSigma,\Sigma}{d}{\pi_1,\ldots,\pi_n}$.
\item There exists a group homomorphism $\map{\psi}{\pi_1(\holed{\Sigma},x_0)}{\op{\symgroup[d]}}$ such that $\psi(\pi_1(\holed{\Sigma},x_0))\subgroup\op{\symgroup[d]}$ acts transitively on $\{1,\ldots,d\}$ and $[\psi(\gamma_i)]=\pi_i$ for every $1\le i\le n$.
\end{enumroman}
\end{proposition}
\begin{proof}
We show the two implications.
\begin{twoimplications}
\rightimplication
Assume that we are given a branched covering $\map{f}{\tSigma}{\Sigma}$ with $\DD(f)=\datum{\tSigma,\Sigma}{d}{\pi_1,\ldots,\pi_n}$. Since surfaces are homogeneous, we can assume that the branching points are exactly $x_1,\ldots,x_n$, with associated partitions $\pi_1,\ldots,\pi_n$ respectively. Let $\map{\holed{f}}{\holed{\tSigma}}{\holed{\Sigma}}$ be the induced covering map, and let $\map{\Mon}{\pi_1(\holed{\Sigma},x_0)}{\op{\symgroup[d]}}$ be the monodromy homomorphism. By \cref{hurwitz:th:monodromy-permutation-matches-partition} we have that $[\Mon(\gamma_i)]=\pi_i$ for every $1\le i\le n$, and the action of $\Mon(\pi_1(\holed{\Sigma},x_0))$ on $\{1,\ldots,d\}$ is transitive since $\holed{\tSigma}$ is connected. Therefore setting $\psi=\Mon$ gives the desired homomorphism.
\leftimplication
Assume conversely that we are given a group homomorphism $\map{\psi}{\pi_1(\holed{\Sigma},x_0)}{\op{\symgroup[d]}}$. \Cref{hurwitz:th:covering-with-given-monodromy} gives a covering map $\map{\holed{f}}{\holed{\tSigma}}{\holed{\Sigma}}$ of degree $d$ with monodromy $\Mon=\psi$. The non-compact surface $\holed{\tSigma}$ is connected by the transitivity hypothesis. We can ``fill the holes'' in $\holed{\Sigma}$ and $\holed{\tSigma}$ to get a branched covering between compact surfaces; here are the details. Consider one of the points $x_i$, and fix an open neighborhood $U\subs\Sigma$ of $x_i$ homeomorphic to $\RR^2$. Now $(\holed{f})^{-1}(U\setminus\{x_i\})$ is a covering space of $U\setminus\{x_i\}\iso\RR^2\setminus\{0\}$. By the classification of covering spaces, we get that $(\holed{f})^{-1}(U\setminus\{x_i\})$ is a disjoint union of punctured disks $\wtilde{V}_1\sqcup\ldots\sqcup\wtilde{V}_r$; moreover, for each $1\le j\le r$ we have charts $\map{\phi_j}{U\setminus\{x_i\}}{\CC\setminus\{0\}}$, $\map{\wtilde{\phi}_j}{\wtilde{V}_j}{\CC\setminus\{0\}}$ and a positive integer $k_j$ such that the diagram
\begin{diagram}
\wtilde{V}_j\rar{\holed{f}}\dar{\wtilde{\phi}_j}&U\setminus\{x_i\}\dar{\phi_j}\\
\CC\setminus\{0\}\rar{F_{k_j}}&\CC\setminus\{0\}
\end{diagram}
commutes, where $F_{k_j}(\xi)=\xi^{k_j}$. We can now ``fill the hole'' in $\wtilde{V}_j$ by considering the surface $\holed{\tSigma}_{\text{fill}}=\holed{\tSigma}\sqcup\CC/\sim$, where $y\sim \wtilde{\phi}_j(y)$ for each $y\in\wtilde{V}_j$. It is clear that $\holed{f}$ extends to a continuous function $\umap{\holed{\tSigma}_{\text{fill}}}{\Sigma}$ by sending $0\in\CC$ to $x_i$; this map is locally modeled on $\xi\mapsto \xi^{k_j}$ near $0$. If we ``fill the hole'' in $\wtilde{V}_j$ for $1\le j\le r$, and then repeat the process for each $x_i$, we end up with a surface $\tSigma$ and a map $\map{f}{\tSigma}{\Sigma}$. It is easy to see from the construction that $\tSigma$ is compact and connected, and that $f$ is a branched covering. Since the associated covering map is exactly $\map{\holed{f}}{\holed{\tSigma}}{\holed{\Sigma}}$, \cref{hurwitz:th:monodromy-permutation-matches-partition} implies that $\pi(x_i)$ matches $\Mon(\gamma_i)=\psi(\gamma_i)$, so $\pi(x_i)=\pi_i$.\qedhere
\end{twoimplications}
\end{proof}

By combining \cref{hurwitz:th:monodromy-realizability-general} with the classification of surfaces, we obtain the following criteria for realizability.

\bgroup
\tikzset{loop around point/.pic={
\draw[black edge] (0,0) to[in=180,out=45,out looseness=.1] (.75,.02) to[out=0,in=180] (.9,.1) to[out=0,in=90] (1,0) node[above] {$a_{#1}$} to[out=-90,in=0] (.9,-.1) to[out=180,in=0] (.75,-0.02) to[out=180,in=-45,in looseness=.1] (0,0);
\filldraw[surf boundary,fill=white] (.9,0) circle(.05);
}}
\begin{corollary}\label{hurwitz:th:monodromy-realizability-orientable}
Let $\surf{g}$ be the connected sum of $g\ge 0$ tori, $d\ge 1$ be an integer, and $\pi_1,\ldots,\pi_n\in\Partitions{d}$ be partitions of $d$. Let $\tSigma$ be an orientable surface with
\[
\chi(\tSigma)=d\chi(\surf{g})-v(\pi_1)-\ldots-v(\pi_n).
\]
Then a combinatorial datum $\DD=\datum{\tSigma,\surf{g}}{d}{\pi_1,\ldots,\pi_n}$ is realizable if and only if there exist permutations $\alpha_1,\ldots,\alpha_n,\beta_1,\ldots,\beta_g,\gamma_1,\ldots,\gamma_g\in\symgroup[d]$ such that:
\begin{enumroman}
\item\label{hurwitz:it:monodomy-realizability-orientable-first} $[\alpha_i]=\pi_i$ for each $1\le i\le n$;
\item $[\beta_1,\gamma_1]\cdots[\beta_g,\gamma_g]\cdot\alpha_1\cdots\alpha_n=1$;
\item\label{hurwitz:it:monodomy-realizability-orientable-last} the subgroup $\angled{\alpha_1,\ldots,\alpha_n,\beta_1,\ldots,\beta_g,\gamma_1,\ldots,\gamma_g}\subgroup\symgroup[d]$ acts transitively on $\{1,\ldots,d\}$.
\end{enumroman}
\end{corollary}
\begin{proof}
Let $\holed{\surf{g}}$ be the non-compact surface obtained by removing $n$ points $x_1,\ldots,x_n$ from $\Sigma_g$. Fix a base-point $x_0\in\holed{\surf{g}}$. Once we observe that $\pi_1(\holed{\surf{g}},x_0)$ has a presentation
\[
\pi_1(\holed{\surf{g}},x_0)=\angled{a_1,\ldots,a_n,b_1,\ldots,b_g,c_1,\ldots,c_g\mid [b_1,c_1]\cdots[b_g,c_g]\cdot a_1\cdots a_n},
\]
where $a_i$ is a loop around $x_i$, \cref{hurwitz:th:monodromy-realizability-general} implies that the existence of permutations satisfying conditions \ref{hurwitz:it:monodomy-realizability-orientable-first}--\ref{hurwitz:it:monodomy-realizability-orientable-last} is equivalent to the existence of a realizable combinatorial datum $\DD_1=\datum{\tSigma_1,\surf{g}}{d}{\pi_1,\ldots,\pi_n}$.
\begin{center}
\tikzsetnextfilename{monodromy-fundamental-group-orientable}
\begin{tikzpicture}[x={(2.5,0)},y={(0,2.5)},pics/commutator/.style n args={3}{code={
\tikzmath{\i0=#1;\i1=#1+1;\i2=#1+2;\i3=#1+3;\i4=#1+4;}
\begin{scope}[every path/.style={line width=\edgelinewidth}]
\draw[#2,postaction={decorate,decoration={markings,mark=at position.5 with {\arrow[xshift=3.3pt]{Stealth[]}}}}] (\i0) -- (\i1) node[midway,auto,colored label={#2},swap] {$b_{#3}$};
\draw[#2,postaction={decorate,decoration={markings,mark=at position.5 with {\arrow[xshift=3.3pt]{Stealth[]}}}}] (\i3) -- (\i2) node[midway,auto,colored label={#2}] {$b_{#3}$};
\draw[#2,postaction={decorate,decoration={markings,mark=at position.5 with {\arrow[xshift=6.6pt]{Stealth[] Stealth[]}}}}] (\i1) -- (\i2) node[midway,auto,colored label={#2},swap] {$c_{#3}$};
\draw[#2,postaction={decorate,decoration={markings,mark=at position.5 with {\arrow[xshift=6.6pt]{Stealth[] Stealth[]}}}}] (\i4) -- (\i3) node[midway,auto,colored label={#2}] {$c_{#3}$};
\end{scope}
}}]
\pgfsetlayers{main,graph vertex}
\begin{pgfonlayer}{graph vertex}
\foreach \i[evaluate=\i as \i using int(\i)] in {0,...,14} {
\fill ({(\i-4)*360/14}:1) coordinate (\i) circle(1pt);
}\end{pgfonlayer}
\fill[disk 1] (0) \foreach \i[evaluate=\i as \i using int(\i)] in {1,...,13} { -- (\i)} -- cycle;
%\node[below=4pt] at (0) {$x_0$};
\foreach \i/\j in {4/5,9/10} {\draw[surf boundary dashed] (\i) -- (\j);}
\pic{commutator={0}{purple}{1}};
\pic{commutator={5}{teal}{i}};
\pic{commutator={10}{green}{g}};
\pic[rotate=100] at (0) {loop around point=1};
\pic[rotate=50] at (0) {loop around point=n};
\foreach \a in {65,75,85} {\filldraw[surf boundary,fill=white] ($(0)+(\a:.9)$) circle (.02);}
\end{tikzpicture}
\end{center}
As discussed in \cref{hurwitz:rm:orientability-of-tsigma}, the surface $\tSigma_1$ is necessarily orientable. Moreover, by the \RH{} formula, we have that
\[
\chi(\tSigma_1)=d\chi(\surf{g})-v(\pi_1)-\ldots-v(\pi_n)=\chi(\tSigma),
\]
hence $\tSigma_1=\tSigma$ and $\DD_1=\DD$.
\end{proof}
\begin{comment}
\begin{remark}\label{hurwitz:rm:sigma-tilde-unique-orientable}
Given $\surf{g}$, $d$ and $\pi_1,\ldots,\pi_n$, there is at most one surface $\tSigma$ such that $\DD=\datum{\tSigma,\surf{g}}{d}{\pi_1,\ldots,\pi_n}$ is a candidate datum. In fact, the \RH{} formula gives
\[
\chi(\tSigma)=d\chi(\surf{g})-v(\pi_1)-\ldots-v(\pi_n)
\]
which, in turn, uniquely determines the orientable surface $\tSigma$. As an application, assume that we have a candidate datum $\DD=\datum{\tSigma,\Sigma}{d}{\pi_1,\ldots,\pi_n}$, and we find permutations in $\symgroup[d]$ satisfying conditions \ref{hurwitz:it:monodomy-realizability-orientable-first}--\ref{hurwitz:it:monodomy-realizability-orientable-last} of \cref{hurwitz:th:monodromy-realizability-orientable}: this immediately implies that $\DD$ is realizable.
\end{remark}
\end{comment}

Before stating the monodromy criterion for non-orientable surfaces, we recall a few basic facts about covering maps and orientability.

Given a non-orientable topological manifold $M$, let $\map{\omega}{\what{M}}{M}$ be its orientable double covering. Fix a base-point $x_0\in M$, and consider the monodromy homomorphism
\[
\map{w}{\pi_1(M,x_0)}{\op{\symgroup(\omega^{-1}(x_0))}\iso\ZZ/2}.
\]
We denote the kernel of $w$ (which is a subgroup of $\pi_1(M,x_0)$ of index $2$) by $W(M,x_0)$. Note that $W(M,x_0)=\omega_\ast(\pi_1(\what{M},\what{x}_0))$ for every base-point $\what{x}_0\in\omega^{-1}(x_0)$, where $\map{\omega_\ast}{\pi_1(\what{M},\what{x}_0)}{\pi_1(M,x_0)}$ denotes the map induced by $\omega$ on the fundamental groups. Given a connected topological manifold $N$ and a covering map $\map{f}{N}{M}$, elementary properties of covering spaces imply that the following are equivalent:
\begin{enumroman}
\item $N$ is orientable;
\item $f_\ast(\pi_1(N,y_0))\subgroup W(M,x_0)$ for every base-point $y_0\in f^{-1}(x_0)$;
\item there exists a covering map $\map{\what{f}}{N}{\what{M}}$ such that $f$ factors as $f=\omega\circ\what{f}$.
\end{enumroman}

\begin{corollary}\label{hurwitz:th:monodromy-realizability-non-orientable}
Let $\nosurf{g}$ be the connected sum of $g\ge 1$ projective planes, $d\ge 1$ an integer, $\pi_1,\ldots,\pi_n\in\Partitions{d}$ partitions of $d$. Let $\tSigma$ be a surface with
\[
\chi(\tSigma)=d\chi(\surf{g})-v(\pi_1)-\ldots-v(\pi_n).
\]
 Then there exists a realizable combinatorial datum $\DD=\datum{\tSigma,\nosurf{g}}{d}{\pi_1,\ldots,\pi_n}$ if and only if there exist permutations $\alpha_1,\ldots,\alpha_n,\beta_1,\ldots,\beta_g\in\symgroup[d]$ such that:
\begin{enumroman}
\item\label{hurwitz:it:monodomy-realizability-non-orientable-first} $[\alpha_i]=\pi_i$ for each $1\le i\le n$;
\item $\beta_1^2\cdots\beta_g^2\cdot\alpha_1\cdots\alpha_n=1$;
\item\label{hurwitz:it:monodomy-realizability-non-orientable-third} the subgroup $\angled{\alpha_1,\ldots,\alpha_n,\beta_1,\ldots,\beta_g}\subgroup\symgroup[d]$ acts transitively on $\{1,\ldots,d\}$;
\item\label{hurwitz:it:monodomy-realizability-non-orientable-fourth} $\tSigma$ is non-orientable if and only if there exists a permutation $\gamma\in\symgroup[d]$ which has a fixed point and can be written as the product of an odd number of $\beta_j$ and any number of $\alpha_i$.
\end{enumroman}
\end{corollary}
\begin{proof}
The first part of the proof is identical to that of \cref{hurwitz:th:monodromy-realizability-orientable}: simply let $\holed{\nosurf{g}}=\nosurf{g}\setminus\{x_1,\ldots,x_n\}$ and observe that there is a presentation
\[
\pi_1(\holed{\nosurf{g}},x_0)=\angled{a_1,\ldots,a_n,b_1,\ldots,b_g\mid b_1^2\cdots b_g^2\cdot a_1\cdots a_n},
\]
where $a_i$ is a loop around $x_i$.
\begin{center}
\tikzsetnextfilename{monodromy-fundamental-group-non-orientable}
\begin{tikzpicture}[x={(2.5,0)},y={(0,2.5)},pics/square/.style n args={3}{code={
\tikzmath{\i0=#1;\i1=#1+1;\i2=#1+2;}
\begin{scope}[every path/.style={line width=\edgelinewidth}]
\draw[#2,postaction={decorate,decoration={markings,mark=at position.5 with {\arrow[xshift=3.3pt]{Stealth[]}}}}] (\i0) -- (\i1) node[midway,auto,colored label={#2},swap] {$b_{#3}$};
\draw[#2,postaction={decorate,decoration={markings,mark=at position.5 with {\arrow[xshift=3.3pt]{Stealth[]}}}}] (\i1) -- (\i2) node[midway,auto,colored label={#2},swap] {$b_{#3}$};
\end{scope}
}}]
\pgfsetlayers{main,graph vertex}
\begin{pgfonlayer}{graph vertex}
\foreach \i[evaluate=\i as \i using int(\i)] in {0,...,8} {
\fill ({(\i-2)*360/8}:1) coordinate (\i) circle(1pt);
}\end{pgfonlayer}
\fill[disk 1] (0) \foreach \i[evaluate=\i as \i using int(\i)] in {1,...,7} { -- (\i)} -- cycle;
\foreach \i/\j in {2/3,5/6} {\draw[surf boundary dashed] (\i) -- (\j);}
\pic{square={0}{purple}{1}};
\pic{square={3}{teal}{i}};
\pic{square={6}{green}{g}};
\pic[rotate=115] at (0) {loop around point=1};
\pic[rotate=65] at (0) {loop around point=n};
\foreach \a in {100,90,80} {\filldraw[surf boundary,fill=white] ($(0)+(\a:.9)$) circle (.02);}
\end{tikzpicture}
\end{center}
Then, \cref{hurwitz:th:monodromy-realizability-general} implies that the existence of permutations satisfying conditions \ref{hurwitz:it:monodomy-realizability-non-orientable-first}--\ref{hurwitz:it:monodomy-realizability-non-orientable-third} is equivalent to the existence of a realizable combinatorial datum $\DD_1=\datum{\tSigma_1,\nosurf{g}}{d}{\pi_1,\ldots,\pi_n}$. As far as condition \ref{hurwitz:it:monodomy-realizability-non-orientable-fourth} is concerned, we just have to prove that $\tSigma_1$ is orientable if and only if there exists a permutation in $\symgroup[d]$ which has a fixed point and can be written as the product of an odd number of $\beta_j$ and any number of $\alpha_i$; since $\chi(\tSigma)=\chi(\tSigma_1)$ by the \RH{} formula, this will conclude the proof.

Note that $w(a_1)=\ldots=w(a_n)=0$, while $w(b_1)=\ldots=w(b_g)=1$. Let $\map{f_1}{\tSigma_1}{\nosurf{g}}$ be a branched covering realizing $\DD_1$; let $\map{\holed{f}}{\holed{\tSigma_1}}{\holed{\nosurf{g}}}$ be the covering map associated to the branched covering, and $\map{\Mon}{\pi_1(\holed{\nosurf{g}},x_0)}{\op{\symgroup[d]}}$ be the monodromy homomorphism. Fix a base-point $\wtilde{x}_0\in f^{-1}(x_0)$; we have that $\holed{\tSigma}$ is orientable if and only if $f_\ast(\pi_1(\holed{\tSigma_1},\wtilde{x}_0))\subgroup W(\holed{\nosurf{g}},x_0)$. Since $f_\ast(\pi_1(\holed{\tSigma_1},\wtilde{x}_0))=\stab{\pi_1(\holed{\nosurf{g}},x_0)}{\wtilde{x}_0}$, it follows that $\holed{\tSigma}$ is non-orientable if and only if there exists a loop $c\in\pi_1(\holed{\nosurf{g}},x_0)$ such that $w(c)=1$ and $\wtilde{x}_0\mon c=\wtilde{x}_0$. By applying $\Mon$, we see that this is equivalent to the existence of a permutation $\gamma\in\symgroup[d]$ which has a fixed point and can be written as the product of an odd number of $\beta_j$ and an arbitrary number of $\alpha_i$; note that the exact fixed point does not matter, since the subgroup generated by $\alpha_1,\ldots,\alpha_n,\beta_1,\ldots,\beta_g$ acts transitively on $\{1,\ldots,d\}$. Finally, observe that $\holed{\tSigma_1}$ is orientable if and only if $\tSigma_1$ is.
\end{proof}
\egroup
\begin{comment}
\begin{remark}\label{hurwitz:rm:sigma-tilde-unique-non-orientable}
Given $\nosurf{g}$, $d$ and $\pi_1,\ldots,\pi_n$, there are at most two surfaces $\tSigma$, one orientable and one non-orientable, such that $\DD=\datum{\tSigma,\nosurf{g}}{d}{\pi_1,\ldots,\pi_n}$ is a candidate datum: like in \cref{hurwitz:rm:sigma-tilde-unique-orientable}, the \RH{} formula fixes $\chi(\tSigma)$ which, together with orientability, uniquely determines the surface $\tSigma$.
\end{remark}
\end{comment}

We will now show that realizability in the case where $\Sigma$ is non-orientable and $\tSigma$ is orientable can be reduced to a situation where both surfaces are orientable.

\begin{proposition}\label{hurwitz:th:monodromy-realizability-double-covering}
Let $\DD=\datum{\tSigma,\Sigma}{d}{\pi_1,\ldots\pi_n}$ be a combinatorial datum, with $\Sigma$ non-orientable and $\tSigma$ orientable. Then $\DD$ is realizable if and only if $d$ is even and there exist partitions $\pi'_1,\ldots,\pi'_n,\pi''_1,\ldots,\pi''_n\in\Partitions{d/2}$ with $\pi_i=\pi'_i\cup\pi''_i$ for each $1\le i\le n$, such that the combinatorial datum
\[
\DD'=\datum{\tSigma,\hSigma}{d/2}{\pi'_1,\ldots,\pi'_n,\pi''_i,\ldots,\pi''_n}
\]
is realizable, where $\hSigma$ is the orientable double covering of $\Sigma$.
\end{proposition}
%\todo{However, not every splitting leads to a realizable datum.}
\begin{proof}
We show the two implications.
\begin{twoimplications}
\rightimplication
Assume that $\DD$ is realized by a branched covering $\map{f}{\tSigma}{\Sigma}$. Let $\map{\holed{f}}{\holed{\tSigma}}{\holed{\Sigma}}$ be the associated covering map, and let $\map{\omega}{\hSigma}{\Sigma}$ be the orientable double covering. Define $\holed{\hSigma}=\omega^{-1}(\holed{\Sigma})$, and denote by $\holed{\omega}$ the restriction of $\omega$ to $\holed{\tSigma}$; clearly, $\map{\holed{\omega}}{\holed{\hSigma}}{\holed{\Sigma}}$ is the orientable double covering of $\holed{\Sigma}$. Since $\holed{\tSigma}$ is orientable while $\holed{\Sigma}$ is not, $\holed{f}$ factors as $\holed{f}=\holed{\what{f}}\circ\holed{\omega}$ for some covering map $\map{\holed{\what{f}}}{\holed{\tSigma}}{\holed{\hSigma}}$; in particular, $d$ is even and $\holed{\what{f}}$ has degree $d/2$. The conclusion follows from a routine topological argument; for the sake of completeness, we will now report the details. Fix a branching point $x\in\Sigma\setminus\holed{\Sigma}$; let $f^{-1}(x)=\{\wtilde{x}_1,\ldots,\wtilde{x}_r\}$ and $\omega^{-1}(x)=\{\what{x}_1,\what{x}_2\}$. There exists a small open neighborhood $U\subs\Sigma$ of $x$ homeomorphic to $\RR^2$ such that:
\begin{itemize}
\item $f^{-1}(U)=\wtilde{U}_1\sqcup\ldots\sqcup\wtilde{U}_r$, where $\wtilde{U}_i$ is a trivializing neighborhood of $\wtilde{x}_i$;
\item $\omega^{-1}(U)=\what{U}_1\sqcup\what{U}_2$, where $\what{U}_i$ is a trivializing neighborhood of $\what{x}_i$.
\end{itemize}
It is now clear that $\map{\holed{\what{f}}}{\holed{\tSigma}}{\holed{\hSigma}}$ extends to a branched covering $\map{\what{f}}{\tSigma}{\hSigma}$: for each $1\le i\le r$, we simply set $\what{f}(\wtilde{x}_i)=\what{x}_1$ if $\holed{\what{f}}(\wtilde{U}_i\setminus\{\wtilde{x}_i\})=\what{U}_1\setminus\{\what{x}_1\}$, and $\what{f}(\wtilde{x}_i)=\what{x}_2$ otherwise; obviously, we must repeat this process for every branching point $x$. The branched covering $\what{f}$ has degree $d/2$; moreover, it is easy to see from the construction that $\pi_{f}(x)=\pi_{\what{f}}(\what{x}_1)\cup\pi_{\what{f}}(\what{x}_2)$, where we have used the subscript in order to clarify which branched covering we were referring to. Therefore the realizable combinatorial datum $\DD(\what{f})$ has the required form.
\leftimplication
Conversely, assume that $d$ is even and we are given a branched covering $\map{\what{f}}{\tSigma}{\hSigma}$ with branching datum
\[
\DD(\what{f})=\datum{\tSigma,\hSigma}{d/2}{\pi_1',\ldots,\pi_n',\pi_1'',\ldots,\pi_n''},
\]
where $\pi_i=\pi_i'\cup\pi_i''$ for every $1\le i\le n$. Let $x_1',\ldots,x_n',x_1'',\ldots,x_n''\in\hSigma$ be the branching points corresponding to the similarly named partitions of $\DD(\what{f})$. Since surfaces are homogeneous, we can assume that $\omega(x_i')=\omega(x_i'')$ for each $1\le i\le n$, where $\map{\omega}{\hSigma}{\Sigma}$ is the covering map. It is now easy to see that
\[
\DD(\omega\circ\what{f})=\datum{\tSigma,\Sigma}{d}{\pi_1'\cup\pi_1'',\ldots,\pi_n'\cup\pi_n''}.\qedhere\sdlendhere
\]
\end{twoimplications}
\end{proof}

We conclude this introductory chapter by showing that the four conditions described in \cref{hurwitz:df:candidate-datum} are actually necessary for a combinatorial datum to be realizable.

\begin{proposition}\label{hurwitz:th:candidate-datum-necessary-conditions}
Let $\map{f}{\tSigma}{\Sigma}$ be a branched covering. Then its branching datum $\DD(f)$ is a candidate datum.
\end{proposition}
\begin{proof}
Conditions \ref{hurwitz:it:candidate-datum-def:1} and \ref{hurwitz:it:candidate-datum-def:3} of \cref{hurwitz:df:candidate-datum} were already addressed, respectively, in \cref{hurwitz:th:riemann-hurwitz-formula} and \cref{hurwitz:rm:orientability-of-tsigma}. Condition \ref{hurwitz:it:candidate-datum-def:4} follows immediately from \cref{hurwitz:th:monodromy-realizability-double-covering}, so we only have to show that the total branching number $v(\pi_1)+\ldots+v(\pi_n)$ is even. Using the notation of \cref{hurwitz:th:monodromy-realizability-orientable,hurwitz:th:monodromy-realizability-non-orientable}, by \cref{hurwitz:th:branching-number-permutations-product} we have that
\[
v(\alpha_1\cdots\alpha_n)\equiv
\begin{dcases*}
v([\beta_1,\gamma_1]\cdots[\beta_g,\gamma_g])&if $\Sigma$ is orientable,\\
v(\beta_1^2\cdots\beta_g^2)&if $\Sigma$ is non-orientable
\end{dcases*}
\pmod{2}.
\]
But commutators and squares are even permutations, therefore
\[
v(\pi_1)+\ldots+v(\pi_n)\equiv v(\alpha_1)+\ldots v(\alpha_n)\equiv v(\alpha_1\cdots\alpha_n)\equiv 0\pmod{2}.\qedhere
\]
\end{proof}

\chapter{Monodromy}

\section{Symmetric group and partitions}

\section{Branched covering action of the fundamental group}

\section{Monodromy and realizability}

\begin{proposition}\label{monodromy:th:monodromy-realizability-orientable}
Let $\surf{g}$ be the connected sum of $g\ge 0$ tori, $d\ge 1$ an integer, $\pi_1,\ldots,\pi_n\in\Partitions{d}$ partitions of $d$. Then there exists a realizable combinatorial datum $\DD=\datum{\tSigma,\surf{g}}{d}{\pi_1,\ldots,\pi_n}$ if and only if there exist permutations $\alpha_1,\ldots,\alpha_g,\beta_1,\ldots,\beta_n,\gamma_1,\ldots,\gamma_n\in\symgroup[g]$ such that:
\begin{enumerate}
\item $[\alpha_i]=\pi_i$ for each $1\le i\le n$;
\item $[\beta_1,\gamma_1]\cdots[\beta_g,\gamma_g]\cdot\alpha_1\cdots\alpha_n=1$;
\item the subgroup $\angled{\alpha_1,\ldots,\alpha_n,\beta_1,\ldots,\beta_g,\gamma_1,\ldots,\gamma_g}\subgroup\symgroup[d]$ acts transitively on $\{1,\ldots,d\}$.
\end{enumerate}
In this case, $\tSigma$ is necessarily orientable.
\end{proposition}
\begin{remark}\label{monodromy:rm:sigma-tilde-unique-orientable}
Given $\surf{g}$, $d$ and $\pi_1,\ldots,\pi_n$, there is at most one surface $\tSigma$ such that $\DD=\datum{\tSigma,\surf{g}}{d}{\pi_1,\ldots,\pi_n}$ is a candidate datum. In fact, the \RH{} formula gives
\[
\chi(\tSigma)=d\chi(\surf{g})-v(\pi_1)-\ldots-v(\pi_n)
\]
which, in turn, uniquely determines the orientable surface $\tSigma$.
\end{remark}

\begin{proposition}\label{monodromy:th:monodromy-realizability-non-orientable}
Let $\nosurf{g}$ be the connected sum of $g\ge 1$ projective planes, $d\ge 1$ an integer, $\pi_1,\ldots,\pi_n\in\Partitions{d}$ partitions of $d$. Then there exists a realizable combinatorial datum $\DD=\datum{\tSigma,\nosurf{g}}{d}{\pi_1,\ldots,\pi_n}$ if and only if there exist permutations $\alpha_1,\ldots,\alpha_n,\beta_1,\ldots,\beta_g\in\symgroup[d]$ such that:
\begin{enumerate}
\item $[\alpha_i]=\pi_i$ for each $1\le i\le n$;
\item $\beta_1^2\cdots\beta_g^2\cdot\alpha_1\cdots\alpha_n=1$;
\item the subgroup $\angled{\alpha_1,\ldots,\alpha_n,\beta_1,\ldots,\beta_g}\subgroup\symgroup[d]$ acts transitively on $\{1,\ldots,d\}$.
\end{enumerate}
In this case, $\tSigma$ is orientable if and only if ??.
\end{proposition}
\begin{remark}\label{monodromy:rm:sigma-tilde-unique-non-orientable}
Given $\nosurf{g}$, $d$ and $\pi_1,\ldots,\pi_n$, there are at most two surfaces $\tSigma$, one orientable and one non-orientable, such that $\DD=\datum{\tSigma,\nosurf{g}}{d}{\pi_1,\ldots,\pi_n}$ is a candidate datum: like in \cref{monodromy:rm:sigma-tilde-unique-orientable}, the \RH{} formula fixes $\chi(\tSigma)$ which, together with orientability, uniquely determines $\tSigma$.
\end{remark}

\begin{proposition}\label{monodromy:th:monodromy-realizability-double-covering}
Let $\DD=\datum{\tSigma,\Sigma}{d}{\pi_1,\ldots\pi_n}$ be a combinatorial datum, with $\Sigma$ non-orientable and $\tSigma$ orientable. Then $\DD$ is realizable if and only if $d$ is even and there exist partitions $\pi'_1,\ldots,\pi'_n,\pi''_1,\ldots,\pi''_n\in\Partitions{d/2}$ with $\pi_i=\pi'_i\cup\pi''_i$ for each $1\le i\le n$, such that the combinatorial datum
\[
\DD'=\datum{\tSigma,\hSigma}{d/2}{\pi'_1,\ldots,\pi'_n,\pi''_i,\ldots,\pi''_n}
\]
is realizable, where $\hSigma$ is the orientable double covering of $\Sigma$.
\end{proposition}

\section{Non-positive Euler characteristic}

\begin{lemma}\label{monodromy:th:product-of-two-cycles}
Let $\alpha\in\symgroup[d]$ be a permutation. Set $r=d-v(\alpha)$, and let $t\ge 0$ be an integer such that $2t\le v(\alpha)$. Then $\alpha$ can be written as the product of a $(r+2t)$\=/cycle and a $d$\=/cycle.
\end{lemma}
\begin{proof}
Without loss of generality, assume that
\[
\alpha=\cycle{1,\ldots,d_1}\cycle{d_1+1,\ldots,d_2}\cdots\cycle{d_{r-1}+1,\ldots,d_r},
\]
where $d_r=d$. Fix
\begin{align*}
\beta_0=\cycle{1,b_1,b_2,\ldots,b_{2t}},&&\beta_1=\cycle{1,d_1+1,d_2+2,\ldots,d_{r-1}+1}.
\end{align*}
Set
\[
\beta=\beta_0\beta_1=\cycle{1,d_1+1,d_2+1,\ldots,d_{r-1}+1,b_1,b_2,\ldots,b_{2t}}.
\]
An easy computation shows that
\begin{align*}
\beta\alpha&=\beta_0\beta_1\alpha\\
&=\cycle{1,b_1,b_2,\ldots,b_{2t}}\cycle{1,2,\ldots,d}\\
&=\cycle{1,\ldots,b_1-1,b_2,\ldots,b_3-1,\ldots,b_4,\ldots,b_{2t-1}-1,b_{2t},\ldots,d,b_1,\ldots,b_2-1,b_3,\ldots,b_{2t}-1}.
\end{align*}
Writing $\alpha=\beta^{-1}(\beta\alpha)$ gives the desired decomposition.
\end{proof}

\begin{corollary}\label{monodromy:th:even-permutation-commutator-or-squares}
Let $\alpha\in\altgroup[d]$ be an even permutation. Then $\alpha$ can be written as:
\begin{enumerate}
\item a commutator $[\beta,\gamma]$, where $\gamma$ is a $d$\=/cycle;
\item a product of two squares $\delta^2\epsilon^2$, where $\delta\epsilon$ is a $d$\=/cycle.
\end{enumerate}
\end{corollary}
\begin{proof}
Since $\alpha$ is an even permutation, its branching number $v(\alpha)$ is even. By \cref{monodromy:th:product-of-two-cycles}, there exist two $d$\=/cycles $\tau,\sigma\in\symgroup[d]$ such that $\alpha=\tau\sigma$.
\begin{enumerate}
\item Since $\tau$ and $\sigma^{-1}$ are conjugated, there exists a permutation $\beta\in\symgroup[d]$ such that $\tau=\beta\sigma^{-1}\beta^{-1}$. Setting $\gamma=\sigma^{-1}$, we immediately get that
\[
\alpha=\tau\sigma=\beta\sigma^{-1}\beta^{-1}\sigma=\beta\gamma\beta^{-1}\gamma^{-1}=[\beta,\gamma].
\]
\item Since $\tau$ and $\sigma$ are conjugated, there exists a permutation $\delta\in\symgroup[d]$ such that $\tau=\delta\sigma\delta^{-1}$. Setting $\epsilon=\delta^{-1}\sigma$, we have that
\[
\alpha=\tau\sigma=\delta\sigma\delta^{-1}\sigma=\delta^2(\delta^{-1}\sigma)^2=\delta^2\epsilon^2.\qedhere
\]
\end{enumerate}
\end{proof}

\begin{theorem}
Let $\DD=\datum{\tSigma,\Sigma}{d}{\pi_1,\ldots,\pi_n}$ be a candidate datum. If $\chi(\Sigma)\le 0$, then $\DD$ is realizable.
\end{theorem}
\begin{proof}
Let us first assume that $\Sigma$ is orientable; this means that $\Sigma=\surf{g}$ is the connected sum of $g\ge 1$ tori, and that $\tSigma$ is orientable as well. Choose permutations $\alpha_1,\ldots,\alpha_n\in\symgroup[d]$ with $[\alpha_i]=\pi_i$ for each $1\le i\le n$. Let $\alpha=\alpha_1\cdots\alpha_n$. Since $\DD$ is a candidate datum, we have that
\[
v(\alpha)\equiv v(\alpha_1)+\ldots+v(\alpha_n)\equiv 0\pmod{2}.
\]
By \cref{monodromy:th:even-permutation-commutator-or-squares}, we can find permutations $\beta_1,\gamma_1\in\symgroup[d]$ such that $\alpha=[\gamma_1,\beta_1]$ and $\beta_1$ is a $d$\=/cycle. Set $\beta_2=\ldots=\beta_g=\gamma_2=\ldots=\gamma_g=\id\in\symgroup[d]$. All the conditions of \cref{monodromy:th:monodromy-realizability-orientable} are satisfied; since $\tSigma$ is orientable, this implies that $\DD$ is realizable (see \cref{monodromy:rm:sigma-tilde-unique-orientable}).

Assume now that $\Sigma$ and $\tSigma$ are both non-orientable; this means that $\Sigma=\nosurf{g}$ is the connected sum of $g\ge 2$ projective planes. Choose permutations $\alpha_1,\ldots,\alpha_n\in\symgroup[d]$ with $[\alpha_i]=\pi_i$ for each $1\le i\le n$. Let $\alpha=\alpha_1\cdots\alpha_n$. Similarly to what we did for the previous case, we can find $\beta_1,\beta_2\in\symgroup[d]$ such that $\alpha=\beta_2^{-2}\beta_1^{-2}$ and $\beta_2\beta_1$ is a $d$\=/cycle. By setting $\beta_3=\ldots=\beta_g=\id\in\symgroup[d]$, \cref{monodromy:th:monodromy-realizability-non-orientable} (together with \cref{monodromy:rm:sigma-tilde-unique-non-orientable}) implies the realizability of $\DD$.

Finally, consider the case where $\Sigma$ is non-orientable and $\tSigma$ is orientable. Since $\DD$ is a candidate datum, $d$ is even and there exist partitions $\pi'_1,\ldots,\pi'_n,\pi''_1,\ldots,\pi''_n\in\Partitions{d/2}$ with $\pi_i=\pi'_1\cup\pi''_i$ for each $1\le i\le n$. Let $\hSigma$ be the double orientable covering of $\Sigma$; by the first case we analyzed, the candidate datum
\[
\DD'=\datum{\tSigma,\hSigma}{d/2}{\pi'_1,\ldots,\pi'_n,\pi''_1,\ldots,\pi''_n}
\]
is realizable. By \cref{monodromy:th:monodromy-realizability-double-covering}, $\DD$ is realizable as well.
\end{proof}

\section{Products in symmetric groups}
\begin{lemma}\label{monodromy:th:same-number-of-cycles}
Let $X,Y$ be finite sets; denote by $h$ the cardinality of $Y$, and by $k$ the cardinality of $X\cap Y$. Let $\alpha\in\symgroup(X)$, $\beta\in\symgroup(Y)$, $\gamma\in\symgroup(X\cap Y)$. Assume that $\beta=\cycle{b_1,\ldots,b_h}$ is a $h$\=/cycle, and that $\gamma$ is a $k$\=/cycle of the form $\gamma=\cycle{b_{i_1},\ldots,b_{i_k}}$ with $1\le i_1\le\ldots\le i_k\le h$. Then $\alpha\in\symgroup(X)$ and $\alpha\gamma^{-1}\beta\in\symgroup(X\cup Y)$ have the same number of cycles.
\end{lemma}
\begin{proof}
Write $\gamma=\cycle{u_1,\ldots,u_k}$, where $u_j=b_{i_j}$. Without loss of generality, assume that $i_1=1$. Then we can write
\[
\beta=\cycle{u_1,\ldots,w_1,u_2,\ldots,w_2,u_3,\ldots,w_{k-1},u_k,\ldots,w_k},
\]
possibly with $u_j=w_j$ for some values of $j$. We immediately get that
\[
\gamma^{-1}\beta=\cycle{u_1,\ldots,w_1}\cycle{u_2,\ldots,w_2}\cdots\cycle{u_k,\ldots,w_k}.
\]
If we denote by $A_1,\ldots,A_r\subs X$ the orbits of $\alpha$, it is then easy to see that the orbits of $\alpha\gamma^{-1}\beta$ are $A_1',\ldots,A_r'\subs X\cup Y$, where
\[
A_j'=A_j\cup\bigcup_{u_l\in A_j}\{u_l,\ldots,w_l\}.\qedhere
\]
\end{proof}

\begin{comment}
\begin{lemma}
\todo{Maybe useless?}Let $\alpha,\beta\in\symgroup[d]$ be permutations. Then $v(\alpha\beta)\le v(\angled{\alpha,\beta})\le v(\alpha)+v(\beta)$. Moreover if $v(\angled{\alpha,\beta})=v(\alpha)+v(\beta)$ then $v(\alpha\beta)=v(\alpha)+v(\beta)$.
\end{lemma}
\begin{proof}
Let $q=v(\angled{\alpha,\beta})$, and let $\gamma=\alpha\beta$. Since $\gamma\in\angled{\alpha,\beta}$, the inequality $v(\gamma)\le q$ trivially holds. We have that $\alpha\beta\gamma^{-1}=1$; by \cref{?}, this implies that the combinatorial datum $\datum{\tSigma,S^2}{d}{[\alpha],[\beta],[\gamma]}$ is realizable for some orientable surface $\tSigma$. The \RH{} formula for this datum is
\[
v(\alpha)+v(\beta)+v(\gamma)=2d-\chi(\tSigma).
\]
Given that $\chi(\tSigma)\le 2\le 2(d-q)$, we find that
\[
v(\alpha)+v(\beta)=(v(\alpha)+v(\beta)+v(\gamma))-v(\gamma)\stackrel{(*)}{\ge} 2q-q=q\ge v(\gamma),
\]
which proves the first part of the lemma. Moreover, if $v(\alpha)+v(\beta)=q$, inequality $(*)$ must be an equality, hence $v(\gamma)=q$.
\end{proof}
\end{comment}

\begin{proposition}\label{monodromy:th:product-reduction-small-v}
Let $\pi,\rho\in\Partitions{d}$ be partitions of $d$. Assume that $v(\pi)+v(\rho)<d$. Then there exist permutations $\alpha,\beta\in\symgroup[d]$ with $[\alpha]=\pi$ and $[\beta]=\rho$ such that $v(\alpha\beta)=v(\alpha)+v(\beta)$.
\end{proposition}
\begin{proof}
First of all, note that the conclusion is trivial whenever $v(\pi)=0$ or $v(\rho)=0$. This already solves the cases $d=1$ and $d=2$. We now proceed by induction on $d\ge 3$, assuming that $v(\pi)>0$ and $v(\rho)>0$. Write $\pi=[a_1,\ldots,a_r]$, $\rho=[b_1,\ldots,b_s]$; without loss of generality, assume that $b_1>1$. Fix
\[
\beta=\cycle{1,\ldots,d_1}\cycle{d_1+1,\ldots,d_2}\cdots\cycle{d_{s-1}+1,\ldots,d_s},
\]
where $b_1=d_1\ge 2$, $b_i=d_i-d_{i-1}$ for $2\le i\le s$ (in particular, $d_s=d$). Note that
\[
d-1\ge v(\pi)+v(\rho)=(a_1-1)+\ldots+(a_r-1)+d-s\ge a_1-1+d-s,
\]
hence $a_1\le s$. Fix $\alpha_1=\cycle{1,d_1+1,\ldots,d_{a_1-1}+1}$, and let $A=\{1,d_1+1,\ldots,d_{a_1-1}+1\}$ be the support of $\alpha_1$. Define $Q=\{1,\ldots,d_{a_1}\}\setminus A$; note that $Q_1$ is non-empty, since $d_1+1\ge 3$ implies that $2\in Q_1$. Consider the partitions
\begin{align*}
\pi'=[a_2,a_3,\ldots,a_r],&&\rho'=[\card{Q},b_{a_1+1},\ldots,b_s].
\end{align*}
We have that
\begin{align*}
\sum\pi'=\sum\rho'=d-a_1,&&v(\pi')=v(\pi)-a_1+1,&&v(\rho')=v(\rho)-1.
\end{align*}
Since $d-a_1<d$ and $v(\pi')+v(\rho')=v(\pi)+v(\rho)-a_1<d-a_1$, by induction we find $\alpha',\beta'\in\symgroup(\{1,\ldots,d\}\setminus A)$ with $[\alpha']=\pi'$ and $[\beta']=\rho'$ such that $v(\alpha'\beta')=v(\alpha')+v(\beta')$. Up to conjugation, we may assume that
\[
\beta'=\beta_1\cycle{d_{a_1}+1,\ldots,d_{a_1+1}}\cycle{d_{a_1+1}+1,\ldots,d_{a_1+2}}\cdots\cycle{d_{s-1}+1,\ldots,d_s},
\]
where $\beta_1$ is the $\card{Q}$\=/cycle whose entries are the elements of $Q$ in increasing order. An easy computation shows that
\[
\alpha_1\beta=\cycle{1,\ldots,d_{a_1}}\cycle{d_{a_1}+1,\ldots,d_{a_1+1}}\cdots\cycle{d_{s-1}+1,\ldots,d_s}.
\]
Therefore, setting $\alpha=\alpha'\alpha_1$, we have that
\begin{align*}
\alpha\beta&=\alpha_1\alpha'\beta\\
&=\alpha'\cycle{1,\ldots,d_{a_1}}\cycle{d_{a_1}+1,\ldots,d_{a_1+1}}\cdots\cycle{d_{s-1}+1,\ldots,d_s}\\
&=\alpha'\beta'\beta_1^{-1}\cycle{1,\ldots,d_{a_1}}
\end{align*}
By \cref{monodromy:th:same-number-of-cycles}, this implies that $\alpha\beta$ has the same number of cycles as $\alpha'\beta'$, so that
\begin{align*}
v(\alpha\beta)&=a_1+v(\alpha'\beta')\\
&=a_1+v(\alpha')+v(\beta')\\
&=a_1+(v(\pi)-a_1+1)+(v(\rho)-1)\\
&=v(\pi)+v(\rho).
\end{align*}
Since $[\alpha]=\pi$ and $[\beta]=\rho$, the conclusion follows.
\end{proof}

\begin{proposition}\label{monodromy:th:product-reduction-large-v-odd}
Let $\pi,\rho\in\Partitions{d}$ be partitions of $d$. Assume that $v(\pi)+v(\rho)\ge d$, and let $t=v(\pi)+v(\rho)-d+1$. Fix an integer $0\le k\le t$ such that $k\equiv t\pmod{2}$. Then there exist permutations $\alpha,\beta\in\symgroup[d]$ with $[\alpha]=\pi$ and $[\beta]=\rho$ such that $v(\alpha\beta)=d-1-k$ and the action of $\angled{\alpha,\beta}$ on $\{1,\ldots,d\}$ is transitive.\todo{Maybe only $k=0$ needed (in this case, transitivity is trivial)?}
\end{proposition}
\begin{proof}
Write $\pi=[a_1,\ldots,a_r]$. Since $v(\rho)\le d-1$ and $v(\pi)+v(\rho)\ge d$, there exists a largest integer $0\le i\le r$ such that $(a_1-1)+\ldots(a_i-1)+v(\rho)\le d-1$. Define
\[
z=d-v(\rho)-(a_1-1)-\ldots-(a_i-1).
\]
Consider the partition $\pi'=[a_1,\ldots,a_i,z,1,\ldots,1]\in\Partitions{d}$. Since by construction $v(\pi')+v(\rho)=d-1$, thanks to \cref{monodromy:th:product-reduction-small-v} we can find permutations $\alpha',\beta\in\symgroup[d]$ with $[\alpha']=\pi'$ and $[\beta]=\rho$ such that $v(\alpha'\beta)=d-1$; in other words, $\alpha'\beta$ is a $d$\=/cycle. Consider now the partition $\pi''=[a_{i+1}-z+1,a_{i+2},\ldots,a_r]$, whose branching number\todo{Terminology?} is $v(\pi'')=t$. Let $n=\sum\pi''$; fix an element $u_1$ of the $z$\=/cycle of $\alpha'$, and let $u_2,\ldots,u_n$ be the fixed points of $\alpha'$ corresponding to the last ones of $\pi'$ (it is easy to see that there are exactly $n-1$ such ones). Since $k\le t=v(\rho'')$ and $k\equiv t\pmod{2}$, \cref{monodromy:th:product-of-two-cycles} gives permutations $\alpha'',\gamma\in\symgroup(\{u_1,\ldots,u_n\})$ such that $[\alpha'']=\rho''$, $\gamma$ is a $n$\=/cycle and $\alpha''\gamma$ is a $(n-k)$\=/cycle. Up to conjugation, we can assume that $\gamma=\cycle{u_1,\ldots,u_n}$. Moreover, it is not restrictive to assume that $u_1,\ldots,u_n$ appear in this order in the $d$\=/cycle $\alpha'\beta$. Therefore, setting $\alpha=\alpha''\alpha'$, we have that
\[
\alpha\beta=(\alpha''\gamma)\gamma^{-1}(\alpha'\beta);
\]
by \cref{monodromy:th:same-number-of-cycles}, this implies that $\alpha\beta$ has the same number of cycles as $\alpha''\gamma\in\symgroup(\{u_1,\ldots,u_n\})$, that is $v(\alpha\beta)=d-(k+1)$. Since $[\alpha]=\pi$ and $[\beta]=\rho$, the only thing left to show is that the action of $\angled{\alpha,\beta}$ on $\{1,\ldots,d\}$ is transitive. Write
\[
\alpha'\beta=\cycle{u_1,\ldots,w_1,u_2,\ldots,w_2,u_3,\ldots,w_{n-1},u_n,\ldots,w_n}
\]
as in the proof of \cref{monodromy:th:same-number-of-cycles}, where an explicit description of the orbits of $\alpha\beta=\alpha''\gamma\gamma^{-1}\alpha'\beta$ is given; from that description, it is clear that for each $1\le j\le n$ the elements $u_j,\ldots,w_j$ all belong to the same orbit. Moreover, since $\beta=(\alpha')^{-1}\cycle{u_1,\ldots,w_1,u_2,\ldots,w_n}$ and $\alpha'$ fixes $u_2,\ldots,u_n$, it follows that $w_j$ and $u_{j+1}$ belong to the same orbit for each $1\le j\le n-1$; this completes the proof.
\end{proof}

\begin{corollary}\label{monodromy:th:product-reduction-odd}
Let $\pi,\rho\in\Partitions{d}$ be partitions of $d$. Assume that $v(\pi)+v(\rho)\ge d-1$ and $v(\pi)+v(\rho)\equiv d-1\pmod{2}$. Then there exist permutations $\alpha,\beta\in\symgroup[d]$ with $[\alpha]=\pi$ and $[\beta]=\rho$ such that $\alpha\beta$ is a $d$\=/cycle.
\end{corollary}
\begin{proof}
The conclusion immediately follows from \cref{monodromy:th:product-reduction-small-v} if $v(\pi)+v(\rho)=d-1$, or from \cref{monodromy:th:product-reduction-large-v-odd} if $v(\pi)+v(\rho)\ge d$.
\end{proof}

\begin{remark}\label{monodromy:rm:product-reduction-odd-prescribed-cycles}
Write $\pi=[a_1,\ldots,a_r]$, $\rho=[b_1,\ldots,b_s]$; assume that $b_1\ge 2$. By directly examining the proof of \cref{monodromy:th:product-reduction-small-v}, we see that the proposed construction yields permutations $\alpha,\beta\in\symgroup[d]$ such that $1$ belongs to the $a_1$\=/the cycle of $\alpha$ and to the $b_1$\=/cycle of $\beta$. It is not hard to see\todo{Maybe explain why?}, once again by inspecting the proof, that the same can be said for \cref{monodromy:th:product-reduction-large-v-odd}. As a consequence, the statement of \cref{monodromy:th:product-reduction-odd} can be enhanced by adding the following line: \emph{$\alpha$ and $\beta$ can be chosen in such a way that $1$ belongs to the $a_1$\=/the cycle of $\alpha$ and to the $b_1$\=/cycle of $\beta$, provided that $b_1\ge 2$}. We will need this improvement for the upcoming proof.
\end{remark}

\begin{proposition}\label{monodromy:th:product-reduction-large-v-even}
Let $\pi,\rho\in\Partitions{d}$ be partitions of $d$. Assume that $v(\pi)+v(\rho)\ge d$ and $v(\pi)+v(\rho)\equiv d\pmod{2}$. Then there exist permutations $\alpha,\beta\in\symgroup[d]$ with $[\alpha]=\pi$ and $[\beta]=\rho$ such that $\angled{\alpha,\beta}$ acts transitively on $\{1,\ldots,d\}$ and
\[
[\alpha\beta]=\begin{dcases*}
[d/2,d/2]&if $\pi=\rho=[2,\ldots,2]$,\\
[1,d-1]&otherwise.
\end{dcases*}
\]
\end{proposition}
\begin{proof}
Assume first that $\pi=\rho=[2,\ldots,d]$. We can choose
\begin{align*}
\alpha=\cycle{2,3}\cycle{4,5}\cdots\cycle{d,1}&&\beta=\cycle{1,2}\cycle{3,4}\cdots\cycle{d-1,d}.
\end{align*}
The action of $\angled{\alpha,\beta}$ is obviously transitive, and
\[
\alpha\beta=\cycle{1,3,\ldots,d-1}\cycle{2,4,\ldots,2}.
\]
Otherwise, since $v(\pi)+v(\rho)\ge d$, at least one of $\pi$ and $\rho$ has an entry which is greater than $2$; without loss of generality, we can assume it is $\rho$. Write $\pi=[a_1,\ldots,a_r]$, $\rho=[b_1,\ldots,b_s]$ with $a_1\ge 2$ (since $v(\pi)\ge 1$) and $b_1\ge 3$. Fix
\[
\beta=\cycle{1,\ldots,d_1}\cycle{d_1+1,\ldots,d_2}\cdots\cycle{d_{s-1}+1,\ldots,d_s},
\]
where $b_1=d_1\ge 3$, $b_i=d_i-d_{i-1}$ for $2\le i\le s$ (in particular, $d_s=d$). Consider the partitions
\begin{align*}
\pi'=[a_1-1,\ldots,a_r],&&\rho'=[b_1-1,\ldots,b_s].
\end{align*}
Since $\sum\pi'=\sum\rho'=d-1$ and $v(\pi')+v(\rho')=v(\pi)+v(\rho)-2$, by \cref{monodromy:th:product-reduction-odd} we can find permutations $\alpha',\beta'\in\symgroup(\{2,\ldots,d\})$ with $[\alpha']=\pi'$ and $[\beta']=\rho'$ such that $\alpha'\beta'$ is a $(d-1)$\=/cycle. Up to conjugation, we can assume that
\[
\beta'=\cycle{2,\ldots,d_1}\cycle{d_1+1,\ldots,d_2}\cdots\cycle{d_{s-1}+1,\ldots,d_s}
\]
or, in other words, $\beta=\cycle{1,2}\beta'$; moreover, as explained in \cref{monodromy:rm:product-reduction-odd-prescribed-cycles}, we can choose $\alpha'$ in such a way that its $(a_1-1)$\=/cycle contains $2$. By setting $\alpha=\alpha'\cycle{1,2}$, we immediately get that $\alpha\beta=\alpha'\beta'$ is a $(d-1)$\=/cycle fixing $1$. Finally, the action of $\angled{\alpha,\beta}$ is transitive since $\alpha$ does not fix $1$.
\end{proof}

\begin{corollary}\label{monodromy:th:product-reduction-large-v}
Let $\pi,\rho\in\Partitions{d}$ be partitions of $d$. Assume that $v(\pi)+v(\rho)\ge d-1$. Then there exist permutations $\alpha,\beta\in\symgroup[d]$ with $[\alpha]=\pi$, $[\beta]=\rho$ such that $\angled{\alpha,\beta}$ acts transitively on $\{1,\ldots,d\}$ and
\[
[\alpha\beta]\in\{[d],[1,d-1],[d/2,d/2]\}.
\]
\end{corollary}
\begin{proof}
The conclusion follows immediately from \cref{monodromy:th:product-reduction-large-v-odd} or \cref{monodromy:th:product-reduction-large-v-even} depending on the parity of $v(\pi)+v(\rho)+d$.
\end{proof}

\section{Projective plane}

\begin{theorem}
Let $\DD=\datum{\tSigma,\RP[2]}{d}{\pi_1,\ldots,\pi_n}$ be a candidate datum. If $\tSigma$ is non-orientable, then $\DD$ is realizable.
\end{theorem}
\begin{proof}
First of all, since $\DD$ is a candidate datum, the \RH{} formula implies that
\[
v(\pi_1)+\ldots+v(\pi_n)=d\chi(\RP[2])-\chi(\tSigma)\ge d-1
\]
(recall that $\tSigma$ is non-orientable, so $\chi(\tSigma)\le 1$). Moreover, the total branching $v(\pi_1)+\ldots+v(\pi_n)$ is even. In order to apply \cref{monodromy:th:monodromy-realizability-non-orientable}, we will now inductively define representatives $\alpha_i\in\symgroup[d]$ with $[\alpha_i]=\pi_i$, satisfying the following invariant: for every $0\le i\le n$, either
\[
v(\alpha_1\cdots\alpha_i)=v(\pi_1)+\ldots+v(\pi_i)
\]
or
\[
\text{$[\alpha_1\cdots\alpha_i]\in\{[d],[1,d-1],[d/2,d/2]\}$ and $\angled{\alpha_1,\ldots,\alpha_i}$ acts transitively.}
\]
Assume we have already defined $\alpha_1,\ldots,\alpha_{i-1}$; we want to suitably choose $\alpha_i$. Let $\alpha=\alpha_1\cdots\alpha_{i-1}$; there are two cases.
\begin{itemize}
\item If $v(\alpha)+v(\pi_i)<d$, by \cref{monodromy:th:product-reduction-small-v} we can find $\alpha_i\in\symgroup[d]$ with $[\alpha_i]=\pi_i$ such that $v(\alpha\alpha_i)=v(\alpha)+v(\alpha_i)$. The invariant is still satisfied: if $v(\alpha)=v(\pi_1)+\ldots+v(\pi_{i-1})$ then obviously $v(\alpha\alpha_i)=v(\pi_1)+\ldots+v(\pi_i)$. If instead $[\alpha]\in\{[d],[1,d-1],[d/2,d/2]\}$, then either $\alpha_i$ is the identity, or $\alpha\alpha_i$ is a $d$\=/cycle; either way, $[\alpha\alpha_i]\in\{[d],[1,d-1],[d/2,d/2]\}$. Note that if the action of $\angled{\alpha_1,\ldots,\alpha_{i-1}}$ is transitive, then the action of $\angled{\alpha_1,\ldots,\alpha_i}$ is transitive as well.
\item If $v(\alpha)+v(\alpha_i)\ge d$, \cref{monodromy:th:product-reduction-large-v} gives a permutation $\alpha_i\in\symgroup[d]$ with $[\alpha_i]=\pi_i$ such that $[\alpha\alpha_i]\in\{[d],[1,d-1],[d/2,d/2]\}$ and the action of $\angled{\alpha,\alpha_i}$ is transitive. The invariant is obviously satisfied.
\end{itemize}
By induction, we can find $\alpha_1,\ldots,\alpha_n\in\symgroup[d]$ with $[\alpha_i]=\pi_i$ such that
\[
[\alpha_1\cdots\alpha_n]\in\{[d],[1,d-1],[d/2,d/2]\}
\]
and $\angled{\alpha_1,\ldots,\alpha_n}$ acts transitively on $\{1,\ldots,d\}$ (note that $v(\alpha_1\cdots\alpha_n)=v(\alpha_1)+\ldots+v(\alpha_n)$ also implies that $[\alpha_1\cdots\alpha_n]=[d]$). Note that\todo{Show that $v(\alpha\beta)\equiv v(\alpha)+v(\beta)\pmod{2}$.}
\[
v(\alpha_1\cdots\alpha_n)\equiv v(\alpha_1)+\ldots+v(\alpha_n)\equiv 0\pmod{2}.
\]
We now prove that $\alpha=\alpha_1\cdots\alpha_n$ is a square.
\begin{itemize}
\item If $[\alpha]=[d]$, then $d$ is odd, so $\alpha$ is the square of $\alpha^{(d+1)/2}$.
\item If $[\alpha]=[1,d-2]$, then $d$ is even, so $\alpha$ is the square of $\alpha^{d/2}$.
\item If $[\alpha]=[d/2,d/2]$, then $d$ is even, and it is easy to see that $\alpha$ is the square of a $d$-cycle.
\end{itemize}
By \cref{monodromy:th:monodromy-realizability-non-orientable}, this implies that there exists a realizable candidate datum $\DD'=\datum{\tSigma',\RP[2]}{d}{\pi_1,\ldots,\pi_n}$. To see that $\tSigma'$ is non-orientable (and, therefore, equal to $\tSigma$, as shown in \cref{monodromy:rm:sigma-tilde-unique-non-orientable}), simply note that ??.
\end{proof}


\section{Prime-degree conjecture}
\begin{proposition}
Let $d$ be a positive integer. Assume that every candidate datum $\datum{\surf{g},\sphere{}}{d}{\pi_1,\pi_2,\pi_3}$ is realizable. Then every candidate datum $\datum{\surf{g},\sphere{}}{d}{\pi_1,\ldots,\pi_n}$ with $n\ge 3$ is realizable.
\end{proposition}
\begin{proof}
We proceed by induction on $n\ge 4$. Assume that every candidate datum with at most $n-1$ partitions is realizable. Fix a candidate datum $\DD=\datum{\surf{g},\sphere{}}{d}{\pi_1,\ldots,\pi_{n}}$; there are two cases.
\begin{itemize}
\item Assume that there are two indices $1\le i<j\le n$ such that $v(\pi_i)+v(\pi_j)\le d-1$; without loss of generality, consider $i=1$ and $j=2$. By \cref{monodromy:th:product-reduction-small-v}, we can find permutations $\alpha_1,\alpha_2\in\symgroup[d]$ with $[\alpha_1]=\pi_1$ and $[\alpha_2]=\pi_2$ such that $v(\alpha_1\alpha_2)=v(\alpha_1)+v(\alpha_2)$. Consider the candidate datum $\DD'=\datum{\surf{g},\sphere{}}{d}{[\alpha_1\alpha_2],\pi_3,\ldots,\pi_n}$. By induction, $\DD'$ is realizable; \cref{monodromy:th:monodromy-realizability-orientable} then gives permutations $\alpha_3,\ldots,\alpha_n\in\symgroup[d]$ with $[\alpha_i]=\pi_i$ such that $(\alpha_1\alpha_2)\alpha_3\cdots\alpha_n=1$ and the action of $\angled{\alpha_1\alpha_2,\alpha_2,\ldots,\alpha_n}$ on $\{1,\ldots,d\}$ is transitive. It is trivial to see that the permutations $\alpha_1,\alpha_2,\alpha_3,\ldots,\alpha_n$ imply the realizability of $\DD$, again by \cref{monodromy:th:monodromy-realizability-orientable}.
\item Otherwise, $v(\pi_i)+v(\pi_j)\ge d$ for every $1\le i<j\le n$. By \cref{monodromy:th:product-reduction-large-v}, we can find permutations $\alpha_1,\alpha_2$ with $[\alpha_1]=\pi_1$ and $[\alpha_2]=\pi_2$ such that $v(\alpha_1\alpha_2)\ge d-2$. Let\todo{Here it is crucial that $n\ge 4$.}
\[
g'=\frac{1}{2}(v(\alpha_1\alpha_2)+v(\alpha_3)+\ldots+v(\alpha_n))-d+1\ge\frac{1}{2}(d-2+d)-d+1\ge 0.
\]
It is easy to see that $\DD'=\datum{\surf{g'},\sphere{}}{d}{[\alpha_1\alpha_2],\pi_3,\ldots,\pi_n}$ is a candidate datum, so it is realizable by induction. Similarly to the case above, this implies that $\DD$ is realizable.
\end{itemize}
\end{proof}
\chapter{\texorpdfstring{\Dessins{}}{Dessins d'enfant}}

\section{Child's drawings on surfaces}

In \cref{monodromy:sc:combinatorial-moves}, we discussed how the Hurwitz existence problem can be reduced to the analysis of candidate data on the sphere. Moreover, thanks to \cref{combinatorial-move:a:small-v,combinatorial-move:a:large-v}, we have devised a relatively reliable technique to decrease the number $n$ of partitions; this technique was successfully employed in \cref{monodromy:sc:results-sphere} to show the realizability of a wide variety of candidate data by induction on $n$, starting from the base case $n=3$. Ignoring the cases where $n\le 2$, which were fully analyzed in \cref{monodromy:sc:combinatorial-moves}, it should come as no surprise that candidate data with $n=3$ play a very important role in the study of the existence problem.

Up to this point, we have only approached the Hurwitz existence problem from a group-theoretic point of view, showing realizability by looking for elements of $\symgroup[d]$ with certain properties. In this section, we will present a totally different tool, of a more topological and combinatorial nature, for attacking the same problem. The concept of \emph{\dessins{}}\footnote{``\emph{\Dessin{}}'' is French for ``child's drawing'', hence the title of this section.} was popularized by Grothendieck in \cite{grothendieck}, in a setting related to, but different from, the Hurwitz existence problem. \Dessins{} provide a strikingly elementary tool for showing the realizability of candidate data with $n=3$ partitions, although they generalize quite nicely to the case $n\ge 4$. However, we will not deal with said generalization, since the reduction technique will prove to be sufficient for our purposes; we refer the interested reader to \resultcite{section}{3}{pervova-methods}.

We start by introducing some basic terminology about graphs. Given a surface $\Sigma$, a \emph{graph} embedded in $\Sigma$ (or, simply, a graph on $\Sigma$) is a closed subspace $\Gamma\subs\Sigma$ consisting of:
\begin{itemize}
\item a finite number of points $x_1,\ldots,x_r\in\Sigma$, called \emph{vertices};
\item a finite number of segments (subspaces homeomorphic to $[0,1]$) $e_1,\ldots,e_d\subs\Sigma$, called \emph{edges}; we require that each edge connects two (not necessarily distinct) vertices, and that the interiors of two edges are disjoint; in other words, two edges may intersect at most at their endpoints; moreover, a vertex cannot lie on the interior of an edge.
\end{itemize}
The \emph{degree} of a vertex $x$ is the number of edges having $x$ as an endpoint; edges connecting $x$ to itself are counted twice; we denote the degree of $x$ by $k(x)$. In order to avoid unpleasant corner cases, we will always require that there are no \emph{isolated vertices} or, in other words, that $k(x)\ge 1$ for every vertex $x$.

A \emph{bipartite graph} is a graph whose vertices are colored either black or white, and each edge connects a black vertex and a white one. If we denote the black vertices by $x_1,\ldots,x_r$ and the white vertices by $y_1,\ldots,y_s$, an easy counting argument shows that
\[
k(x_1)+\ldots+k(x_r)=k(y_1)+\ldots+k(y_s)=d,
\]
where $d$ is the number of edges.

Given a graph $\Gamma$ on a surface $\Sigma$, the space $\Sigma\setminus\Gamma$ is a disjoint union of a finite number of non-compact surfaces $S_1\sqcup\ldots\sqcup S_h$, called \emph{complementary regions} of $\Gamma$. We are finally ready to give the definition of the much anticipated \dessins{}.

\begin{definition}
Let $\Sigma$ be a surface. A \emph{\dessin{}} on $\Sigma$ is a bipartite graph $\Gamma\subs\Sigma$ whose complementary regions are topological disks.
\end{definition}

\todo{Examples of \dessins{}.}

Let $\Gamma$ be a \dessin{}, and fix one a complementary region $D$. By traveling along its boundary, always keeping $D$ to the left, we get a cyclic sequence of edges of $\Gamma$, which we call \emph{combinatorial boundary} of $D$, and denote by $\partial D$. Note that the same edge $e$ can be traveled along twice, once for each direction; in this case, it will appear twice in $\partial D$, and we will say that $e$ is \emph{enveloped} by $D$. The number of edges (with multiplicity) of $\partial D$ is the \emph{perimeter} of $D$, denoted by $\card{\partial D}$.\todo{The definition is not very formal, examples incoming.} If $D_1,\ldots,D_h$ are the complementary region of $\Gamma$, a counting argument shows that
\[
\card{\partial D_1}+\ldots+\card{\partial D_h}=2d.
\]

It is also easy to see that the perimeter of each complementary region is even: in fact, when traveling along the boundary of a region, we alternately encounter black and white vertices, so an even number of edges is required to get back to the starting color.

Finally, note that every \dessin{} is necessarily connected, otherwise there would be some complementary region with two or more boundary components.

\begin{definition}
Let $\Gamma$ be a \dessin{} on a surface $\Sigma$; let $d$ be the number of edges. Let $x_1,\ldots,x_r$ be the black vertices, $y_1,\ldots,y_s$ the white ones. Denote by $D_1,\ldots,D_h$ the complementary regions of $\Gamma$. The \emph{branching datum} of $\Gamma$ is the tuple
\[
\DD(\Gamma)=\datum{\Sigma,\sphere{}}{d}{[k(x_1),\ldots,k(x_r)],[k(y_1),\ldots,k(y_s)],[\card{\partial D_1}/2,\ldots,\card{\partial D_h}/2]}.
\]
\end{definition}

From the discussion above, we immediately see that $\DD(\Gamma)$ is a combinatorial datum, since
\[
k(x_1)+\ldots+k(x_r)=k(y_1)+\ldots+k(y_s)=\card{\partial D_1}/2+\ldots+\card{\partial D_h}/2=d.
\]
Actually, if $\Sigma$ is orientable, $\DD(\Gamma)$ is a candidate datum: by the Euler formula,
\[
\chi(\Sigma)=r+s-d+h=2d-v(\pi_1)-v(\pi_2)-v(\pi_3),
\]
where $\pi_1=[k(x_1),\ldots,k(x_r)]$, $\pi_2=[k(y_1),\ldots,k(y_s)]$ and $\pi_3=[\card{\partial D_1}/2,\ldots,\card{\partial D_h}/2]$. This is no coincidence, just like the name ``branching datum of $\Gamma$'' was not picked at random: the following result establishes a strong connection between \dessins{} and realizable combinatorial data.

\begin{proposition}\label{dessins:th:dessins-realizability}
Let $\DD=\datum{\surf{g}}{d}{\pi_1,\pi_2,\pi_3}$ be a combinatorial datum. Then $\DD$ is realizable if and only if there exists a \dessin{} $\Gamma\subs\Sigma_g$ with $\DD(\Gamma)=\DD$.
\end{proposition}
\begin{proof}
Assume that $\DD$ is realized by a branched covering $\map{f}{\surf{g}}{\sphere{}}$. Let $\{\wtilde{x}_1,\ldots,\wtilde{x}_r\}=f^{-1}(x)$, $\{\wtilde{y}_1,\ldots,\wtilde{y}_s\}=f^{-1}(y)$, $\{\wtilde{z}_1,\ldots,\wtilde{z}_h\}=f^{-1}(z)$. Fix a segment $e\subs\sphere$ connecting $x$ and $y$ (and avoiding $z$); we claim that $\Gamma=f^{-1}(e)$ is the desired \dessin{}. Let $\interior{e}$ be the interior of $e$ (that is, $\interior{e}=e\setminus\{x,y\}$). First of all, note that $f^{-1}(\interior{e})$ is the disjoint union of $d$ open segments $\interior{e}_1,\ldots,\interior{e}_d$, since the restriction of $f$ to $\sphere{}\setminus\{x,y,z\}$ is a covering map of degree $d$. Moreover, it is easy to see that the closure of each $\interior{e}_i$ is a closed segment $e_i$ connecting one point in $f^{-1}(x)$ and one point of $f^{-1}(y)$; it follows that $\Gamma$ is a bipartite graph on $\surf{g}$, with black vertices $\wtilde{x}_1,\ldots,\wtilde{x}_r$ and white vertices $\wtilde{y}_1,\ldots,\wtilde{y}_s$. Consider a vertex $\wtilde{x}_i$; recall that $f$ is locally modeled on the complex map $\xi\mapsto\xi^k$, where $k=k(\wtilde{x}_i)$ is the local degree of $\wtilde{x}_i$. As a consequence, we immediately see that there are exactly $k(\wtilde{x}_i)$ edges of $\Gamma$ with $\wtilde{x}_i$ as an endpoint; of course, the same holds for every $\wtilde{y}_j$. Finally, we turn to the complementary regions of $\Gamma$. Let $D=\sphere{}\setminus e\iso\RR^2$, $\holed{D}=D\setminus\{z\}\iso\RR^2\setminus\{0\}$. The restriction of $f$ to $f^{-1}(\holed{D})=\surf{g}\setminus(\Gamma\cup\{\wtilde{z}_1,\ldots,\wtilde{z}_h\})$ is a covering map of the punctured disk $\holed{D}$. It is then easy to see that the complementary regions of $\Gamma$ are discs $\wtilde{D}_1,\ldots,\wtilde{D}_h$, with $\wtilde{z}_i\in\wtilde{D}_i$ for each $1\le i\le h$, and that the restriction $\map{f}{\wtilde{D}_i}{D}$ is modeled the complex map $\xi\mapsto\xi^{k(\wtilde{z}_i)}$. Since the perimeter of $D$ is $2$, we have that $\card{\partial\wtilde{D}_i}=2 k(\wtilde{z}_i)$; this concludes the proof of the equality $\DD(f)=\DD(\Gamma)$.

Conversely, assume that we are given a \dessin{} $\Gamma\subs\surf{g}$ with $\DD(\Gamma)=\DD$. Fix three arbitrary points $x,y,z\in\sphere{}$, and let $e\subs\sphere{}$ be a segment connecting $x$ and $y$ (and avoiding $z$). First of all, we define $f$ on $\Gamma$, sending black vertices to $x$ and white vertices to $y$, and mapping edges homeomorphically to $e$. Extending $f$ to all of $\surf{g}$ is a relatively easy task: here are the details. Consider the standard closed disk $K=\{a\in\RR^2:\lVert a\rVert\le 1\}$, and take a complementary region $\wtilde{D}\subs\surf{g}$; let $\map{\phi}{K}{\surf{g}}$ be a continuous map which restricts to a homeomorphism $\map{\phi}{\interior{K}}{\wtilde{D}}$, where $\interior{K}$ denotes the interior of $K$. There exists a map\todo{Picture much needed.} $\map{\psi}{K}{\sphere{}}$ such that $\psi(0)=z$, the diagram
\begin{diagram}
\partial K\dar{\phi}\ar[dr,"\psi"]\\
\Gamma\rar{f}&e
\end{diagram}
commutes, $\psi$ is a local homeomorphism in $\interior{K}\setminus\{0\}$ and it is modeled on $\xi\mapsto\xi^k$ in a neighborhood of $0\in K$; in particular $k$ will necessarily be equal to half the perimeter of $\wtilde{D}$. We can now extend $f$ to $\wtilde{D}$ by setting $f(\wtilde{x})=\psi(\phi^{-1}(\wtilde{x}))$ for every $\wtilde{x}\in\wtilde{D}$. After repeating the process for all the complementary regions, it is not hard to verify that the map $\map{f}{\tSigma}{\sphere{}}$ we have obtained is a branched covering with branching points $x,y,z\in\sphere{}$. Since $\Gamma=f^{-1}(e)$, the first\todo{Namely, $\Rightarrow$.} part of the proof implies that $\DD(\Gamma)=\DD(f)$.
\end{proof}

\section{Unwinding, joining and fattening}

In the next section we will introduce a new kind of combinatorial moves, which operate on \dessins{} rather than permutations. In this context, the importance of visual intuition cannot be overstated. Therefore, we will now spend some time describing in detail two operations that will play a major role in the topological explanation of the upcoming combinatorial moves.

\paragraph{Unwinding the boundary.} Let $\Gamma$ be a graph on a surface $\Sigma$. Take a complementary region $D$, and assume $D$ is a topological disk. Intuitively, when we \emph{unwind the boundary} of $D$, we represent $D$ as the standard closed disk $K$ embedded in $\RR^2$; the edges of the combinatorial boundary of $D$ are placed sequentially on the topological boundary of $K$, possibly with repetitions. For a more formal description, we can follow the strategy presented in the second part of the proof of \cref{dessins:th:dessins-realizability}: we consider a continuous map $\map{\phi}{K}{\surf{g}}$ which restricts to a homeomorphism $\map{\phi}{\interior{K}}{D}$; edges on the topological boundary of $D$ can be pulled back by $\phi$, thus unwinding the combinatorial boundary of $D$ on $\partial K$.

\paragraph{Joining vertices along edges.} Let $\Gamma$ be a graph on a surface $\Sigma$. Consider an edge $e$, and let $x$, $y$ be its (distinct) endpoints. \emph{Joining} $x$ and $y$ along $e$ means shrinking $e$ to a single point, so that $x$ and $y$ are merged into a single vertex, say $z$; it is immediate to check that $k(z)=k(x)+k(y)-2$, while the degrees of the other vertices are left unchanged. The topology of the complementary regions does not change either. To be more precise, there is a natural one-to-one correspondence between regions of $\Sigma\setminus\Gamma$ and regions of $\Sigma\setminus\Gamma'$, where $\Gamma'$ is the graph obtained after joining $x$ and $y$ along $e$, and corresponding regions are homeomorphic. The edge $e$ disappears from the combinatorial boundaries, so the perimeters of the two regions touching $e$ decrease by $1$ (if the two regions were actually the same, then the perimeter decreases by $2$); the other perimeters do not change. Of course, the joining operation can be performed along more edges simultaneously, by joining vertices along one edge at a time.

\paragraph{Fattening graphs.} Representing \dessins{} on the sphere is easy, since graphs on $\sphere{}$ naturally embed in $\RR^2$; unfortunately, this is not the case for surfaces of genus $g\ge 1$. However, for a specific class of graphs (including \dessins{}), there is a trick we can exploit in order to represent them as diagrams on the plane. Let $\Gamma$ be a graph on a surface $\surf{g}$; assume that the complementary regions of $\Gamma$ are disks. Note that the topology of the embedding $\Gamma\subs\surf{g}$ can be completely recovered if we are given $\Gamma$ as an abstract graph, plus a tubular neighborhood of $\Gamma$ in $\surf{g}$; we call such a datum a \emph{fat graph}. A fat graph can be represented as a diagram with a finite number of transverse crossings on the plane; each crossing is equipped with the additional information of which edge goes over and which goes under.

In order to reconstruct the embedding of $\Gamma$ in $\surf{g}$, we simply have to thicken the edges of the diagram, keeping in mind that the two edges involved in a crossing actually go one under the other. This operation yields a fat graph, from which $\surf{g}$ can be recovered by gluing a disk along each boundary component.

We will employ this technique in order to represent \dessins{} (and more generally, graphs whose complementary regions are disks) embedded in higher genus surfaces.

\section{Genus-reducing combinatorial moves}\label{dessins:sc:combinatorial-moves}

As we have already anticipated, the goal of this thesis is a complete classification of the exceptional data with a partition of length $2$. \Cref{combinatorial-move:a:small-v,combinatorial-move:a:large-v} are often able to reduce the existence problem to instances with $n=3$ partitions. We will now introduce a few more combinatorial moves, which heavily exploit the machinery of \dessins{}. Unlike the aforementioned ones, these moves only work under very restrictive assumptions, namely that $n=3$ and $\len{\pi_3}=2$; on the other hand, they allow a much finer control on the partitions involved, and are often versatile enough to reduce an instance of the existence problem to the case where $\tSigma=\sphere{}$.

In this section, we will only be dealing with candidate data of the form $\DD=\datum{\surf{g}}{d}{\pi_1,\pi_2,[s,d-s]}$ with $1\le s\le d-1$. In this setting, the \RH{} formula can simply be written as
\[
\len{\pi_1}+\len{\pi_2}=d-2g.
\]
We will adopt the following conventions:
\begin{itemize}
\item vertices corresponding to the entries of $\pi_1$ (or $\pi_1'$) will be colored black;
\item vertices corresponding to the entries of $\pi_2$ (or $\pi_2'$) will be colored white;
\item unnamed vertices will be labeled with their degrees;
\item the complementary disk associated to the first entry of $\pi_3$ (or $\pi_3'$) will be denoted by $D_1$ and will be colored orange;
\item the complementary disk associated to the second entry of $\pi_3$ (or $\pi_3'$) will be denoted by $D_2$ and will be colored blue.
\end{itemize}

\begin{combinatorialmoveb}\label{combinatorial-move:b:[1 1 3]}
Let $\DD=\datum{\surf{g}}{d}{\pi_1,\pi_2,[s,d-s]}$ be a candidate datum with $g\ge 1$. Assume that $[1,1,3]\subs\pi_1$. Consider the candidate datum
\[
\DD'=\datum{\surf{g-1}}{d}{\pi_1',\pi_2,[s,d-s]},
\]
where $\pi_1'=\pi_1\setminus[3]\cup[1,1,1]$. Then $\DD\cmove\DD'$.
\end{combinatorialmoveb}
\begin{proof}
Assume that $\DD'$ is realizable; by \cref{dessins:th:dessins-realizability}, there exists a \dessin{} $\Gamma'\subs\surf{g-1}$ with $\DD(\Gamma')=\DD'$. Our aim will be to construct a new \dessin{} $\Gamma\subs{g}$ with $\DD(\Gamma)=\DD$; by \cref{dessins:th:dessins-realizability}, this will imply that $\DD$ is realizable as well.

Note that $[1,1,1,1,1]\subs\pi_1'$; therefore, without loss of generality, we can assume that $\Gamma$ has three black vertices of degree $1$ lying on the boundary of the complementary region $D_1$. Let us represent $D_1$ with its boundary unwound, and focus on the three black vertices of degree $1$. We perform the following operations on $\Gamma'$.
\begin{enumerate}[(1)]
\item Attach a tube to $\surf{g-1}$ with both endpoints in $D_1$; to be more precise, remove two disjoint open disks contained in the interior of $D_1$, and glue a tube $S^1\times[0,1]$ along the two new boundary components. This effectively increases the genus by $1$.
\item Connect the three black vertices with two new edges, as shown in red in the picture; note that the orange complementary region of the new graph is still a disk.
\item Join the three black vertices along the red edges.
\end{enumerate}
After these operations, we get a new \dessin{} $\Gamma$ embedded in $\surf{g}$. It is easy to check that $\DD(\Gamma)=\DD$, therefore $\DD$ is realizable.
\end{proof}

In the upcoming proofs, we will often represent complementary disks with their boundaries unwound, without explicitly saying so. The pictures should be clear enough to avoid any ambiguity.

\begin{combinatorialmoveb}\label{combinatorial-move:b:4 2}
Let $\DD=\datum{\surf{g}}{d}{\pi_1,\pi_2,[s,d-s]}$ be a candidate datum with $g\ge 1$. Assume that:
\begin{assumptions}
\item $2\le s\le d-s$;
\item $x\in\pi_1$ for some $x\ge 4$;
\item $2\in\pi_2$.
\end{assumptions}
Let $x_1$, $x_2$ be positive integers whose sum equals $x-2$, and consider the candidate datum
\[
\DD'=\datum{\surf{g-1}}{d-2}{\pi_1',\pi_2',[s-1,d-s-1]},
\]
where $\pi_1'=\pi_1\setminus[x]\cup[x_1,x_2]$ and $\pi_2'=\pi_2\setminus[2]$. Then $\DD\cmove\DD'$.
\end{combinatorialmoveb}
\begin{proof}
Consider a \dessin{} $\Gamma'\subs\surf{g-1}$ realizing $\DD'$. There are two cases.
\paragraph{Case 1:} the black vertex of degree $x_1$ lies on $\partial D_1$ and the one with degree $x_2$ lies on $\partial D_2$ (or vice versa). Then we perform the following operations on $\Gamma'$.
\begin{enumerate}[(1)]
\item Attach a tube to $\surf{g-1}$ with one endpoint in $D_1$ and the other one in $D_2$.
\item Add one black vertex, one white vertex and two edges as shown in the picture.
\item Draw the two red edges shown in the picture.
\item Perform the joining operation along the red edges.
\end{enumerate}

\paragraph{Case 2:} the two black vertices of degrees $x_1$ and $x_2$ lie (say) on $\partial D_1$. Fix an edge $e\subs\Gamma'$ which lies on the boundaries of both disks. We perform the following operations on $\Gamma'$.
\begin{enumerate}[(1)]
\item Add one black vertex and one white vertex on $e$.
\item Attach a tube to $\surf{g-1}$ with both endpoints in $D_1$.
\item Draw the two red edges shown in the picture.
\item Perform the joining operation along the red edges.
\end{enumerate}

In both cases, we get a new \dessin{} $\Gamma$ embedded in $\surf{g}$. It is easy to check that $\Gamma$ realizes the candidate datum $\DD$.
\end{proof}

\begin{combinatorialmoveb}\label{combinatorial-move:b:[3 3] [2 2]}
Let $\DD=\datum{\surf{g}}{d}{\pi_1,\pi_2,[s,d-s]}$ be a candidate datum with $g\ge 1$. Assume that:
\begin{assumptions}
\item $3\le s\le d-3$;
\item $[x,y]\subs\pi_1$ for some $x\ge 3$, $y\ge 3$;
\item $[2,2]\subs\pi_2$.
\end{assumptions}
Consider the candidate datum
\[
\DD'=\datum{\surf{g-1}}{d-4}{\pi_1',\pi_2',[s-2,d-s-2]},
\]
where $\pi_1'=\pi_1\setminus[x,y]\cup[x-2,y-2]$ and $\pi_2'=\pi_2\setminus[2,2]$. Then $\DD\cmove\DD'$.
\end{combinatorialmoveb}
\begin{proof}
Consider a \dessin{} $\Gamma'\subs\surf{g-1}$ realizing $\DD'$. There are two cases.
\paragraph{Case 1:} the black vertex of degree $x-2$ lies on $\partial D_1$ and the one with degree $y-2$ lies on $\partial D_2$ (or vice versa). Then we perform the following operations on $\Gamma'$.
\begin{enumerate}[(1)]
\item Attach a tube to $\surf{g-1}$ with one endpoint in $D_1$ and the other one in $D_2$.
\item Add two black vertices, two white vertices and four edges as shown in the picture.
\item Draw the red edge as shown in the picture.
\item Perform the join operation along the red edges.
\end{enumerate}

\paragraph{Case 2:} the two black vertices of degrees $x-2$ and $y-2$ lie (say) on $\partial D_1$. We perform the following operations on $\Gamma'$.
\begin{enumerate}[(1)]
\item Add two black vertices and two white vertices on $e$.
\item Attach a tube to $\surf{g-1}$ with both endpoints in $D_1$.
\item Draw the two red edges shown in the picture.
\item Perform the joining operation along the red edges.
\end{enumerate}

In both cases, we get a new \dessin{} $\Gamma$ embedded in $\surf{g}$. It is easy to check that $\Gamma$ realizes the candidate datum $\DD$.
\end{proof}

\begin{combinatorialmoveb}\label{combinatorial-move:b:4 3}
Let $\DD=\datum{\surf{g}}{d}{\pi_1,\pi_2,[s,d-s]}$ be a candidate datum with $g\ge 1$. Assume that:
\begin{assumptions}
\item $2\le s\le d-2$;
\item $x\in\pi_1$ for some $x\ge 4$;
\item $y\in\pi_2$ for some $y\ge 3$.
\end{assumptions}
Consider the candidate datum
\[
\DD'=\datum{\surf{g-1}}{d-2}{\pi_1',\pi_2',[s-1,d-s-1]},
\]
where $\pi_1'=\pi_1\setminus[x]\cup[x-2]$ and $\pi_2'=\pi_2\setminus[y]\cup[y-2]$. Then $\DD\cmove\DD'$.
\end{combinatorialmoveb}
\begin{proof}
Consider a \dessin{} $\Gamma'\subs\surf{g-1}$ realizing $\DD'$. Let $u$ be the black vertex of degree $x-2$, and let $v$ be the white vertex of degree $y-2$; there are two cases.
\paragraph{Case 1:} $u$ lies on $\partial D_1$ and $v$ lies on $\partial D_2$ (or vice versa). Then we perform the following operations on $\Gamma'$.
\begin{enumerate}[(1)]
\item Attach a tube to $\surf{g-1}$ with one endpoint in $D_1$ and the other in $D_2$.
\item Add one black vertex, one white vertex and two edges as shown in the picture.
\item Draw the two red edges shown in the picture.
\item Perform the joining operation along the red edges.
\end{enumerate}

\paragraph{Case 2:} neither $u$ nor $v$ lie (say) on $\partial D_2$; analyzing this case will be more involved than usual. We will say that an edge is \emph{shared} if it is not enveloped by $D_1$ or by $D_2$; in other words, an edge is shared if it appears exactly once in $\partial D_1$; in the following pictures, shared edges will be colored green. Our goal will be to prove that we can add two vertices of degree $2$ -- one black and one white -- on a shared edge, in such a way that the vertices $\{u,2,2,v\}$ appear in this order on $\partial D_1$. Let us unwind the boundary of $D_1$; since $u$ does not lie on the boundary of $D_2$, it appears exactly $x-2\ge 2$ times on $\partial D_1$; similarly, $v$ appears exactly $y-2$ times.
\begin{itemize}
\item Assume that $\{u,v,u,\text{shared edge}\}$ appear in this order on $\partial D_1$. Then, by adding a black and a white vertex on this shared edge, we get the desired result.
\item Otherwise, consider the pictures below.
\begin{enumerate}[(1)]
\item We are in the following situation: there is a contiguous segment $A$ of $\partial D_1$ that contains all the occurrences of $u$, and does not contain any occurrence of $v$ or any shared edge. Similarly, there is a segment $B$ of $\partial D_1$ that contains all the occurrences of $v$, no occurrence of $u$ and no shared edge. Note that $u$ and $v$ are never adjacent to a shared edge, since by assumption they do not lie on $\partial D_2$.
\item Choose an orientation of $\partial D_1$ (counterclockwise in the picture) and consider the first occurrence of $u$ in $A$; let $\alpha$ be the edge immediately afterwards in $\partial D_1$. Since $\alpha$ is not shared, it must appear once more on the combinatorial boundary of $D_1$, with the opposite orientation. Note that $\alpha$ it cannot occur immediately before the first appearance of $u$, otherwise $u$ would have degree $1$, therefore it will occur somewhere else on $A$
\item Consider the first occurrence of $v$ in $B$; let $\beta$ be the edge immediately before in $\partial D_1$. Since $\beta$ is not shared, it will also occur somewhere else on the combinatorial perimeter of $D_1$
\item Let $a$ be the other endpoint of $\alpha$, and let $b$ be the other endpoint of $\beta$. Erase the edges $\alpha$ and $\beta$ and draw two new ones, connecting $u$ to $v$ and $a$ to $b$ as shown in the picture.
\item It is now easy to see that the orange region is still a complementary disk, and that its perimeter has not changed. Moreover, by traveling along its boundary, we encounter $\{u,v,u\}$ in this order, without any shared edges in between; since there must be at least one shared edge on the boundary of the new orange region, the argument of the first bullet point applies.
\end{enumerate}

Once we have added the two vertices of degree $2$ as explained above, we can perform the following operations.
\begin{enumerate}[(1)]
\item Attach a tube to $\surf{g-1}$ with both endpoints in $D_1$.
\item Draw the two red edges shown in the picture.
\item Perform the joining operation along the red edges.
\end{enumerate}
\end{itemize}

In both cases, we get a new \dessin{} $\Gamma$ embedded in $\surf{g}$. It is easy to check that $\Gamma$ realizes the candidate datum $\DD$.
\end{proof}

We will make extensive use of these combinatorial moves in the next chapter, where a full classification of the exceptional data with $n=3$ and $\len{\pi_3}=2$ will be provided (see \cref{short-partition:th:realizability-on-sphere-n-3,short-partition:th:realizability-on-torus-n-3,short-partition:th:realizability-on-higher-genus-n-3}).

\section{Realizability by \texorpdfstring{\dessins{}}{dessins d'enfant}}

We conclude this chapter by proving the realizability of a few families of candidate data by means of \dessins{}; these results, while interesting by themselves, will be useful in the next chapter for addressing some cases which are not covered by the combinatorial moves we have introduced.

\begin{proposition}\label{dessins:th:special-case-[2 d-2]}
Let $\DD=\datum{\surf{g}}{d}{\pi_1,\pi_2,[2,d-2]}$ be a candidate datum. Assume that:
\begin{assumptions}
\item $[x,y]\subs\pi_1$ for some $x\ge 2$, $y\ge 3$;
\item $[2,2]\subs\pi_2$.
\end{assumptions}
Then $\DD$ is realizable.
\end{proposition}
\begin{proof}
We will show that, under the stated assumptions, there is a combinatorial move\footnote{To be precise, when $d=5$ the tuple $\DD'$ is not a combinatorial datum according to our definition. However, the only candidate datum $\DD$ of degree $5$ satisfying the assumptions is $\DD=\datum{\sphere{}}{5}{[2,3],[1,2,2],[2,3]}$, whose realizability can be easily checked by hand with a suitable \dessin{}.}
\[
\DD\cmove\DD'=\datum{\surf{g}}{d-4}{\pi_1\setminus[x,y]\cup[x+y-4],\pi_2\setminus[2,2],[d-4]}.
\]
Note that $\DD'$ is realizable by \cref{monodromy:th:sphere-[d]}; let $\Gamma'\subs\surf{g}$ be a \dessin{} realizing it. We perform the following operations on $\Gamma'$.
\begin{enumerate}[(1)]
\item Consider the black vertex of degree $x+y-4$ and split it into two vertices of degrees $x-2$ and $y-2$.
\item Add two white vertices and four edges as shown in the picture. This creates a new complementary disk with perimeter $4$.
\end{enumerate}
After these operations, we get a new \dessin{} $\Gamma$ embedded in $\surf{g}$. It is easy to check that $\Gamma$ realizes the candidate datum $\DD$.
\end{proof}


\begin{proposition}\label{dessins:th:special-families}
The following families of candidate data are realizable for every $g\ge 2$.
\begin{enumerate}[(1)]
\item $\datum{\surf{g}}{6g}{[3,\ldots,3],[3,\ldots,3],[s,6g-s]}$.
\item $\datum{\surf{g}}{6g+2}{[2,3,\ldots,3],[2,3\ldots,3],[s,6g+2-s]}$.
\item $\datum{\surf{g}}{6g+3}{[3,\ldots,3],[1,2,3,\ldots,3],[s,6g+3-s]}$.
\item $\datum{\surf{g}}{6g+4}{[1,3,\ldots,3],[1,3,\ldots,3],[s,6g+4-s]}$.
\item $\datum{\surf{g}}{6g+6}{[1,2,3,\ldots,3],[1,2,3,\ldots,3],[s,6g+6-s]}$.
\end{enumerate}
\end{proposition}
\begin{proof}
\def\env#1{\underline{#1}}
In the scope of this proof we define an \emph{augmented combinatorial datum} as a combinatorial datum $\DD=\datum{\surf{g}}{d}{\pi_1,\pi_2,\pi_3}$ where some distinguished elements of $\pi_3$ are called \emph{enveloping}; we will underline the enveloping elements in order to recognize them. An augmented combinatorial datum $\DD$ is \emph{realizable} if there exists a \dessin{} $\Gamma$ with $\DD(\Gamma)=\DD$ such that, for every complementary disk $D$ corresponding to an enveloping element of $\pi_3$, there is an edge of $\Gamma$ which is enveloped by $D$.
\paragraph{Step 1.} The augmented datum
\[
\DD=\datum{\surf{2}}{12}{[3,3,3,3],[3,3,3,3],\pi_3}
\]
is realizable for $\pi_3\in\{[1,\env{11}],[2,\env{10}],[\env{3},\env{9}],[\env{4},\env{8}],[\env{5},\env{7}],[\env{6},\env{6}]\}$. The following pictures display \dessins{} realizing each of these augmented data; as usual, the disk associated to the first element of $\pi_3$ is colored orange and the other one is colored blue; enveloped edges are drawn in the same color as the corresponding disk.
\paragraph{Step 2.} Let $\DD=\datum{\surf{g}}{d}{\pi_1,\pi_2,\pi_3}$ be a realizable augmented datum, and let $\env{x}\in\pi_3$ be an enveloping element. Then the augmented datum
\[
\DD'=\datum{\surf{g+1}}{d+6}{\pi_1\cup[3,3],\pi_2\cup[3,3],\pi_3\setminus[\env{x}]\cup[\env{x+6}]}
\]
is realizable as well. In fact, consider a \dessin{} $\Gamma\subs\surf{g}$ realizing $\DD$, and fix an edge $e$ enveloped by the disk $D$ associated to $\env{x}$. We perform the following operations on $\Gamma$.
\begin{enumerate}[(1)]
\item Add one black vertex and one white vertex on $e$.
\item Attach a tube to $\surf{g}$ with both endpoints on $D$.
\item Add one black vertex, one white vertex and four edges as shown in the picture.
\end{enumerate}
After these operations, we get a new \dessin{} $\Gamma'$ embedded in $\surf{g+1}$. It is easy to check that $\Gamma'$ realizes the augmented datum $\DD'$.
\paragraph{Step 3.} For every $g\ge 2$, the augmented datum 
\[
\datum{\surf{g}}{6g}{[3,\ldots,3],[3,\ldots,3],\pi_3}
\]
is realizable for $\pi_3\in\{[1,\env{6g-1}],[2,\env{6g-2}]\}\cup\{[\env{s},\env{6g-s}]\colon 3\le s\le 6g-3\}$. This can be easily shown by induction on $g$, using step 1 as the base case and step 2 for the induction.
\paragraph{Step 4.} Let $\DD=\datum{\surf{g}}{d}{\pi_1,\pi_2,\pi_3}$ be a realizable augmented datum, and let $\env{x}\in\pi_3$ be an enveloping element. Then the augmented datum
\[
\DD'=\datum{\surf{g}}{d+2}{\pi_1\cup[2],\pi_2\cup[2],\pi_3\setminus[\env{x}]\cup[\env{x+2}]}
\]
is realizable as well. In fact, consider a \dessin{} $\Gamma\subs\surf{g}$ realizing $\DD$, and fix an edge $e$ enveloped by the disk $D$ associated to $\env{x}$. Then, add one black vertex and one white vertex on $e$. The new \dessin{} realizes the augmented datum $\DD'$.
\paragraph{Step 5.} Let $\DD=\datum{\surf{g}}{d}{\pi_1,\pi_2,\pi_3}$ be a realizable augmented datum, and let $\env{x}\in\pi_3$ be an enveloping element. Then the augmented datum
\[
\DD'=\datum{\surf{g}}{d+3}{\pi_1\cup[3],\pi_2\cup[1,2],\pi_3\setminus[\env{x}]\cup[\env{x+3}]}
\]
is realizable as well. In fact, consider a \dessin{} $\Gamma\subs\surf{g}$ realizing $\DD$, and fix an edge $e$ enveloped by the disk $D$ associated to $\env{x}$. Then, add one black vertex and two white vertices as shown in the picture. The new \dessin{} realizes the augmented datum $\DD'$.
\paragraph{Step 6.} Let $\DD=\datum{\surf{g}}{d}{\pi_1,\pi_2,\pi_3}$ be a realizable augmented datum, and let $\env{x}\in\pi_3$ be an enveloping element. Then the augmented datum
\[
\DD'=\datum{\surf{g}}{d+4}{\pi_1\cup[1,3],\pi_2\cup[1,3],\pi_3\setminus[\env{x}]\cup[\env{x+4}]}
\]
is realizable as well. In fact, consider a \dessin{} $\Gamma\subs\surf{g}$ realizing $\DD$, and fix an edge $e$ enveloped by the disk $D$ associated to $\env{x}$. Then, add two black vertices and two white vertices as shown in the picture. The new \dessin{} realizes the augmented datum $\DD'$.

Finally, it is easy to see that the five families listed in the statement can be obtained by applying steps 4, 5 and 6 zero or more times to an augmented datum which is realizable by step 3, and then forgetting about the augmentation; here are the details.
\begin{enumerate}[(1)]
\item $\datum{\surf{g}}{6g}{[3,\ldots,3],[3,\ldots,3],[s,6g-s]}$ is already realizable by step 3.
\item If we assume that $s\le 3g+1$, step 4 gives
\begin{multline*}
\datum{\surf{g}}{6g+2}{[2,3,\ldots,3],[2,3\ldots,3],[s,\env{6g+2-s}]}\\
\cmove\datum{\surf{g}}{6g}{[3,\ldots,3],[3,\ldots,3],[s,\env{6g-s}]},
\end{multline*}
which is realizable by step 3.
\item If we assume that $s\le 3g+1$, step 5 gives 
\begin{multline*}
\datum{\surf{g}}{6g+3}{[3,\ldots,3],[1,2,3,\ldots,3],[s,\env{6g+3-s}]}\\
\cmove\datum{\surf{g}}{6g}{[3,\ldots,3],[3,\ldots,3],[s,\env{6g-s}]},
\end{multline*}
which is realizable by step 3.
\item If we assume that $s\le 3g+2$, step 6 gives
\begin{multline*}
\datum{\surf{g}}{6g+4}{[1,3,\ldots,3],[1,3,\ldots,3],[s,\env{6g+4-s}]}\\
\cmove\datum{\surf{g}}{6g}{[3,\ldots,3],[3,\ldots,3],[s,\env{6g-s}]},
\end{multline*}
which is realizable by step 3.
\item If we assume that $s\le 3g+3$, step 4 and 6 give
\begin{multline*}
\datum{\surf{g}}{6g+6}{[1,2,3,\ldots,3],[1,2,3,\ldots,3],[s,\env{6g+6-s}]}\\
\begin{aligned}
&\cmove\datum{\surf{g}}{6g+4}{[1,3,\ldots,3],[1,3,\ldots,3],[s,\env{6g+4-s}]}\\
&\cmove\datum{\surf{g}}{6g}{[3,\ldots,3],[3,\ldots,3],[s,\env{6g-s}]},
\end{aligned}
\end{multline*}
which is realizable by step 3.\qedhere
\end{enumerate}
\end{proof}
\chapter{Exceptional data with a short partition}\label{short-partition:ch}
\smallvertices{}

\section{Realizability on the sphere}

In this final chapter, we will give a full solution of the Hurwitz existence problem for candidate data containing a partition of length $2$; as usual, we will assume that $\Sigma=\sphere{}$ and $n\ge 3$. This specific instance of the existence problem had already received some interest in the literature, leading to a few partial results.
\begin{itemize}
\item \cref{monodromy:th:sphere-[1 d-1]} addresses the cases where $\pi_n=[1,d-1]$; the proof was actually borrowed from \resultcite{proposition}{5.3}{edmonds}.
\item \textcite{pervova-existence-ii} dealt with the cases where $n=3$ and $\pi_3=[2,d-2]$.
\item \textcite{pakovich} solved the existence problem for $\len{\pi_n}=2$ and $\tSigma=\sphere{}$.
\end{itemize}
In particular, we will consistently exploit the results by \citeauthor{pakovich} as a base case for genus-reducing combinatorial moves. We will now state the relevant theorems, but we decide to omit the proofs: while the core ideas are very ingenious, filling in the details is quite tedious and time-consuming\footnote{The same could probably be said about the other proofs in this chapter, which cannot be omitted for obvious reasons.}. We refer the interested reader to \cite{pakovich}.

\begin{theorem}\label{short-partition:th:realizability-on-sphere-n-3}
Let $\DD=\datum{\sphere{}}{d}{\pi_1,\pi_2,[s,d-s]}$ be a candidate datum. Then $\DD$ is realizable unless it satisfies one of the following.
\begin{enumarabic}
\item $\DD=\datum{\sphere{}}{12}{[2,2,2,2,2,2],[1,1,1,3,3,3],[6,6]}$.
\item $\DD=\datum{\sphere{}}{2k}{[2,\ldots,2],[2,\ldots,2],[s,2k-s]}$ with $k\ge 2$, $s\neq k$.
\item $\DD=\datum{\sphere{}}{2k}{[2,\ldots,2],[1,2,\ldots,2,3],[k,k]}$ with $k\ge2$.
\item $\DD=\datum{\sphere{}}{4k+2}{[2,\ldots,2],[1,\ldots,1,k+1,k+2],[2k+1,2k+1]}$ with $k\ge 1$.
\item $\DD=\datum{\sphere{}}{4k}{[2,\ldots,2],[1,\ldots,1,k+1,k+1],[2k-1,2k+1]}$ with $k\ge2$.
\item\label{short-partition:th:realizability-on-sphere-n-3:it:6} $\DD=\datum{\sphere{}}{kh}{[h,\ldots,h],[1,\ldots,1,k+1],[lh,(k-l)h]}$ with $h\ge 2$, $k\ge 2$, $1\le l\le k-1$.
\end{enumarabic}
\end{theorem}

\begin{theorem}\label{short-partition:th:realizability-on-sphere-n-ge-4}
Let $\DD=\datum{\sphere{}}{d}{\pi_1,\ldots,\pi_{n-1},[s,d-s]}$ be a candidate datum with $n\ge 4$. Then $\DD$ is realizable.
\end{theorem}

In the upcoming proofs, we will also make extensive use of the computational results from \cref{computational-results:ch}; although delegating work to the computer is never necessary (and in theory we could prove the same results by hand), doing so will save us a lot of effort in dealing with tricky corner-cases, and allow us to focus more on the reduction-oriented part of the proofs.

\section{Realizability on the torus for \texorpdfstring{$n=3$}{n=3}}

As anticipated in the title, this section deals with the cases where $n=3$ and $\tSigma=\surf{1}$. Due to the relatively large number of families of exceptional data listed in \cref{short-partition:th:realizability-on-sphere-n-3}, this is the instance of the problem which will require the heaviest casework.

\begin{theorem} \label{short-partition:th:realizability-on-torus-n-3}
Let $\DD=\datum{\surf{1}}{d}{\pi_1,\pi_2,[s,d-s]}$ be a candidate datum. Then $\DD$ is realizable unless it satisfies one of the following.
\begin{enumarabic}
\item $\DD=\datum{\surf{1}}{6}{[3,3],[3,3],[2,4]}$.
\item $\DD=\datum{\surf{1}}{8}{[2,2,2,2],[4,4],[3,5]}$.
\item $\DD=\datum{\surf{1}}{12}{[2,2,2,2,2,2],[3,3,3,3],[5,7]}$.
\item $\DD=\datum{\surf{1}}{16}{[2,2,2,2,2,2,2,2],[1,3,3,3,3,3],[8,8]}$.
\item $\DD=\datum{\surf{1}}{2k}{[2,\ldots,2],[2,\ldots,2,3,5],[k,k]}$ with $k\ge 5$.
\end{enumarabic}
\end{theorem}
\begin{proof}
Thanks to \cref{monodromy:th:sphere-[1 d-1]} we can assume that $2\le s\le d-2$. If $d\le 16$, a computer-aided search (see the results in \cref{computational-results:sc:lists}) shows that the only exceptional cases are:
\begin{enumarabic}
\item $\DD=\datum{\surf{1}}{6}{[3, 3],[3, 3],[2, 4]}$;
\item $\DD=\datum{\surf{1}}{8}{[2, 2, 2, 2],[4, 4],[3, 5]}$;
\item $\DD=\datum{\surf{1}}{10}{[2, 2, 2, 2, 2],[2, 3, 5],[5, 5]}$;
\item $\DD=\datum{\surf{1}}{12}{[2, 2, 2, 2, 2, 2],[3, 3, 3, 3],[5, 7]}$;
\item $\DD=\datum{\surf{1}}{12}{[2, 2, 2, 2, 2, 2],[2, 2, 3, 5],[6, 6]}$;
\item $\DD=\datum{\surf{1}}{14}{[2, 2, 2, 2, 2, 2, 2],[2, 2, 2, 3, 5],[7, 7]}$;
\item $\DD=\datum{\surf{1}}{16}{[2, 2, 2, 2, 2, 2, 2, 2],[1, 3, 3, 3, 3, 3],[8, 8]}$;
\item $\DD=\datum{\surf{1}}{16}{[2, 2, 2, 2, 2, 2, 2, 2],[2, 2, 2, 2, 3, 5],[8, 8]}$.

\end{enumarabic}
This is in agreement with the theorem statement, so we can assume that $d\ge 17$. We now analyze several cases.
\begin{sideline}{Case 1.} Assume that:
\begin{assumptions}
\item $x\in\pi_1$ for some $x\ge 4$;
\item $2\in\pi_2$;
\item $\pi_2\neq[2,\ldots,2]$.
\end{assumptions}
\Cref{combinatorial-move:b:4 2} gives
\[
\DD\cmove\datum{\sphere{}}{d-2}{\pi_1\setminus[x]\cup[1,x-3],\pi_2\setminus[2],[s-1,d-s-1]},
\]
which is realizable by \cref{short-partition:th:realizability-on-sphere-n-3} unless one of the following holds\footnote{Since this proof is long enough as it is, we will refrain from explaining in detail why candidate data of a certain form are realizable by \cref{short-partition:th:realizability-on-sphere-n-3}; we leave the tedious yet elementary casework required to the motivated reader. In this situation, for instance, one could simply note that $\pi_1\setminus[x]\cup[1,x-3]$ and $\pi_2\setminus[2]$ are both different from $[2,\ldots,2]$, therefore the only exceptional data to consider among those listed in \cref{short-partition:th:realizability-on-sphere-n-3} are those belonging to family \ref{short-partition:th:realizability-on-sphere-n-3:it:6}.}.
\begin{itemize}
\item $\DD=\datum{\surf{1}}{kh+2}{[1,\ldots,1,k+4],[2,h,\ldots,h],[lh+1,(k-l)h+1]}$ with $k\ge 2$, $h\ge 3$, $1\le l\le k-1$. By applying \cref{combinatorial-move:b:4 2} we get
\[
\DD\cmove\datum{\sphere{}}{kh}{[1,\ldots,1,2,k],[h,\ldots,h],[lh,(k-l)h]},
\]
which is realizable by \cref{short-partition:th:realizability-on-sphere-n-3}.
\item $\DD=\datum{\surf{1}}{kh+2}{[1,\ldots,1,4,k+1],[2,h,\ldots,h],[lh+1,(k-l)h+1]}$ with $k\ge 2$, $h\ge 3$, $1\le l\le k-1$. By applying \cref{combinatorial-move:b:4 3} we get
\[
\DD\cmove\datum{\sphere{}}{kh}{[1,\ldots,1,2,k+1],[2,h-2,h,\ldots,h],[lh,(k-l)h]},
\]
which is realizable by \cref{short-partition:th:realizability-on-sphere-n-3}.\sdlendhere
\end{itemize}
\end{sideline}
\begin{sideline}{Case 2.} Assume that:
\begin{assumptions}
\item $x\in\pi_1$ for some $x\ge 4$;
\item $y\in\pi_2$ for some $y\ge 4$;
\item $2\not\in\pi_1$ and $2\not\in\pi_2$.
\end{assumptions}
\Cref{combinatorial-move:b:4 3} gives
\[
\DD\cmove\datum{\sphere{}}{d-2}{\pi_1\setminus[x]\cup[x-2],\pi_2\setminus[y]\cup[y-2],[s-1,d-s-1]},
\]
which is realizable by \cref{short-partition:th:realizability-on-sphere-n-3} unless
\[
\DD=\datum{\surf{1}}{kh+2}{[1,\ldots,1,k+3],[h,\ldots,h,h+2],[lh+1,(k-l)h+1]}
\]
for some $k\ge 2$, $h\ge 3$, $1\le l\le k-1$. If this is the case, applying \cref{combinatorial-move:b:4 3} yields
\[
\DD\cmove\datum{\sphere{}}{kh}{[1,\ldots,1,k+1],[h-2,h,\ldots,h,h+2],[lh,(k-l)h]},
\]
which is realizable by \cref{short-partition:th:realizability-on-sphere-n-3}.
\end{sideline}
\begin{sideline}{Case 3.} Assume that:
\begin{assumptions}
\item $x\in\pi_1$ for some $x\ge 4$;
\item $\max(\pi_2)=3$;
\item $2\not\in\pi_2$.
\end{assumptions}
\Cref{combinatorial-move:b:4 3} gives
\[
\DD\cmove=\datum{\sphere{}}{d-2}{\pi_1\setminus[x]\cup[x-2],\pi_2\setminus[3]\cup[1],[s-1,d-s-1]},
\]
which is realizable by \cref{short-partition:th:realizability-on-sphere-n-3} unless
\[
\DD=\datum{\surf{1}}{2h+2}{[h,h+2],[1,\ldots,1,3,3],[h+1,h+1]}
\]
for some $h\ge 8$ (recall that we are assuming $d\ge 17$). If this is the case, applying \Cref{combinatorial-move:b:[1 1 3]} yields
\[
\DD\cmove\datum{\sphere{}}{2h+2}{[h,h+2],[1,\ldots,1,3],[h+1,h+1]},
\]
which is realizable by \cref{short-partition:th:realizability-on-sphere-n-3}.
\end{sideline}
\begin{sideline}[1.8cm]{Case 4.} Assume that:
\begin{assumptions}
\item $\max(\pi_1)=3$;
\item $\max(\pi_2)\le 3$;
\item $\pi_2\neq[2,\ldots,2]$.
\end{assumptions}
We analyze a few sub-cases.
\begin{itemize}
\item \textbf{Case 4.1:} $[1,1]\subs\pi_1$. \Cref{combinatorial-move:b:[1 1 3]} gives
\[
\DD\cmove\datum{\sphere{}}{d}{\pi_1\setminus[3]\cup[1,1,1],\pi_2,[s,d-s]},
\]
which is realizable by \cref{short-partition:th:realizability-on-sphere-n-3}.
\item \textbf{Case 4.2:} $[3,3]\subs\pi_1$ and $[2,2]\subs\pi_2$. If $s=2$ or $s=d-2$ then $\DD$ is realizable by \cref{dessins:th:special-case-[2 d-2]}. Otherwise, \cref{combinatorial-move:b:[3 3] [2 2]} gives
\[
\DD\cmove\datum{\sphere{}}{d-4}{\pi_1\setminus[3,3]\cup[1,1],\pi_2\setminus[2,2],[s-2,d-s-2]},
\]
which is realizable by \cref{short-partition:th:realizability-on-sphere-n-3}.
\item \textbf{Case 4.3:} $[2,2]\not\subs\pi_2$. The \RH{} formula immediately implies that $3\in\pi_2$. Assume by contradiction that $\DD$ is not realizable. If this is the case, $[1,1]\not\subs\pi_2$ by Case 4.1, but then $[3,3]\subs\pi_2$. It follows (Case 4.2) that $[2,2]\not\subs\pi_1$; moreover, Case 4.1 also implies that $[1,1]\not\subs\pi_1$. In other words, both $\pi_1$ and $\pi_2$ can be written as $\rho\cup[3,\ldots,3]$, where $\rho\subs[1,2]$ (possibly different for $\pi_1$ and $\pi_2$). As a consequence, we have the inequalities
\begin{align*}
d\ge 3\len{\pi_1}-3,&&d\ge 3\len{\pi_2}-3,
\end{align*}
which contradict the \RH{} formula if $d\ge 13$.
\item \textbf{Case 4.4:} $[3,3]\not\subs\pi_1$. From the \RH{} formula it follows that $[3,3]\subs\pi_2$, but then $\DD$ is realizable by Case 4.1.\sdlendhere
\end{itemize}
\end{sideline}
\begin{sideline}{Case 5.} Assume that:
\begin{assumptions}
\item $\max(\pi_1)\ge 4$;
\item $\pi_2=[2,\ldots,2]$.
\end{assumptions}
Let $x=\max(\pi_1)$; \cref{combinatorial-move:b:4 2} gives
\[
\DD\cmove\datum{\sphere{}}{d-2}{\pi_1\setminus[x]\cup[1,x-3],[2,\ldots,2],[s-1,d-s-1]},
\]
which is realizable by \cref{short-partition:th:realizability-on-sphere-n-3} unless one of the following holds.
\begin{itemize}
\item $\DD=\datum{\surf{1}}{2k+2}{[2,\ldots,2,3,5],[2,\ldots,2],[k+1,k+1]}$ with $k\ge 7$. This is in fact one of the exceptional data listed in the statement.
\item $\DD=\datum{\surf{1}}{2k+2}{[2,\ldots,2,6],[2,\ldots,2],[k+1,k+1]}$ with $k\ge 7$. By applying \cref{combinatorial-move:b:4 2} we get
\[
\DD\cmove\datum{\sphere{}}{2k}{[2,\ldots,2],[2,\ldots,2],[k,k]},
\]
which is realizable by \cref{short-partition:th:realizability-on-sphere-n-3}.
\item $\DD=\datum{\surf{1}}{4k+4}{[1,\ldots,1,k+2,k+4],[2,\ldots,2],[2k+2,2k+2]}$ with $k\ge 3$. By applying \cref{combinatorial-move:b:4 2} we get
\[
\DD\cmove\datum{\sphere{}}{4k+2}{[1,\ldots,1,2,k,k+2],[2,\ldots,2],[2k+1,2k+1]},
\]
which is realizable by \cref{short-partition:th:realizability-on-sphere-n-3}.
\item $\DD=\datum{\surf{1}}{4k+4}{[1,\ldots,1,k+1,k+5],[2,\ldots,2],[2k+2,2k+2]}$ with $k\ge 3$. By applying \cref{combinatorial-move:b:4 2} we get
\[
\DD\cmove\datum{\sphere{}}{4k+2}{[1,\ldots,1,2,k+1,k+1],[2,\ldots,2],[2k+1,2k+1]},
\]
which is realizable by \cref{short-partition:th:realizability-on-sphere-n-3}.
\item $\DD=\datum{\surf{1}}{4k+2}{[1,\ldots,1,k+1,k+4],[2,\ldots,2],[2k,2k+2]}$ with $k\ge 4$. By applying \cref{combinatorial-move:b:4 2} we get
\[
\DD\cmove\datum{\sphere{}}{4k}{[1,\ldots,1,2,k,k+1],[2,\ldots,2],[2k-1,2k+1]},
\]
which is realizable by \cref{short-partition:th:realizability-on-sphere-n-3}.
\item $\DD=\datum{\surf{1}}{2k+2}{[1,\ldots,1,k+4],[2,\ldots,2],[2l+1,2(k-l)+1]}$ with $k\ge 7$, $1\le l\le k-1$. By applying \cref{combinatorial-move:b:4 2} we get
\[
\DD\cmove\datum{\sphere{}}{2k}{[1,\ldots,1,2,k],[2,\ldots,2],[2l,2(k-l)]},
\]
which is realizable by \cref{short-partition:th:realizability-on-sphere-n-3}.\sdlendhere
\end{itemize}
\end{sideline}
\begin{sideline}{Case 6.} Assume that:
\begin{assumptions}
\item $\max(\pi_1)=3$;
\item $\pi_2=[2,\ldots,2]$.
\end{assumptions}
The \RH{} formula immediately implies that $[3,3,3,3]\subs\pi_1$. If $s=2$ or $s=d-2$ then $\DD$ is realizable by \cref{dessins:th:special-case-[2 d-2]}. Otherwise, \cref{combinatorial-move:b:[3 3] [2 2]} gives
\[
\DD\cmove\datum{\sphere{}}{d-4}{\pi_1\setminus[3,3]\cup[1,1],[2,\ldots,2],[s-2,d-s-2]},
\]
which is realizable by \cref{short-partition:th:realizability-on-sphere-n-3}, since we are assuming that $d\ge 17$.
\end{sideline}
The cases we have analyzed, up to swapping $\pi_1$ and $\pi_2$, cover all the candidate data of the form $\datum{\surf{1}}{d}{\pi_1,\pi_2,[s,d-s]}$. We have shown that every datum which is not listed in the statement is realizable, therefore the proof is complete.
\end{proof}

\section{Realizability on higher genus surfaces for \texorpdfstring{$n=3$}{n=3}}

It turns out that, for $n=3$ and $\len{\pi_3}=2$, there are no exceptional data on surfaces with genus $g\ge 2$.

\begin{theorem}\label{short-partition:th:realizability-on-higher-genus-n-3}
Let $\DD=\datum{\surf{g}}{d}{\pi_1,\pi_2,[s,d-s]}$ be a candidate datum with $g\ge 2$. Then $\DD$ is realizable.
\end{theorem}
\begin{proof}
Thanks to \cref{monodromy:th:sphere-[1 d-1]} we can assume that $2\le s\le d-2$. For $d\le 18$, a computer-aided search shows that there are no exceptional data (see the results in \cref{computational-results:sc:lists}). Therefore, we can further assume that $d\ge 19$. We proceed by induction on $g\ge 2$, analyzing several cases.
\begin{sideline}{Case 1.} Assume that:
\begin{assumptions}
\item $x\in\pi_1$ for some $x\ge 4$;
\item $2\in\pi_2$.
\end{assumptions}
\Cref{combinatorial-move:b:4 2} gives
\[
\DD\cmove\datum{\surf{g-1}}{d-2}{\pi_1\setminus[x]\cup[1,x-3],\pi_2\setminus[2],[s-1,d-s-1]},
\]
which is realizable by \cref{short-partition:th:realizability-on-torus-n-3} if $g=2$, or by induction if $g\ge 3$.
\end{sideline}
\begin{sideline}{Case 2.} Assume that:
\begin{assumptions}
\item $x\in\pi_1$ for some $x\ge 4$;
\item $y\in\pi_2$ for some $y\ge 3$;
\item $2\not\in\pi_1$ and $2\not\in\pi_2$.
\end{assumptions}
\Cref{combinatorial-move:b:4 3} gives
\[
\DD\cmove\datum{\surf{g-1}}{d-2}{\pi_1\setminus[x]\cup[x-2],\pi_2\setminus[y]\cup[y-2],[s-1,d-s-1]},
\]
which is realizable by \cref{short-partition:th:realizability-on-torus-n-3} if $g=2$, or by induction if $g\ge 3$.
\end{sideline}
\begin{sideline}{Case 3.} Assume that:
\begin{assumptions}
\item $\max(\pi_1)=3$;
\item $\max(\pi_2)\le 3$.
\end{assumptions}
We analyze a few sub-cases.
\begin{itemize}
\item \textbf{Case 3.1:} $[1,1]\subs\pi_1$. \Cref{combinatorial-move:b:[1 1 3]} gives
\[
\DD\cmove\datum{\surf{g-1}}{d}{\pi_1\setminus[3]\cup[1,1,1],\pi_2,[s,d-s]},
\]
which is realizable by \cref{short-partition:th:realizability-on-torus-n-3} if $g=2$, or by induction if $g\ge 3$.
\item\textbf{Case 3.2:} $[3,3]\subs\pi_1$ and $[2,2]\subs\pi_2$. If $s=2$ or $s=d-2$ then $\DD$ is realizable by \cref{dessins:th:special-case-[2 d-2]}. Otherwise, \cref{combinatorial-move:b:[3 3] [2 2]} gives
\[
\DD\cmove\datum{\surf{g-1}}{d-4}{\pi_1\setminus[3,3]\cup[1,1],\pi_2\setminus[2,2],[s-2,d-s-2]},
\]
which is realizable by \cref{short-partition:th:realizability-on-torus-n-3} if $g=2$, or by induction if $g\ge 3$.
\item \textbf{Case 3.3:} $[2,2]\not\subs\pi_1$. The \RH{} formula immediately implies that $[3,3]\subs\pi_2$. Assume by contradiction that $\DD$ is not realizable. Then $[1,1]\not\subs\pi_1$ and $[1,1]\not\subs\pi_2$ (Case 3.1), and moreover $[2,2]\not\subs\pi_1$ (Case 3.2). In other words, both $\pi_1$ and $\pi_2$ can be written as $\rho\cup[3,\ldots,3]$, where $\rho\subs[1,2]$ (possibly different for $\pi_1$ and $\pi_2$). It is then easy to see that $\DD$ belongs to one of the families listed in \cref{dessins:th:special-families} and, therefore, is realizable.
\item \textbf{Case 3.4:} $[3,3]\not\subs\pi_1$. From the \RH{} formula it follows that $[3,3]\subs\pi_2$, but then $\DD$ is realizable by Cases 3.2 and 3.3.\sdlendhere
\end{itemize}
\end{sideline}
The cases we have analyzed, up to swapping $\pi_1$ and $\pi_2$, cover all the candidate data of the form $\datum{\surf{g}}{d}{\pi_1,\pi_2,[s,d-s]}$ with $g\ge 2$. We have shown that every datum is realizable, therefore the proof is complete.
\end{proof}

\section{Realizability for \texorpdfstring{$n\ge 4$}{n≥4}}

After solving the existence problem for $n=3$ and $\len{\pi_3}=2$, we turn to candidate data with $n\ge 4$ partitions. As we are about to see, exceptional data in this setting are very rare: there is an infinite family with $d=4$, which we already encountered in \cref{monodromy:th:sphere-d-equals-4}, and a single datum with $n=4$ and $d=8$.

\begin{theorem}\label{short-partition:th:realizability-on-higher-genus-n-ge-4}
Let $\DD=\datum{\surf{g}}{d}{\pi_1,\ldots,\pi_{n-1},[s,d-s]}$ be a candidate datum with $n\ge 4$. Then $\DD$ is realizable unless it satisfies one of the following.
\begin{enumarabic}
\item $\DD=\datum{\surf{2}}{8}{[2,2,2,2],[2,2,2,2],[2,2,2,2],[3,5]}$.
\item $\DD=\datum{\surf{n-3}}{4}{[2,2],\ldots,[2,2],[1,3]}$.
\end{enumarabic}
\end{theorem}
\begin{proof}
If $d\le 16$, a computer-aided search (see the results in \cref{computational-results:sc:lists}) shows that the only exceptional cases are:
\begin{enumarabic}
\item $\DD=\datum{\surf{1}}{4}{[2, 2],[2, 2],[2, 2],[1, 3]}$;
\item $\DD=\datum{\surf{2}}{8}{[2, 2, 2, 2],[2, 2, 2, 2],[2, 2, 2, 2],[3, 5]}$.
\end{enumarabic}
This is in agreement with the statement, so we can assume that $d\ge 17$. Moreover, every candidate datum with $g=0$ is realizable by \cref{short-partition:th:realizability-on-sphere-n-ge-4}; as a consequence, we only have to consider the cases where $g\ge 1$.

We will proceed by induction on $n$. We start with the base case $n=4$, which requires the heaviest casework. Fix a candidate datum $\DD=\datum{\surf{g}}{d}{\pi_1,\pi_2,\pi_3,[s,d-s]}$.
\begin{sideline}{Case 1.} Assume that the inequality $v(\pi_i)+v(\pi_j)<d$ holds for a pair of indices $1\le i<j\le 3$; up to reindexing, we can assume that $v(\pi_1)+v(\pi_2)<d$. \Cref{combinatorial-move:a:small-v} gives
\[
\DD\cmove\DD'=\datum{\surf{g}}{d}{\pi_1',\pi_3,[s,d-s]}
\]
for a suitable candidate datum $\DD'$, where $v(\pi_1')=v(\pi_1)+v(\pi_2)$. If $g\ge 2$, then $\DD'$ is realizable by \cref{short-partition:th:realizability-on-higher-genus-n-3}. If instead $g=1$, then $\DD'$ is realizable by \cref{short-partition:th:realizability-on-torus-n-3} unless
\[
\text{$\DD'=\datum{\surf{1}}{2k}{[2,\ldots,2],[2,\ldots,2,3,5],[k,k]}$ with $2k=d$.}
\]
If this is the case, then $\pi_4=[k,k]$ and $\{\pi_1',\pi_3\}=\{[2,\ldots,2],[2,\ldots,2,3,5]\}$. Some more casework is required to show that $\DD'$ can actually be chosen to be realizable; since $\DD\cmove\DD'$, this will imply that $\DD$ is realizable as well.
\begin{itemize}
\item If $\pi_1'=[2,\ldots,2]$, then $v(\pi_1')=k$ and $v(\pi_3)=k+2$. Assume without loss of generality that $v(\pi_1)\le k/2$. We have that
\[
k+2<1+k+2\le v(\pi_1)+v(\pi_3)\le \frac{k}{2}+k+2<d.
\]
Repeating the construction with $i=1$ and $j=3$ will yield a realizable $\DD'$.
\item If $\pi_3=[2,\ldots,2]$ and $v(\pi_1)\not\in\{2,k,k+1\}$ then $v(\pi_1)+v(\pi_3)<d$ and $v(\pi_1)+v(\pi_3)\not\in\{k,k+2\}$. Therefore, repeating the construction with $i=1$ and $j=3$ will yield a realizable $\DD'$.
\item If $\pi_3=[2,\ldots,2]$, $v(\pi_1)=2$ and $\pi_2\neq [2,\ldots,2]$, then repeating the construction with $i=1$ and $j=3$ will yield a realizable $\DD'$.
\item If $\pi_2=\pi_3=[2,\ldots,2]$ and $\pi_1\neq[1,\ldots,1,3]$, we follow a different approach. By applying \cref{combinatorial-move:a:large-v} to the partitions $\pi_2$ and $\pi_3$ we get
\[
\DD\cmove\datum{\sphere{}}{2k}{\pi_1,[k,k],[k,k]},
\]
which is realizable by \cref{short-partition:th:realizability-on-sphere-n-3}.
\item Finally, if $\pi_1=[1,\ldots,1,3]$ and $\pi_2=\pi_3=[2,\ldots,2]$, we have to work explicitly with permutations. Consider
\begin{align*}
\alpha_1=\cycle{1,3,5},&&\alpha_2=\cycle{1,2}\cycle{3,4}\cdots\cycle{2k-1,2k}.
\end{align*}
Clearly $[\alpha_1]=\pi_1$ and $[\alpha_2]=\pi_2$; moreover,
\[
\alpha_1\alpha_2=\cycle{1,2,3,4,5,6}\cycle{7,8}\cdots\cycle{2k-1,2k},
\]
so $[\alpha_1\alpha_2]=[2,\ldots,2,6]$. Note that
\[
v(\alpha_1)+v(\alpha_2)=2+k=v(\alpha_1\alpha_2),
\]
so \cref{monodromy:rm:combinatorial-move:a:small-v} gives
\[
\DD\cmove\datum{\surf{1}}{2k}{[2,\ldots,2,6],[2,\ldots,2],[k,k]},
\]
which is realizable by \cref{short-partition:th:realizability-on-torus-n-3}.
\end{itemize}
Up to swapping $\pi_1$ and $\pi_2$, this analysis covers all the possible cases.
\end{sideline}
\begin{sideline}{Case 2.} Otherwise, the inequality $v(\pi_i)+v(\pi_j)\ge d$ holds for every $1\le i<j\le 3$. In particular, up to reindexing, we can assume that $v(\pi_3)\ge d/2$. Note that $v(\pi_1)+v(\pi_2)\ge d$ and $v(\pi_3)+v([s,d-s])\ge 1+d-2=d-1$, so \cref{combinatorial-move:a:large-v} gives
\[
\DD\cmove\DD'=\datum{\surf{g'}}{d}{\pi_1',\pi_3,[s,d-s]}
\]
for a suitable candidate datum $\DD'$ with $v(\pi_1')\ge d-2$. We can actually compute
\[
g'=\frac{1}{2}(v(\pi_1')+v(\pi_3)+v([s,d-s])-d+1\ge\frac{1}{2}\left(d-2+\frac{d}{2}+d-2\right)-d+1=\frac{d}{4}-1\ge 2.
\]
Therefore $\DD'$ is realizable by \cref{short-partition:th:realizability-on-higher-genus-n-3}, and $\DD$ is realizable as well.
\end{sideline}

We now turn to the case $n\ge 5$; we show by induction that every candidate datum $\DD=\datum{\surf{g}}{d}{\pi_1,\ldots,\pi_{n-1},[s,d-s]}$  different from $\datum{\surf{n-3}}{4}{[2,2],\ldots,[2,2],[1,3]}$ is realizable. The case $d=4$ is addressed by \cref{monodromy:th:sphere-d-equals-4}. If $n=5$ and $d=8$, the computational results in \cref{computational-results:sc:lists} imply that $\DD$ is realizable. Otherwise, the routine reduction argument relying on \cref{combinatorial-move:a:small-v,combinatorial-move:a:large-v} shows that $\DD$ is realizable. \qedhere
\end{proof}


\section{Exceptionality}

This section is devoted to showing that the candidate data listed in the statements of \cref{short-partition:th:realizability-on-sphere-n-3,short-partition:th:realizability-on-torus-n-3,short-partition:th:realizability-on-higher-genus-n-ge-4} are in fact exceptional. First of all, we address the special cases with small degree ($d\le 16$). The computational results in \cref{computational-results:sc:lists} show that the following candidate data are exceptional.
\begin{enumarabic}
\item $\DD=\datum{\sphere{}}{12}{[2,2,2,2,2,2],[1,1,1,3,3,3],[6,6]}$.
\item $\DD=\datum{\surf{1}}{6}{[3,3],[3,3],[2,4]}$.
\item $\DD=\datum{\surf{1}}{8}{[2,2,2,2],[4,4],[3,5]}$.
\item $\DD=\datum{\surf{1}}{12}{[2,2,2,2,2,2],[3,3,3,3],[5,7]}$.
\item $\DD=\datum{\surf{1}}{16}{[2,2,2,2,2,2,2,2],[1,3,3,3,3,3],[8,8]}$.
\item $\DD=\datum{\surf{2}}{8}{[2,2,2,2],[2,2,2,2],[2,2,2,2],[3,5]}$.
\end{enumarabic}
We now turn to infinite families of exceptional data.

\begin{proposition}\label{short-partition:th:exceptional-d-4}
Let $n\ge 3$ be a positive integer. Then the candidate datum
\[
\DD=\datum{\surf{n-3}}{4}{[2,2],\ldots,[2,2],[1,3]}
\]
is exceptional.
\end{proposition}
\begin{proof}
In order to show the exceptionality of $\DD$, we will employ the monodromy approach. Let $\alpha_1,\ldots,\alpha_{n-1}\in\symgroup[4]$ be permutations matching $[2,2]$. It is easy to see that the three permutations matching $[2,2]$, together with the identity, form a subgroup
\[
\{\id,\cycle{1,2}\cycle{3,4},\cycle{1,3}\cycle{2,4},\cycle{1,4}\cycle{2,3}\}\subgroup\symgroup[4],
\]
which does not contain any $3$\=/cycles. As a consequence, the product $\alpha_1\cdots\alpha_{n-1}$ cannot match $[1,3]$; by \cref{hurwitz:th:monodromy-realizability-orientable}, this implies that $\DD$ is exceptional.
\end{proof}

For the other families of exceptional data, we will instead make use of \dessins{}.

\bgroup
\tikzset{
quick/.style={graph edge={below}{black edge}},
bunch of nodes settings/quick/.style={first edge={contour edge={quick={#1}{#1},begin=0.5,end=0.5,left ecap=360}},last edge={contour edge={quick={#1}{#1},left ecap=360,,begin=0.5,end=0.5}}},
contour edge settings/quick/.style 2 args={left=disk #1 boundary,right=disk #2 boundary,edge=black edge}
}
\NewDocumentEnvironment{dessintext}{}{\begin{longtable}{@{}p{0.6\linewidth}@{}p{0.4\linewidth}@{}}}{\end{longtable}}
\begin{proposition}\label{short-partition:th:exceptional-composite}
Let $h\ge 2$, $k\ge 2$, $1\le l\le k-1$ be integers. Then the candidate datum
\[
\DD=\datum{\sphere{}}{kh}{[h,\ldots,h],[1,\ldots,1,k+1],[lh,(k-l)h]}
\]
is exceptional.
\end{proposition}
\begin{proof}
Assume by contradiction that there is a \dessin{} $\Gamma\subs\sphere{}$ realizing $\DD$. Let $v$ be the white vertex of degree $k+1$. Since $\Gamma$ is connected, each of the $k$ black vertices must have an edge connecting it to $v$. Therefore there are exactly $k-1$ black vertices with one edge between them and $v$, and one black vertex with two edges between it and $v$; we call this special black vertex $u$. The two edges connecting $u$ and $v$ define the two complementary disks of $\Gamma$. As shown in the picture below, there are (excluding $u$) $a$ black vertices inside $D_1$ and $k-1-a$ black vertices inside $D_2$, for some integer $0\le a\le k-1$; of the $h-2$ white vertices of degree $1$ connected to $u$, $b$ lie inside $D_1$, while $h-2-b$ lie inside $D_2$, for some $0\le b\le h-2$.
\begin{center}
\tikzsetnextfilename{exceptionality-composite}
\begin{tikzpicture}[graph picture,x={(4,0)},y={(0,4)}]
\path[use as bounding box] (-.94,-.7) rectangle (1.7,.85);
\path (0,0) coordinate (w) pic{white vertex} node[shift={(115:1.6em)}] {$v$};
\path (1,0) coordinate (b) pic{black vertex} node[shift={(65:1.6em)}] {$u$};
\path[contour edge={quick={2}{1}}] (w) to[out=90,in=90,looseness=2.5]  (b);
\path[contour edge={quick={1}{2}}] (w) to[out=-90,in=-90,looseness=2] (b);
\pic[rotate=180] at (w) {bunch of nodes={quick={2},size=1.4cm,first node={\pic{black vertex};
\pic[rotate=90]{bunch of nodes={size=0.6cm,quick={2},first node={\pic{white vertex};},last node={\pic{white vertex};},brace={$h-1$}}};
},last node={\pic{black vertex};\pic[rotate=-90]{bunch of nodes={size=0.6cm,quick={2},first node={\pic{white vertex};},last node={\pic{white vertex};}}};},brace={$k-1-a$}}};

\pic at (w) {bunch of nodes={quick={1},size=1.4cm,first node={\pic{black vertex};
\pic[rotate=-90]{bunch of nodes={size=0.6cm,quick={1},first node={\pic{white vertex};},last node={\pic{white vertex};}}};
},last node={\pic{black vertex};\pic[rotate=90]{bunch of nodes={size=0.6cm,quick={1},first node={\pic{white vertex};},brace={$h-1$},last node={\pic{white vertex};}}};},brace={$a$}}};

\pic at (b) {bunch of nodes={quick={2},size=0.6cm,first node={\pic{white vertex};},last node={\pic{white vertex};},brace={$h-2-b$}}};
\pic[rotate=180] at (b) {bunch of nodes={quick={1},size=0.6cm,first node={\pic{white vertex};},last node={\pic{white vertex};},brace={$b$}}};
\end{tikzpicture}
\end{center}
It is easy to compute the perimeters:
\begin{align*}
\frac{1}{2}\card{\partial D_1}=ah+b+1,&&\frac{1}{2}\card{\partial D_2}=(k-1-a)h+(h-2-b)+1.
\end{align*}
In particular, we see that $\card{\partial D_1}/2$ and $\card{\partial D_2}/2$ are not divisible by $h$, hence $\Gamma$ cannot be a \dessin{} realizing $\DD$.
\end{proof}

The following result will allow us to show the exceptionality of some candidate data of the form $\datum{\surf{g}}{d}{[2,\ldots,2],\pi_2,\pi_3}$.

\begin{lemma}\label{short-partition:th:lemma-[2 ... 2]}
Let $\DD=\datum{\surf{g}}{d}{[2,\ldots,2],\pi_2,\pi_3}$ be a combinatorial datum. Then $\DD$ is realizable if and only if there exists a graph $\Gamma$ embedded in $\surf{g}$ such that:
\begin{itemize}
\item $\pi_2=[k(v_1),\ldots,k(v_r)]$, where $v_1,\ldots,v_r$ are the vertices of $\Gamma$;
\item the complementary regions of $\Gamma$ are topological disks $D_1,\ldots,D_h$;
\item $\pi_3=[\card{\partial D_1},\ldots,\card{\partial D_h}]$.
\end{itemize}
\end{lemma}
\begin{proof}
The key idea is that vertices of degree $2$ do not contribute to the topology of a graph in any meaningful way.
\begin{twoimplications}
\rightimplication
If are given a \dessin{} $\Gamma'\subs\surf{g}$ realizing $\DD$, we can obtain a suitable graph $\Gamma$ simply by removing all the black vertices, and merging the two edges adjacent to each one of them into a single edge. This operations does not change the topology of the complementary regions; all the perimeters are halved.
\leftimplication
Conversely, if we are given a graph $\Gamma\subs\surf{g}$, we can recover a \dessin{} $\Gamma'$ by coloring the vertices of $\Gamma$ with the color white, and inserting a black vertex in the middle of every edge.
\end{twoimplications}
\[ % very ugly hack, but it works :/
\tikzsetnextfilename{dessin-2-2-2-2}
\begin{tikzpicture}[graph picture]
\foreach \i/\j in {-1/1,1/0} {
\begin{scope}[shift={(\i*3.5,0)}]
\path (0,0) coordinate (1) pic{white vertex};
\path (1,1) coordinate (2) pic{white vertex};
\path (2,-1) coordinate (3) pic{white vertex};
\path (3,-0.3) coordinate (4) pic{white vertex};
\path (1.5,-0.1) coordinate (5) pic{white vertex};
\ifnumcomp{\j}{=}{1}{
\tikzset{every path/.style={decoration={markings,mark=at position .5 with {\pic{black vertex};}}}}
}{}
\path[quick,postaction={decorate}] (1) arc(0:360:0.5);
\path[quick,postaction={decorate}] (1) to[bend left] (2);
\path[quick,postaction={decorate}] (2) to[bend left=60,looseness=1.5] (3);
\path[quick,postaction={decorate}] (1) to[bend right] (3);
\path[quick,postaction={decorate}] (3) to (4);
\path[quick,postaction={decorate}] (2) to (5);
\end{scope}
}
\node at (1,0) {$\Longleftrightarrow$};
\end{tikzpicture}\qedhere\]
\end{proof}

\Cref{short-partition:th:lemma-[2 ... 2]} suggests an approach by enumeration for showing the exceptionality of candidate data of the form $\datum{\surf{g}}{d}{[2,\ldots,2],\pi_2,\pi_3}$. In fact, if the number of fat graphs whose degrees are the entries of $\pi_2$ is small, we can check them one by one to see if the associated embedded graph satisfies the conditions of \cref{short-partition:th:lemma-[2 ... 2]}.

\begin{proposition}\label{short-partition:th:exceptional-sphere}
The following families of candidate data are exceptional.
\begin{enumarabic}
\item $\DD=\datum{\sphere{}}{2k}{[2,\ldots,2],[2,\ldots,2],[s,2k-s]}$ with $k\ge 2$, $s\neq k$.
\item $\DD=\datum{\sphere{}}{2k}{[2,\ldots,2],[1,2,\ldots,2,3],[k,k]}$ with $k\ge2$.
\item $\DD=\datum{\sphere{}}{4k+2}{[2,\ldots,2],[1,\ldots,1,k+1,k+2],[2k+1,2k+1]}$ with $k\ge 1$.
\item $\DD=\datum{\sphere{}}{4k}{[2,\ldots,2],[1,\ldots,1,k+1,k+1],[2k-1,2k+1]}$ with $k\ge2$.
\end{enumarabic}
\end{proposition}
\begin{proof}
For each family listed in the statement, we follow the approach by enumeration presented above: we draw all the possible graphs embedded in $\sphere{}$ whose vertices have the entries of $\pi_2$ as degrees and whose complementary regions are two disks; then we compute the perimeters of these disks, showing that they cannot be equal to the entries of $\pi_3$.
\begin{enumarabic}
\item There is only one connected graph embedded in $\sphere{}$ whose vertices have degrees $[2,\ldots,2]$.
\begin{dessintext}
\makecell[c]{
\tikzsetnextfilename{exceptionality-sphere-1-1}
\begin{tikzpicture}[graph picture,x={(4,0)},y={(0,4)}]
\path[contour edge={quick={1}{2}}] circle(0.3);
\foreach \i in {0,45,...,359} {\pic at (\i:0.3) {white vertex};}
\end{tikzpicture}
}&$\begin{aligned}
&\card{\partial D_1}=\card{\partial D_2}=k
\end{aligned}$
\end{dessintext}
By \cref{short-partition:th:lemma-[2 ... 2]}, it follows that $\DD$ is exceptional unless $s=k$.
\item A connected graph embedded in $\sphere{}$ whose vertices have degrees $[1,2,\ldots,2,3]$ can be represented by a diagram like the following, where $a$ denotes the number of edges on the path connecting the vertex of degree $3$ to the vertex of degree $1$.

\begin{dessintext}
\makecell[c]{
\tikzsetnextfilename{exceptionality-sphere-2-1}
\begin{tikzpicture}[graph picture,x={(4,0)},y={(0,4)},
vertices along/.style={postaction={decorate,decoration={markings,mark=between positions 0.2 and 0.81 step 0.2 with {\pic{white vertex};}}}}
]
\path (0,0) coordinate (3) pic{white vertex};
\path (0.5,0) coordinate (1) pic{white vertex};
\path[contour edge={quick={1}{2},left bcap=90},vertices along] (3) arc(0:360:.2);
\path[contour edge={quick={2}{2},right ecap=360},vertices along] (3) -- (1);
\draw[decorate,decoration={mirror,raise=0.5em,pre=moveto,pre length=0.3em,post=moveto,post length=0.3em,brace}] (3) -- (1) node[midway,below=1em] {$a$};
\end{tikzpicture}
}&$\begin{aligned}
&1\le a\le k-1\\
&\card{\partial D_1}=k-a\neq k\\
&\card{\partial D_2}=k+a\neq k
\end{aligned}$
\end{dessintext}
It follows that $\DD$ is exceptional, once again by \cref{short-partition:th:lemma-[2 ... 2]}.
\item There are only three kinds of graphs embedded in $\sphere$ whose vertices have degrees $[1,\ldots,1,k+1,k+2]$ and whose complementary regions are two disks. We label the vertices of degrees $k+1$, $k+2$ with the letters $v$, $u$ respectively.
\begin{dessintext}
\makecell[c]{
\tikzsetnextfilename{exceptionality-sphere-3-1}
\begin{tikzpicture}[graph picture,x={(4,0)},y={(0,4)}]
\path[use as bounding box] (-.48,-.25) rectangle (1.11,.56);
\path (0,0) coordinate (u) pic{white vertex} node[shift={(-105:1.2em)}] {$u$};
\path (0.4,0.3) coordinate (v) pic{white vertex} node[shift={(75:1.2em)}] {$v$};
\path[contour edge={quick={2}{1}}] (u) to[bend left=90,in=80,looseness=2.2] (v);
\path[contour edge={quick={1}{2}}] (u) to[bend right=90,out=-100,looseness=2.2] (v);
\pic[rotate=180] at (u) {bunch of nodes={size=0.6cm,quick={2},first node={\pic{white vertex};},last node={\pic{white vertex};},brace={$k-a$}}};
\pic at (u) {bunch of nodes={size=0.6cm,quick={1},first node={\pic{white vertex};},last node={\pic{white vertex};},brace={$a$}}};
\pic[rotate=180] at (v) {bunch of nodes={size=0.6cm,quick={1},first node={\pic{white vertex};},last node={\pic{white vertex};},brace={$b$}}};
\pic at (v) {bunch of nodes={size=0.6cm,quick={2},first node={\pic{white vertex};},last node={\pic{white vertex};},brace={$k-1-b$}}};
\end{tikzpicture}
}&$\begin{aligned}
&0\le a\le k\\
&0\le b\le k-1\\
&\card{\partial D_1}=2a+2b+2\neq 2k+1\\
&\card{\partial D_2}=4k-2a-2b\neq 2k+1
\end{aligned}$\\
\makecell[c]{
\tikzsetnextfilename{exceptionality-sphere-3-2}
\begin{tikzpicture}[graph picture,x={(4,0)},y={(0,4)}]
\path (0,0) coordinate (u) pic{white vertex} node[shift={(-65:1.2em)}] {$u$};
\path (0.2,0.3) coordinate (v) pic{white vertex} node[shift={(100:1em)}] {$v$};
\path[contour edge={quick={1}{2}}] (u) arc(0:360:0.2);
\path[contour edge={quick={2}{2}}] (u) to[out=60,in=180] (v);
\pic[rotate=180] at (u) {bunch of nodes={size=0.6cm,quick={1},first node={\pic{white vertex};},last node={\pic{white vertex};},brace={$a$}}};
\pic at (u) {bunch of nodes={size=0.6cm,quick={2},first node={\pic{white vertex};},last node={\pic{white vertex};},brace={$k-1-a$}}};
\pic at (v) {bunch of nodes={size=0.6cm,quick={2},first node={\pic{white vertex};},last node={\pic{white vertex};},brace={$k$}}};
\end{tikzpicture}
}&$\begin{aligned}
&0\le a\le k-1\\
&\card{\partial D_1}=2a+1\neq 2k+1\\
&\card{\partial D_2}=4k-2a+1\neq 2k+1
\end{aligned}$\\
\makecell[c]{
\tikzsetnextfilename{exceptionality-sphere-3-3}
\begin{tikzpicture}[graph picture,x={(4,0)},y={(0,4)}]
\path (0,0) coordinate (u) pic{white vertex} node[shift={(-65:1.2em)}] {$v$};
\path (0.2,0.3) coordinate (v) pic{white vertex} node[shift={(100:1em)}] {$u$};
\path[contour edge={quick={1}{2}}] (u) arc(0:360:0.2);
\path[contour edge={quick={2}{2}}] (u) to[out=60,in=180] (v);
\pic[rotate=180] at (u) {bunch of nodes={size=0.6cm,quick={1},first node={\pic{white vertex};},last node={\pic{white vertex};},brace={$b$}}};
\pic at (u) {bunch of nodes={size=0.6cm,quick={2},first node={\pic{white vertex};},last node={\pic{white vertex};},brace={$k-2-b$}}};
\pic at (v) {bunch of nodes={size=0.6cm,quick={2},first node={\pic{white vertex};},last node={\pic{white vertex};},brace={$k+1$}}};
\end{tikzpicture}
}&$\begin{aligned}
&0\le b\le k-2\\
&\card{\partial D_1}=2b+1\neq 2k+1\\
&\card{\partial D_2}=4k-2b+1\neq 2k+1
\end{aligned}$
\end{dessintext}
In all three cases, we have that $\card{\partial D_1}\neq 2k+1$, therefore $\DD$ is exceptional.
\item There are only two kinds of graphs embedded in $\sphere$ whose vertices have degrees $[1,\ldots,1,k+1,k+1]$ and whose complementary regions are two disks. We label the two vertices of degree $k+1$ with the letters $u$ and $v$.
\begin{dessintext}
\makecell[c]{
\tikzsetnextfilename{exceptionality-sphere-4-1}
\begin{tikzpicture}[graph picture,x={(4,0)},y={(0,4)}]
\path[use as bounding box] (-.63,-.25) rectangle (1.11,.56);
\path (0,0) coordinate (u) pic{white vertex} node[shift={(-105:1.2em)}] {$u$};
\path (0.4,0.3) coordinate (v) pic{white vertex} node[shift={(75:1.2em)}] {$v$};
\path[contour edge={quick={2}{1}}] (u) to[bend left=90,in=80,looseness=2.2] (v);
\path[contour edge={quick={1}{2}}] (u) to[bend right=90,out=-100,looseness=2.2] (v);
\pic[rotate=180] at (u) {bunch of nodes={size=0.6cm,quick={2},first node={\pic{white vertex};},last node={\pic{white vertex};},brace={$k-1-a$}}};
\pic at (u) {bunch of nodes={size=0.6cm,quick={1},first node={\pic{white vertex};},last node={\pic{white vertex};},brace={$a$}}};
\pic[rotate=180] at (v) {bunch of nodes={size=0.6cm,quick={1},first node={\pic{white vertex};},last node={\pic{white vertex};},brace={$b$}}};
\pic at (v) {bunch of nodes={size=0.6cm,quick={2},first node={\pic{white vertex};},last node={\pic{white vertex};},brace={$k-1-b$}}};
\end{tikzpicture}
}&$\begin{aligned}
&0\le a\le k-1\\
&0\le b\le k-1\\
&\card{\partial D_1}=2a+2b+2\neq 2k\pm 1\\
&\card{\partial D_2}=4k-2a-2b-2\neq 2k\pm 1
\end{aligned}$\\
\makecell[c]{
\tikzsetnextfilename{exceptionality-sphere-4-2}
\begin{tikzpicture}[graph picture,x={(4,0)},y={(0,4)}]
\path (0,0) coordinate (u) pic{white vertex} node[shift={(-65:1.2em)}] {$u$};
\path (0.2,0.3) coordinate (v) pic{white vertex} node[shift={(100:1em)}] {$v$};
\path[contour edge={quick={1}{2}}] (u) arc(0:360:0.2);
\path[contour edge={quick={2}{2}}] (u) to[out=60,in=180] (v);
\pic[rotate=180] at (u) {bunch of nodes={size=0.6cm,quick={1},first node={\pic{white vertex};},last node={\pic{white vertex};},brace={$a$}}};
\pic at (u) {bunch of nodes={size=0.6cm,quick={2},first node={\pic{white vertex};},last node={\pic{white vertex};},brace={$k-2-a$}}};
\pic at (v) {bunch of nodes={size=0.6cm,quick={2},first node={\pic{white vertex};},last node={\pic{white vertex};},brace={$k$}}};
\end{tikzpicture}
}&$\begin{aligned}
&0\le a\le k-2\\
&\card{\partial D_1}=2a+1\neq 2k\pm 1\\
&\card{\partial D_2}=4k-2a-1\neq 2k\pm 1
\end{aligned}$
\end{dessintext}
In both cases, we have that $\card{\partial D_1}\neq 2k\pm 1$, therefore $\DD$ is exceptional.\qedhere
\end{enumarabic}
\end{proof}
\egroup

\begin{proposition}\label{short-partition:th:exceptional-torus}
Let $k\ge 5$ be an integer. Then the candidate datum
\[
\DD=\datum{\surf{1}}{2k}{[2,\ldots,2],[2,\ldots,2,3,5],[k,k]}
\]
is exceptional.
\end{proposition}
\begin{proof}
We start by enumerating all the abstract graphs whose vertices have degrees $[2,\ldots,2,3,5]$, without considering the embedding in $\surf{1}$. Once again, we observe that vertices of degree $2$ do not contribute to the topology of the graph, so we can reduce the number of cases to two.

\def\myfirstradius{1.2}
\def\mysecondradius{1.8}
\def\drawtwovertices{
\path (0,0) coordinate (w1) pic{white vertex};
\path (\myradius,0) coordinate (w2) pic{white vertex};
}
\begin{center}
\setlength\tabcolsep{3em}
\tikzset{
quick/.style={graph edge={below}{black edge}}
}
\begin{tabular}{@{}cc@{}}
(1)&(2)\\
\tikzsetnextfilename{exceptionality-torus-first-graph}
\begin{tikzpicture}[baseline=0pt,graph picture]
\def\myradius{\myfirstradius}
\drawtwovertices
\path[quick] (w1) -- (w2);
\path[quick,every to/.style={distance=1.5*\myradius cm}] (w1) to[out=135,in=-135] (w1) (w2) to[out=120,in=30] (w2) to[out=-120,in=-30] (w2);
\begin{scope}[x={(\myradius,0)},y={(0,\myradius)}]
\node at (-0.5,0.5) {$a$};
\node at (0.4,0.2) {$b$};
\node at (1.7,0.7) {$c$};
\node at (1.7,-0.7) {$e$};
\end{scope}
\end{tikzpicture}&
\tikzsetnextfilename{exceptionality-torus-second-graph}
\begin{tikzpicture}[baseline=0pt,graph picture]
\def\myradius{\mysecondradius}
\drawtwovertices
\path[quick] (w1) -- (w2);
\path[quick] (w1) arc(180:0:{0.5*\myradius});
\path[quick] (w1) arc(-180:0:{0.5*\myradius});
\path[quick] (w2) to[out=45,in=-45,distance=\myradius cm] (w2);
\begin{scope}[x={(\myradius,0)},y={(0,\myradius)}]
\node at (1.5,0.3) {$a$};
\node at (0.3,0.6) {$b$};
\node at (0.5,0.1) {$c$};
\node at (0.3,-0.6) {$e$};
\end{scope}
\end{tikzpicture}
\end{tabular}
\end{center}

In the pictures above, the arcs are labeled with a letter representing their length (in terms of edges) or, in other words, the number of degree-$2$ vertices that lie on them plus $1$. In particular, $a+b+c+e=k$. We now analyze the two cases separately.
\NewDocumentEnvironment{dessintext}{}{\begin{longtable}{@{}p{0.5\linewidth}@{}p{0.5\linewidth}@{}}}{\end{longtable}}
\begin{enumarabic}
\tikzset{
contour edge settings/quick/.style 2 args={left=disk #1 boundary,right=disk #2 boundary,edge=black edge}
}
\item There is only one embedding of the first graph in $\surf{1}$ whose complementary regions are two discs.
\begin{dessintext}
\makecell[c]{
\tikzsetnextfilename{exceptionality-torus-first-graph-1}
\begin{tikzpicture}[graph picture,every to/.style={distance=1.5*\myradius cm}]
\def\myradius{\myfirstradius}
\drawtwovertices
\begin{scope}[x={(\myradius,0)},y={(0,\myradius)}]
\node at (-0.6,0.6) {$a$};
\node at (0.5,0.3) {$b$};
\node at (1.6,0.7) {$c$};
\node at (1.6,-0.7) {$e$};
\path[use as bounding box] (-1,0) rectangle (2,0);
\end{scope}
\path[use as bounding box];
\drawtwovertices
\path[contour edge={quick={2}{2}}] (w1) -- (w2);
\path[contour edge={quick={1}{2}}] (w1) to[out=135,in=-135] (w1);
\path[contour edge={quick={2}{2}}] (w2) to[out=70,in=-30] (w2);
\path[contour edge={quick={2}{2},above}] (w2) to[out=-70,in=30] (w2);
\end{tikzpicture}}&$\begin{aligned}
&a,b,c,e\ge 1\\
&a+b+c+e=k\\
&\card{\partial D_1}=a<k
\end{aligned}$
\end{dessintext}
\item There are three topologically different embedding of the second graph in $\surf{1}$ whose complementary regions are two disks.
\begin{dessintext}
\makecell[c]{
\tikzsetnextfilename{exceptionality-torus-second-graph-1}
\begin{tikzpicture}[graph picture]
\def\myradius{\mysecondradius}
\drawtwovertices
\begin{scope}[x={(\myradius,0)},y={(0,\myradius)}]
\node at (1.5,0.3) {$a$};
\node at (0.2,0.45) {$b$};
\node at (0.8,0.45) {$c$};
\node at (0.3,-0.6) {$e$};
\path[use as bounding box] (-0.1,0) rectangle (1.7,0);
\end{scope}
\path[use as bounding box];
\path[contour edge={above,quick={2}{2},left bcap=90}] (w1) to[out=60,in=-150,looseness=2.5] (w2);
\path[contour edge={quick={2}{2}}] (w1) to[out=-30,in=120,looseness=2.5] (w2);
\path[contour edge={quick={2}{2}}] (w1) arc(-180:0:{0.5*\myradius});
\path[contour edge={quick={2}{1}}] (w2) to[out=45,in=-45,distance=\myradius cm] (w2);
\end{tikzpicture}}&$\begin{aligned}
&a,b,c,e\ge 1\\
&a+b+c+e=k\\
&\card{\partial D_1}=a<k
\end{aligned}$\\
\makecell[c]{
\tikzsetnextfilename{exceptionality-torus-second-graph-2}
\begin{tikzpicture}[graph picture]
\def\myradius{\mysecondradius}
\drawtwovertices
\begin{scope}[x={(\myradius,0)},y={(0,\myradius)}]
\node at (1.5,0.3) {$a$};
\node at (0.3,0.65) {$b$};
\node at (0.5,0.15) {$c$};
\node at (0.3,-0.6) {$e$};
\path[use as bounding box] (-0.1,0) rectangle (1.7,0);
\end{scope}
\path[use as bounding box];
\path[contour edge={above,quick={1}{2},left bcap=90}] (w1) to (w2);
\path[contour edge={quick={2}{1}}] (w1) arc(180:0:{0.5*\myradius});
\path[contour edge={above,quick={2}{2}}] (w1) to[out=-90,in=-30,out looseness=2,in looseness=2.5] (w2);
\path[contour edge={quick={2}{2}}] (w2) to[out=30,in=-90,distance=1.5*\myradius cm] (w2);
\end{tikzpicture}
}&$\begin{aligned}
&a,b,c,e\ge 1\\
&a+b+c+e=k\\
&\card{\partial D_1}=b+c<k
\end{aligned}$\\
\makecell[c]{
\tikzsetnextfilename{exceptionality-torus-second-graph-3}
\begin{tikzpicture}[graph picture]
\def\myradius{\mysecondradius}
\drawtwovertices
\begin{scope}[x={(\myradius,0)},y={(0,\myradius)}]
\node at (1.5,0.3) {$a$};
\node at (0.2,0.45) {$b$};
\node at (0.8,0.45) {$c$};
\node at (0.3,-0.6) {$e$};
\path[use as bounding box] (-0.1,0) rectangle (1.7,0);
\end{scope}
\path[use as bounding box];
\path[contour edge={above,quick={2}{1},left bcap=90}] (w1) to[out=60,in=-150,looseness=2.5] (w2);
\path[contour edge={quick={1}{2}}] (w1) to[out=-30,in=120,looseness=2.5] (w2);
\path[contour edge={above,quick={2}{2}}] (w1) to[out=-90,in=-30,out looseness=2,in looseness=2.5] (w2);
\path[contour edge={quick={1}{2}}] (w2) to[out=30,in=-90,distance=1.5*\myradius cm] (w2);
\end{tikzpicture}
}&$\begin{aligned}
&a,b,c,e\ge 1\\
&a+b+c+e=k\\
&\card{\partial D_1}=a+b+c<k
\end{aligned}$
\end{dessintext}
\end{enumarabic}

In all of the cases, we see that $\card{\partial D_1}\neq k$; if follows that there is no \dessin{} $\Gamma\subs\surf{1}$ such that $\DD(\Gamma)=\DD$.
\end{proof}


\section{Final result}

By combining all the results in this chapter, we are finally able to compile the full list of exceptional data with a partition of length $2$.

\begin{solution-hurwitz*}
Let
\[
\DD=\datum{\surf{g}}{d}{\pi_1,\ldots,\pi_{n-1},[s,d-s]}
\]
be a candidate datum with $n\ge 3$. Then $\DD$ is exceptional if and only if one of the following holds.
\begin{enumarabic}
\item $\DD=\datum{\sphere{}}{12}{[2,2,2,2,2,2],[1,1,1,3,3,3],[6,6]}$.
\item $\DD=\datum{\sphere{}}{2k}{[2,\ldots,2],[2,\ldots,2],[s,2k-s]}$ with $k\ge 2$, $s\neq k$.
\item $\DD=\datum{\sphere{}}{2k}{[2,\ldots,2],[1,2,\ldots,2,3],[k,k]}$ with $k\ge2$.
\item $\DD=\datum{\sphere{}}{4k+2}{[2,\ldots,2],[1,\ldots,1,k+1,k+2],[2k+1,2k+1]}$ with $k\ge 1$.
\item $\DD=\datum{\sphere{}}{4k}{[2,\ldots,2],[1,\ldots,1,k+1,k+1],[2k-1,2k+1]}$ with $k\ge2$.
\item $\DD=\datum{\sphere{}}{kh}{[h,\ldots,h],[1,\ldots,1,k+1],[lh,(k-l)h]}$ with $h\ge 2$, $k\ge 2$, $1\le l\le k-1$.
\item $\DD=\datum{\surf{1}}{6}{[3,3],[3,3],[2,4]}$.
\item $\DD=\datum{\surf{1}}{8}{[2,2,2,2],[4,4],[3,5]}$.
\item $\DD=\datum{\surf{1}}{12}{[2,2,2,2,2,2],[3,3,3,3],[5,7]}$.
\item $\DD=\datum{\surf{1}}{16}{[2,2,2,2,2,2,2,2],[1,3,3,3,3,3],[8,8]}$.
\item $\DD=\datum{\surf{1}}{2k}{[2,\ldots,2],[2,\ldots,2,3,5],[k,k]}$ with $k\ge 5$.
\item $\DD=\datum{\surf{2}}{8}{[2,2,2,2],[2,2,2,2],[2,2,2,2],[3,5]}$.
\item $\DD=\datum{\surf{n-3}}{4}{[2,2],\ldots,[2,2],[1,3]}$.
\end{enumarabic}
\end{solution-hurwitz*}

\begin{appendices}
\crefalias{chapter}{appendix}
\chapter{Computational results}\label{computational-results:ch}
\bgroup
\def\mytablewidth{0.8\textwidth}
\def\tt{\mathbf{t}}

\section{Zheng's formula}

This chapter is devoted to the presentation of a computational approach to the Hurwitz existence problem, which was first introduced by \citeauthor{zheng} in \cite{zheng}, and to the discussion of some experimental results we have obtained by exploiting it. In this short section, we will merely report the formula used for the computation, without any explanation. For an in-depth derivation of said formula, we refer the reader to \citeauthor{zheng}'s article.

We will require some basic notions about the representation theory of $\symgroup[d]$; all the material we will need is extensively covered in \cite{sagan}. Let $d$ be a positive integer. Given a partition $\pi\in\Partitions{d}$, we denote by $\rho^\pi$, $\zeta^\pi$ and $C_\pi$ the irreducible representation of $\symgroup[d]$, the character and the conjugacy class of $\symgroup[d]$ associated to $\pi$.

A \emph{secondary partition} of $d$ is an unordered multiset $\mu=[\pi_1,\ldots,\pi_k]$, where $\pi_1,\ldots,\pi_k$ are partitions such that $\sum\pi_1+\ldots+\sum\pi_k=d$. We denote the set of all secondary partitions of $d$ by $\SPartitions{d}$.

Fix a positive integer $n$. We will work in the polynomial ring $\QQ[\mathcal{T}]$, where $\mathcal{T}$ is the infinite set of variables $\mathcal{T}=\{t_{i,j}\colon 1\le i\le n,j\ge 1\}$. Given an integer $1\le i\le n$ and a partition $\pi=[a_1,\ldots,a_p]\in\Partitions{d}$, we define $\tt_i^\pi=t_{i,a_1}\cdots t_{i,a_p}$.

Consider now a secondary partition $\mu=[\pi_1,\ldots,\pi_k]\in\SPartitions{d}$; let $d_i=\sum\pi_i$ for $1\le i\le k$, and let $m_1,\ldots,m_l$ be the multiplicities of the elements of $[\pi_1,\ldots,\pi_k]$ (in particular, $m_1+\ldots+m_l=k$). We define:
\begin{itemize}
\item the rational number $\displaystyle r(\mu)=(-1)^{k-1}\frac{(k-1)!}{m_1!\cdots m_l!}\prod_{j=1}^k\left(\frac{\dim\rho^{\pi_j}}{d_j!}\right)^2$;
\item the polynomials $\displaystyle s(\mu)(t_{i,1},\ldots,t_{i,d})=\prod_{j=1}^k\sum_{\nu\in\Partitions{d_j}}\frac{\zeta^{\pi_j}(C_\nu)\cdot\card{C_\nu}}{\dim\rho^{\pi_j}}\cdot\tt_j^\nu$ for $1\le i\le n$.
\end{itemize}

Finally, for fixed positive integers $n$ and $d$, we define the polynomial
\[
P_{n,d}(t_{1,1},\ldots,t_{1,d},t_{2,1},\ldots,t_{n,d})=\sum_{\mu\in\SPartitions{d}}r(\mu)\prod_{i=1}^ks(\mu)(t_{i,1},\ldots,t_{i,d}).
\]
As shown in \cite{zheng}, given a candidate datum $\DD=\datum{\surf{g}}{d}{\pi_1,\ldots,\pi_n}$, the coefficient of the monomial $\tt_1^{\pi_1}\cdots\tt_n^{\pi_n}$ in $P_{n,d}$ is non-zero if and only if $\DD$ is realizable. This property gives a computationally efficient method for finding all the exceptional data for given values of $n$ and $d$.

\section{Implementation}\label{computational-results:sc:implementation}
We wrote a \Cpp{} program for efficiently listing the exceptional data using Zheng's formula. Here are a few implementation details.
\begin{itemize}
\item For a given partition $\pi=[a_1,\ldots,a_p]\in\Partitions{d}$, computing $\card{C_\pi}$ is very easy: we have the formula
\[
\card{C_\pi}=\frac{d!}{a_1!\cdots a_p!}.
\]
\item The characters $\zeta^\pi$ can be recursively computed by means of the Murnaghan–Nakayama formula (see \resultcite{theorem}{4.10.2}{sagan}).
\item In order to compute the dimension of $\rho^\pi$, we can exploit the fact that $\dim\rho^\pi=\zeta^\pi(\id)$, since we need the characters anyway.
\item Instead of carrying out the exact computation in $\QQ$, which is quite expensive, we can work in a finite field $\FF_q$ for some prime number $q$. If a candidate datum is realizable according the the reduced computation in $\FF_q$, then it is realizable; if it is exceptional for sufficiently many primes $q$, then it is exceptional.
\item The computation of $P_{n,d}$ is very well-suited to parallelization, since it involves a summation whose terms can be evaluated independently.
\end{itemize}

\section{Results}

To the best of the author's knowledge, the only available computational results date back to \citeyear{zheng}, when \citeauthor{zheng} managed to compute the polynomials $P_{n,d}$ for $n=3$ and $d\le 20$. Thanks to the resources kindly provided by the University of Pisa (and the undeniable advancement in hardware technology in the past 15 years), we were able to carry out the computation up to $d=29$ which, very conveniently, is a prime number. The results are displayed in the following table (for obvious reasons, we cannot include the full list).

\bgroup
\setlength{\tabcolsep}{1.5em}
\pgfplotstabletypeset[
column type=,
begin table={\begin{tabularx}{\textwidth}{@{}S[table-format=2]S[table-format=5]S[table-format=1]S[table-format=5]@{}}},
end table={\end{tabularx}},
string type,
every head row/.style={output empty row,before row={\toprule\multicolumn{4}{c}{Number of exceptional data with $n=3$, $d\le 29$}\\[0.25em]
\multicolumn{1}{c}{$d$}&\multicolumn{1}{c}{$\tSigma=\sphere{}$}&\multicolumn{1}{c}{$\tSigma=\surf{1}$}&\multicolumn{1}{c}{Total}\\},after row={\midrule}},
every last row/.style={after row={\bottomrule}},
empty cells with={\textendash},
]{table-n3.txt}
\egroup

\begin{remark}
For the sake of efficiency, as explained in the previous section, the computation was carried out in the finite fields $\FF_{q_1}$ and $\FF_{q_2}$, where $q_1=\numprint{1000000007}$ and $q_2=\numprint{1000000009}$. As a consequence, the numbers reported in the table are merely an upper bound, rather than exact values. However, heuristic arguments suggests that, at least for $d\le 29$, working in these two finite fields is enough to avoid any false positives.
\end{remark}

A few considerations are in order. First of all, one of the most prominent goals of the computer-assisted approach to the Hurwitz existence problem is providing experimental evidence supporting the \hyperref[prime-degree-conjecture]{prime-degree conjecture}. Thanks to \cref{monodromy:th:reduction-to-n-3}, the verification can be reduced to the $n=3$ case. Therefore, our results prove that the conjecture holds for prime numbers up to $29$ (recall that the algorithm is able to prove realizability even when working in finite fields).

Moreover, the results reveal that most of the exceptional data have $\tSigma=\sphere{}$; there are a handful with $\tSigma=\surf{1}$, and none with $\tSigma=\surf{g}$ for $g\ge 2$. This curious pattern calls for a more in-depth investigation: the following tables show the results we have obtained by running \citeauthor{zheng}'s algorithm for $n=4$ and $n=5$.

\bgroup
\setlength{\tabcolsep}{1.5em}
\pgfplotstabletypeset[
column type=,
begin table={\begin{tabularx}{\textwidth}{@{}S[table-format=2]S[table-format=4]S[table-format=1]S[table-format=1]S[table-format=4]@{}}},
end table={\end{tabularx}},
string type,
every head row/.style={output empty row,before row={\toprule\multicolumn{5}{c}{Number of exceptional data with $n=4$, $d\le 20$}\\[0.25em]
\multicolumn{1}{c}{$d$}&\multicolumn{1}{c}{$\tSigma=\sphere{}$}&\multicolumn{1}{c}{$\tSigma=\surf{1}$}&\multicolumn{1}{c}{$\tSigma=\surf{2}$}&\multicolumn{1}{c}{Total}\\},after row={\midrule}},
every last row/.style={after row={\bottomrule}},
empty cells with={\textendash},
]{table-n4.txt}

\pgfplotstabletypeset[
column type=,
begin table={\begin{tabularx}{\textwidth}{@{}S[table-format=2]S[table-format=3]S[table-format=1]S[table-format=1]S[table-format=3]@{}}},
end table={\end{tabularx}},
string type,
every head row/.style={output empty row,before row={\toprule\multicolumn{5}{c}{Number of exceptional data with $n=5$, $d\le 15$}\\[0.25em]
\multicolumn{1}{c}{$d$}&\multicolumn{1}{c}{$\tSigma=\sphere{}$}&\multicolumn{1}{c}{$\tSigma=\surf{1}$}&\multicolumn{1}{c}{$\tSigma=\surf{2}$}&\multicolumn{1}{c}{Total}\\},after row={\midrule}},
every last row/.style={after row={\bottomrule}},
empty cells with={\textendash},
]{table-n5.txt}
\egroup

From the experimental results it looks like, even for $n\ge 4$, the majority of the exceptional data occurs when $\tSigma=\sphere$. We note that, unlike in the $n=3$ case, when $n\ge 4$ we also find a few exceptional data on surfaces of genus $g\ge 2$; however,
all of them are covered by the following families:
\begin{enumarabic}
\item $\DD=\datum{\surf{2}}{16}{[2,\ldots,2],[2,\ldots,2],[2,\ldots,2],[1,3,3,3,3,3]}$;
\item $\DD=\datum{\surf{2}}{2k}{[2,\ldots,2],[2,\ldots,2],[2,\ldots,2],[2,\ldots,2,3,5]}$ with $k\ge 4$;
\item $\DD=\datum{\surf{n-3}}{4}{[2,2],\ldots,[2,2],[1,3]}$ with $n\ge 5$.
\end{enumarabic}
This observation leads to the following conjecture, which was already proposed in \cite{zheng}.
\begin{conjecture*}[Zheng]
Let $\DD=\datum{\surf{g}}{d}{\pi_1,\ldots,\pi_n}$ be a candidate datum with $g\ge 2$. Then $\DD$ is realizable unless it belongs to one of the aforementioned families.
\end{conjecture*}
According to \citeauthor{zheng}, the statement can be reduced to the $n=3$ case, just like the prime-degree conjecture. Therefore, our computation shows that Zheng's conjecture holds at least for $d\le 29$.

\section{Lists of exceptional data with a short partition}\label{computational-results:sc:lists}

We conclude this appendix by enumerating all the exceptional data $\DD=\datum{\surf{g}}{d}{\pi_1,\ldots,\pi_n}$ with $\len{\pi_n}=2$ for small values of $n$ and $d$. These lists, which were computed by our \Cpp{} program, were used in the proofs of \cref{short-partition:th:realizability-on-torus-n-3,short-partition:th:realizability-on-higher-genus-n-3,short-partition:th:realizability-on-higher-genus-n-ge-4} to dramatically reduce the number of cases we had to handle. In the unlikely (but not impossible) event that running the algorithm in finite fields yields some false positives, the results in this section were double-checked by carrying out the exact computation in the field $\QQ$ of rational numbers.

\begin{tabularx}{\mytablewidth}{l}
\toprule
\multicolumn{1}{l}{Exceptional data with $n=3$, $\len{\pi_3}=2$, $\tSigma=\sphere{}$ and $d\le 12$}\\
\midrule
$\datum{\sphere}{4}{[2, 2],[2, 2],[1, 3]}$\\
$\datum{\sphere}{6}{[2, 2, 2],[2, 2, 2],[2, 4]}$\\
$\datum{\sphere}{6}{[2, 2, 2],[2, 2, 2],[1, 5]}$\\
$\datum{\sphere}{6}{[2, 2, 2],[1, 2, 3],[3, 3]}$\\
$\datum{\sphere}{6}{[2, 2, 2],[1, 1, 4],[2, 4]}$\\
$\datum{\sphere}{6}{[1, 1, 1, 3],[3, 3],[3, 3]}$\\
$\datum{\sphere}{8}{[2, 2, 2, 2],[2, 2, 2, 2],[3, 5]}$\\
$\datum{\sphere}{8}{[2, 2, 2, 2],[2, 2, 2, 2],[2, 6]}$\\
$\datum{\sphere}{8}{[2, 2, 2, 2],[2, 2, 2, 2],[1, 7]}$\\
$\datum{\sphere}{8}{[2, 2, 2, 2],[1, 2, 2, 3],[4, 4]}$\\
$\datum{\sphere}{8}{[2, 2, 2, 2],[1, 1, 3, 3],[3, 5]}$\\
$\datum{\sphere}{8}{[2, 2, 2, 2],[1, 1, 1, 5],[4, 4]}$\\
$\datum{\sphere}{8}{[2, 2, 2, 2],[1, 1, 1, 5],[2, 6]}$\\
$\datum{\sphere}{8}{[1, 1, 1, 1, 1, 3],[4, 4],[4, 4]}$\\
$\datum{\sphere}{9}{[1, 1, 1, 1, 1, 4],[3, 3, 3],[3, 6]}$\\
$\datum{\sphere}{10}{[2, 2, 2, 2, 2],[2, 2, 2, 2, 2],[4, 6]}$\\
$\datum{\sphere}{10}{[2, 2, 2, 2, 2],[2, 2, 2, 2, 2],[3, 7]}$\\
$\datum{\sphere}{10}{[2, 2, 2, 2, 2],[2, 2, 2, 2, 2],[2, 8]}$\\
$\datum{\sphere}{10}{[2, 2, 2, 2, 2],[2, 2, 2, 2, 2],[1, 9]}$\\
$\datum{\sphere}{10}{[2, 2, 2, 2, 2],[1, 2, 2, 2, 3],[5, 5]}$\\
$\datum{\sphere}{10}{[2, 2, 2, 2, 2],[1, 1, 1, 3, 4],[5, 5]}$\\
$\datum{\sphere}{10}{[2, 2, 2, 2, 2],[1, 1, 1, 1, 6],[4, 6]}$\\
$\datum{\sphere}{10}{[2, 2, 2, 2, 2],[1, 1, 1, 1, 6],[2, 8]}$\\
$\datum{\sphere}{10}{[1, 1, 1, 1, 1, 1, 1, 3],[5, 5],[5, 5]}$\\
$\datum{\sphere}{12}{[2, 2, 2, 2, 2, 2],[2, 2, 2, 2, 2, 2],[5, 7]}$\\
$\datum{\sphere}{12}{[2, 2, 2, 2, 2, 2],[2, 2, 2, 2, 2, 2],[4, 8]}$\\
$\datum{\sphere}{12}{[2, 2, 2, 2, 2, 2],[2, 2, 2, 2, 2, 2],[3, 9]}$\\
$\datum{\sphere}{12}{[2, 2, 2, 2, 2, 2],[2, 2, 2, 2, 2, 2],[2, 10]}$\\
$\datum{\sphere}{12}{[2, 2, 2, 2, 2, 2],[2, 2, 2, 2, 2, 2],[1, 11]}$\\
$\datum{\sphere}{12}{[2, 2, 2, 2, 2, 2],[1, 2, 2, 2, 2, 3],[6, 6]}$\\
$\datum{\sphere}{12}{[2, 2, 2, 2, 2, 2],[1, 1, 1, 3, 3, 3],[6, 6]}$\\
$\datum{\sphere}{12}{[2, 2, 2, 2, 2, 2],[1, 1, 1, 1, 4, 4],[5, 7]}$\\
$\datum{\sphere}{12}{[2, 2, 2, 2, 2, 2],[1, 1, 1, 1, 1, 7],[6, 6]}$\\
$\datum{\sphere}{12}{[2, 2, 2, 2, 2, 2],[1, 1, 1, 1, 1, 7],[4, 8]}$\\
$\datum{\sphere}{12}{[2, 2, 2, 2, 2, 2],[1, 1, 1, 1, 1, 7],[2, 10]}$\\
$\datum{\sphere}{12}{[1, 1, 1, 1, 1, 1, 1, 1, 1, 3],[6, 6],[6, 6]}$\\
$\datum{\sphere}{12}{[1, 1, 1, 1, 1, 1, 1, 5],[3, 3, 3, 3],[6, 6]}$\\
$\datum{\sphere}{12}{[1, 1, 1, 1, 1, 1, 1, 5],[3, 3, 3, 3],[3, 9]}$\\
$\datum{\sphere}{12}{[1, 1, 1, 1, 1, 1, 1, 1, 4],[4, 4, 4],[4, 8]}$\\
\bottomrule\\[1em]
\toprule
\multicolumn{1}{l}{Exceptional data with $n=3$, $\len{\pi_3}=2$, $g\ge 1$ and $d\le 20$}\\
\midrule
$\datum{\surf{1}}{6}{[3, 3],[3, 3],[2, 4]}$\\
$\datum{\surf{1}}{8}{[2, 2, 2, 2],[4, 4],[3, 5]}$\\
$\datum{\surf{1}}{10}{[2, 2, 2, 2, 2],[2, 3, 5],[5, 5]}$\\
$\datum{\surf{1}}{12}{[2, 2, 2, 2, 2, 2],[3, 3, 3, 3],[5, 7]}$\\
$\datum{\surf{1}}{12}{[2, 2, 2, 2, 2, 2],[2, 2, 3, 5],[6, 6]}$\\
$\datum{\surf{1}}{14}{[2, 2, 2, 2, 2, 2, 2],[2, 2, 2, 3, 5],[7, 7]}$\\
$\datum{\surf{1}}{16}{[2, 2, 2, 2, 2, 2, 2, 2],[1, 3, 3, 3, 3, 3],[8, 8]}$\\
$\datum{\surf{1}}{16}{[2, 2, 2, 2, 2, 2, 2, 2],[2, 2, 2, 2, 3, 5],[8, 8]}$\\
$\datum{\surf{1}}{18}{[2, 2, 2, 2, 2, 2, 2, 2, 2],[2, 2, 2, 2, 2, 3, 5],[9, 9]}$\\
\bottomrule\\[1em]
\toprule
\multicolumn{1}{l}{Exceptional data with $n=4$, $\len{\pi_4}=2$ and $d\le 16$}\\
\midrule
$\datum{\surf{1}}{4}{[2, 2],[2, 2],[2, 2],[1, 3]}$\\
$\datum{\surf{2}}{8}{[2, 2, 2, 2],[2, 2, 2, 2],[2, 2, 2, 2],[3, 5]}$\\
\bottomrule\\[1em]
\toprule
\multicolumn{1}{l}{Exceptional data with $n=5$, $\len{\pi_5}=2$ and $d\le 8$}\\
\midrule
$\datum{\surf{2}}{4}{[2, 2],[2, 2],[2, 2],[2, 2],[1, 3]}$\\
\bottomrule
\end{tabularx}

\egroup
\end{appendices}

\printbibliography[heading=bibintoc]
\end{document}