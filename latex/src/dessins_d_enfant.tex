\chapter{\texorpdfstring{\Dessins{}}{Dessins d'enfant}}

\section{Child's drawings on surfaces}

In \cref{monodromy:sc:combinatorial-moves}, we discussed how the Hurwitz existence problem can be reduced to the analysis of candidate data on the sphere. Moreover, thanks to \cref{combinatorial-move:a:small-v,combinatorial-move:a:large-v}, we have devised a relatively reliable technique to decrease the number $n$ of partitions; this technique was successfully employed in \cref{monodromy:sc:results-sphere} to show the realizability of a wide variety of candidate data by induction on $n$, starting from the base case $n=3$. Ignoring the cases where $n\le 2$, which were fully analyzed in \cref{monodromy:sc:combinatorial-moves}, it should come as no surprise that candidate data with $n=3$ play a very important role in the study of the existence problem.

Up to this point, we have only approached the Hurwitz existence problem from a group-theoretic point of view, showing realizability by looking for elements of $\symgroup[d]$ with certain properties. In this section, we will present a totally different tool, of a more topological and combinatorial nature, for attacking the same problem. The concept of \emph{\dessins{}}\footnote{``\emph{\Dessin{}}'' is French for ``child's drawing'', hence the title of this section.} was popularized by Grothendieck in \cite{grothendieck}, in a setting related to, but different from, the Hurwitz existence problem. \Dessins{} provide a strikingly elementary tool for showing the realizability of candidate data with $n=3$ partitions, although they generalize quite nicely to the case $n\ge 4$. However, we will not deal with said generalization, since the reduction technique will prove to be sufficient for our purposes; we refer the interested reader to \resultcite{section}{3}{pervova-methods}.

We start by introducing some basic terminology about graphs. Given a surface $\Sigma$, a \emph{graph} embedded in $\Sigma$ (or, simply, a graph on $\Sigma$) is a closed subspace $\Gamma\subs\Sigma$ consisting of:
\begin{itemize}
\item a finite number of points $x_1,\ldots,x_r\in\Sigma$, called \emph{vertices};
\item a finite number of segments (subspaces homeomorphic to $[0,1]$) $e_1,\ldots,e_d\subs\Sigma$, called \emph{edges}; we require that each edge connects two (not necessarily distinct) vertices, and that the interiors of two edges are disjoint; in other words, two edges may intersect at most at their endpoints; moreover, a vertex cannot lie on the interior of an edge.
\end{itemize}
The \emph{degree} of a vertex $x$ is the number of edges having $x$ as an endpoint; edges connecting $x$ to itself are counted twice; we denote the degree of $x$ by $k(x)$. In order to avoid unpleasant corner cases, we will always require that there are no \emph{isolated vertices} or, in other words, that $k(x)\ge 1$ for every vertex $x$.

A \emph{bipartite graph} is a graph whose vertices are colored either black or white, and each edge connects a black vertex and a white one. If we denote the black vertices by $x_1,\ldots,x_r$ and the white vertices by $y_1,\ldots,y_s$, an easy counting argument shows that
\[
k(x_1)+\ldots+k(x_r)=k(y_1)+\ldots+k(y_s)=d,
\]
where $d$ is the number of edges.

Given a graph $\Gamma$ on a surface $\Sigma$, the space $\Sigma\setminus\Gamma$ is a disjoint union of a finite number of non-compact surfaces $S_1\sqcup\ldots\sqcup S_h$, called \emph{complementary regions} of $\Gamma$. We are finally ready to give the definition of the much anticipated \dessins{}.

\begin{definition}
Let $\Sigma$ be a surface. A \emph{\dessin{}} on $\Sigma$ is a bipartite graph $\Gamma\subs\Sigma$ whose complementary regions are topological disks.
\end{definition}

\todo{Examples of \dessins{}.}

Let $\Gamma$ be a \dessin{}, and fix one a complementary region $D$. By traveling along its boundary, always keeping $D$ to the left, we get a cyclic sequence of edges of $\Gamma$, which we call \emph{combinatorial boundary} of $D$, and denote by $\partial D$. Note that the same edge $e$ can be traveled along twice, once for each direction; in this case, it will appear twice in $\partial D$, and we will say that $e$ is \emph{enveloped} by $D$. The number of edges (with multiplicity) of $\partial D$ is the \emph{perimeter} of $D$, denoted by $\card{\partial D}$.\todo{The definition is not very formal, examples incoming.} If $D_1,\ldots,D_h$ are the complementary region of $\Gamma$, a counting argument shows that
\[
\card{\partial D_1}+\ldots+\card{\partial D_h}=2d.
\]

It is also easy to see that the perimeter of each complementary region is even: in fact, when traveling along the boundary of a region, we alternately encounter black and white vertices, so an even number of edges is required to get back to the starting color.

Finally, note that every \dessin{} is necessarily connected, otherwise there would be some complementary region with two or more boundary components.

\begin{definition}
Let $\Gamma$ be a \dessin{} on a surface $\Sigma$; let $d$ be the number of edges. Let $x_1,\ldots,x_r$ be the black vertices, $y_1,\ldots,y_s$ the white ones. Denote by $D_1,\ldots,D_h$ the complementary regions of $\Gamma$. The \emph{branching datum} of $\Gamma$ is the tuple
\[
\DD(\Gamma)=\datum{\Sigma,\sphere{}}{d}{[k(x_1),\ldots,k(x_r)],[k(y_1),\ldots,k(y_s)],[\card{\partial D_1}/2,\ldots,\card{\partial D_h}/2]}.
\]
\end{definition}

From the discussion above, we immediately see that $\DD(\Gamma)$ is a combinatorial datum, since
\[
k(x_1)+\ldots+k(x_r)=k(y_1)+\ldots+k(y_s)=\card{\partial D_1}/2+\ldots+\card{\partial D_h}/2=d.
\]
Actually, if $\Sigma$ is orientable, $\DD(\Gamma)$ is a candidate datum: by the Euler formula,
\[
\chi(\Sigma)=r+s-d+h=2d-v(\pi_1)-v(\pi_2)-v(\pi_3),
\]
where $\pi_1=[k(x_1),\ldots,k(x_r)]$, $\pi_2=[k(y_1),\ldots,k(y_s)]$ and $\pi_3=[\card{\partial D_1}/2,\ldots,\card{\partial D_h}/2]$. This is no coincidence, just like the name ``branching datum of $\Gamma$'' was not picked at random: the following result establishes a strong connection between \dessins{} and realizable combinatorial data.

\begin{proposition}\label{dessins:th:dessins-realizability}
Let $\DD=\datum{\surf{g}}{d}{\pi_1,\pi_2,\pi_3}$ be a combinatorial datum. Then $\DD$ is realizable if and only if there exists a \dessin{} $\Gamma\subs\Sigma_g$ with $\DD(\Gamma)=\DD$.
\end{proposition}
\begin{proof}
Assume that $\DD$ is realized by a branched covering $\map{f}{\surf{g}}{\sphere{}}$. Let $\{\wtilde{x}_1,\ldots,\wtilde{x}_r\}=f^{-1}(x)$, $\{\wtilde{y}_1,\ldots,\wtilde{y}_s\}=f^{-1}(y)$, $\{\wtilde{z}_1,\ldots,\wtilde{z}_h\}=f^{-1}(z)$. Fix a segment $e\subs\sphere$ connecting $x$ and $y$ (and avoiding $z$); we claim that $\Gamma=f^{-1}(e)$ is the desired \dessin{}. Let $\interior{e}$ be the interior of $e$ (that is, $\interior{e}=e\setminus\{x,y\}$). First of all, note that $f^{-1}(\interior{e})$ is the disjoint union of $d$ open segments $\interior{e}_1,\ldots,\interior{e}_d$, since the restriction of $f$ to $\sphere{}\setminus\{x,y,z\}$ is a covering map of degree $d$. Moreover, it is easy to see that the closure of each $\interior{e}_i$ is a closed segment $e_i$ connecting one point in $f^{-1}(x)$ and one point of $f^{-1}(y)$; it follows that $\Gamma$ is a bipartite graph on $\surf{g}$, with black vertices $\wtilde{x}_1,\ldots,\wtilde{x}_r$ and white vertices $\wtilde{y}_1,\ldots,\wtilde{y}_s$. Consider a vertex $\wtilde{x}_i$; recall that $f$ is locally modeled on the complex map $\xi\mapsto\xi^k$, where $k=k(\wtilde{x}_i)$ is the local degree of $\wtilde{x}_i$. As a consequence, we immediately see that there are exactly $k(\wtilde{x}_i)$ edges of $\Gamma$ with $\wtilde{x}_i$ as an endpoint; of course, the same holds for every $\wtilde{y}_j$. Finally, we turn to the complementary regions of $\Gamma$. Let $D=\sphere{}\setminus e\iso\RR^2$, $\holed{D}=D\setminus\{z\}\iso\RR^2\setminus\{0\}$. The restriction of $f$ to $f^{-1}(\holed{D})=\surf{g}\setminus(\Gamma\cup\{\wtilde{z}_1,\ldots,\wtilde{z}_h\})$ is a covering map of the punctured disk $\holed{D}$. It is then easy to see that the complementary regions of $\Gamma$ are discs $\wtilde{D}_1,\ldots,\wtilde{D}_h$, with $\wtilde{z}_i\in\wtilde{D}_i$ for each $1\le i\le h$, and that the restriction $\map{f}{\wtilde{D}_i}{D}$ is modeled the complex map $\xi\mapsto\xi^{k(\wtilde{z}_i)}$. Since the perimeter of $D$ is $2$, we have that $\card{\partial\wtilde{D}_i}=2 k(\wtilde{z}_i)$; this concludes the proof of the equality $\DD(f)=\DD(\Gamma)$.

Conversely, assume that we are given a \dessin{} $\Gamma\subs\surf{g}$ with $\DD(\Gamma)=\DD$. Fix three arbitrary points $x,y,z\in\sphere{}$, and let $e\subs\sphere{}$ be a segment connecting $x$ and $y$ (and avoiding $z$). First of all, we define $f$ on $\Gamma$, sending black vertices to $x$ and white vertices to $y$, and mapping edges homeomorphically to $e$. Extending $f$ to all of $\surf{g}$ is a relatively easy task: here are the details. Consider the standard closed disk $K=\{a\in\RR^2:\lVert a\rVert\le 1\}$, and take a complementary region $\wtilde{D}\subs\surf{g}$; let $\map{\phi}{K}{\surf{g}}$ be a continuous map which restricts to a homeomorphism $\map{\phi}{\interior{K}}{\wtilde{D}}$, where $\interior{K}$ denotes the interior of $K$. There exists a map\todo{Picture much needed.} $\map{\psi}{K}{\sphere{}}$ such that $\psi(0)=z$, the diagram
\begin{diagram}
\partial K\dar{\phi}\ar[dr,"\psi"]\\
\Gamma\rar{f}&e
\end{diagram}
commutes, $\psi$ is a local homeomorphism in $\interior{K}\setminus\{0\}$ and it is modeled on $\xi\mapsto\xi^k$ in a neighborhood of $0\in K$; in particular $k$ will necessarily be equal to half the perimeter of $\wtilde{D}$. We can now extend $f$ to $\wtilde{D}$ by setting $f(\wtilde{x})=\psi(\phi^{-1}(\wtilde{x}))$ for every $\wtilde{x}\in\wtilde{D}$. After repeating the process for all the complementary regions, it is not hard to verify that the map $\map{f}{\tSigma}{\sphere{}}$ we have obtained is a branched covering with branching points $x,y,z\in\sphere{}$. Since $\Gamma=f^{-1}(e)$, the first\todo{Namely, $\Rightarrow$.} part of the proof implies that $\DD(\Gamma)=\DD(f)$.
\end{proof}

\section{Unwinding, joining and fattening}

In the next section we will introduce a new kind of combinatorial moves, which operate on \dessins{} rather than permutations. In this context, the importance of visual intuition cannot be overstated. Therefore, we will now spend some time describing in detail two operations that will play a major role in the topological explanation of the upcoming combinatorial moves.

\paragraph{Unwinding the boundary.} Let $\Gamma$ be a graph on a surface $\Sigma$. Take a complementary region $D$, and assume $D$ is a topological disk. Intuitively, when we \emph{unwind the boundary} of $D$, we represent $D$ as the standard closed disk $K$ embedded in $\RR^2$; the edges of the combinatorial boundary of $D$ are placed sequentially on the topological boundary of $K$, possibly with repetitions. For a more formal description, we can follow the strategy presented in the second part of the proof of \cref{dessins:th:dessins-realizability}: we consider a continuous map $\map{\phi}{K}{\surf{g}}$ which restricts to a homeomorphism $\map{\phi}{\interior{K}}{D}$; edges on the topological boundary of $D$ can be pulled back by $\phi$, thus unwinding the combinatorial boundary of $D$ on $\partial K$.

\paragraph{Joining vertices along edges.} Let $\Gamma$ be a graph on a surface $\Sigma$. Consider an edge $e$, and let $x$, $y$ be its (distinct) endpoints. \emph{Joining} $x$ and $y$ along $e$ means shrinking $e$ to a single point, so that $x$ and $y$ are merged into a single vertex, say $z$; it is immediate to check that $k(z)=k(x)+k(y)-2$, while the degrees of the other vertices are left unchanged. The topology of the complementary regions does not change either. To be more precise, there is a natural one-to-one correspondence between regions of $\Sigma\setminus\Gamma$ and regions of $\Sigma\setminus\Gamma'$, where $\Gamma'$ is the graph obtained after joining $x$ and $y$ along $e$, and corresponding regions are homeomorphic. The edge $e$ disappears from the combinatorial boundaries, so the perimeters of the two regions touching $e$ decrease by $1$ (if the two regions were actually the same, then the perimeter decreases by $2$); the other perimeters do not change. Of course, the joining operation can be performed along more edges simultaneously, by joining vertices along one edge at a time.

\paragraph{Fattening graphs.} Representing \dessins{} on the sphere is easy, since graphs on $\sphere{}$ naturally embed in $\RR^2$; unfortunately, this is not the case for surfaces of genus $g\ge 1$. However, for a specific class of graphs (including \dessins{}), there is a trick we can exploit in order to represent them as diagrams on the plane. Let $\Gamma$ be a graph on a surface $\surf{g}$; assume that the complementary regions of $\Gamma$ are disks. Note that the topology of the embedding $\Gamma\subs\surf{g}$ can be completely recovered if we are given $\Gamma$ as an abstract graph, plus a tubular neighborhood of $\Gamma$ in $\surf{g}$; we call such a datum a \emph{fat graph}. A fat graph can be represented as a diagram with a finite number of transverse crossings on the plane; each crossing is equipped with the additional information of which edge goes over and which goes under.

In order to reconstruct the embedding of $\Gamma$ in $\surf{g}$, we simply have to thicken the edges of the diagram, keeping in mind that the two edges involved in a crossing actually go one under the other. This operation yields a fat graph, from which $\surf{g}$ can be recovered by gluing a disk along each boundary component.

We will employ this technique in order to represent \dessins{} (and more generally, graphs whose complementary regions are disks) embedded in higher genus surfaces.

\section{Genus-reducing combinatorial moves}\label{dessins:sc:combinatorial-moves}

As we have already anticipated, the goal of this thesis is a complete classification of the exceptional data with a partition of length $2$. \Cref{combinatorial-move:a:small-v,combinatorial-move:a:large-v} are often able to reduce the existence problem to instances with $n=3$ partitions. We will now introduce a few more combinatorial moves, which heavily exploit the machinery of \dessins{}. Unlike the aforementioned ones, these moves only work under very restrictive assumptions, namely that $n=3$ and $\len{\pi_3}=2$; on the other hand, they allow a much finer control on the partitions involved, and are often versatile enough to reduce an instance of the existence problem to the case where $\tSigma=\sphere{}$.

In this section, we will only be dealing with candidate data of the form $\DD=\datum{\surf{g}}{d}{\pi_1,\pi_2,[s,d-s]}$ with $1\le s\le d-1$. In this setting, the \RH{} formula can simply be written as
\[
\len{\pi_1}+\len{\pi_2}=d-2g.
\]
We will adopt the following conventions:
\begin{itemize}
\item vertices corresponding to the entries of $\pi_1$ (or $\pi_1'$) will be colored black;
\item vertices corresponding to the entries of $\pi_2$ (or $\pi_2'$) will be colored white;
\item unnamed vertices will be labeled with their degrees;
\item the complementary disk associated to the first entry of $\pi_3$ (or $\pi_3'$) will be denoted by $D_1$ and will be colored orange;
\item the complementary disk associated to the second entry of $\pi_3$ (or $\pi_3'$) will be denoted by $D_2$ and will be colored blue.
\end{itemize}

\begin{combinatorialmoveb}\label{combinatorial-move:b:[1 1 3]}
Let $\DD=\datum{\surf{g}}{d}{\pi_1,\pi_2,[s,d-s]}$ be a candidate datum with $g\ge 1$. Assume that $[1,1,3]\subs\pi_1$. Consider the candidate datum
\[
\DD'=\datum{\surf{g-1}}{d}{\pi_1',\pi_2,[s,d-s]},
\]
where $\pi_1'=\pi_1\setminus[3]\cup[1,1,1]$. Then $\DD\cmove\DD'$.
\end{combinatorialmoveb}
\begin{proof}
Assume that $\DD'$ is realizable; by \cref{dessins:th:dessins-realizability}, there exists a \dessin{} $\Gamma'\subs\surf{g-1}$ with $\DD(\Gamma')=\DD'$. Our aim will be to construct a new \dessin{} $\Gamma\subs\surf{g}$ with $\DD(\Gamma)=\DD$; by \cref{dessins:th:dessins-realizability}, this will imply that $\DD$ is realizable as well.

Note that $[1,1,1,1,1]\subs\pi_1'$; therefore, without loss of generality, we can assume that $\Gamma$ has three black vertices of degree $1$ lying on the boundary of the complementary region $D_1$. Let us represent $D_1$ with its boundary unwound , and focus on the three black vertices of degree $1$ (see the picture labeled (0) below). We perform the following operations on $\Gamma'$.
\begin{enumerate}[(1)]
\item Attach a tube to $\surf{g-1}$ with both endpoints in $D_1$; to be more precise, remove two disjoint open disks contained in the interior of $D_1$, and glue a tube $S^1\times[0,1]$ along the two new boundary components. This effectively increases the genus by $1$.
\item Connect the three black vertices with two new edges, as shown in red in the picture; note that the orange complementary region of the new graph is still a disk.
\item Join the three black vertices along the red edges.
\end{enumerate}
\bgroup
\def\picturesetupzero#1{
\pic {cmove setting one disk=1};
\path \surfcirclepoint{d1}{-30} coordinate (1-1);
\path \surfcirclepoint{d1}{-150} coordinate (1-3);
\path \surfcirclepoint{d1}{-90} coordinate (1-2) pic{black vertex};
\ifnumcomp{#1}{=}{1}{\pic at (1-1) {black vertex};\pic at (1-3) {black vertex};\node[below right] at (1-1) {$1$};\node[below=5pt] at (1-2) {$1$};\node[below left] at (1-3) {$1$};}{}
}
\def\picturesetupone#1{
\picturesetupzero{#1}
\pic {cmove setting one disk tube=1};
\tubefill{disk 1};
}
\def\picturesetuptwo#1{
\picturesetupone{#1}
\ifnum#1=1
\tikzset{myedgestyle/.style={surf edge={##1}{red edge}}}\else
\tikzset{myedgestyle/.style={after join={##1}{d1}{white}}}\fi
\path[myedgestyle={behind}] (1-3) to[out=90,in=60,out looseness=3.1,in looseness=2] (1-2);
\path[myedgestyle={front}] let \p1=\tuberightpoint{-60},\p2=\tubeleftpoint{-120},\n1={(\x1-\x2)/2} in (1-1) to[bend left] (\p1) arc(0:180:\n1) to[bend right] (1-2);
}
\def\picturesetupthree{
\picturesetuptwo{0}
\node[below=5pt] at (1-2) {$3$};
}
\tabcolsep=0pt
\begin{longtable}{*{2}{>{\centering\arraybackslash}p{.5\linewidth}}}
(0)&(1)\\*
\tikzsetnextfilename{cmove-1-0}
\begin{tikzpicture}[surf picture]
\picturesetupzero{1}
\end{tikzpicture}
&
\tikzsetnextfilename{cmove-1-1}
\begin{tikzpicture}[surf picture]
\picturesetupone{1}
\end{tikzpicture}
\\
(2)&(3)\\*
\tikzsetnextfilename{cmove-1-2}
\begin{tikzpicture}[surf picture]
\picturesetuptwo{1}
\end{tikzpicture}
&
\tikzsetnextfilename{cmove-1-3}
\begin{tikzpicture}[surf picture]
\picturesetupthree
\end{tikzpicture}
\end{longtable}
\egroup
After these operations, we get a new \dessin{} $\Gamma$ embedded in $\surf{g}$. It is easy to check that $\DD(\Gamma)=\DD$, therefore $\DD$ is realizable, again by \cref{dessins:th:dessins-realizability}.
\end{proof}

In the upcoming proofs, we will often represent complementary disks with their boundaries unwound, without explicitly saying so. The pictures should be clear enough to avoid any ambiguity.

\begin{combinatorialmoveb}\label{combinatorial-move:b:4 2}
Let $\DD=\datum{\surf{g}}{d}{\pi_1,\pi_2,[s,d-s]}$ be a candidate datum with $g\ge 1$. Assume that:
\begin{assumptions}
\item $2\le s\le d-s$;
\item $x\in\pi_1$ for some $x\ge 4$;
\item $2\in\pi_2$.
\end{assumptions}
Let $x_1$, $x_2$ be positive integers whose sum equals $x-2$, and consider the candidate datum
\[
\DD'=\datum{\surf{g-1}}{d-2}{\pi_1',\pi_2',[s-1,d-s-1]},
\]
where $\pi_1'=\pi_1\setminus[x]\cup[x_1,x_2]$ and $\pi_2'=\pi_2\setminus[2]$. Then $\DD\cmove\DD'$.
\end{combinatorialmoveb}
\begin{proof}
Consider a \dessin{} $\Gamma'\subs\surf{g-1}$ realizing $\DD'$. There are two cases.
\paragraph{Case 1:} the black vertex of degree $x_1$ lies on $\partial D_1$ and the one with degree $x_2$ lies on $\partial D_2$ (or vice versa). Then we perform the following operations on $\Gamma'$.
\begin{enumerate}[(1)]
\item Attach a tube to $\surf{g-1}$ with one endpoint in $D_1$ and the other one in $D_2$.
\item Add one black vertex, one white vertex and two edges as shown in the picture.
\item Draw the two red edges shown in the picture.
\item Perform the joining operation along the red edges.
\end{enumerate}
\bgroup
\def\picturesetupone#1{
\pic{cmove setting two disks};
\pic{cmove setting two disks tube};
\tubefill{white};
\path \surfcirclepoint{d1}{-90} coordinate (x1);
\path \surfcirclepoint{d2}{-90} coordinate (x2);
\ifnum#1=0
\path (x1) pic{black vertex} node[below=3pt] {$x_1$};
\path (x2) pic{black vertex} node[below=3pt] {$x_2$};
\fi
}
\def\picturesetuptwo#1{
\picturesetupone{#1}
\tubebelt{black edge}{black edge dashed}
\path \tubemiddlepoint{120} coordinate (b) pic{black vertex};
\path \tubemiddlepoint{240} coordinate (w) pic{white vertex};
\tubeleftfill{disk 1}
\tuberightfill{disk 2}
\ifnum#1=0
\node[above=5pt] at (b) {$2$};
\fi
\node[below=5pt] at (w) {$2$};
}
\def\picturesetupthree#1{
\picturesetuptwo{#1}
\ifnum#1=0
\tikzset{myedgestyle/.style={surf edge={front}{red edge}}}\else
\tikzset{myedgestyle/.style={after join={front}{##1}{white}}}\fi
\path[myedgestyle={d1}] let \p1=\tubeleftpoint{240} in (x1) to[bend left] (\p1) to[out=90,in=180] (b);
\path[myedgestyle={d2}] let \p1=\tuberightpoint{-60} in (x2) to[bend right] (\p1) to[out=90,in=0] (b);
}
\def\picturesetupfour{
\picturesetupthree{1}
\node[above=5pt] at (b) {$x$};
}
\tabcolsep=0pt
\begin{longtable}{*{2}{>{\centering\arraybackslash}p{.5\linewidth}}}
(1)&(2)\vspace{0.5em}\\*
\tikzsetnextfilename{cmove-2-1-1}
\begin{tikzpicture}[surf picture]
\picturesetupone{1}
\end{tikzpicture}
&
\tikzsetnextfilename{cmove-2-1-2}
\begin{tikzpicture}[surf picture]
\picturesetuptwo{1}
\end{tikzpicture}
\\
(3)&(4)\\*
\tikzsetnextfilename{cmove-2-1-3}
\begin{tikzpicture}[surf picture]
\picturesetupthree{1}
\end{tikzpicture}
&
\tikzsetnextfilename{cmove-2-1-4}
\begin{tikzpicture}[surf picture]
\picturesetupfour
\end{tikzpicture}
\end{longtable}
\egroup

\paragraph{Case 2:} the two black vertices of degrees $x_1$ and $x_2$ lie (say) on $\partial D_1$. Fix an edge $e\subs\Gamma'$ which lies on the boundaries of both disks. We perform the following operations on $\Gamma'$.
\begin{enumerate}[(1)]
\item Add one black vertex and one white vertex on $e$.
\item Attach a tube to $\surf{g-1}$ with both endpoints in $D_1$.
\item Draw the two red edges shown in the picture.
\item Perform the joining operation along the red edges.
\end{enumerate}

In both cases, we get a new \dessin{} $\Gamma$ embedded in $\surf{g}$. It is easy to check that $\Gamma$ realizes the candidate datum $\DD$.
\end{proof}

\begin{combinatorialmoveb}\label{combinatorial-move:b:[3 3] [2 2]}
Let $\DD=\datum{\surf{g}}{d}{\pi_1,\pi_2,[s,d-s]}$ be a candidate datum with $g\ge 1$. Assume that:
\begin{assumptions}
\item $3\le s\le d-3$;
\item $[x,y]\subs\pi_1$ for some $x\ge 3$, $y\ge 3$;
\item $[2,2]\subs\pi_2$.
\end{assumptions}
Consider the candidate datum
\[
\DD'=\datum{\surf{g-1}}{d-4}{\pi_1',\pi_2',[s-2,d-s-2]},
\]
where $\pi_1'=\pi_1\setminus[x,y]\cup[x-2,y-2]$ and $\pi_2'=\pi_2\setminus[2,2]$. Then $\DD\cmove\DD'$.
\end{combinatorialmoveb}
\begin{proof}
Consider a \dessin{} $\Gamma'\subs\surf{g-1}$ realizing $\DD'$. There are two cases.
\paragraph{Case 1:} the black vertex of degree $x-2$ lies on $\partial D_1$ and the one with degree $y-2$ lies on $\partial D_2$ (or vice versa). Then we perform the following operations on $\Gamma'$.
\begin{enumerate}[(1)]
\item Attach a tube to $\surf{g-1}$ with one endpoint in $D_1$ and the other one in $D_2$.
\item Add two black vertices, two white vertices and four edges as shown in the picture.
\item Draw the red edge as shown in the picture.
\item Perform the join operation along the red edges.
\end{enumerate}

\paragraph{Case 2:} the two black vertices of degrees $x-2$ and $y-2$ lie (say) on $\partial D_1$. We perform the following operations on $\Gamma'$.
\begin{enumerate}[(1)]
\item Add two black vertices and two white vertices on $e$.
\item Attach a tube to $\surf{g-1}$ with both endpoints in $D_1$.
\item Draw the two red edges shown in the picture.
\item Perform the joining operation along the red edges.
\end{enumerate}

In both cases, we get a new \dessin{} $\Gamma$ embedded in $\surf{g}$. It is easy to check that $\Gamma$ realizes the candidate datum $\DD$.
\end{proof}

\begin{combinatorialmoveb}\label{combinatorial-move:b:4 3}
Let $\DD=\datum{\surf{g}}{d}{\pi_1,\pi_2,[s,d-s]}$ be a candidate datum with $g\ge 1$. Assume that:
\begin{assumptions}
\item $2\le s\le d-2$;
\item $x\in\pi_1$ for some $x\ge 4$;
\item $y\in\pi_2$ for some $y\ge 3$.
\end{assumptions}
Consider the candidate datum
\[
\DD'=\datum{\surf{g-1}}{d-2}{\pi_1',\pi_2',[s-1,d-s-1]},
\]
where $\pi_1'=\pi_1\setminus[x]\cup[x-2]$ and $\pi_2'=\pi_2\setminus[y]\cup[y-2]$. Then $\DD\cmove\DD'$.
\end{combinatorialmoveb}
\begin{proof}
Consider a \dessin{} $\Gamma'\subs\surf{g-1}$ realizing $\DD'$. Let $u$ be the black vertex of degree $x-2$, and let $v$ be the white vertex of degree $y-2$; there are two cases.
\paragraph{Case 1:} $u$ lies on $\partial D_1$ and $v$ lies on $\partial D_2$ (or vice versa). Then we perform the following operations on $\Gamma'$.
\begin{enumerate}[(1)]
\item Attach a tube to $\surf{g-1}$ with one endpoint in $D_1$ and the other in $D_2$.
\item Add one black vertex, one white vertex and two edges as shown in the picture.
\item Draw the two red edges shown in the picture.
\item Perform the joining operation along the red edges.
\end{enumerate}

\paragraph{Case 2:} neither $u$ nor $v$ lie (say) on $\partial D_2$; analyzing this case will be more involved than usual. We will say that an edge is \emph{shared} if it is not enveloped by $D_1$ or by $D_2$; in other words, an edge is shared if it appears exactly once in $\partial D_1$; in the following pictures, shared edges will be colored green. Our goal will be to prove that we can add two vertices of degree $2$ -- one black and one white -- on a shared edge, in such a way that the vertices $\{u,2,2,v\}$ appear in this order on $\partial D_1$. Let us unwind the boundary of $D_1$; since $u$ does not lie on the boundary of $D_2$, it appears exactly $x-2\ge 2$ times on $\partial D_1$; similarly, $v$ appears exactly $y-2$ times.
\begin{itemize}
\item Assume that $\{u,v,u,\text{shared edge}\}$ appear in this order on $\partial D_1$. Then, by adding a black and a white vertex on this shared edge, we get the desired result.
\item Otherwise, consider the pictures below.
\begin{enumerate}[(1)]
\item We are in the following situation: there is a contiguous segment $A$ of $\partial D_1$ that contains all the occurrences of $u$, and does not contain any occurrence of $v$ or any shared edge. Similarly, there is a segment $B$ of $\partial D_1$ that contains all the occurrences of $v$, no occurrence of $u$ and no shared edge. Note that $u$ and $v$ are never adjacent to a shared edge, since by assumption they do not lie on $\partial D_2$.
\item Choose an orientation of $\partial D_1$ (counterclockwise in the picture) and consider the first occurrence of $u$ in $A$; let $\alpha$ be the edge immediately afterwards in $\partial D_1$. Since $\alpha$ is not shared, it must appear once more on the combinatorial boundary of $D_1$, with the opposite orientation. Note that $\alpha$ it cannot occur immediately before the first appearance of $u$, otherwise $u$ would have degree $1$, therefore it will occur somewhere else on $A$
\item Consider the first occurrence of $v$ in $B$; let $\beta$ be the edge immediately before in $\partial D_1$. Since $\beta$ is not shared, it will also occur somewhere else on the combinatorial perimeter of $D_1$
\item Let $a$ be the other endpoint of $\alpha$, and let $b$ be the other endpoint of $\beta$. Erase the edges $\alpha$ and $\beta$ and draw two new ones, connecting $u$ to $v$ and $a$ to $b$ as shown in the picture.
\item It is now easy to see that the orange region is still a complementary disk, and that its perimeter has not changed. Moreover, by traveling along its boundary, we encounter $\{u,v,u\}$ in this order, without any shared edges in between; since there must be at least one shared edge on the boundary of the new orange region, the argument of the first bullet point applies.
\end{enumerate}

Once we have added the two vertices of degree $2$ as explained above, we can perform the following operations.
\begin{enumerate}[(1)]
\item Attach a tube to $\surf{g-1}$ with both endpoints in $D_1$.
\item Draw the two red edges shown in the picture.
\item Perform the joining operation along the red edges.
\end{enumerate}
\end{itemize}

In both cases, we get a new \dessin{} $\Gamma$ embedded in $\surf{g}$. It is easy to check that $\Gamma$ realizes the candidate datum $\DD$.
\end{proof}

We will make extensive use of these combinatorial moves in the next chapter, where a full classification of the exceptional data with $n=3$ and $\len{\pi_3}=2$ will be provided (see \cref{short-partition:th:realizability-on-sphere-n-3,short-partition:th:realizability-on-torus-n-3,short-partition:th:realizability-on-higher-genus-n-3}).

\section{Realizability by \texorpdfstring{\dessins{}}{dessins d'enfant}}

We conclude this chapter by proving the realizability of a few families of candidate data by means of \dessins{}; these results, while interesting by themselves, will be useful in the next chapter for addressing some cases which are not covered by the combinatorial moves we have introduced.

\begin{proposition}\label{dessins:th:special-case-[2 d-2]}
Let $\DD=\datum{\surf{g}}{d}{\pi_1,\pi_2,[2,d-2]}$ be a candidate datum. Assume that:
\begin{assumptions}
\item $[x,y]\subs\pi_1$ for some $x\ge 2$, $y\ge 3$;
\item $[2,2]\subs\pi_2$.
\end{assumptions}
Then $\DD$ is realizable.
\end{proposition}
\begin{proof}
We will show that, under the stated assumptions, there is a combinatorial move\footnote{To be precise, when $d=5$ the tuple $\DD'$ is not a combinatorial datum according to our definition. However, the only candidate datum $\DD$ of degree $5$ satisfying the assumptions is $\DD=\datum{\sphere{}}{5}{[2,3],[1,2,2],[2,3]}$, whose realizability can be easily checked by hand with a suitable \dessin{}.}
\[
\DD\cmove\DD'=\datum{\surf{g}}{d-4}{\pi_1\setminus[x,y]\cup[x+y-4],\pi_2\setminus[2,2],[d-4]}.
\]
Note that $\DD'$ is realizable by \cref{monodromy:th:sphere-[d]}; let $\Gamma'\subs\surf{g}$ be a \dessin{} realizing it. We perform the following operations on $\Gamma'$.
\begin{enumerate}[(1)]
\item Consider the black vertex of degree $x+y-4$ and split it into two vertices of degrees $x-2$ and $y-2$.
\item Add two white vertices and four edges as shown in the picture. This creates a new complementary disk with perimeter $4$.
\end{enumerate}
After these operations, we get a new \dessin{} $\Gamma$ embedded in $\surf{g}$. It is easy to check that $\Gamma$ realizes the candidate datum $\DD$.
\end{proof}


\begin{proposition}\label{dessins:th:special-families}
The following families of candidate data are realizable for every $g\ge 2$.
\begin{enumerate}[(1)]
\item $\datum{\surf{g}}{6g}{[3,\ldots,3],[3,\ldots,3],[s,6g-s]}$.
\item $\datum{\surf{g}}{6g+2}{[2,3,\ldots,3],[2,3\ldots,3],[s,6g+2-s]}$.
\item $\datum{\surf{g}}{6g+3}{[3,\ldots,3],[1,2,3,\ldots,3],[s,6g+3-s]}$.
\item $\datum{\surf{g}}{6g+4}{[1,3,\ldots,3],[1,3,\ldots,3],[s,6g+4-s]}$.
\item $\datum{\surf{g}}{6g+6}{[1,2,3,\ldots,3],[1,2,3,\ldots,3],[s,6g+6-s]}$.
\end{enumerate}
\end{proposition}
\begin{proof}
\def\env#1{\underline{#1}}
In the scope of this proof we define an \emph{augmented combinatorial datum} as a combinatorial datum $\DD=\datum{\surf{g}}{d}{\pi_1,\pi_2,\pi_3}$ where some distinguished elements of $\pi_3$ are called \emph{enveloping}; we will underline the enveloping elements in order to recognize them. An augmented combinatorial datum $\DD$ is \emph{realizable} if there exists a \dessin{} $\Gamma$ with $\DD(\Gamma)=\DD$ such that, for every complementary disk $D$ corresponding to an enveloping element of $\pi_3$, there is an edge of $\Gamma$ which is enveloped by $D$.
\paragraph{Step 1.} The augmented datum
\[
\DD=\datum{\surf{2}}{12}{[3,3,3,3],[3,3,3,3],\pi_3}
\]
is realizable for $\pi_3\in\{[1,\env{11}],[2,\env{10}],[\env{3},\env{9}],[\env{4},\env{8}],[\env{5},\env{7}],[\env{6},\env{6}]\}$. The following pictures display \dessins{} realizing each of these augmented data; as usual, the disk associated to the first element of $\pi_3$ is colored orange and the other one is colored blue; enveloped edges are drawn in the same color as the corresponding disk.\todo{Do this!}

\bgroup

\def\myradius{1.2}
\tikzset{
contour edge settings/quick/.style 2 args={left=disk #1 boundary,right=disk #2 boundary,edge=black edge}
}
\def\draweightvertices{
\path[use as bounding box] circle(1.6*\myradius);
\path (0:\myradius) coordinate (w1) pic{white vertex};
\path (90:\myradius) coordinate (w2) pic{white vertex};
\path (180:\myradius) coordinate (w3) pic{white vertex};
\path (270:\myradius) coordinate (w4) pic{white vertex};
\path (45:\myradius) coordinate (b1) pic{black vertex};
\path (135:\myradius) coordinate (b2) pic{black vertex};
\path (225:\myradius) coordinate (b3) pic{black vertex};
\path (315:\myradius) coordinate (b4) pic{black vertex};
}

\begin{tabular}{ccc}
\makecell[c]{
$\pi_3=[1,\env{11}]$\\
\tikzsetnextfilename{dessin-special-families-small-1}
\begin{tikzpicture}[graph picture]
\draweightvertices
\path[contour edge={quick={2}{2}}] (w4) to[bend left] (b3);
\path[contour edge={quick={2}{2}}] (b3) to[bend left] (w3);
\path[contour edge={quick={2}{1}}] (w3) to[bend left] (b2);
\path[contour edge={quick={2}{2}}] (b2) to[bend left] (w2);
\path[contour edge={quick={2}{2},left ecap=90}] (w2) to[bend left] (b1);
\path[contour edge={quick={2}{2}}] (w4) to[out=110,looseness=2.2,in=170] (b4);
\path[contour edge={quick={2}{2}}] (b3) to[out=60,out looseness=1.5,in=-135] (w1);
\path[contour edge={quick={2}{2}}] (b4) to[bend right=90] (w1);
\path[contour edge={above,quick={2}{2}}] (w1) to[out=140,in=120,looseness=2.2] (b4);
\path[contour edge={above,quick={2}{2},end=0.85}] (w4) to[out=10,in=-160,out looseness=1.3,in looseness=2] (b1);
\path[contour edge={quick={2}{2}}] (b1) to[out=-90,in=-90,looseness=2.2] (w2);
\path[contour edge={quick={1}{2}}] (w3) to[bend right=90,looseness=2.2] (b2);
\end{tikzpicture}
}&\makecell[c]{
$\pi_3=[2,\env{10}]$\\
\tikzsetnextfilename{dessin-special-families-small-2}
\begin{tikzpicture}[graph picture]
\draweightvertices
\path[contour edge={quick={1}{2}}] (w4) to[bend left] (b3);
\path[contour edge={quick={1}{2}}] (b3) to[bend left] (w3);
\path[contour edge={quick={1}{2}}] (w3) to[bend left] (b2);
\path[contour edge={quick={1}{2}}] (b2) to[bend left=110] (w4);
\path[contour edge={above,quick={2}{2}}] (w3) to[out=-20,out looseness=1.4,in=-180] (b1);
\path[contour edge={above,quick={2}{2},end=0.75}] (b2) to[out=-60,in=-120,looseness=2.2] (w2);
\path[contour edge={quick={2}{2}}] (w2) to[bend left=90] (b1);
\path[contour edge={quick={2}{2}}] (b1) to[out=-90,in=-90,looseness=2.2] (w2);
\path[contour edge={above,quick={2}{2}}] (w4) to[out=110,looseness=2.2,in=170] (b4);
\path[contour edge={above,quick={2}{2}}] (b3) to[out=60,out looseness=1.4,in=-135] (w1);
\path[contour edge={quick={2}{2}}] (b4) to[bend right=90] (w1);
\path[contour edge={quick={2}{2}}] (w1) to[out=140,in=120,looseness=2.2] (b4);
\end{tikzpicture}
}&\makecell[c]{
$\pi_3=[\env{3},\env{9}]$\\
\tikzsetnextfilename{dessin-special-families-small-3}
\begin{tikzpicture}[graph picture]
\draweightvertices
\path[contour edge={quick={2}{2}}] (w4) to[bend left] (b3);
\path[contour edge={quick={2}{2}}] (b3) to[bend left] (w3);
\path[contour edge={quick={2}{1}}] (w3) to[bend left] (b2);
\path[contour edge={quick={1}{1},left ecap=90}] (b2) to[out=-40,out looseness=2,in=-170] (w2);
\path[contour edge={quick={1}{2},left ecap=90}] (w2) to[bend left] (b1);
\path[contour edge={quick={2}{2}}] (w4) to[out=110,looseness=2.2,in=170] (b4);
\path[contour edge={quick={2}{2}}] (b3) to[out=60,out looseness=1.5,in=-135] (w1);
\path[contour edge={quick={2}{2}}] (b4) to[bend right=90] (w1);
\path[contour edge={above,quick={2}{2}}] (w1) to[out=140,in=120,looseness=2.2] (b4);
\path[contour edge={above,quick={2}{2},end=0.85}] (w4) to[out=10,in=-160,out looseness=1.3,in looseness=2] (b1);
\path[contour edge={quick={1}{2}}] (b1) to[out=-90,in=-90,looseness=2.2] (w2);
\path[contour edge={above,quick={1}{2}}] (w3) to[out=-20,in=50,looseness=3.5] (b2);
\end{tikzpicture}
}\\
\makecell[c]{
$\pi_3=[\env{4},\env{8}]$\\\tikzsetnextfilename{dessin-special-families-small-4}
\begin{tikzpicture}[graph picture]
\draweightvertices
\path[contour edge={quick={1}{2}}] (w4) to[bend left] (b3);
\path[contour edge={quick={1}{2}}] (b3) to[bend left] (w3);
\path[contour edge={quick={1}{2}}] (w3) to[bend left] (b2);
\path[contour edge={quick={1}{2}}] (b2) to[out=-30,in=0] (w4);
\path[contour edge={above,quick={2}{2}}] (w3) to[out=10,in=-180] (b1);
\path[contour edge={quick={1}{1}}] (b2) to[bend left] (w2);
\path[contour edge={quick={1}{2},left ecap=90}] (w2) to[bend left] (b1);
\path[contour edge={quick={1}{2}}] (b1) to[out=-90,in=-90,looseness=2.2] (w2);
\path[contour edge={above,quick={2}{2}}] (w4) to[out=110,looseness=2.2,in=170] (b4);
\path[contour edge={above,quick={2}{2}}] (b3) to[out=60,out looseness=1.5,in=-135] (w1);
\path[contour edge={quick={2}{2}}] (b4) to[bend right=90] (w1);
\path[contour edge={quick={2}{2}}] (w1) to[out=140,in=120,looseness=2.2] (b4);
\end{tikzpicture}
}&\makecell[c]{
$\pi_3=[\env{5},\env{7}]$\\
\tikzsetnextfilename{dessin-special-families-small-5}
\begin{tikzpicture}[graph picture]
\draweightvertices
\path[contour edge={quick={1}{2},left bcap=90}] (w4) to[bend left] (b3);
\path[contour edge={quick={1}{2}}] (b3) to[bend left] (w3);
\path[contour edge={quick={1}{2}}] (w3) to[bend left] (b2);
\path[contour edge={quick={1}{2},right bcap=90,begin=.5}] (b2) to[bend left] (w2);
\path[contour edge={quick={1}{2},end=.5,left ecap=90}] (w2) to[bend left] (b1);
\path[contour edge={above,quick={2}{2}}] (w4) to[out=120,looseness=3,in=60] (b4);
\path[contour edge={quick={1}{1}}] (b2) to[out=100,looseness=2,in=40] (w1);
\path[contour edge={quick={2}{1},right bcap=90,left ecap=90}] (b4) to[bend right=90] (w1);
\path[contour edge={quick={2}{1}}] (w1) to[out=140,in=150,looseness=2.2] (b4);
\path[contour edge={quick={2}{1}}] (w4) to[out=30,in=-140,in looseness=2] (b1);
\path[contour edge={above,quick={2}{2}}] (b1) to[out=170,in=10] (w3);
\path[contour edge={quick={2}{2}}] (b3) to[bend right] (w2);
\end{tikzpicture}
}&\makecell[c]{
$\pi_3=[\env{6},\env{6}]$\\
\tikzsetnextfilename{dessin-special-families-small-6}
\begin{tikzpicture}[graph picture]
\draweightvertices
\path[contour edge={quick={1}{2}}] (w4) to[bend left] (b3);
\path[contour edge={quick={1}{2}}] (b3) to[bend left] (w3);
\path[contour edge={quick={1}{2}}] (w3) to[bend left] (b2);
\path[contour edge={quick={1}{2}}] (b2) to[bend left] (w4);
\path[contour edge={above,quick={2}{2}}] (w3) to[out=-20,in=-180] (b1);
\path[contour edge={above,quick={1}{1}}] (b2) to[bend left] (w2);
\path[contour edge={quick={1}{2}}] (w2) to[bend left=90] (b1);
\path[contour edge={quick={1}{2}}] (b1) to[out=-90,in=-90,looseness=2.2] (w2);
\path[contour edge={above,quick={1}{1}}] (w4) to[bend right] (b4);
\path[contour edge={above,quick={2}{2}}] (b3) to[out=60,in=-135] (w1);
\path[contour edge={quick={2}{1}}] (b4) to[bend right=90] (w1);
\path[contour edge={quick={2}{1}}] (w1) to[out=140,in=120,looseness=2.2] (b4);
\end{tikzpicture}
}
\end{tabular}
\egroup


\paragraph{Step 2.} Let $\DD=\datum{\surf{g}}{d}{\pi_1,\pi_2,\pi_3}$ be a realizable augmented datum, and let $\env{x}\in\pi_3$ be an enveloping element. Then the augmented datum
\[
\DD'=\datum{\surf{g+1}}{d+6}{\pi_1\cup[3,3],\pi_2\cup[3,3],\pi_3\setminus[\env{x}]\cup[\env{x+6}]}
\]
is realizable as well. In fact, consider a \dessin{} $\Gamma\subs\surf{g}$ realizing $\DD$, and fix an edge $e$ enveloped by the disk $D$ associated to $\env{x}$. We perform the following operations on $\Gamma$.
\begin{enumerate}[(1)]
\item Add one black vertex and one white vertex on $e$.
\item Attach a tube to $\surf{g}$ with both endpoints on $D$.
\item Add one black vertex, one white vertex and four edges as shown in the picture.
\end{enumerate}
After these operations, we get a new \dessin{} $\Gamma'$ embedded in $\surf{g+1}$. It is easy to check that $\Gamma'$ realizes the augmented datum $\DD'$.
\paragraph{Step 3.} For every $g\ge 2$, the augmented datum 
\[
\datum{\surf{g}}{6g}{[3,\ldots,3],[3,\ldots,3],\pi_3}
\]
is realizable for $\pi_3\in\{[1,\env{6g-1}],[2,\env{6g-2}]\}\cup\{[\env{s},\env{6g-s}]\colon 3\le s\le 6g-3\}$. This can be easily shown by induction on $g$, using step 1 as the base case and step 2 for the induction.
\paragraph{Step 4.} Let $\DD=\datum{\surf{g}}{d}{\pi_1,\pi_2,\pi_3}$ be a realizable augmented datum, and let $\env{x}\in\pi_3$ be an enveloping element. Then the augmented datum
\[
\DD'=\datum{\surf{g}}{d+2}{\pi_1\cup[2],\pi_2\cup[2],\pi_3\setminus[\env{x}]\cup[\env{x+2}]}
\]
is realizable as well. In fact, consider a \dessin{} $\Gamma\subs\surf{g}$ realizing $\DD$, and fix an edge $e$ enveloped by the disk $D$ associated to $\env{x}$. Then, add one black vertex and one white vertex on $e$. The new \dessin{} realizes the augmented datum $\DD'$.
\paragraph{Step 5.} Let $\DD=\datum{\surf{g}}{d}{\pi_1,\pi_2,\pi_3}$ be a realizable augmented datum, and let $\env{x}\in\pi_3$ be an enveloping element. Then the augmented datum
\[
\DD'=\datum{\surf{g}}{d+3}{\pi_1\cup[3],\pi_2\cup[1,2],\pi_3\setminus[\env{x}]\cup[\env{x+3}]}
\]
is realizable as well. In fact, consider a \dessin{} $\Gamma\subs\surf{g}$ realizing $\DD$, and fix an edge $e$ enveloped by the disk $D$ associated to $\env{x}$. Then, add one black vertex and two white vertices as shown in the picture. The new \dessin{} realizes the augmented datum $\DD'$.
\paragraph{Step 6.} Let $\DD=\datum{\surf{g}}{d}{\pi_1,\pi_2,\pi_3}$ be a realizable augmented datum, and let $\env{x}\in\pi_3$ be an enveloping element. Then the augmented datum
\[
\DD'=\datum{\surf{g}}{d+4}{\pi_1\cup[1,3],\pi_2\cup[1,3],\pi_3\setminus[\env{x}]\cup[\env{x+4}]}
\]
is realizable as well. In fact, consider a \dessin{} $\Gamma\subs\surf{g}$ realizing $\DD$, and fix an edge $e$ enveloped by the disk $D$ associated to $\env{x}$. Then, add two black vertices and two white vertices as shown in the picture. The new \dessin{} realizes the augmented datum $\DD'$.

Finally, it is easy to see that the five families listed in the statement can be obtained by applying steps 4, 5 and 6 zero or more times to an augmented datum which is realizable by step 3, and then forgetting about the augmentation; here are the details.
\begin{enumerate}[(1)]
\item $\datum{\surf{g}}{6g}{[3,\ldots,3],[3,\ldots,3],[s,6g-s]}$ is already realizable by step 3.
\item If we assume that $s\le 3g+1$, step 4 gives
\begin{multline*}
\datum{\surf{g}}{6g+2}{[2,3,\ldots,3],[2,3\ldots,3],[s,\env{6g+2-s}]}\\
\cmove\datum{\surf{g}}{6g}{[3,\ldots,3],[3,\ldots,3],[s,\env{6g-s}]},
\end{multline*}
which is realizable by step 3.
\item If we assume that $s\le 3g+1$, step 5 gives 
\begin{multline*}
\datum{\surf{g}}{6g+3}{[3,\ldots,3],[1,2,3,\ldots,3],[s,\env{6g+3-s}]}\\
\cmove\datum{\surf{g}}{6g}{[3,\ldots,3],[3,\ldots,3],[s,\env{6g-s}]},
\end{multline*}
which is realizable by step 3.
\item If we assume that $s\le 3g+2$, step 6 gives
\begin{multline*}
\datum{\surf{g}}{6g+4}{[1,3,\ldots,3],[1,3,\ldots,3],[s,\env{6g+4-s}]}\\
\cmove\datum{\surf{g}}{6g}{[3,\ldots,3],[3,\ldots,3],[s,\env{6g-s}]},
\end{multline*}
which is realizable by step 3.
\item If we assume that $s\le 3g+3$, step 4 and 6 give
\begin{multline*}
\datum{\surf{g}}{6g+6}{[1,2,3,\ldots,3],[1,2,3,\ldots,3],[s,\env{6g+6-s}]}\\
\begin{aligned}
&\cmove\datum{\surf{g}}{6g+4}{[1,3,\ldots,3],[1,3,\ldots,3],[s,\env{6g+4-s}]}\\
&\cmove\datum{\surf{g}}{6g}{[3,\ldots,3],[3,\ldots,3],[s,\env{6g-s}]},
\end{aligned}
\end{multline*}
which is realizable by step 3.\qedhere
\end{enumerate}
\end{proof}