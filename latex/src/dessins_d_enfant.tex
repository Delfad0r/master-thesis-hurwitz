\chapter{\texorpdfstring{\Dessins{}}{Dessins d'enfant}}

\section{Child's drawings on surfaces}
\smallvertices

In \cref{monodromy:sc:combinatorial-moves}, we discussed how the Hurwitz existence problem can be reduced to the analysis of candidate data on the sphere. Moreover, thanks to \cref{combinatorial-move:a:small-v,combinatorial-move:a:large-v}, we have devised a relatively reliable technique to decrease the number $n$ of partitions; this technique was successfully employed in \cref{monodromy:sc:results-sphere} to show the realizability of a wide variety of candidate data by induction on $n$, starting from the base case $n=3$. Ignoring the cases where $n\le 2$, which were fully analyzed in \cref{monodromy:sc:combinatorial-moves}, it should come as no surprise that candidate data with $n=3$ play a very important role in the study of the existence problem.

Up to this point, we have only approached the Hurwitz existence problem from a group-theoretic point of view, showing realizability by looking for elements of $\symgroup[d]$ with certain properties. In this section we will present a totally different tool, of a more topological and combinatorial nature, for attacking the same problem. The concept of \emph{\dessin{}}\footnote{``\emph{\Dessin{}}'' is French for ``child's drawing'', hence the title of this section.} was popularized by Grothendieck in \cite{grothendieck}, in a setting related to, but different from, the Hurwitz existence problem. \Dessins{} provide a strikingly elementary tool for showing the realizability or exceptionality of candidate data with $n=3$ partitions, although they generalize quite nicely to the case $n\ge 4$. However, we will not deal with this generalization, since the reduction technique will prove to be sufficient for our purposes; we refer the interested reader to \resultcite{section}{4}{pervova-existence-i}.

We start by introducing some basic terminology about graphs. Given a surface $\Sigma$, a \emph{graph} embedded in $\Sigma$ (or, simply, a graph on $\Sigma$) is a closed subspace $\Gamma\subs\Sigma$ consisting of:
\begin{itemize}
\item a finite number of points $x_1,\ldots,x_r\in\Sigma$, called \emph{vertices};
\item a finite number of segments (subspaces homeomorphic to $[0,1]$) $e_1,\ldots,e_d\subs\Sigma$, called \emph{edges}; we require that each edge connects two (not necessarily distinct) vertices, and that the interiors of two edges are disjoint; in other words, two edges may intersect at most at their endpoints; moreover, a vertex cannot lie on the interior of an edge.
\end{itemize}
The \emph{degree} of a vertex $x$ is the number of edges having $x$ as an endpoint; edges connecting $x$ to itself are counted twice; we denote the degree of $x$ by $k(x)$. In order to avoid unpleasant corner cases, we will always require that there are no \emph{isolated vertices} or, in other words, that $k(x)\ge 1$ for every vertex $x$.

A \emph{bipartite graph} is a graph whose vertices are colored either black or white, and each edge connects a black vertex to a white one. If we denote the black vertices by $x_1,\ldots,x_r$ and the white vertices by $y_1,\ldots,y_s$, an easy counting argument shows that
\[
k(x_1)+\ldots+k(x_r)=k(y_1)+\ldots+k(y_s)=d,
\]
where $d$ is the number of edges.

Given a graph $\Gamma$ on a surface $\Sigma$, the space $\Sigma\setminus\Gamma$ is a disjoint union of a finite number of non-compact surfaces $S_1\sqcup\ldots\sqcup S_h$, called \emph{complementary regions} of $\Gamma$. We are finally ready to give the definition of the much anticipated \dessins{}.

\begin{definition}
Let $\Sigma$ be a surface. A \emph{\dessin{}} on $\Sigma$ is a bipartite graph $\Gamma\subs\Sigma$ whose complementary regions are topological disks.
\end{definition}

The picture below presents an example of a \dessin{} embedded in the torus $\surf{1}$, with $2$ black vertices, $3$ white vertices, $7$ edges and $2$ complementary regions.

\begin{center}
\tikzsetnextfilename{dessin-first-example}
\begin{tikzpicture}[tdplot_main_coords,declare function={interp(\a,\b,\t)=\a+\t*(\b-\a);},every plot/.style={smooth,samples=\torusprecision}]
\pgfsetlayers{main,graph vertex}
\draw[surf boundary,fill=disk 1,even odd rule,3d torus];
\draw[surf boundary,3d torus stretch];
\fill[disk 2] [torus graph small region];
\draw[black edge dashed] [torus graph behind];
\drawtorusgraph
\end{tikzpicture}
\end{center}

Let $\Gamma$ be a \dessin{}, and fix a complementary region $D$. By traveling along its boundary, always keeping $D$ to the left, we get a cyclic sequence of edges of $\Gamma$, which we call \emph{combinatorial boundary} of $D$, and denote by $\partial D$. Note that the same edge $e$ can be traveled along twice, once in each direction; in this case, it will appear twice in $\partial D$, and we will say that $e$ is \emph{enveloped} by $D$. The number of edges (with multiplicity) of $\partial D$ is the \emph{perimeter} of $D$, denoted by $\card{\partial D}$. For instance, let us consider the \dessin{} of the previous example. Let $D_1$ be the orange complementary disk, and let $D_2$ be the blue one; label the edges $e_1,\ldots,e_7$ as shown in the following picture. By traveling along the boundary of $D_1$, we get that
\[
\partial D_1=\{e_1,e_2,e_4,e_5,e_7,e_6,e_5,e_4,e_3,e_1,e_6,e_7\}.
\]
In particular, note that $\card{\partial D_1}=12$, and that $e_1$, $e_4$, $e_5$, $e_6$ and $e_7$ are enveloped by $D_1$. Similarly, we see that $\partial D_2=\{e_2,e_3\}$.
\begin{center}
\largevertices{}
\def\xac{2}
\def\yac{10}
\def\radiusA{2.55}
\def\radiusB{.75}
\tikzsetnextfilename{dessin-boundary-example}
\begin{tikzpicture}[tdplot_main_coords,declare function={interp(\a,\b,\t)=\a+\t*(\b-\a);},every plot/.style={smooth,samples=\torusprecision}]
\pgfsetlayers{main,graph vertex}
\draw[surf boundary,fill=disk 1,even odd rule,3d torus];
\draw[surf boundary,3d torus stretch];

\draw[disk 1 boundary dashed] [torus graph behind right] [torus graph behind left];
\draw[black edge dashed] [torus graph behind];
\draw[disk 1 boundary,fill=white]
	[torus graph upper contour]
	[torus graph lower contour];
\begin{scope}[/pgf/fpu/install only={reciprocal}]
\path[decorate,decoration={markings,mark=between positions 0.015 and .99 step 1.5em with {\arrow{Stealth[disk 1 boundary,scale=.6]}}}] [torus graph upper contour] [torus graph behind left] [torus graph lower contour] [torus graph behind right];
\draw[disk 2 boundary,fill=disk 2,postaction={decorate,decoration={markings,mark=between positions 0 and .99 step 1.5em with {\arrowreversed{Stealth[disk 2 boundary,scale=.6]}}}}] [torus graph small contour];
\end{scope}
\drawtorusgraph
\node[above right,xshift=2pt] at (\ontorus{255}{45}) {$e_1$};
\node[above=3pt] at (\ontorus{290}{75}) {$e_2$};
\node[below=7pt] at (\ontorus{290}{-15}) {$e_3$};
\node[below left] at (\ontorus{15}{90}) {$e_4$};
\node[below right] at (\ontorus{160}{105}) {$e_5$};
\node[left=2pt] at (\ontorus{240}{30}) {$e_6$};
\node[below=3pt] at (\ontorus{240}{-90}) {$e_7$};
\end{tikzpicture}
\end{center}


If $D_1,\ldots,D_h$ are the complementary regions of $\Gamma$, a counting argument shows that
\[
\card{\partial D_1}+\ldots+\card{\partial D_h}=2d.
\]
It is also easy to see that the perimeter of each complementary region is even: in fact, when traveling along the boundary of a region, we alternately encounter black and white vertices, so an even number of edges is required to get back to the starting color.

Finally, note that every \dessin{} is necessarily connected, otherwise there would be some complementary region with two or more boundary components.

\begin{definition}
Let $\Gamma$ be a \dessin{} on a surface $\Sigma$; let $d$ be the number of edges. Let $x_1,\ldots,x_r$ be the black vertices, $y_1,\ldots,y_s$ the white ones. Denote by $D_1,\ldots,D_h$ the complementary disks of $\Gamma$. The \emph{branching datum} of $\Gamma$ is the tuple
\[
\DD(\Gamma)=\datum{\Sigma,\sphere{}}{d}{[k(x_1),\ldots,k(x_r)],[k(y_1),\ldots,k(y_s)],[\card{\partial D_1}/2,\ldots,\card{\partial D_h}/2]}.
\]
\end{definition}

From the discussion above, we immediately see that $\DD(\Gamma)$ is a combinatorial datum, since
\[
k(x_1)+\ldots+k(x_r)=k(y_1)+\ldots+k(y_s)=\card{\partial D_1}/2+\ldots+\card{\partial D_h}/2=d.
\]
Actually, if $\Sigma$ is orientable, $\DD(\Gamma)$ is a candidate datum: by the Euler formula,
\begin{align*}
\chi(\Sigma)&=\card{\{\text{vertices}\}}-\card{\{\text{edges}\}}+\card{\{\text{regions}\}}\\
&=(r+s)-d+h\\
&=2d-v(\pi_1)-v(\pi_2)-v(\pi_3),
\end{align*}
where $\pi_1=[k(x_1),\ldots,k(x_r)]$, $\pi_2=[k(y_1),\ldots,k(y_s)]$ and $\pi_3=[\card{\partial D_1}/2,\ldots,\card{\partial D_h}/2]$. This is no coincidence, just like the name ``branching datum of $\Gamma$'' was not picked at random: the following result establishes a strong connection between \dessins{} and realizable combinatorial data.

\begin{proposition}\label{dessins:th:dessins-realizability}
Let $\DD=\datum{\surf{g}}{d}{\pi_1,\pi_2,\pi_3}$ be a combinatorial datum. Then $\DD$ is realizable if and only if there exists a \dessin{} $\Gamma\subs\Sigma_g$ with $\DD(\Gamma)=\DD$.
\end{proposition}
\begin{proof}
We show the two implications.
\begin{twoimplications}
\rightimplication
Assume that $\DD$ is realized by a branched covering $\map{f}{\surf{g}}{\sphere{}}$ with branching points $x,y,z\in\sphere{}$. Let
\begin{align*}
\{\wtilde{x}_1,\ldots,\wtilde{x}_r\}=f^{-1}(x),&& \{\wtilde{y}_1,\ldots,\wtilde{y}_s\}=f^{-1}(y),&&\{\wtilde{z}_1,\ldots,\wtilde{z}_h\}=f^{-1}(z).
\end{align*}
Fix a segment $e\subs\sphere$ connecting $x$ to $y$ and avoiding $z$; we claim that $\Gamma=f^{-1}(e)$ is the desired \dessin{}. Let $\interior{e}$ be the interior of $e$ (that is, $\interior{e}=e\setminus\{x,y\}$). First of all, note that $f^{-1}(\interior{e})$ is the disjoint union of $d$ open segments $\interior{e}_1,\ldots,\interior{e}_d$, since the restriction of $f$ to $f^{-1}(\sphere{}\setminus\{x,y,z\})$ is a covering map of degree $d$. Moreover, it is easy to see that the closure of each $\interior{e}_i$ is a closed segment $e_i$ connecting one point in $f^{-1}(x)$ to one point of $f^{-1}(y)$; it follows that $\Gamma$ is a bipartite graph on $\surf{g}$, with black vertices $\wtilde{x}_1,\ldots,\wtilde{x}_r$ and white vertices $\wtilde{y}_1,\ldots,\wtilde{y}_s$. Consider a vertex $\wtilde{x}_i$; recall that $f$ is locally modeled at $\wtilde{x}_i$ on the complex map $\xi\mapsto\xi^k$, where $k=k(\wtilde{x}_i)$ is the local degree of $f$ at $\wtilde{x}_i$. As a consequence, we immediately see that there are exactly $k(\wtilde{x}_i)$ edges of $\Gamma$ with $\wtilde{x}_i$ as an endpoint, counted with multiplicity; of course, the same holds for every $\wtilde{y}_j$. Finally, we turn to the complementary regions of $\Gamma$. Let $D=\sphere{}\setminus e\iso\RR^2$, $\holed{D}=D\setminus\{z\}\iso\RR^2\setminus\{0\}$. The restriction of $f$ to $f^{-1}(\holed{D})=\surf{g}\setminus(\Gamma\cup\{\wtilde{z}_1,\ldots,\wtilde{z}_h\})$ is a covering map of the punctured disk $\holed{D}$. It is then easy to see that the complementary regions of $\Gamma$ are discs $\wtilde{D}_1,\ldots,\wtilde{D}_h$, with $\wtilde{z}_i\in\wtilde{D}_i$ for each $1\le i\le h$, and that the restriction $\map{f}{\wtilde{D}_i}{D}$ is modeled the complex map $\xi\mapsto\xi^{k(\wtilde{z}_i)}$. Since the perimeter of $D$ is $2$, we have that $\card{\partial\wtilde{D}_i}=2 k(\wtilde{z}_i)$; this concludes the proof of the equality $\DD(f)=\DD(\Gamma)$.
\leftimplication
Conversely, assume that we are given a \dessin{} $\Gamma\subs\surf{g}$ with $\DD(\Gamma)=\DD$. Fix three arbitrary points $x,y,z\in\sphere{}$, and let $e\subs\sphere{}$ be a segment connecting $x$ to $y$ and avoiding $z$. First of all, we define $f$ on $\Gamma$, sending black vertices to $x$ and white vertices to $y$, and mapping edges homeomorphically to $e$. Extending $f$ to all of $\surf{g}$ is a relatively easy task: the idea is to send each complementary region to the disk $\sphere{}\setminus e$ with the map $\xi\mapsto\xi^k$, for some $k\ge 1$ depending on the perimeter; here are the details. Consider the standard closed disk $K=\{a\in\CC:\lvert a\rvert\le 1\}$, and fix a complementary region $\wtilde{D}\subs\surf{g}\setminus\Gamma$. Let $\map{\phi}{K}{\surf{g}}$ be a continuous map which restricts to a homeomorphism $\map{\phi}{\interior{K}}{\wtilde{D}}$, where $\interior{K}$ denotes the interior of $K$. Let the perimeter of $\wtilde{D}$ be $2k$; then there exists a map $\map{\psi}{K}{\sphere}$ such that $\psi(0)=z$, the diagram
\begin{diagram}
\partial K\dar{\phi}\ar[dr,"\psi"]\\
\Gamma\rar{f}&e
\end{diagram}
commutes, $\psi$ is a local homeomorphism in $\interior{K}\setminus\{0\}$ and it is modeled on $\xi\mapsto\xi^k$ in a neighborhood of $0\in K$. We can now extend $f$ to $\wtilde{D}$ by setting $f(\wtilde{x})=\psi(\phi^{-1}(\wtilde{x}))$ for every $\wtilde{x}\in\wtilde{D}$. After repeating the process for all the complementary regions, it is not hard to verify that the map $\map{f}{\tSigma}{\sphere{}}$ we have obtained is a branched covering with branching points $x,y,z\in\sphere{}$. Since $\Gamma=f^{-1}(e)$, the ($\Rightarrow$) part of the proof implies that $\DD(\Gamma)=\DD(f)$.\qedhere
\end{twoimplications}
\end{proof}

\section{Unwinding, collapsing and fattening}\label{dessins:sc:unwind-join-fatten}

In the next section we will introduce a new kind of combinatorial moves, which operate on \dessins{} rather than permutations. In this context, the importance of visual intuition cannot be overstated. Therefore, we will now spend some words describing in detail three operations that will play a major role in the topological explanation of the upcoming combinatorial moves.

\paragraph{Unwinding the boundary.} Let $\Gamma$ be a graph on a surface $\Sigma$. Take a complementary region $D$, and assume $D$ is a topological disk. Intuitively, when we \emph{unwind the boundary} of $D$, we represent $D$ as the standard closed disk $K$ embedded in $\RR^2$; the vertices and the edges of the combinatorial boundary of $D$ are placed sequentially on the topological boundary of $K$, possibly with repetitions. For a more formal description, we can follow the strategy presented in the second part of the proof of \cref{dessins:th:dessins-realizability}. Consider a continuous map $\map{\phi}{K}{\surf{g}}$ such that:
\begin{enumroman}
\item its restriction to $\interior{K}$ is a homeomorphism $\map{\phi}{\interior{K}}{D}$;
\item for each edge $e$ of $\Gamma$, the restriction $\map{\phi}{\phi^{-1}(\interior{e})}{\interior{e}}$ is a covering map;
\item for each vertex $v$ of $\Gamma$, the preimages $\phi^{-1}(v)$ form a discrete subset of $\partial K$.
\end{enumroman}
The vertices and the edges on the topological boundary of $D$ can be pulled back by $\phi$, thus unwinding the combinatorial boundary of $D$ on $\partial K$.
\begin{center}
\largevertices{}
\def\radiusA{2.55}
\def\radiusB{.75}
\tikzsetnextfilename{dessin-unwinding-example}
\begin{tikzpicture}
\pgfsetlayers{surf surrounding,main,graph vertex}
\surroundingradius=.5cm
\begin{scope}[shift={(-4,0)}]
\draw[black edge,fill=disk 1] circle(2);
\foreach \i[evaluate=\i as \i using int(\i),evaluate=\i as \col using {ifthenelse(mod(\i,2),"white ","black ")},evaluate=\i as \a using \i*360/12,evaluate=\i as \l using {{1,2,4,5,7,6,5,4,3,1,6,7}[\i]},evaluate=\l as \s using {or(\l==2,\l==3)}] in {0,...,11} { \pic at (\a:2) {\col vertex}; \node at ({\a+360/24}:2.25) {$e_{\l}$}; \ifnum\s=1\path[surrounding=disk 2] (\a:2) arc(\a:{\a+360/12}:2);\fi }
\end{scope}
\draw[->,violet] (-1.1,0) -- (.1,0) node[above,midway] {$\phi$};
\begin{scope}[tdplot_main_coords,declare function={interp(\a,\b,\t)=\a+\t*(\b-\a);},every plot/.style={smooth,samples=\torusprecision},shift={(4,0)}]
\draw[surf boundary,fill=disk 1,even odd rule,3d torus];
\draw[surf boundary,3d torus stretch];

\draw[black edge dashed] [torus graph behind];
\fill[disk 2] [torus graph small region];

\drawtorusgraph
\node[above right,xshift=2pt] at (\ontorus{255}{45}) {$e_1$};
\node[above=3pt] at (\ontorus{290}{75}) {$e_2$};
\node[below=7pt] at (\ontorus{290}{-15}) {$e_3$};
\node[below left] at (\ontorus{15}{90}) {$e_4$};
\node[below right] at (\ontorus{160}{105}) {$e_5$};
\node[left=2pt] at (\ontorus{240}{30}) {$e_6$};
\node[below=3pt] at (\ontorus{240}{-90}) {$e_7$};
\end{scope}
\end{tikzpicture}
\end{center}

\paragraph{Collapsing edges.} Let $\Gamma$ be a graph on a surface $\Sigma$. Consider an edge $e$, and let $x$, $y$ be its (distinct) endpoints. \emph{Collapsing} the edge $e$ means producing a new graph $\Gamma'$ by shrinking $e$ to a single point, so that $x$ and $y$ are merged into a single vertex, say $z$; note that $\Gamma'$ is defined only up to isotopy. It is immediate to check that $k(z)=k(x)+k(y)-2$, while the degrees of the other vertices are left unchanged. The topology of the complementary regions does not change either. To be more precise, there is a natural one-to-one correspondence between the regions of $\Sigma\setminus\Gamma$ and the regions of $\Sigma\setminus\Gamma'$, and corresponding regions are homeomorphic to each other.
\begin{center}
\tikzsetnextfilename{dessin-join-example}
\begin{tikzpicture}[tdplot_main_coords,declare function={interp(\a,\b,\t)=\a+\t*(\b-\a);},every plot/.style={smooth,samples=\torusprecision}]
\pgfsetlayers{main,graph vertex}
\begin{scope}[shift={(-3.5,0)}]
\draw[surf boundary,fill=disk 1,even odd rule,3d torus];
\draw[surf boundary,3d torus stretch];
\draw[black edge dashed] [torus graph behind];
\draw[black edge] plot[variable=\x,domain={vcrit2(240)}:{vcrit1(240)}] (\ontorus{240}{\x});
\draw[red edge] [torus straight={240}{60}{270}{30}];
\draw[black edge,fill=disk 2] [torus graph small region];
\draw[black edge] [torus straight={-50}{30}{80}{150}];
\draw[black edge] [torus straight={80}{150}{240}{60}];
\path (\ontorus{240}{60}) pic {white vertex} node[below left,yshift=1pt] {$x$};
\pic at (\ontorus{240}{0}) {white vertex};
\pic at (\ontorus{310}{30}) {white vertex};
\path (\ontorus{270}{30}) pic {white vertex} node[right=2pt] {$y$};
\pic at (\ontorus{80}{150}) {white vertex};
\node[colored label=red,below left] at (\ontorus{265}{40}) {$e$};
\end{scope}
\pic {my fancy arrow};
\begin{scope}[shift={(3.5,0)}]
\draw[surf boundary,fill=disk 1,even odd rule,3d torus];
\draw[surf boundary,3d torus stretch];
\draw[black edge dashed] [torus graph behind];
\draw[black edge] plot[variable=\x,domain={vcrit2(240)}:{vcrit1(240)}] (\ontorus{240}{\x});
\draw[black edge,fill=disk 2] plot[variable=\t,domain=-1:1] (\ontorus{interp(240,310,.5*\t+.5)}{interp(interp(60,30,.5*\t+.5),30+60*sqrt(1-\t*\t),.5*\t+.5)}) -- plot[variable=\t,domain=1:-1] (\ontorus{interp(240,310,.5*\t+.5)}{interp(interp(60,30,.5*\t+.5),30-60*sqrt(1-\t*\t),.5*\t+.5)});
\draw[black edge] [torus straight={-50}{30}{80}{150}];
\draw[black edge] [torus straight={80}{150}{240}{60}];
\path (\ontorus{240}{60}) pic {white vertex} node[below left,yshift=1pt] {$z$};
\pic at (\ontorus{240}{0}) {white vertex};
\pic at (\ontorus{310}{30}) {white vertex};
\pic at (\ontorus{80}{150}) {white vertex};
\end{scope}
\end{tikzpicture}
\end{center}
The edge $e$ disappears from the combinatorial boundaries, so the perimeters of the two regions touching $e$ decrease by $1$ (if the two regions were actually the same, then the perimeter decreases by $2$); the other perimeters do not change. Of course, if we collapse more than one edge, the resulting graph does not depend on the order in which we perform the operations.

\paragraph{Fattening graphs.} Representing \dessins{} on the sphere is easy, since graphs on $\sphere{}$ naturally embed in $\RR^2$; unfortunately, this is not the case for surfaces of genus $g\ge 1$. However, for a specific class of graphs (including \dessins{}), there is a trick we can exploit in order to represent them as diagrams on the plane. Let $\Gamma$ be a graph on a surface $\surf{g}$; assume that the complementary regions of $\Gamma$ are disks. Note that the topology of the embedding $\Gamma\subs\surf{g}$ can be completely recovered if we are given $\Gamma$ as an abstract graph, plus a tubular neighborhood of $\Gamma$ in $\surf{g}$; we call such a datum a \emph{fat graph}. A fat graph can be represented as a planar diagram with a finite number of transverse crossings; each crossing is equipped with the additional information of which edge goes over and which goes under (although this choice is immaterial).

In order to reconstruct the embedding of $\Gamma$ in $\surf{g}$, we simply have to thicken the edges of the diagram, keeping in mind that the two edges involved in a crossing actually go one under the other. This operation yields a fat graph, from which $\surf{g}$ can be recovered by gluing a disk along each boundary component.

\begin{center}
\def\xac{2}
\def\yac{10}
\tikzsetnextfilename{dessin-fattening-example}
\begin{tikzpicture}[tdplot_main_coords,declare function={interp(\a,\b,\t)=\a+\t*(\b-\a);},every plot/.style={smooth,samples=\torusprecision}]
\pgfsetlayers{main,graph vertex}
\begin{scope}[shift={(-3.5,0)}]
\draw[surf boundary,fill=disk 1,even odd rule,3d torus];
\draw[surf boundary,3d torus stretch];

\draw[disk 1 boundary dashed] [torus graph behind right] [torus graph behind left];
\draw[black edge dashed] [torus graph behind];
\draw[disk 1 boundary,fill=white]
	[torus graph upper contour]
	[torus graph lower contour];
\draw[disk 2 boundary,fill=disk 2] [torus graph small contour];
\drawtorusgraph
\end{scope}
\pic at (.2,0) {my fancy double arrow};
\begin{scope}[shift={(3.5,0)}]
\draw[disk 1 boundary,fill=gray!30] plot[variable=\x,domain={vcrit2(240-\xac)}:{360+vcrit1(240-\xac)}] (\ontorus{240-\xac}{\x}) --plot[variable=\x,domain={360+vcrit1(240)+\xac)}:{vcrit2(240+\xac)}] (\ontorus{240+\xac}{\x});
\draw[black edge] [torus graph behind];
\draw[disk 1 boundary,fill=white]
	[torus graph upper contour]
	[torus graph lower contour];
\draw[disk 2 boundary] [torus graph small contour];
\drawtorusgraph
\end{scope}
\end{tikzpicture}
\end{center}

We will employ this technique in order to represent \dessins{} (and, more generally, graphs whose complementary regions are disks) embedded in surfaces of positive genus.

\section{Genus-reducing combinatorial moves}\label{dessins:sc:combinatorial-moves}
\largevertices{}

As we have already anticipated, the goal of this thesis is a complete classification of the exceptional data with a partition of length $2$. \Cref{combinatorial-move:a:small-v,combinatorial-move:a:large-v} are often able to reduce the existence problem to instances with $n=3$ partitions. We will now introduce a few more combinatorial moves, which heavily exploit the machinery of \dessins{}. Unlike the aforementioned ones, these moves only work under very restrictive assumptions, namely that $n=3$ and $\len{\pi_3}=2$; on the other hand, they allow a much finer control on the partitions involved, and are often versatile enough to reduce an instance of the existence problem to the case where $\tSigma=\sphere{}$.

In this section, we will only be dealing with candidate data of the form $\DD=\datum{\surf{g}}{d}{\pi_1,\pi_2,[s,d-s]}$ with $1\le s\le d-1$. In this setting, the \RH{} formula can simply be written as
\[
\len{\pi_1}+\len{\pi_2}=d-2g.
\]
In the upcoming pictures concerning \dessins{}, we will adopt the following conventions:
\begin{itemize}
\item vertices corresponding to the entries of $\pi_1$ (or $\pi_1'$) will be colored black;
\item vertices corresponding to the entries of $\pi_2$ (or $\pi_2'$) will be colored white;
\item unnamed vertices will be labeled with their degrees;
\item the complementary disk associated to the first entry of $\pi_3$ (or $\pi_3'$) will be denoted by $D_1$ and will be colored orange;
\item the complementary disk associated to the second entry of $\pi_3$ (or $\pi_3'$) will be denoted by $D_2$ and will be colored blue.
\end{itemize}

\begin{combinatorialmoveb}\label{combinatorial-move:b:[1 1 3]}
Let $\DD=\datum{\surf{g}}{d}{\pi_1,\pi_2,[s,d-s]}$ be a candidate datum with $g\ge 1$. Assume that $[1,1,3]\subs\pi_1$. Consider the candidate datum
\[
\DD'=\datum{\surf{g-1}}{d}{\pi_1',\pi_2,[s,d-s]},
\]
where $\pi_1'=\pi_1\setminus[3]\cup[1,1,1]$. Then $\DD\cmove\DD'$.
\end{combinatorialmoveb}
\begin{proof}
Assume that $\DD'$ is realizable; by \cref{dessins:th:dessins-realizability}, there exists a \dessin{} $\Gamma'\subs\surf{g-1}$ with $\DD(\Gamma')=\DD'$. Our aim will be to construct a new \dessin{} $\Gamma\subs\surf{g}$ with $\DD(\Gamma)=\DD$; by \cref{dessins:th:dessins-realizability}, this will imply that $\DD$ is realizable as well.

Note that $[1,1,1,1,1]\subs\pi_1'$; therefore, without loss of generality, we can assume that $\Gamma$ has three black vertices of degree $1$ lying on the boundary of the complementary region $D_1$. Let us represent $D_1$ with its boundary unwound, and focus on the three black vertices of degree $1$ (see the picture labeled \tikzenumlabel{0} below). We perform the following operations on $\surf{g-1}$ and $\Gamma'$.
\begin{enumarabic}
\item Attach a tube to $\surf{g-1}$ with both endpoints in $D_1$; namely, remove two disjoint open disks contained in the interior of $D_1$, and glue a tube $S^1\times[0,1]$ along the two new boundary components. This effectively increases the genus by $1$.
\item Connect the three black vertices with two new edges, colored red in the picture; note that the orange complementary region of the new graph is still a disk.
\item Collapse the red edges.
\end{enumarabic}
\bgroup
\def\picturesetupzero#1{
\pic {cmove setting one disk=1};
\path \surfcirclepoint{d1}{-30} coordinate (1-1);
\path \surfcirclepoint{d1}{-150} coordinate (1-3);
\path \surfcirclepoint{d1}{-90} coordinate (1-2) pic{black vertex};
\ifnumcomp{#1}{=}{1}{\pic at (1-1) {black vertex};\pic at (1-3) {black vertex};\node[below right] at (1-1) {$1$};\node[below=5pt] at (1-2) {$1$};\node[below left] at (1-3) {$1$};}{}
}
\def\picturesetupone#1{
\picturesetupzero{#1}
\pic {cmove setting one disk tube=1};
\tubefill{disk 1};
}
\def\picturesetuptwo#1{
\picturesetupone{#1}
\ifnum#1=1
\tikzset{myedgestyle/.style={surf edge={##1}{red edge}}}\else
\tikzset{myedgestyle/.style={after join={##1}{d1}{white}}}\fi
\path[myedgestyle={behind}] (1-3) to[out=90,in=60,out looseness=3.1,in looseness=2] (1-2);
\path[myedgestyle={front}] let \p1=\tuberightpoint{-60},\p2=\tubeleftpoint{-120},\n1={(\x1-\x2)/2} in (1-1) to[bend left] (\p1) arc(0:180:\n1) to[bend right] (1-2);
}
\def\picturesetupthree{
\picturesetuptwo{0}
\node[below=5pt] at (1-2) {$3$};
}
\tabcolsep=0pt
\begin{longtable}{*{2}{>{\centering\arraybackslash}p{.5\linewidth}}}
\tikzenumlabel{0}&\tikzenumlabel{1}\\*
\tikzsetnextfilename{cmove-1-0}
\begin{tikzpicture}[surf picture]
\picturesetupzero{1}
\end{tikzpicture}
&
\tikzsetnextfilename{cmove-1-1}
\begin{tikzpicture}[surf picture]
\picturesetupone{1}
\end{tikzpicture}
\\\addlinespace[2em]
\tikzenumlabel{2}&\tikzenumlabel{3}\\*
\tikzsetnextfilename{cmove-1-2}
\begin{tikzpicture}[surf picture]
\picturesetuptwo{1}
\end{tikzpicture}
&
\tikzsetnextfilename{cmove-1-3}
\begin{tikzpicture}[surf picture]
\picturesetupthree
\end{tikzpicture}
\end{longtable}
\egroup
After these operations, we get a new \dessin{} $\Gamma$ embedded in $\surf{g}$. It is easy to check that $\DD(\Gamma)=\DD$, therefore $\DD$ is realizable.
\end{proof}

In the upcoming proofs, we will often represent complementary disks with their boundaries unwound, without explicitly saying so. The pictures should be clear enough to avoid any ambiguity.

\begin{combinatorialmoveb}\label{combinatorial-move:b:4 2}
Let $\DD=\datum{\surf{g}}{d}{\pi_1,\pi_2,[s,d-s]}$ be a candidate datum with $g\ge 1$. Assume that:
\begin{assumptions}
\item $2\le s\le d-2$;
\item $x\in\pi_1$ for some $x\ge 4$;
\item $2\in\pi_2$.
\end{assumptions}
Let $x_1$, $x_2$ be positive integers whose sum equals $x-2$, and consider the candidate datum
\[
\DD'=\datum{\surf{g-1}}{d-2}{\pi_1',\pi_2',[s-1,d-s-1]},
\]
where $\pi_1'=\pi_1\setminus[x]\cup[x_1,x_2]$ and $\pi_2'=\pi_2\setminus[2]$. Then $\DD\cmove\DD'$.
\end{combinatorialmoveb}
\begin{proof}
Consider a \dessin{} $\Gamma'\subs\surf{g-1}$ realizing $\DD'$. There are two cases.
\begin{sideline}{Case 1:}
the black vertex of degree $x_1$ lies on $\partial D_1$ and the one with degree $x_2$ lies on $\partial D_2$ (or vice versa). Then we perform the following operations on $\surf{g-1}$ and $\Gamma'$.
\begin{enumarabic}
\item Attach a tube to $\surf{g-1}$ with one endpoint in $D_1$ and the other one in $D_2$.
\item Add one black vertex, one white vertex and two edges as shown in the picture.
\item Draw the two red edges shown in the picture.
\item Collapse the red edges.
\end{enumarabic}
\bgroup
\def\picturesetupone#1{
\pic{cmove setting two disks};
\pic{cmove setting two disks tube};
\tubefill{white};
\path \surfcirclepoint{d1}{-90} coordinate (x1);
\path \surfcirclepoint{d2}{-90} coordinate (x2);
\ifnum#1=0
\path (x1) pic{black vertex} node[below=3pt] {$x_1$};
\path (x2) pic{black vertex} node[below=3pt] {$x_2$};
\fi
}
\def\picturesetuptwo#1{
\picturesetupone{#1}
\tubebelt{black edge}{black edge dashed}
\path \tubemiddlepoint{150} coordinate (b) pic{black vertex};
\path \tubemiddlepoint{-150} coordinate (w) pic{white vertex};
\tubeleftfill{disk 1}
\tuberightfill{disk 2}
}
\def\picturesetupthree#1{
\picturesetuptwo{#1}
\ifnum#1=0
\tikzset{myedgestyle/.style={surf edge={front}{red edge}}}\else
\tikzset{myedgestyle/.style={after join={front}{##1}{white}}}\fi
\path[myedgestyle={d1}] let \p1=\tubeleftpoint{240} in (x1) to[bend left] (\p1) to[out=90,in=180] (b);
\path[myedgestyle={d2}] let \p1=\tuberightpoint{-60} in (x2) to[bend right] (\p1) to[out=90,in=0] (b);
}
\def\picturesetupfour{
\picturesetupthree{1}
\node[above=5pt] at (b) {$x$};
}
\tabcolsep=0pt
\begin{longtable}{*{2}{>{\centering\arraybackslash}p{.5\linewidth}}}
\tikzenumlabel{1}&\tikzenumlabel{2}\\*
\tikzsetnextfilename{cmove-2-1-1}
\begin{tikzpicture}[surf picture]
\picturesetupone{0}
\end{tikzpicture}
&
\tikzsetnextfilename{cmove-2-1-2}
\begin{tikzpicture}[surf picture]
\picturesetuptwo{0}
\end{tikzpicture}
\\\addlinespace[2em]
\tikzenumlabel{3}&\tikzenumlabel{4}\\*
\tikzsetnextfilename{cmove-2-1-3}
\begin{tikzpicture}[surf picture]
\picturesetupthree{0}
\end{tikzpicture}
&
\tikzsetnextfilename{cmove-2-1-4}
\begin{tikzpicture}[surf picture]
\picturesetupfour
\end{tikzpicture}
\end{longtable}
\egroup
\end{sideline}
\begin{sideline}{Case 2:}
the two black vertices of degrees $x_1$ and $x_2$ lie (say) on $\partial D_1$. Fix an edge $e\subs\Gamma'$ which lies on the boundaries of both disks. We perform the following operations on $\surf{g-1}$ and $\Gamma'$.
\begin{enumarabic}
\item Add one black vertex and one white vertex on $e$.
\item Attach a tube to $\surf{g-1}$ with both endpoints in $D_1$.
\item Draw the two red edges shown in the picture.
\item Collapse the red edges.
\end{enumarabic}
\bgroup
\def\picturesetupone#1#2{
\pic {cmove setting one disk=1};
\path \surfcirclepoint{d1}{-30} coordinate (x2);
\path \surfcirclepoint{d1}{-90} coordinate (x1) pic{black vertex};
\ifnum#2=0
\path \surfcirclepoint{d1}{180} node[right,colored label=green] {$e$};
\tikzset{myedgestyle/.style={surf edge={behind}{green edge}}}\else
\tikzset{myedgestyle/.style={}}\fi
\path[myedgestyle,surrounding=disk 2,postaction={decorate,decoration={markings,mark=at position .25 with {\coordinate (2b);},mark=at position .75 with {\coordinate (2w);}}}] \surfcirclepath{d1}{-180}{-120};
\path (2w) pic {white vertex};
\ifnum#1=0
\pic at (x2) {black vertex};
\pic at (2b) {black vertex};
\node[below right] at (x2) {$x_2$};
\node[below=5pt] at (x1) {$x_1$};
\fi
}
\def\picturesetuptwo#1{
\picturesetupone{#1}{1}
\pic {cmove setting one disk tube=1};
\tubefill{disk 1};
}
\def\picturesetupthree#1{
\picturesetuptwo{#1}
\ifnum#1=0
\tikzset{myedgestyle/.style 2 args={surf edge={##1}{red edge}}}\else
\tikzset{myedgestyle/.style 2 args={after join={##1}{d1}{##2}}}\fi
\path[myedgestyle={behind}{disk 2}] (2b) to[out=75,in=60,out looseness=2.5,in looseness=2] (x1);
\path[myedgestyle={front}{white}] let \p1=\tuberightpoint{-60},\p2=\tubeleftpoint{-120},\n1={(\x1-\x2)/2} in (x2) to[bend left] (\p1) arc(0:180:\n1) to[bend right] (x1);
}
\def\picturesetupfour{
\picturesetupthree{1}
\node[below=5pt] at (x1) {$x$};
}
\tabcolsep=0pt
\begin{longtable}{*{2}{>{\centering\arraybackslash}p{.5\linewidth}}}
\tikzenumlabel{1}&\tikzenumlabel{2}\\*
\tikzsetnextfilename{cmove-2-2-1}
\begin{tikzpicture}[surf picture]
\picturesetupone{0}{0}
\end{tikzpicture}
&
\tikzsetnextfilename{cmove-2-2-2}
\begin{tikzpicture}[surf picture]
\picturesetuptwo{0}
\end{tikzpicture}
\\\addlinespace[2em]
\tikzenumlabel{3}&\tikzenumlabel{4}\\*
\tikzsetnextfilename{cmove-2-2-3}
\begin{tikzpicture}[surf picture]
\picturesetupthree{0}
\end{tikzpicture}
&
\tikzsetnextfilename{cmove-2-2-4}
\begin{tikzpicture}[surf picture]
\picturesetupfour
\end{tikzpicture}
\end{longtable}
\egroup
\end{sideline}

In both cases, we get a new \dessin{} $\Gamma$ embedded in $\surf{g}$. It is easy to check that $\Gamma$ realizes the candidate datum $\DD$.
\end{proof}

\begin{combinatorialmoveb}\label{combinatorial-move:b:[3 3] [2 2]}
Let $\DD=\datum{\surf{g}}{d}{\pi_1,\pi_2,[s,d-s]}$ be a candidate datum with $g\ge 1$. Assume that:
\begin{assumptions}
\item $3\le s\le d-3$;
\item $[x,y]\subs\pi_1$ for some $x\ge 3$, $y\ge 3$;
\item $[2,2]\subs\pi_2$.
\end{assumptions}
Consider the candidate datum
\[
\DD'=\datum{\surf{g-1}}{d-4}{\pi_1',\pi_2',[s-2,d-s-2]},
\]
where $\pi_1'=\pi_1\setminus[x,y]\cup[x-2,y-2]$ and $\pi_2'=\pi_2\setminus[2,2]$. Then $\DD\cmove\DD'$.
\end{combinatorialmoveb}
\begin{proof}
Consider a \dessin{} $\Gamma'\subs\surf{g-1}$ realizing $\DD'$. There are two cases.
\begin{sideline}{Case 1:}
the black vertex of degree $x-2$ lies on $\partial D_1$ and the one with degree $y-2$ lies on $\partial D_2$ (or vice versa). Then we perform the following operations on $\surf{g-1}$ and $\Gamma'$.
\begin{enumarabic}
\item Attach a tube to $\surf{g-1}$ with one endpoint in $D_1$ and the other one in $D_2$.
\item Add two black vertices, two white vertices and four edges as shown in the picture.
\item Draw the two red edges shown in the picture.
\item Collapse the red edges.
\end{enumarabic}
\bgroup
\def\picturesetupone#1{
\pic{cmove setting two disks};
\pic{cmove setting two disks tube};
\tubefill{white};
\path \surfcirclepoint{d1}{-90} coordinate (x-2);
\path \surfcirclepoint{d2}{-90} coordinate (y-2);
\ifnum#1=0
\path (x-2) pic{black vertex} node[below=3pt] {$x-2$};
\path (y-2) pic{black vertex} node[below=3pt] {$y-2$};
\fi
}
\def\picturesetuptwo#1{
\picturesetupone{#1}
\tubebelt{black edge}{black edge dashed}
\path \tubemiddlepoint{90} coordinate (w1) pic{white vertex};
\path \tubemiddlepoint{135} coordinate (b1) pic{black vertex};
\path \tubemiddlepoint{180} coordinate (w2) pic {white vertex};
\path \tubemiddlepoint{225} coordinate (b2) pic {black vertex};
\tubeleftfill{disk 1}
\tuberightfill{disk 2}
}
\def\picturesetupthree#1{
\picturesetuptwo{#1}
\ifnum#1=0
\tikzset{myedgestyle/.style={surf edge={front}{red edge}}}\else
\tikzset{myedgestyle/.style={after join={front}{##1}{white}}}\fi
\path[myedgestyle={d1}] let \p1=\tubeleftpoint{240} in (x-2) to[bend left] (\p1) to[out=90,in=-170] (b1);
\path[myedgestyle={d2}] let \p1=\tuberightpoint{-60} in (y-2) to[bend right] (\p1) to[out=90,in=10] (b2);
}
\def\picturesetupfour{
\picturesetupthree{1}
\node[above left] at (b1) {$x$};
\node[below=3pt] at (b2) {$y$};
}
\tabcolsep=0pt
\begin{longtable}{*{2}{>{\centering\arraybackslash}p{.5\linewidth}}}
\tikzenumlabel{1}&\tikzenumlabel{2}\\*
\tikzsetnextfilename{cmove-3-1-1}
\begin{tikzpicture}[surf picture]
\picturesetupone{0}{0}
\end{tikzpicture}
&
\tikzsetnextfilename{cmove-3-1-2}
\begin{tikzpicture}[surf picture]
\picturesetuptwo{0}
\end{tikzpicture}
\\\addlinespace[2em]
\tikzenumlabel{3}&\tikzenumlabel{4}\\*
\tikzsetnextfilename{cmove-3-1-3}
\begin{tikzpicture}[surf picture]
\picturesetupthree{0}
\end{tikzpicture}
&
\tikzsetnextfilename{cmove-3-1-4}
\begin{tikzpicture}[surf picture]
\picturesetupfour
\end{tikzpicture}
\end{longtable}
\egroup
\end{sideline}
\begin{sideline}{Case 2:}
the two black vertices of degrees $x-2$ and $y-2$ lie (say) on $\partial D_1$. Fix an edge $e\subs\Gamma'$ which lies on the boundaries of both disks. We perform the following operations on $\surf{g-1}$ and $\Gamma'$.
\begin{enumarabic}
\item Add two black vertices and two white vertices on $e$.
\item Attach a tube to $\surf{g-1}$ with both endpoints in $D_1$.
\item Draw the two red edges shown in the picture.
\item Collapse the red edges.
\end{enumarabic}
\bgroup
\def\picturesetupone#1#2{
\pic {cmove setting one disk=1};
\path \surfcirclepoint{d1}{-30} coordinate (x2) pic{black vertex};
\path \surfcirclepoint{d1}{-90} coordinate (x1) pic{black vertex};
\ifnum#2=0
\path \surfcirclepoint{d1}{150} node[below right,colored label=green] {$e$};
\tikzset{myedgestyle/.style={surf edge={behind}{green edge}}}\else
\tikzset{myedgestyle/.style={}}\fi
\path[myedgestyle,surrounding=disk 2,postaction={decorate,decoration={markings,mark=at position .2 with {\coordinate (2b-1);},mark=at position .4 with {\coordinate (2w-1);},,mark=at position .6 with {\coordinate (2b-2);},mark=at position .8 with {\coordinate (2w-2);}}}] \surfcirclepath{d1}{150}{240};
\path (2w-1) pic {white vertex} (2w-2) pic {white vertex};
\ifnum#1=0
\pic at (2b-1) {black vertex};
\pic at (2b-2) {black vertex};
\node[below right] at (x2) {$y-2$};
\node[below=5pt] at (x1) {$x-2$};
\fi
}
\def\picturesetuptwo#1{
\picturesetupone{#1}{1}
\pic {cmove setting one disk tube=1};
\tubefill{disk 1};
}
\def\picturesetupthree#1{
\picturesetuptwo{#1}
\ifnum#1=0
\tikzset{myedgestyle/.style={surf edge={##1}{red edge}}}\else
\tikzset{myedgestyle/.style={after join={##1}{d1}{disk 2}}}\fi
\path[myedgestyle={behind}] (2b-1) to[out=30,in=80,looseness=1.8] (x1);
\path[myedgestyle={front}] let \p1=\tubeleftpoint{-120},\p2=\tuberightpoint{-60},\n1={(\x2-\x1)/2} in (2b-2) to[bend right] (\p1) arc(180:0:\n1) to[bend right] (x2);
}
\def\picturesetupfour{
\picturesetupthree{1}
\node[below=5pt] at (x1) {$x$};
\node[below right] at (x2) {$y$};
}
\tabcolsep=0pt
\begin{longtable}{*{2}{>{\centering\arraybackslash}p{.5\linewidth}}}
\tikzenumlabel{1}&\tikzenumlabel{2}\\*
\tikzsetnextfilename{cmove-3-2-1}
\begin{tikzpicture}[surf picture]
\picturesetupone{0}{0}
\end{tikzpicture}
&
\tikzsetnextfilename{cmove-3-2-2}
\begin{tikzpicture}[surf picture]
\picturesetuptwo{0}
\end{tikzpicture}
\\\addlinespace[2em]
\tikzenumlabel{3}&\tikzenumlabel{4}\\*
\tikzsetnextfilename{cmove-3-2-3}
\begin{tikzpicture}[surf picture]
\picturesetupthree{0}
\end{tikzpicture}
&
\tikzsetnextfilename{cmove-3-2-4}
\begin{tikzpicture}[surf picture]
\picturesetupfour
\end{tikzpicture}
\end{longtable}
\egroup
\end{sideline}

In both cases, we get a new \dessin{} $\Gamma$ embedded in $\surf{g}$. It is easy to check that $\Gamma$ realizes the candidate datum $\DD$.
\end{proof}

\begin{combinatorialmoveb}\label{combinatorial-move:b:4 3}
Let $\DD=\datum{\surf{g}}{d}{\pi_1,\pi_2,[s,d-s]}$ be a candidate datum with $g\ge 1$. Assume that:
\begin{assumptions}
\item $2\le s\le d-2$;
\item $x\in\pi_1$ for some $x\ge 4$;
\item $y\in\pi_2$ for some $y\ge 3$.
\end{assumptions}
Consider the candidate datum
\[
\DD'=\datum{\surf{g-1}}{d-2}{\pi_1',\pi_2',[s-1,d-s-1]},
\]
where $\pi_1'=\pi_1\setminus[x]\cup[x-2]$ and $\pi_2'=\pi_2\setminus[y]\cup[y-2]$. Then $\DD\cmove\DD'$.
\end{combinatorialmoveb}
\begin{proof}
Consider a \dessin{} $\Gamma'\subs\surf{g-1}$ realizing $\DD'$. Let $u$ be the black vertex of degree $x-2$, and let $v$ be the white vertex of degree $y-2$; there are two cases.
\begin{sideline}{Case 1:}
$u$ lies on $\partial D_1$ and $v$ lies on $\partial D_2$ (or vice versa). Then we perform the following operations on $\surf{g-1}$ and $\Gamma'$.
\begin{enumarabic}
\item Attach a tube to $\surf{g-1}$ with one endpoint in $D_1$ and the other in $D_2$.
\item Add one black vertex, one white vertex and two edges as shown in the picture.
\item Draw the two red edges shown in the picture.
\item Collapse the red edges.
\end{enumarabic}

\bgroup
\def\picturesetupone#1{
\pic{cmove setting two disks};
\pic{cmove setting two disks tube};
\tubefill{white};
\path \surfcirclepoint{d1}{-90} coordinate (x1);
\path \surfcirclepoint{d2}{-90} coordinate (x2);
\ifnum#1=0
\path (x1) pic{black vertex} node[below=3pt] {$x-2$};
\path (x2) pic{white vertex} node[below=3pt] {$y-2$};
\fi
}
\def\picturesetuptwo#1{
\picturesetupone{#1}
\tubebelt{black edge}{black edge dashed}
\path \tubemiddlepoint{150} coordinate (b) pic{black vertex};
\path \tubemiddlepoint{-150} coordinate (w) pic{white vertex};
\tubeleftfill{disk 1}
\tuberightfill{disk 2}
}
\def\picturesetupthree#1{
\picturesetuptwo{#1}
\ifnum#1=0
\tikzset{myedgestyle/.style={surf edge={front}{red edge}}}\else
\tikzset{myedgestyle/.style={after join={front}{##1}{white}}}\fi
\path[myedgestyle={d1}] let \p1=\tubeleftpoint{240} in (x1) to[bend left] (\p1) to[out=90,in=180] (b);
\path[myedgestyle={d2}] let \p1=\tuberightpoint{-60} in (x2) to[bend right] (\p1) to[out=90,in=10] (w);
}
\def\picturesetupfour{
\picturesetupthree{1}
\node[above=5pt] at (b) {$x$};
\node[below=5pt] at (w) {$y$};
}
\tabcolsep=0pt
\begin{longtable}{*{2}{>{\centering\arraybackslash}p{.5\linewidth}}}
\tikzenumlabel{1}&\tikzenumlabel{2}\\*
\tikzsetnextfilename{cmove-4-1-1}
\begin{tikzpicture}[surf picture]
\picturesetupone{0}
\end{tikzpicture}
&
\tikzsetnextfilename{cmove-4-1-2}
\begin{tikzpicture}[surf picture]
\picturesetuptwo{0}
\end{tikzpicture}
\\\addlinespace[2em]
\tikzenumlabel{3}&\tikzenumlabel{4}\\*
\tikzsetnextfilename{cmove-4-1-3}
\begin{tikzpicture}[surf picture]
\picturesetupthree{0}
\end{tikzpicture}
&
\tikzsetnextfilename{cmove-4-1-4}
\begin{tikzpicture}[surf picture]
\picturesetupfour
\end{tikzpicture}
\end{longtable}
\egroup
After these operations, we get a new \dessin{} $\Gamma$ embedded in $\surf{g}$. It is easy to check that $\Gamma$ realizes the candidate datum $\DD$.
\end{sideline}
\begin{sideline}{Case 2:}
neither $u$ nor $v$ lie (say) on $\partial D_2$; analyzing this case will be more involved than usual. We will say that an edge is \emph{shared} if it is not enveloped by $D_1$ or by $D_2$; in other words, an edge is shared if it appears exactly once in $\partial D_1$. Our goal will be to prove that we can add two vertices of degree $2$ -- one black and one white -- on a shared edge, in such a way that the vertices 
\begin{center}
\tikzsetnextfilename{cmove-4-2-boundary-order}
\begin{tikzpicture}[graph picture]
\path[graph edge={below}{black edge}] (0,0) -- (0.5,0) (1.5,0) -- (2,0) (4,0) -- (4.5,0) (5.5,0) -- (6,0);
\path[graph edge={below}{black edge dashed}] (0.5,0) -- (1.5,0) (4.5,0) -- (5.5,0);
\path[graph edge={below}{green edge}] (2,0) -- (4,0);
\path[ab/.style={above=3pt}] (0,0) pic{black vertex} node[ab] {$u$} (2.5,0) pic{white vertex} node[ab] {$2$} (3.5,0) pic{black vertex} node[ab] {$2$} (6,0) pic{white vertex} node[ab] {$v$};
\end{tikzpicture}
\end{center}
appear in this order (either clockwise or counterclockwise) on $\partial D_1$. Let us unwind the boundary of $D_1$; since $u$ does not lie on the boundary of $D_2$, it appears exactly $x-2\ge 2$ times on $\partial D_1$; similarly, $v$ appears exactly $y-2$ times.
\def\myradius{1.3cm}
If there is a shared edge $e$ such that $\{u,v,u,e\}$ appear in this order (but not necessarily consecutively) on $\partial D_1$, then, by adding a black vertex and a white vertex on $e$, we immediately get the desired result.
\begin{center}
\def\picturesetupbase{
\begin{pgfonlayer}{graph edge below}
\fill[disk 1,postaction={draw,surf boundary}] circle(1);
\end{pgfonlayer}
}
\def\picturesetup{
\picturesetupbase
\path (0:1) pic{black vertex} node[right] {$u$};
\path (180:1) pic {black vertex} node[left] {$u$};
\path (-90:1) pic {white vertex} node[below] {$v$};
\path[graph edge={below}{green edge}] (45:1) arc (45:135:1);
}
\tikzsetnextfilename{cmove-4-2-1}
\begin{tikzpicture}[graph picture,x={(\myradius,0)},y={(0,\myradius)}]
\begin{scope}[shift={(-\myradius*2.5,0)}]
\picturesetup
\node[above=2pt,colored label=green] at (90:1) {$e$};
\end{scope}
\pic {my fancy arrow};
\begin{scope}[shift={(\myradius*2.5,0)}]
\picturesetup
\path (60:1) pic{black vertex} node[above right] {$2$};
\path (120:1) pic{white vertex} node[above left] {$2$};
\end{scope}
\end{tikzpicture}
\end{center}

Otherwise, we have to modify the \dessin{} $\Gamma'$; consider the pictures below.
\begin{enumarabic}
\item We are in the following situation: there is a contiguous segment $A$ of $\partial D_1$ that contains all the occurrences of $u$, and does not contain any occurrence of $v$ or any shared edge. Similarly, there is a segment $B$ of $\partial D_1$ that contains all the occurrences of $v$, no occurrence of $u$ and no shared edge. Note that $u$ and $v$ are never adjacent to a shared edge, since by assumption they do not lie on $\partial D_2$.
\item Choose an orientation of $\partial D_1$ (counterclockwise in the picture) and consider the first occurrence of $u$ in $A$; let $\alpha$ be the edge immediately afterwards in $\partial D_1$. Since $\alpha$ is not shared, it must appear once more on the combinatorial boundary of $D_1$, with the opposite orientation. Note that $\alpha$ cannot occur immediately before the first appearance of $u$, otherwise $u$ would have degree $1$, therefore it will occur somewhere else on $A$.
\item Consider the first occurrence of $v$ in $B$; let $\beta$ be the edge immediately before in $\partial D_1$. Since $\beta$ is not shared, it will also occur somewhere else on the combinatorial perimeter of $D_1$.
\item Let $a$ be the other endpoint of $\alpha$, and let $b$ be the other endpoint of $\beta$. Erase the edges $\alpha$ and $\beta$ and draw two new ones, connecting $u$ to $v$ and $a$ to $b$ as shown in the picture. It is now easy to see that the orange region is still a complementary disk, and that its perimeter has not changed. Moreover, by traveling along its boundary, we encounter $\{u,v,u\}$ in this order, without any shared edges in between (recall that there are no shared edges on $B$); since there must be at least one shared edge $e$ on the boundary of the new orange region, we find $\{u,v,u,e\}$ in this order; as explained above, adding one black vertex and one white vertex on $e$ gives the desired result.
\end{enumarabic}
\def\picturesetupbase{
\path[use as bounding box] (2,1.4) rectangle (-2,-1.4);
\begin{pgfonlayer}{graph edge below}
\fill[disk 1,postaction={draw,surf boundary}] circle(1);
\end{pgfonlayer}
\path (120:1) pic{black vertex} node[above left] {$u$};
\path (210:1) pic{black vertex} node[left=3pt] {$u$};
\path (30:1) pic{white vertex} node[above right] {$v$};
\path (-30:1) pic{white vertex} node[below right] {$v$};
}
\def\picturesetupalpha{
\tikzset{alpha/.style={graph edge={below}{violet,line width=\edgelinewidth,middle arrow}},
middle arrow/.style={postaction={decorate,decoration={markings,mark=at position .5 with{\arrow[xshift=3.3pt]{Stealth[]}}}}}}
\begin{pgfonlayer}{graph edge below}
\draw[alpha] (120:1) arc (120:150:1) node[midway,auto,colored label=violet] {$\alpha$} pic{white vertex};
\draw[alpha] (210:1) arc (210:180:1) node[midway,auto,swap,colored label=violet] {$\alpha$}  pic{white vertex};
\end{pgfonlayer}
}
\def\picturesetupbeta{
\tikzset{beta/.style={graph edge={below}{teal,line width=\edgelinewidth,middle arrow}}}
\begin{pgfonlayer}{graph edge below}
\draw[beta] (-60:1) pic{black vertex} arc (-60:-30:1) node[midway,auto,colored label=teal] {$\beta$} ;
\draw[beta] (60:1) pic{black vertex} arc (60:30:1) node[midway,auto,swap,colored label=teal] {$\beta$};
\end{pgfonlayer}
}
\tabcolsep=0pt
\begin{longtable}{*{2}{>{\centering\arraybackslash}p{.5\linewidth}}}
\tikzenumlabel{1}&\tikzenumlabel{2}\\*
\begin{tikzpicture}[graph picture,x={(\myradius,0)},y={(0,\myradius)},bar/.tip={Bar[width=3pt]}]
\picturesetupbase
\path (240:1) pic{black vertex} node[below left] {$u$};
\draw[violet,{bar}-{bar}] (120:0.85) arc(120:240:0.85) node[midway,right,colored label=violet] {$A$};
\draw[teal,{bar}-{bar}] (30:0.85) arc(30:-30:0.85) node[midway,left,colored label=teal] {$B$};
\end{tikzpicture}
&
\begin{tikzpicture}[graph picture,x={(\myradius,0)},y={(0,\myradius)}]
\picturesetupbase
\picturesetupalpha
\end{tikzpicture}
\\\addlinespace[2em]
\tikzenumlabel{3}&\tikzenumlabel{4}\\*
\begin{tikzpicture}[graph picture,x={(\myradius,0)},y={(0,\myradius)}]
\picturesetupbase
\picturesetupalpha
\picturesetupbeta
\end{tikzpicture}
&
\begin{tikzpicture}[graph picture,x={(\myradius,0)},y={(0,\myradius)}]
\picturesetupbase
\pic at (60:1) {black vertex};
\pic at (-60:1) {black vertex};
\pic at (180:1) {white vertex};
\pic at (150:1) {white vertex};
\begin{pgfonlayer}{graph edge below}
\foreach \a/\b in {135/195,45/-45} {
\draw[disk 1,line width={2*sin(15)*1cm}] (\a:{cos(15)}) to[out=\a,in=\b,looseness=3] (\b:{cos(15)});
}
\foreach \a/\col in {-60/teal,30/teal,120/violet,180/violet} {
\path[decorate,decoration={markings,mark=between positions .2 and .85 step .2 with {\draw[\col,scale=.75] (45:1pt) -- (-135:1pt) (135:1pt) -- (-45:1pt);}}] (\a:1) arc (\a:\a+30:1);
}
\end{pgfonlayer}
\path[graph edge={below}{black edge}] (120:1) to[bend right] (30:1);
\path[graph edge={below}{black edge}] (180:1) to[bend left] (-60:1);
\node[left=3pt] at (150:1) {$a$};
\node[left=3pt] at (180:1) {$a$};
\node[above=3pt] at (60:1) {$b$};
\node[below=3pt] at (-60:1) {$b$};
\end{tikzpicture}
\end{longtable}

Once we have added the two vertices of degree $2$ on a shared edge, we are in the situation depicted below in \tikzenumlabel{0}. We then perform the following operations.
\begin{enumarabic}
\item Attach a tube to $\surf{g-1}$ with both endpoints in $D_1$.
\item Draw the two red edges shown in the picture.
\item Collapse the red edges.
\end{enumarabic}

\bgroup
\def\picturesetupzero#1#2{
\pic {cmove setting one disk=1};
\path \surfcirclepoint{d1}{-30} coordinate (x2) pic{white vertex};
\path \surfcirclepoint{d1}{-90} coordinate (x1) pic{black vertex};
\ifnum#2=0
\tikzset{myedgestyle/.style={surf edge={behind}{green edge}}}\else
\tikzset{myedgestyle/.style={}}\fi
\path[myedgestyle,surrounding=disk 2,postaction={decorate,decoration={markings,mark=at position .25 with {\coordinate (2b);},mark=at position .75 with {\coordinate (2w);}}}] \surfcirclepath{d1}{-180}{-120};
\ifnum#1=0
\pic at (2w) {white vertex};
\pic at (2b) {black vertex};
\node[below right] at (x2) {$y-2$};
\node[below=5pt] at (x1) {$x-2$};
\fi
}
\def\picturesetupone#1{
\picturesetupzero{#1}{1}
\pic {cmove setting one disk tube=1};
\tubefill{disk 1};
}
\def\picturesetuptwo#1{
\picturesetupone{#1}
\ifnum#1=0
\tikzset{myedgestyle/.style={surf edge={##1}{red edge}}}\else
\tikzset{myedgestyle/.style={after join={##1}{d1}{disk 2}}}\fi
\path[myedgestyle={behind}] (2b) to[out=75,in=60,out looseness=2.5,in looseness=2] (x1);
\path[myedgestyle={front}] let \p1=\tubeleftpoint{-120},\p2=\tuberightpoint{-60},\n1={(\x2-\x1)/2} in (2w) to[bend left=15] (\p1) arc(180:0:\n1) to[bend right] (x2);
}
\def\picturesetupthree{
\picturesetuptwo{1}
\node[below=5pt] at (x1) {$x$};
\node[below right] at (x2) {$y$};
}
\tabcolsep=0pt
\begin{longtable}{*{2}{>{\centering\arraybackslash}p{.5\linewidth}}}
\tikzenumlabel{0}&\tikzenumlabel{1}\\*
\tikzsetnextfilename{cmove-4-2-2-0}
\begin{tikzpicture}[surf picture]
\picturesetupzero{0}{0}
\end{tikzpicture}
&
\tikzsetnextfilename{cmove-4-2-2-1}
\begin{tikzpicture}[surf picture]
\picturesetupone{0}
\end{tikzpicture}
\\\addlinespace[2em]
\tikzenumlabel{2}&\tikzenumlabel{3}\\*
\tikzsetnextfilename{cmove-4-2-2-2}
\begin{tikzpicture}[surf picture]
\picturesetuptwo{0}
\end{tikzpicture}
&
\tikzsetnextfilename{cmove-4-2-2-3}
\begin{tikzpicture}[surf picture]
\picturesetupthree
\end{tikzpicture}
\end{longtable}
\egroup
Finally, we get a new \dessin{} $\Gamma$ embedded in $\surf{g}$. It is easy to check that $\Gamma$ realizes the candidate datum $\DD$.\qedhere
\end{sideline}
\end{proof}

We will make extensive use of these combinatorial moves in the next chapter, where a full classification of the exceptional data with $n=3$ and $\len{\pi_3}=2$ will be provided (see \cref{short-partition:th:realizability-on-sphere-n-3,short-partition:th:realizability-on-torus-n-3,short-partition:th:realizability-on-higher-genus-n-3}).


\section{Realizability by \texorpdfstring{\dessins{}}{dessins d'enfant}}

We conclude this chapter by proving the realizability of a few families of candidate data by means of \dessins{}; these results, while interesting by themselves, will be useful in the next chapter for addressing some cases which are not covered by the combinatorial moves we have introduced.

\begin{proposition}\label{dessins:th:special-case-[2 d-2]}
Let $\DD=\datum{\surf{g}}{d}{\pi_1,\pi_2,[2,d-2]}$ be a candidate datum. Assume that:
\begin{assumptions}
\item $[x,y]\subs\pi_1$ for some $x\ge 2$, $y\ge 3$;
\item $[2,2]\subs\pi_2$.
\end{assumptions}
Then $\DD$ is realizable.
\end{proposition}
\begin{proof}
We will show that, under the stated assumptions, there is a combinatorial move\footnote{To be precise, when $d=5$ the tuple $\DD'$ is not a combinatorial datum according to our definition. However, the only candidate datum $\DD$ of degree $5$ satisfying the assumptions is $\DD=\datum{\sphere{}}{5}{[2,3],[1,2,2],[2,3]}$, whose realizability can be easily checked by hand with a suitable \dessin{}.}
\[
\DD\cmove\DD'=\datum{\surf{g}}{d-4}{\pi_1\setminus[x,y]\cup[x+y-4],\pi_2\setminus[2,2],[d-4]}.
\]
Note that $\DD'$ is realizable by \cref{monodromy:th:sphere-[d]}; let $\Gamma'\subs\surf{g}$ be a \dessin{} realizing it. We perform the following operations on $\Gamma'$.
\begin{enumarabic}
\item Consider the black vertex of degree $x+y-4$ and split it into two vertices of degrees $x-2$ and $y-2$.
\item Add two white vertices and four edges as shown in the picture. This creates a new complementary disk with perimeter $4$.
\end{enumarabic}
\bgroup
\def\myscale{1}
\def\picturesetupbackground{
\clip circle(1.5 and 1);
\clip circle(1.5 and 1);
\fill[disk 2] circle (2);
}
\tikzset{myleftvertex/.pic={
\path[black edge] (0,0) to[bend right] (180:1.5) (0,0) to[bend right=10] (195:1.5) (0,0) to[bend left] (210:1.5);
},
myrightvertex/.pic={
\path[black edge] (0,0) to[bend left] (15:1.5) (0,0) to[bend right] (-15:1.5);
}
}
\tabcolsep=0pt
\begin{longtable}{>{\centering\arraybackslash}p{.6\linewidth}>{\centering\arraybackslash}p{.4\linewidth}}
\tikzenumlabel{1}&\tikzenumlabel{2}\\*\addlinespace[1em]
\tikzsetnextfilename{dessin-[2_d-2]-1}
\begin{tikzpicture}[graph picture,x={(\myscale,0)},y={(0,\myscale)}]
\begin{pgfonlayer}{graph edge below}
\begin{scope}[shift={(-2.2,0)}]
\picturesetupbackground
\path (0,0) pic{myleftvertex} pic{myrightvertex} pic{black vertex} node[above=10pt] {$x+y-4$};
\end{scope}
\pic{my fancy arrow};
\begin{scope}[shift={(2.2,0)}]
\picturesetupbackground
\path (-.7,0) pic {myleftvertex} pic {black vertex} node[above right] {$x-2$} (.7,0) pic {myrightvertex} pic {black vertex} node[below left] {$y-2$};
\end{scope}
\end{pgfonlayer}
\end{tikzpicture}
&
\tikzsetnextfilename{dessin-[2_d-2]-2}
\begin{tikzpicture}[graph picture,x={(\myscale,0)},y={(0,\myscale)}]
\begin{pgfonlayer}{graph edge below}
\picturesetupbackground
\path (-.7,0) pic {myleftvertex} pic {black vertex} node[above=3pt] {$x$} (.7,0) pic {myrightvertex} pic {black vertex} node[below=3pt] {$y$};
\path[preaction={fill=disk 1},black edge] (-.7,0) to[bend left=60] pic[pos=.5] {white vertex} node[pos=.5,above=2pt] {$2$} (.7,0) to[bend left=60] pic[pos=.5] {white vertex} node[pos=.5,below left] {$2$} (-.7,0);
\end{pgfonlayer}
\end{tikzpicture}
\end{longtable}
\egroup
After these operations, we get a new \dessin{} $\Gamma$ embedded in $\surf{g}$. It is easy to check that $\Gamma$ realizes the candidate datum $\DD$.
\end{proof}


\begin{proposition}\label{dessins:th:special-families}
The following families of candidate data are realizable for every $g\ge 2$.
\begin{enumarabic}
\item $\datum{\surf{g}}{6g}{[3,\ldots,3],[3,\ldots,3],[s,6g-s]}$.
\item $\datum{\surf{g}}{6g+2}{[2,3,\ldots,3],[2,3\ldots,3],[s,6g+2-s]}$.
\item $\datum{\surf{g}}{6g+3}{[3,\ldots,3],[1,2,3,\ldots,3],[s,6g+3-s]}$.
\item $\datum{\surf{g}}{6g+4}{[1,3,\ldots,3],[1,3,\ldots,3],[s,6g+4-s]}$.
\item $\datum{\surf{g}}{6g+6}{[1,2,3,\ldots,3],[1,2,3,\ldots,3],[s,6g+6-s]}$.
\end{enumarabic}
\end{proposition}
\begin{proof}
\def\env#1{\underline{#1}}
In the scope of this proof we define an \emph{augmented combinatorial datum} as a combinatorial datum $\DD=\datum{\surf{g}}{d}{\pi_1,\pi_2,\pi_3}$ where some distinguished elements of $\pi_3$ are called \emph{enveloping}; we will underline the enveloping elements in order to recognize them. An augmented combinatorial datum $\DD$ is \emph{realizable} if there exists a \dessin{} $\Gamma$ with $\DD(\Gamma)=\DD$ such that, for every complementary disk $D$ corresponding to an enveloping element of $\pi_3$, there is an edge of $\Gamma$ which is enveloped by $D$.
\begin{sideline}{Step 1.}
The augmented datum
\[
\DD=\datum{\surf{2}}{12}{[3,3,3,3],[3,3,3,3],\pi_3}
\]
is realizable for $\pi_3\in\{[1,\env{11}],[2,\env{10}],[\env{3},\env{9}],[\env{4},\env{8}],[\env{5},\env{7}],[\env{6},\env{6}]\}$. The following pictures display \dessins{} realizing each of these augmented data; as usual, the disk associated to the first element of $\pi_3$ is colored orange and the other one is colored blue; enveloped edges are drawn in the same color as the corresponding disk. In order to represent these graphs embedded in $\surf{2}$, we employ the ``fat graph'' trick described in \cref{dessins:sc:unwind-join-fatten}.

\bgroup

\smallvertices{}
\def\myradius{1.33}
\tikzset{
contour edge settings/quick/.style 2 args={/tikz/evaluate={\col=ifthenelse(#1==#2,"disk #1 boundary enveloped","black edge");},left=disk #1 boundary,right=disk #2 boundary,edge=\col}
}
\def\draweightvertices{
\path[use as bounding box] (-1.8,-1.5) rectangle (1.8,2);
\path (0:\myradius) coordinate (w1) pic{white vertex};
\path (90:\myradius) coordinate (w2) pic{white vertex};
\path (180:\myradius) coordinate (w3) pic{white vertex};
\path (270:\myradius) coordinate (w4) pic{white vertex};
\path (45:\myradius) coordinate (b1) pic{black vertex};
\path (135:\myradius) coordinate (b2) pic{black vertex};
\path (225:\myradius) coordinate (b3) pic{black vertex};
\path (315:\myradius) coordinate (b4) pic{black vertex};
}

\tabcolsep=0pt
\centering
\begin{longtable}{*{3}{>{\centering\arraybackslash}p{.33\linewidth}}}
$\pi_3=[1,\env{11}]$&$\pi_3=[2,\env{10}]$&$\pi_3=[\env{3},\env{9}]$
\\*
\tikzsetnextfilename{dessin-special-families-small-1}
\begin{tikzpicture}[graph picture]
\draweightvertices
\path[contour edge={quick={2}{2},left bcap=90}] (w4) to[bend left] (b3);
\path[contour edge={quick={2}{2}}] (b3) to[bend left] (w3);
\path[contour edge={quick={2}{1}}] (w3) to[bend left] (b2);
\path[contour edge={quick={2}{2}}] (b2) to[bend left] (w2);
\path[contour edge={quick={2}{2},left ecap=90}] (w2) to[bend left] (b1);
\path[contour edge={quick={2}{2}}] (w4) to[out=110,looseness=2.2,in=170] (b4);
\path[contour edge={quick={2}{2}}] (b3) to[out=60,out looseness=1.5,in=-135] (w1);
\path[contour edge={quick={2}{2}}] (b4) to[bend right=90] (w1);
\path[contour edge={above,quick={2}{2}}] (w1) to[out=140,in=120,looseness=2.2] (b4);
\path[contour edge={above,quick={2}{2},end=0.85}] (w4) to[out=10,in=-160,out looseness=1.3,in looseness=2] (b1);
\path[contour edge={quick={2}{2}}] (b1) to[out=-90,in=-90,looseness=2.2] (w2);
\path[contour edge={quick={1}{2}}] (w3) to[bend right=90,looseness=2.2] (b2);
\end{tikzpicture}
&
\tikzsetnextfilename{dessin-special-families-small-2}
\begin{tikzpicture}[graph picture]
\draweightvertices
\path[contour edge={quick={1}{2}}] (w4) to[bend left] (b3);
\path[contour edge={quick={1}{2}}] (b3) to[bend left] (w3);
\path[contour edge={quick={1}{2}}] (w3) to[bend left] (b2);
\path[contour edge={quick={1}{2}}] (b2) to[bend left=110] (w4);
\path[contour edge={above,quick={2}{2}}] (w3) to[out=-20,out looseness=1.4,in=-180] (b1);
\path[contour edge={above,quick={2}{2},end=0.75}] (b2) to[out=-60,in=-120,looseness=2.2] (w2);
\path[contour edge={quick={2}{2}}] (w2) to[bend left=90] (b1);
\path[contour edge={quick={2}{2}}] (b1) to[out=-90,in=-90,looseness=2.2] (w2);
\path[contour edge={above,quick={2}{2}}] (w4) to[out=110,looseness=2.2,in=170] (b4);
\path[contour edge={above,quick={2}{2}}] (b3) to[out=60,out looseness=1.4,in=-135] (w1);
\path[contour edge={quick={2}{2}}] (b4) to[bend right=90] (w1);
\path[contour edge={quick={2}{2}}] (w1) to[out=140,in=120,looseness=2.2] (b4);
\end{tikzpicture}
&
\tikzsetnextfilename{dessin-special-families-small-3}
\begin{tikzpicture}[graph picture]
\draweightvertices
\path[contour edge={quick={2}{2},left bcap=90}] (w4) to[bend left] (b3);
\path[contour edge={quick={2}{2}}] (b3) to[bend left] (w3);
\path[contour edge={quick={2}{1}}] (w3) to[bend left] (b2);
\path[contour edge={quick={1}{1},left ecap=90}] (b2) to[out=-40,out looseness=2,in=-170] (w2);
\path[contour edge={quick={1}{2},left ecap=90}] (w2) to[bend left] (b1);
\path[contour edge={quick={2}{2}}] (w4) to[out=110,looseness=2.2,in=170] (b4);
\path[contour edge={quick={2}{2}}] (b3) to[out=60,out looseness=1.5,in=-135] (w1);
\path[contour edge={quick={2}{2}}] (b4) to[bend right=90] (w1);
\path[contour edge={above,quick={2}{2}}] (w1) to[out=140,in=120,looseness=2.2] (b4);
\path[contour edge={above,quick={2}{2},end=0.85}] (w4) to[out=10,in=-160,out looseness=1.3,in looseness=2] (b1);
\path[contour edge={quick={1}{2}}] (b1) to[out=-90,in=-90,looseness=2.2] (w2);
\path[contour edge={above,quick={1}{2}}] (w3) to[out=-20,in=50,looseness=3.5] (b2);
\end{tikzpicture}
\\\addlinespace[2em]
$\pi_3=[\env{4},\env{8}]$&$\pi_3=[\env{5},\env{7}]$&$\pi_3=[\env{6},\env{6}]$
\\*
\tikzsetnextfilename{dessin-special-families-small-4}
\begin{tikzpicture}[graph picture]
\draweightvertices
\path[contour edge={quick={1}{2}}] (w4) to[bend left] (b3);
\path[contour edge={quick={1}{2}}] (b3) to[bend left] (w3);
\path[contour edge={quick={1}{2}}] (w3) to[bend left] (b2);
\path[contour edge={quick={1}{2}}] (b2) to[out=-30,in=0] (w4);
\path[contour edge={above,quick={2}{2}}] (w3) to[out=10,in=-180] (b1);
\path[contour edge={quick={1}{1}}] (b2) to[bend left] (w2);
\path[contour edge={quick={1}{2},left ecap=90}] (w2) to[bend left] (b1);
\path[contour edge={quick={1}{2}}] (b1) to[out=-90,in=-90,looseness=2.2] (w2);
\path[contour edge={above,quick={2}{2}}] (w4) to[out=110,looseness=2.2,in=170] (b4);
\path[contour edge={above,quick={2}{2}}] (b3) to[out=60,out looseness=1.5,in=-135] (w1);
\path[contour edge={quick={2}{2}}] (b4) to[bend right=90] (w1);
\path[contour edge={quick={2}{2}}] (w1) to[out=140,in=120,looseness=2.2] (b4);
\end{tikzpicture}
&
\tikzsetnextfilename{dessin-special-families-small-5}
\begin{tikzpicture}[graph picture]
\draweightvertices
\path[contour edge={quick={1}{2},left bcap=90}] (w4) to[bend left] (b3);
\path[contour edge={quick={1}{2}}] (b3) to[bend left] (w3);
\path[contour edge={quick={1}{2}}] (w3) to[bend left] (b2);
\path[contour edge={quick={1}{2},right bcap=90,begin=.5}] (b2) to[bend left] (w2);
\path[contour edge={quick={1}{2},end=.5,left ecap=90}] (w2) to[bend left] (b1);
\path[contour edge={above,quick={2}{2}}] (w4) to[out=120,looseness=3,in=60] (b4);
\path[contour edge={quick={1}{1}}] (b2) to[out=100,looseness=2,in=40] (w1);
\path[contour edge={quick={2}{1},right bcap=90,left ecap=90}] (b4) to[bend right=90] (w1);
\path[contour edge={quick={2}{1}}] (w1) to[out=140,in=150,looseness=2.2] (b4);
\path[contour edge={quick={2}{1}}] (w4) to[out=30,in=-140,in looseness=2] (b1);
\path[contour edge={above,quick={2}{2}}] (b1) to[out=170,in=10] (w3);
\path[contour edge={quick={2}{2}}] (b3) to[bend right] (w2);
\end{tikzpicture}
&
\tikzsetnextfilename{dessin-special-families-small-6}
\begin{tikzpicture}[graph picture]
\draweightvertices
\path[contour edge={quick={1}{2}}] (w4) to[bend left] (b3);
\path[contour edge={quick={1}{2}}] (b3) to[bend left] (w3);
\path[contour edge={quick={1}{2}}] (w3) to[bend left] (b2);
\path[contour edge={quick={1}{2}}] (b2) to[bend left] (w4);
\path[contour edge={above,quick={2}{2}}] (w3) to[out=-20,in=-180] (b1);
\path[contour edge={above,quick={1}{1}}] (b2) to[bend left] (w2);
\path[contour edge={quick={1}{2}}] (w2) to[bend left=90] (b1);
\path[contour edge={quick={1}{2}}] (b1) to[out=-90,in=-90,looseness=2.2] (w2);
\path[contour edge={above,quick={1}{1}}] (w4) to[bend right] (b4);
\path[contour edge={above,quick={2}{2}}] (b3) to[out=60,in=-135] (w1);
\path[contour edge={quick={2}{1}}] (b4) to[bend right=90] (w1);
\path[contour edge={quick={2}{1}}] (w1) to[out=140,in=120,looseness=2.2] (b4);
\end{tikzpicture}
\end{longtable}
\egroup
\end{sideline}
\begin{sideline}{Step 2.}
Let $\DD=\datum{\surf{g}}{d}{\pi_1,\pi_2,\pi_3}$ be a realizable augmented datum, and let $\env{x}\in\pi_3$ be an enveloping element. Then the augmented datum
\[
\DD'=\datum{\surf{g+1}}{d+6}{\pi_1\cup[3,3],\pi_2\cup[3,3],\pi_3\setminus[\env{x}]\cup[\env{x+6}]}
\]
is realizable as well. In fact, consider a \dessin{} $\Gamma\subs\surf{g}$ realizing $\DD$, and fix an edge $e$ enveloped by the disk $D$ associated to $\env{x}$; this disk will be colored orange, as shown in the picture below labeled \tikzenumlabel{0}. We perform the following operations on $\Gamma$.
\begin{enumarabic}
\item Add one black vertex and one white vertex on $e$.
\item Attach a tube to $\surf{g}$ with both endpoints on $D$.
\item Add one black vertex, one white vertex and four edges as shown in the picture.
\end{enumarabic}
\bgroup
\def\picturesetupzero#1{
\pic {cmove setting one disk=1};
\path[surf edge={behind}{disk 1 boundary enveloped},surrounding=disk 1,postaction={decorate,decoration={markings,mark=at position .25 with {\coordinate (b1);},mark=at position .75 with {\coordinate (w1);}}}] \surfcirclepath{d1}{-120}{-60} \ifnum#1=0 node[midway,auto,colored label=disk 1 boundary enveloped] {$e$}\fi;
}
\def\picturesetupone#1{
\picturesetupzero{1}
\pic at (b1) {black vertex};
\pic at (w1) {white vertex};
\ifnum#1=0
\node[below=3pt] at (b1) {$2$};
\node[below=3pt] at (w1) {$2$};
\fi
}
\def\picturesetuptwo#1{
\picturesetupone{#1}
\pic {cmove setting one disk tube=1};
\tubefill{disk 1};
}
\def\picturesetupthree{
\picturesetuptwo{1}
\path \tubemiddlepoint{150} coordinate (b2) pic{black vertex};
\path \tubemiddlepoint{-150} coordinate (w2) pic{white vertex};
\tubebelt{black edge}{black edge dashed}
\path[surf edge={front}{black edge}] let \p1=\tubeleftpoint{-90} in (b1) to[bend right] (\p1) to[out=90,in=180] (w2);
\path[surf edge={front}{black edge}] let \p1=\tuberightpoint{-90} in (w1) to[bend left] (\p1) to[out=90,in=0] (b2);
\node[below=3pt] at (b1) {$3$};
\node[below=3pt] at (w1) {$3$};
\node[left] at (b2) {$3$};
\node[right] at (w2) {$3$};
}
\tabcolsep=0pt
\begin{longtable}{*{2}{>{\centering\arraybackslash}p{.5\linewidth}}}
\tikzenumlabel{0}&\tikzenumlabel{1}\\*
\tikzsetnextfilename{dessin-special-families-cmove-0}
\begin{tikzpicture}[surf picture]
\picturesetupzero{0}
\end{tikzpicture}
&
\tikzsetnextfilename{dessin-special-families-cmove-1}
\begin{tikzpicture}[surf picture]
\picturesetupone{0}
\end{tikzpicture}
\\\addlinespace[2em]
\tikzenumlabel{2}&\tikzenumlabel{3}\\*
\tikzsetnextfilename{dessin-special-families-cmove-2}
\begin{tikzpicture}[surf picture]
\picturesetuptwo{0}
\end{tikzpicture}
&
\tikzsetnextfilename{dessin-special-families-cmove-3}
\begin{tikzpicture}[surf picture]
\picturesetupthree
\end{tikzpicture}
\end{longtable}
\egroup
After these operations, we get a new \dessin{} $\Gamma'$ embedded in $\surf{g+1}$. It is easy to check that $\Gamma'$ realizes the augmented datum $\DD'$.
\end{sideline}
\begin{sideline}{Step 3.}
For every $g\ge 2$, the augmented datum 
\[
\datum{\surf{g}}{6g}{[3,\ldots,3],[3,\ldots,3],\pi_3}
\]
is realizable for $\pi_3\in\{[1,\env{6g-1}],[2,\env{6g-2}]\}\cup\{[\env{s},\env{6g-s}]\colon 3\le s\le 6g-3\}$. This can be easily shown by induction on $g$, using Step 1 as the base case and Step 2 for the induction.
\end{sideline}
\def\myscale{1.5}
\begin{sideline}{Step 4.}
Let $\DD=\datum{\surf{g}}{d}{\pi_1,\pi_2,\pi_3}$ be a realizable augmented datum, and let $\env{x}\in\pi_3$ be an enveloping element. Then the augmented datum
\[
\DD'=\datum{\surf{g}}{d+2}{\pi_1\cup[2],\pi_2\cup[2],\pi_3\setminus[\env{x}]\cup[\env{x+2}]}
\]
is realizable as well. In fact, consider a \dessin{} $\Gamma\subs\surf{g}$ realizing $\DD$, and fix an edge $e$ enveloped by the disk $D$ associated to $\env{x}$. Then, add one black vertex and one white vertex on $e$.
\begin{center}
\def\picturesetup{
\begin{pgfonlayer}{graph edge below}
\begin{scope}
\clip (-.8,0) to[bend left=90] (.8,0) to[bend left=90] (-.8,0);
\fill[disk 1] circle(3);
\draw[disk 1 boundary enveloped,line width=\edgelinewidth] (-.5,0) pic{black vertex} to (.5,0) pic {white vertex};
\path[black edge] (-.5,0) to (-1,0);
\path[black edge] (.5,0) to (1,0);
\end{scope}
\end{pgfonlayer}
}
\tikzsetnextfilename{dessin-special-family-step-4}
\begin{tikzpicture}[graph picture,x={(\myscale,0)},y={(0,\myscale)}]
\begin{scope}[shift={(-1.5,0)}]
\picturesetup
\node[below=2pt,colored label=disk 1 boundary enveloped] at (0,0) {$e$};
\end{scope}
\pic {my fancy arrow};
\begin{scope}[shift={(1.5,0)}]
\picturesetup
\path (-1/6,0) pic{white vertex} node[above=2pt] {$2$};
\path (1/6,0) pic{black vertex} node[above=2pt] {$2$};
\end{scope}
\end{tikzpicture}
\end{center}
The new \dessin{} realizes the augmented datum $\DD'$.
\end{sideline}

\def\picturesetup{
\begin{pgfonlayer}{graph edge below}
\begin{scope}
\clip (-.8,0) to[bend left=90,looseness=1.5] (.8,0) to[bend left=90] (-.8,0);
\fill[disk 1] circle(3);
\draw[disk 1 boundary enveloped,line width=\edgelinewidth] (-.5,0) pic{black vertex} to (.5,0) pic {white vertex};
\path[black edge] (-.5,0) to (-1,0);
\path[black edge] (.5,0) to (1,0);
\end{scope}
\end{pgfonlayer}
}
\begin{sideline}{Step 5.}
Let $\DD=\datum{\surf{g}}{d}{\pi_1,\pi_2,\pi_3}$ be a realizable augmented datum, and let $\env{x}\in\pi_3$ be an enveloping element. Then the augmented datum
\[
\DD'=\datum{\surf{g}}{d+3}{\pi_1\cup[3],\pi_2\cup[1,2],\pi_3\setminus[\env{x}]\cup[\env{x+3}]}
\]
is realizable as well. In fact, consider a \dessin{} $\Gamma\subs\surf{g}$ realizing $\DD$, and fix an edge $e$ enveloped by the disk $D$ associated to $\env{x}$. Then, add one black vertex and two white vertices as shown in the picture.
\begin{center}
\tikzsetnextfilename{dessin-special-family-step-5}
\begin{tikzpicture}[graph picture,x={(\myscale,0)},y={(0,\myscale)}]
\begin{scope}[shift={(-1.5,0)}]
\picturesetup
\node[below=2pt,colored label=disk 1 boundary enveloped] at (0,0) {$e$};
\end{scope}
\pic {my fancy arrow};
\begin{scope}[shift={(1.5,0)}]
\picturesetup
\path (-1/6,0) pic{white vertex} node[below=2pt] {$2$};
\path (1/6,0) pic{black vertex} node[below=2pt] {$3$};
\path[graph edge={below}{black edge}] (1/6,0) to (1/6,1/3) pic{white vertex} node[above right] {$1$};
\end{scope}
\end{tikzpicture}
\end{center}
The new \dessin{} realizes the augmented datum $\DD'$.
\end{sideline}
\begin{sideline}{Step 6.}
Let $\DD=\datum{\surf{g}}{d}{\pi_1,\pi_2,\pi_3}$ be a realizable augmented datum, and let $\env{x}\in\pi_3$ be an enveloping element. Then the augmented datum
\[
\DD'=\datum{\surf{g}}{d+4}{\pi_1\cup[1,3],\pi_2\cup[1,3],\pi_3\setminus[\env{x}]\cup[\env{x+4}]}
\]
is realizable as well. In fact, consider a \dessin{} $\Gamma\subs\surf{g}$ realizing $\DD$, and fix an edge $e$ enveloped by the disk $D$ associated to $\env{x}$. Then, add two black vertices and two white vertices as shown in the picture.
\begin{center}
\tikzsetnextfilename{dessin-special-family-step-6}
\begin{tikzpicture}[graph picture,x={(\myscale,0)},y={(0,\myscale)}]
\begin{scope}[shift={(-1.5,0)}]
\picturesetup
\node[below=2pt,colored label=disk 1 boundary enveloped] at (0,0) {$e$};
\end{scope}
\pic {my fancy arrow};
\begin{scope}[shift={(1.5,0)}]
\picturesetup
\path (-1/6,0) pic{white vertex} node[below=2pt] {$3$};
\path (1/6,0) pic{black vertex} node[below=2pt] {$3$};
\path[graph edge={below}{black edge}] (1/6,0) to (1/6,1/3) pic{white vertex} node[above right] {$1$};
\path[graph edge={below}{black edge}] (-1/6,0) to (-1/6,1/3) pic{black vertex} node[above left] {$1$};
\end{scope}
\end{tikzpicture}
\end{center}
The new \dessin{} realizes the augmented datum $\DD'$.
\end{sideline}

Finally, it is easy to see that the five families listed in the statement can be obtained by applying Steps 4, 5 and 6 zero or more times to an augmented datum which is realizable by Step 3, and then forgetting about the augmentation; here are the details.
\begin{enumarabic}
\item $\datum{\surf{g}}{6g}{[3,\ldots,3],[3,\ldots,3],[s,6g-s]}$ is already realizable by Step 3.
\item If we assume that $s\le 3g+1$, Step 4 gives
\begin{multline*}
\datum{\surf{g}}{6g+2}{[2,3,\ldots,3],[2,3\ldots,3],[s,\env{6g+2-s}]}\\
\cmove\datum{\surf{g}}{6g}{[3,\ldots,3],[3,\ldots,3],[s,\env{6g-s}]},
\end{multline*}
which is realizable by Step 3.
\item If we assume that $s\le 3g+1$, Step 5 gives 
\begin{multline*}
\datum{\surf{g}}{6g+3}{[3,\ldots,3],[1,2,3,\ldots,3],[s,\env{6g+3-s}]}\\
\cmove\datum{\surf{g}}{6g}{[3,\ldots,3],[3,\ldots,3],[s,\env{6g-s}]},
\end{multline*}
which is realizable by Step 3.
\item If we assume that $s\le 3g+2$, Step 6 gives
\begin{multline*}
\datum{\surf{g}}{6g+4}{[1,3,\ldots,3],[1,3,\ldots,3],[s,\env{6g+4-s}]}\\
\cmove\datum{\surf{g}}{6g}{[3,\ldots,3],[3,\ldots,3],[s,\env{6g-s}]},
\end{multline*}
which is realizable by Step 3.
\item If we assume that $s\le 3g+3$, Steps 4 and 6 give
\begin{multline*}
\datum{\surf{g}}{6g+6}{[1,2,3,\ldots,3],[1,2,3,\ldots,3],[s,\env{6g+6-s}]}\\
\begin{aligned}
&\cmove\datum{\surf{g}}{6g+4}{[1,3,\ldots,3],[1,3,\ldots,3],[s,\env{6g+4-s}]}\\
&\cmove\datum{\surf{g}}{6g}{[3,\ldots,3],[3,\ldots,3],[s,\env{6g-s}]},
\end{aligned}
\end{multline*}
which is realizable by Step 3.\qedhere
\end{enumarabic}
\end{proof}