\section{Rivestimenti ramificati}
\subsection{Definizione}

\begin{frame}
\begin{columns}[onlytextwidth]
\column{.65\textwidth}
Un \emph{rivestimento ramificato} fra due \sidenotehighlight<1>{superfici}{\draw (mymark) to[out=90,in=180] (.5,.7) node[anchor=west] {chiuse, connesse, orientabili};} $\tSigma$ e $\Sigma$ è una funzione continua
\[
\map{f}{\tSigma}{\Sigma}
\]
che localmente si comporta come
\[
\uMap{\CC}{\CC}{\xi}{\xi^k.}
\]

\begin{itemize}
\item Il \emph{grado locale} $k=k(\wtilde{x})$ dipende da $\wtilde{x}\in\tSigma$, ed è uguale a $1$ per quasi tutti i punti.
\item Un punto $x\in\Sigma$ è \emph{di ramificazione} se $k(\wtilde{x})>1$ per un qualche $\wtilde{x}\in f^{-1}(x)$.
\end{itemize}
\column{.35\textwidth}
\begin{flushright}
\tdplotsetmaincoords{65}{-30}
\tikzsetnextfilename{slides-branched-covering-local-model}
\begin{tikzpicture}[tdplot_main_coords,x={(1.4,0,0)},y={(0,1.4,0)},z={(0,0,1.4)}]
\def\a{30};
\foreach \i in {1,2} {
\coordinate (y-\i) at ({sin(\a)*.7},{-cos(\a)*.7},{(\a/360+\i-2)*.7});
}
\coordinate (y) at ({sin(\a)*.7},{-cos(\a)*.7},-2.5);

\begin{scope}[shift={(0,0,-2.5)}]
\fill[disk 1] circle (1);
\foreach \i in {.1,.2,...,.9} { \draw[gray!80,line width=.1] circle (\i); }
\foreach \i in {0,...,39} {\draw[gray!80,line width=.1] (0,0) -- ({\i*360/40}:1); }
\draw circle (1);
\fill[black] circle (1pt) node[below left] {$x$};
\fill[black] (y) circle (1pt);
\end{scope}

\draw[violet,-{Latex[]}] (0,0,0) -- (0,0,-2.4) node[midway,left=2pt] {$f$};
\draw[violet,-{Latex[]}] (y-1) -- ([shift={(0,0,.1)}]y);

\newcounter{branchedcoveringcounter}
\path[decorate,decoration={show path construction,curveto code={
\filldraw[line width=.1pt,fill=disk 1,fill opacity=.9,draw=gray!80] (0,0,0) -- (\tikzinputsegmentfirst) -- (\tikzinputsegmentlast) -- cycle;
\draw (\tikzinputsegmentfirst) .. controls (\tikzinputsegmentsupporta) and (\tikzinputsegmentsupportb) .. (\tikzinputsegmentlast);
\foreach \i in {.1,.2,...,.9} {
	\draw[line width=.2,gray,opacity=.7] ($\i*(\tikzinputsegmentfirst)$) .. controls ($\i*(\tikzinputsegmentsupporta)$) and ($\i*(\tikzinputsegmentsupportb)$) .. ($\i*(\tikzinputsegmentlast)$);
}
\ifnumcomp{\the\value{branchedcoveringcounter}}{=}{0}{
	\draw[teal,line width=\edgelinewidth] (0,0,0) -- (\tikzinputsegmentfirst);
}{}
\ifnumcomp{\the\value{branchedcoveringcounter}}{=}{15}{
	\draw[violet] (y-2) -- (y-1);
}{}
\stepcounter{branchedcoveringcounter}
}}] plot[smooth,samples=61,variable=\t,domain=-360:360] ({sin(\t)},{-cos(\t)},{\t/360});
\draw[teal,line width=\edgelinewidth] (0,0,0) -- (0,-1,1);

\def\a{30};
\foreach \i in {1,2} {
\fill[black] (y-\i) circle (1pt);
}
\fill[black] (0,0,0) circle (1pt) node[below left] {$\wtilde{x}$};
\end{tikzpicture}
\end{flushright}
\end{columns}
\end{frame}

\subsection{Grado del rivestimento}
\begin{frame}
\begin{columns}[onlytextwidth]
\column{.65\textwidth}
Posto
\begin{itemize}
\item $\holed{\Sigma}=\Sigma\setminus\{\text{punti di ramificazione}\}$,
\item $\holed{\tSigma}=f^{-1}(\holed{\Sigma})$,
\end{itemize}
la restrizione $\map{\holed{f}}{\holed{\tSigma}}{\holed{\Sigma}}$ è un rivestimento di grado $d\ge 1$.

\begin{mybox}
Per ogni $x\in\Sigma$ vale
\[
k(\wtilde{x}_1)+\ldots+k(\wtilde{x}_r)=d,
\]
dove $\{\wtilde{x}_1,\ldots,\wtilde{x}_r\}=f^{-1}(x)$.
\end{mybox}
\begin{itemize}
\item Possiamo associare a $x$ la partizione di $d$
\[
\pi(x)=[k(\wtilde{x}_1),\ldots,k(\wtilde{x}_r)].
\]
\end{itemize}
\column{.35\textwidth}
Picture
\end{columns}
\end{frame}

\subsection{Dati di ramificazione}
\begin{frame}
Le proprietà combinatorie di $\map{f}{\tSigma}{\Sigma}$ sono contenute nel \emph{dato di ramificazione}
\[
\DD(f)=\datum{\tSigma,\Sigma}{d}{\pi(x_1),\ldots,\pi(x_n)},
\]
dove $x_1,\ldots,x_n$ sono i punti di ramificazione.
\begin{mybox}[title=Formula di Riemann-Hurwitz]
Se $\DD(f)=\datum{\tSigma,\Sigma}{d}{\pi_1,\ldots,\pi_n}$ è un dato di ramificazione, allora
\begin{equation}\tag{RH}
d\cdot\chi(\Sigma)-\chi(\tSigma)=d\cdot n-\len{\pi_1}-\ldots-\len{\pi_n}.
\end{equation}
\end{mybox}
\begin{itemize}
\item Una tupla $\DD=\datum{\tSigma,\Sigma}{d}{\pi_1,\ldots,\pi_n}$ che soddisfa la formula di Riemann-Hurwitz si dice \emph{dato compatibile}.
\item Per comodità, richiediamo anche che $n\ge 3$ e $\pi_i\neq[1,\ldots,1]$.
\end{itemize}
\end{frame}

\subsection{Problema di esistenza di Hurwitz}
\begin{frame}
\begin{itemize}
\item Un dato compatibile $\DD=\datum{\tSigma,\Sigma}{d}{\pi_1,\ldots,\pi_n}$ si dice \emph{realizzabile} se è il dato di ramificazione di un qualche rivestimento ramificato $\map{f}{\tSigma}{\Sigma}$, \emph{eccezionale} altrimenti.
\item La formula di Riemann-Hurwitz è condizione necessaria, ma non sufficiente per la realizzabilità.\\
Ad esempio, il dato $\DD=\datum{\sphere{},\sphere{}}{4}{[2,2],[2,2],[3,1]}$ soddisfa
\begin{equation}\tag{\textcolor{black!70}{RH}}
\begin{tikzpicture}[baseline=(m-2-1.base)]
\matrix[ampersand replacement=\&,matrix of math nodes,nodes={inner sep=0pt,text height={height("I")},text depth={depth("y")}},row sep=1.2em,row 2/.style={text opacity=.7}] (m) {
4\&{}\cdot{}\&2\&{}-{}\&2\&{}={}\&4\&{}\cdot{}\& 3\&{}-{}\&2\&{}-{}\&2\&{}-{}\&2\\
d\&{}\cdot{}\&\chi(\Sigma)\&{}-{}\&\chi(\tSigma)\&{}={}\& d\&{}\cdot{}\& n\&{}-{}\& \len{\pi_1}\&{}-{}\& \ldots\&{}-{}\& \len{\pi_n}\&,\\
};
\begin{pgfonlayer}{background}
\foreach \i in {1,3,5,7,9,11,15} {
\foreach \j in {1,2} {
\fill[rounded corners=2pt,violet!15] ($(m-\j-\i.north east)+(2pt,2pt)$) rectangle ($(m-\j-\i.south west)-(2pt,1pt)$);
}
\draw[violet!25,line width=1pt,densely dotted] (m-1-\i) -- (m-2-\i);
}
\end{pgfonlayer}
\end{tikzpicture}
\end{equation}
ma è eccezionale.
\end{itemize}
\begin{mybox}[title=Problema di esistenza di Hurwitz]
Quali dati compatibili sono realizzabili?
\end{mybox}
\begin{itemize}
\item Approcci generali: monodromia e \dessins{}.
\item Soluzione completa nel caso $\len{\pi_n}=2$.
\end{itemize}
\end{frame}