\chapter{Realizablility by monodromy}

\section{Symmetric group and partitions}

\section{Branched covering action of the fundamental group}

\section{Monodromy and realizability}

\begin{proposition}\label{monodromy:th:monodromy-realizability-orientable}
Let $\surf{g}$ be the connected sum of $g\ge 0$ tori, $d\ge 1$ an integer, $\pi_1,\ldots,\pi_n\in\Partitions{d}$ partitions of $d$. Then there exists a realizable combinatorial datum $\DD=\datum{\tSigma,\surf{g}}{d}{\pi_1,\ldots,\pi_n}$ if and only if there exist permutations $\alpha_1,\ldots,\alpha_g,\beta_1,\ldots,\beta_n,\gamma_1,\ldots,\gamma_n\in\symgroup[g]$ such that:
\begin{enumerate}
\item $[\alpha_i]=\pi_i$ for each $1\le i\le n$;
\item $[\beta_1,\gamma_1]\cdots[\beta_g,\gamma_g]\cdot\alpha_1\cdots\alpha_n=1$;
\item the subgroup $\angled{\alpha_1,\ldots,\alpha_n,\beta_1,\ldots,\beta_g,\gamma_1,\ldots,\gamma_g}\subgroup\symgroup[d]$ acts transitively on $\{1,\ldots,d\}$.
\end{enumerate}
In this case, $\tSigma$ is necessarily orientable.
\end{proposition}
\begin{remark}\label{monodromy:rm:sigma-tilde-unique-orientable}
Given $\surf{g}$, $d$ and $\pi_1,\ldots,\pi_n$, there is at most one surface $\tSigma$ such that $\DD=\datum{\tSigma,\surf{g}}{d}{\pi_1,\ldots,\pi_n}$ is a candidate datum. In fact, the \RH{} formula gives
\[
\chi(\tSigma)=d\chi(\surf{g})-v(\pi_1)-\ldots-v(\pi_n)
\]
which, in turn, uniquely determines the orientable surface $\tSigma$.
\end{remark}

\begin{proposition}\label{monodromy:th:monodromy-realizability-non-orientable}
Let $\nosurf{g}$ be the connected sum of $g\ge 1$ projective planes, $d\ge 1$ an integer, $\pi_1,\ldots,\pi_n\in\Partitions{d}$ partitions of $d$. Then there exists a realizable combinatorial datum $\DD=\datum{\tSigma,\nosurf{g}}{d}{\pi_1,\ldots,\pi_n}$ if and only if there exist permutations $\alpha_1,\ldots,\alpha_n,\beta_1,\ldots,\beta_g\in\symgroup[d]$ such that:
\begin{enumerate}
\item $[\alpha_i]=\pi_i$ for each $1\le i\le n$;
\item $\beta_1^2\cdots\beta_g^2\cdot\alpha_1\cdots\alpha_n=1$;
\item the subgroup $\angled{\alpha_1,\ldots,\alpha_n,\beta_1,\ldots,\beta_g}\subgroup\symgroup[d]$ acts transitively on $\{1,\ldots,d\}$.
\end{enumerate}
In this case, $\tSigma$ is orientable if and only if ??.
\end{proposition}
\begin{remark}\label{monodromy:rm:sigma-tilde-unique-non-orientable}
Given $\nosurf{g}$, $d$ and $\pi_1,\ldots,\pi_n$, there are at most two surfaces $\tSigma$, one orientable and one non-orientable, such that $\DD=\datum{\tSigma,\nosurf{g}}{d}{\pi_1,\ldots,\pi_n}$ is a candidate datum: like in \cref{monodromy:rm:sigma-tilde-unique-orientable}, the \RH{} formula fixes $\chi(\tSigma)$ which, together with orientability, uniquely determines $\tSigma$.
\end{remark}

\begin{proposition}\label{monodromy:th:monodromy-realizability-double-covering}
Let $\DD=\datum{\tSigma,\Sigma}{d}{\pi_1,\ldots\pi_n}$ be a combinatorial datum, with $\Sigma$ non-orientable and $\tSigma$ orientable. Then $\DD$ is realizable if and only if $d$ is even and there exist partitions $\pi'_1,\ldots,\pi'_n,\pi''_1,\ldots,\pi''_n\in\Partitions{d/2}$ with $\pi_i=\pi'_i\cup\pi''_i$ for each $1\le i\le n$, such that the combinatorial datum
\[
\DD'=\datum{\tSigma,\hSigma}{d/2}{\pi'_1,\ldots,\pi'_n,\pi''_i,\ldots,\pi''_n}
\]
is realizable, where $\hSigma$ is the orientable double covering of $\Sigma$.
\end{proposition}

\section{Non-positive Euler characteristic}

\begin{lemma}\label{monodromy:th:product-of-two-cycles}
Let $\alpha\in\symgroup[d]$ be a permutation. Set $r=d-v(\alpha)$, and let $t\ge 0$ be an integer such that $2t\le v(\alpha)$. Then $\alpha$ can be written as the product of a $(r+2t)$\=/cycle and a $d$\=/cycle.
\end{lemma}
\begin{proof}
Without loss of generality, assume that
\[
\alpha=\cycle{1,\ldots,d_1}\cycle{d_1+1,\ldots,d_2}\cdots\cycle{d_{r-1}+1,\ldots,d_r},
\]
where $d_r=d$. Fix
\begin{align*}
\beta_0=\cycle{1,b_1,b_2,\ldots,b_{2t}},&&\beta_1=\cycle{1,d_1+1,d_2+2,\ldots,d_{r-1}+1}.
\end{align*}
Set
\[
\beta=\beta_0\beta_1=\cycle{1,d_1+1,d_2+1,\ldots,d_{r-1}+1,b_1,b_2,\ldots,b_{2t}}.
\]
An easy computation shows that
\begin{align*}
\beta\alpha&=\beta_0\beta_1\alpha\\
&=\cycle{1,b_1,b_2,\ldots,b_{2t}}\cycle{1,2,\ldots,d}\\
&=\cycle{1,\ldots,b_1-1,b_2,\ldots,b_3-1,\ldots,b_4,\ldots,b_{2t-1}-1,b_{2t},\ldots,d,b_1,\ldots,b_2-1,b_3,\ldots,b_{2t}-1}.
\end{align*}
Writing $\alpha=\beta^{-1}(\beta\alpha)$ gives the desired decomposition.
\end{proof}

\begin{corollary}\label{monodromy:th:even-permutation-commutator-or-squares}
Let $\alpha\in\altgroup[d]$ be an even permutation. Then $\alpha$ can be written as:
\begin{enumerate}
\item a commutator $[\beta,\gamma]$, where $\gamma$ is a $d$\=/cycle;
\item a product of two squares $\delta^2\epsilon^2$, where $\delta\epsilon$ is a $d$\=/cycle.
\end{enumerate}
\end{corollary}
\begin{proof}
Since $\alpha$ is an even permutation, its branching number $v(\alpha)$ is even. By \cref{monodromy:th:product-of-two-cycles}, there exist two $d$\=/cycles $\tau,\sigma\in\symgroup[d]$ such that $\alpha=\tau\sigma$.
\begin{enumerate}
\item Since $\tau$ and $\sigma^{-1}$ are conjugated, there exists a permutation $\beta\in\symgroup[d]$ such that $\tau=\beta\sigma^{-1}\beta^{-1}$. Setting $\gamma=\sigma^{-1}$, we immediately get that
\[
\alpha=\tau\sigma=\beta\sigma^{-1}\beta^{-1}\sigma=\beta\gamma\beta^{-1}\gamma^{-1}=[\beta,\gamma].
\]
\item Since $\tau$ and $\sigma$ are conjugated, there exists a permutation $\delta\in\symgroup[d]$ such that $\tau=\delta\sigma\delta^{-1}$. Setting $\epsilon=\delta^{-1}\sigma$, we have that
\[
\alpha=\tau\sigma=\delta\sigma\delta^{-1}\sigma=\delta^2(\delta^{-1}\sigma)^2=\delta^2\epsilon^2.\qedhere
\]
\end{enumerate}
\end{proof}

\begin{theorem}
Let $\DD=\datum{\tSigma,\Sigma}{d}{\pi_1,\ldots,\pi_n}$ be a candidate datum. If $\chi(\Sigma)\le 0$, then $\DD$ is realizable.
\end{theorem}
\begin{proof}
Let us first assume that $\Sigma$ is orientable; this means that $\Sigma=\surf{g}$ is the connected sum of $g\ge 1$ tori, and that $\tSigma$ is orientable as well. Choose permutations $\alpha_1,\ldots,\alpha_n\in\symgroup[d]$ with $[\alpha_i]=\pi_i$ for each $1\le i\le n$. Let $\alpha=\alpha_1\cdots\alpha_n$. Since $\DD$ is a candidate datum, we have that
\[
v(\alpha)\equiv v(\alpha_1)+\ldots+v(\alpha_n)\equiv 0\pmod{2}.
\]
By \cref{monodromy:th:even-permutation-commutator-or-squares}, we can find permutations $\beta_1,\gamma_1\in\symgroup[d]$ such that $\alpha=[\gamma_1,\beta_1]$ and $\beta_1$ is a $d$\=/cycle. Set $\beta_2=\ldots=\beta_g=\gamma_2=\ldots=\gamma_g=\id\in\symgroup[d]$. All the conditions of \cref{monodromy:th:monodromy-realizability-orientable} are satisfied; since $\tSigma$ is orientable, this implies that $\DD$ is realizable (see \cref{monodromy:rm:sigma-tilde-unique-orientable}).

Assume now that $\Sigma$ and $\tSigma$ are both non-orientable; this means that $\Sigma=\nosurf{g}$ is the connected sum of $g\ge 2$ projective planes. Choose permutations $\alpha_1,\ldots,\alpha_n\in\symgroup[d]$ with $[\alpha_i]=\pi_i$ for each $1\le i\le n$. Let $\alpha=\alpha_1\cdots\alpha_n$. Similarly to what we did for the previous case, we can find $\beta_1,\beta_2\in\symgroup[d]$ such that $\alpha=\beta_2^{-2}\beta_1^{-2}$ and $\beta_2\beta_1$ is a $d$\=/cycle. By setting $\beta_3=\ldots=\beta_g=\id\in\symgroup[d]$, \cref{monodromy:th:monodromy-realizability-non-orientable} (together with \cref{monodromy:rm:sigma-tilde-unique-non-orientable}) implies the realizability of $\DD$.

Finally, consider the case where $\Sigma$ is non-orientable and $\tSigma$ is orientable. Since $\DD$ is a candidate datum, $d$ is even and there exist partitions $\pi'_1,\ldots,\pi'_n,\pi''_1,\ldots,\pi''_n\in\Partitions{d/2}$ with $\pi_i=\pi'_1\cup\pi''_i$ for each $1\le i\le n$. Let $\hSigma$ be the double orientable covering of $\Sigma$; by the first case we analyzed, the candidate datum
\[
\DD'=\datum{\tSigma,\hSigma}{d/2}{\pi'_1,\ldots,\pi'_n,\pi''_1,\ldots,\pi''_n}
\]
is realizable. By \cref{monodromy:th:monodromy-realizability-double-covering}, $\DD$ is realizable as well.
\end{proof}

\section{Products in symmetric groups}
\begin{lemma}\label{monodromy:th:same-number-of-cycles}
Let $X,Y$ be finite sets; denote by $h$ the cardinality of $Y$, and by $k$ the cardinality of $X\cap Y$. Let $\alpha\in\symgroup(X)$, $\beta\in\symgroup(Y)$, $\gamma\in\symgroup(X\cap Y)$. Assume that $\beta=\cycle{b_1,\ldots,b_h}$ is a $h$\=/cycle, and that $\gamma$ is a $k$\=/cycle of the form $\gamma=\cycle{b_{i_1},\ldots,b_{i_k}}$ with $1\le i_1\le\ldots\le i_k\le h$. Then $\alpha\in\symgroup(X)$ and $\alpha\gamma^{-1}\beta\in\symgroup(X\cup Y)$ have the same number of cycles.
\end{lemma}
\begin{proof}
Write $\gamma=\cycle{u_1,\ldots,u_k}$, where $u_j=b_{i_j}$. Without loss of generality, assume that $i_1=1$. Then we can write
\[
\beta=\cycle{u_1,\ldots,w_1,u_2,\ldots,w_2,u_3,\ldots,w_{k-1},u_k,\ldots,w_k},
\]
possibly with $u_j=w_j$ for some values of $j$. We immediately get that
\[
\gamma^{-1}\beta=\cycle{u_1,\ldots,w_1}\cycle{u_2,\ldots,w_2}\cdots\cycle{u_k,\ldots,w_k}.
\]
If we denote by $A_1,\ldots,A_r\subs X$ the orbits of $\alpha$, it is then easy to see that the orbits of $\alpha\gamma^{-1}\beta$ are $A_1',\ldots,A_r'\subs X\cup Y$, where
\[
A_j'=A_j\cup\bigcup_{u_l\in A_j}\{u_l,\ldots,w_l\}.\qedhere
\]
\end{proof}

\begin{comment}
\begin{lemma}
\todo{Maybe useless?}Let $\alpha,\beta\in\symgroup[d]$ be permutations. Then $v(\alpha\beta)\le v(\angled{\alpha,\beta})\le v(\alpha)+v(\beta)$. Moreover if $v(\angled{\alpha,\beta})=v(\alpha)+v(\beta)$ then $v(\alpha\beta)=v(\alpha)+v(\beta)$.
\end{lemma}
\begin{proof}
Let $q=v(\angled{\alpha,\beta})$, and let $\gamma=\alpha\beta$. Since $\gamma\in\angled{\alpha,\beta}$, the inequality $v(\gamma)\le q$ trivially holds. We have that $\alpha\beta\gamma^{-1}=1$; by \cref{?}, this implies that the combinatorial datum $\datum{\tSigma,S^2}{d}{[\alpha],[\beta],[\gamma]}$ is realizable for some orientable surface $\tSigma$. The \RH{} formula for this datum is
\[
v(\alpha)+v(\beta)+v(\gamma)=2d-\chi(\tSigma).
\]
Given that $\chi(\tSigma)\le 2\le 2(d-q)$, we find that
\[
v(\alpha)+v(\beta)=(v(\alpha)+v(\beta)+v(\gamma))-v(\gamma)\stackrel{(*)}{\ge} 2q-q=q\ge v(\gamma),
\]
which proves the first part of the lemma. Moreover, if $v(\alpha)+v(\beta)=q$, inequality $(*)$ must be an equality, hence $v(\gamma)=q$.
\end{proof}
\end{comment}

\begin{proposition}\label{monodromy:th:product-reduction-small-v}
Let $\pi,\rho\in\Partitions{d}$ be partitions of $d$. Assume that $v(\pi)+v(\rho)<d$. Then there exist permutations $\alpha,\beta\in\symgroup[d]$ with $[\alpha]=\pi$ and $[\beta]=\rho$ such that $v(\alpha\beta)=v(\alpha)+v(\beta)$.
\end{proposition}
\begin{proof}
First of all, note that the conclusion is trivial whenever $v(\pi)=0$ or $v(\rho)=0$. This already solves the cases $d=1$ and $d=2$. We now proceed by induction on $d\ge 3$, assuming that $v(\pi)>0$ and $v(\rho)>0$. Write $\pi=[a_1,\ldots,a_r]$, $\rho=[b_1,\ldots,b_s]$; without loss of generality, assume that $b_1>1$. Fix
\[
\beta=\cycle{1,\ldots,d_1}\cycle{d_1+1,\ldots,d_2}\cdots\cycle{d_{s-1}+1,\ldots,d_s},
\]
where $b_1=d_1\ge 2$, $b_i=d_i-d_{i-1}$ for $2\le i\le s$ (in particular, $d_s=d$). Note that
\[
d-1\ge v(\pi)+v(\rho)=(a_1-1)+\ldots+(a_r-1)+d-s\ge a_1-1+d-s,
\]
hence $a_1\le s$. Fix $\alpha_1=\cycle{1,d_1+1,\ldots,d_{a_1-1}+1}$, and let $A=\{1,d_1+1,\ldots,d_{a_1-1}+1\}$ be the support of $\alpha_1$. Define $Q=\{1,\ldots,d_{a_1}\}\setminus A$; note that $Q_1$ is non-empty, since $d_1+1\ge 3$ implies that $2\in Q_1$. Consider the partitions
\begin{align*}
\pi'=[a_2,a_3,\ldots,a_r],&&\rho'=[\card{Q},b_{a_1+1},\ldots,b_s].
\end{align*}
We have that
\begin{align*}
\sum\pi'=\sum\rho'=d-a_1,&&v(\pi')=v(\pi)-a_1+1,&&v(\rho')=v(\rho)-1.
\end{align*}
Since $d-a_1<d$ and $v(\pi')+v(\rho')=v(\pi)+v(\rho)-a_1<d-a_1$, by induction we find $\alpha',\beta'\in\symgroup(\{1,\ldots,d\}\setminus A)$ with $[\alpha']=\pi'$ and $[\beta']=\rho'$ such that $v(\alpha'\beta')=v(\alpha')+v(\beta')$. Up to conjugation, we may assume that
\[
\beta'=\beta_1\cycle{d_{a_1}+1,\ldots,d_{a_1+1}}\cycle{d_{a_1+1}+1,\ldots,d_{a_1+2}}\cdots\cycle{d_{s-1}+1,\ldots,d_s},
\]
where $\beta_1$ is the $\card{Q}$\=/cycle whose entries are the elements of $Q$ in increasing order. An easy computation shows that
\[
\alpha_1\beta=\cycle{1,\ldots,d_{a_1}}\cycle{d_{a_1}+1,\ldots,d_{a_1+1}}\cdots\cycle{d_{s-1}+1,\ldots,d_s}.
\]
Therefore, setting $\alpha=\alpha'\alpha_1$, we have that
\begin{align*}
\alpha\beta&=\alpha_1\alpha'\beta\\
&=\alpha'\cycle{1,\ldots,d_{a_1}}\cycle{d_{a_1}+1,\ldots,d_{a_1+1}}\cdots\cycle{d_{s-1}+1,\ldots,d_s}\\
&=\alpha'\beta'\beta_1^{-1}\cycle{1,\ldots,d_{a_1}}
\end{align*}
By \cref{monodromy:th:same-number-of-cycles}, this implies that $\alpha\beta$ has the same number of cycles as $\alpha'\beta'$, so that
\begin{align*}
v(\alpha\beta)&=a_1+v(\alpha'\beta')\\
&=a_1+v(\alpha')+v(\beta')\\
&=a_1+(v(\pi)-a_1+1)+(v(\rho)-1)\\
&=v(\pi)+v(\rho).
\end{align*}
Since $[\alpha]=\pi$ and $[\beta]=\rho$, the conclusion follows.
\end{proof}

\begin{proposition}\label{monodromy:th:product-reduction-large-v-odd}
Let $\pi,\rho\in\Partitions{d}$ be partitions of $d$. Assume that $v(\pi)+v(\rho)\ge d$, and let $t=v(\pi)+v(\rho)-d+1$. Fix an integer $0\le k\le t$ such that $k\equiv t\pmod{2}$. Then there exist permutations $\alpha,\beta\in\symgroup[d]$ with $[\alpha]=\pi$ and $[\beta]=\rho$ such that $v(\alpha\beta)=d-1-k$ and the action of $\angled{\alpha,\beta}$ on $\{1,\ldots,d\}$ is transitive.\todo{Maybe only $k=0$ needed (in this case, transitivity is trivial)?}
\end{proposition}
\begin{proof}
Write $\pi=[a_1,\ldots,a_r]$. Since $v(\rho)\le d-1$ and $v(\pi)+v(\rho)\ge d$, there exists a largest integer $0\le i\le r$ such that $(a_1-1)+\ldots(a_i-1)+v(\rho)\le d-1$. Define
\[
z=d-v(\rho)-(a_1-1)-\ldots-(a_i-1).
\]
Consider the partition $\pi'=[a_1,\ldots,a_i,z,1,\ldots,1]\in\Partitions{d}$. Since by construction $v(\pi')+v(\rho)=d-1$, thanks to \cref{monodromy:th:product-reduction-small-v} we can find permutations $\alpha',\beta\in\symgroup[d]$ with $[\alpha']=\pi'$ and $[\beta]=\rho$ such that $v(\alpha'\beta)=d-1$; in other words, $\alpha'\beta$ is a $d$\=/cycle. Consider now the partition $\pi''=[a_{i+1}-z+1,a_{i+2},\ldots,a_r]$, whose branching number\todo{Terminology?} is $v(\pi'')=t$. Let $n=\sum\pi''$; fix an element $u_1$ of the $z$\=/cycle of $\alpha'$, and let $u_2,\ldots,u_n$ be the fixed points of $\alpha'$ corresponding to the last ones of $\pi'$ (it is easy to see that there are exactly $n-1$ such ones). Since $k\le t=v(\rho'')$ and $k\equiv t\pmod{2}$, \cref{monodromy:th:product-of-two-cycles} gives permutations $\alpha'',\gamma\in\symgroup(\{u_1,\ldots,u_n\})$ such that $[\alpha'']=\rho''$, $\gamma$ is a $n$\=/cycle and $\alpha''\gamma$ is a $(n-k)$\=/cycle. Up to conjugation, we can assume that $\gamma=\cycle{u_1,\ldots,u_n}$. Moreover, it is not restrictive to assume that $u_1,\ldots,u_n$ appear in this order in the $d$\=/cycle $\alpha'\beta$. Therefore, setting $\alpha=\alpha''\alpha'$, we have that
\[
\alpha\beta=(\alpha''\gamma)\gamma^{-1}(\alpha'\beta);
\]
by \cref{monodromy:th:same-number-of-cycles}, this implies that $\alpha\beta$ has the same number of cycles as $\alpha''\gamma\in\symgroup(\{u_1,\ldots,u_n\})$, that is $v(\alpha\beta)=d-(k+1)$. Since $[\alpha]=\pi$ and $[\beta]=\rho$, the only thing left to show is that the action of $\angled{\alpha,\beta}$ on $\{1,\ldots,d\}$ is transitive. Write
\[
\alpha'\beta=\cycle{u_1,\ldots,w_1,u_2,\ldots,w_2,u_3,\ldots,w_{n-1},u_n,\ldots,w_n}
\]
as in the proof of \cref{monodromy:th:same-number-of-cycles}, where an explicit description of the orbits of $\alpha\beta=\alpha''\gamma\gamma^{-1}\alpha'\beta$ is given; from that description, it is clear that for each $1\le j\le n$ the elements $u_j,\ldots,w_j$ all belong to the same orbit. Moreover, since $\beta=(\alpha')^{-1}\cycle{u_1,\ldots,w_1,u_2,\ldots,w_n}$ and $\alpha'$ fixes $u_2,\ldots,u_n$, it follows that $w_j$ and $u_{j+1}$ belong to the same orbit for each $1\le j\le n-1$; this completes the proof.
\end{proof}

\begin{corollary}\label{monodromy:th:product-reduction-odd}
Let $\pi,\rho\in\Partitions{d}$ be partitions of $d$. Assume that $v(\pi)+v(\rho)\ge d-1$ and $v(\pi)+v(\rho)\equiv d-1\pmod{2}$. Then there exist permutations $\alpha,\beta\in\symgroup[d]$ with $[\alpha]=\pi$ and $[\beta]=\rho$ such that $\alpha\beta$ is a $d$\=/cycle.
\end{corollary}
\begin{proof}
The conclusion immediately follows from \cref{monodromy:th:product-reduction-small-v} if $v(\pi)+v(\rho)=d-1$, or from \cref{monodromy:th:product-reduction-large-v-odd} if $v(\pi)+v(\rho)\ge d$.
\end{proof}

\begin{remark}\label{monodromy:rm:product-reduction-odd-prescribed-cycles}
Write $\pi=[a_1,\ldots,a_r]$, $\rho=[b_1,\ldots,b_s]$; assume that $b_1\ge 2$. By directly examining the proof of \cref{monodromy:th:product-reduction-small-v}, we can see that the proposed construction yields permutations $\alpha,\beta\in\symgroup[d]$ such that $1$ belongs to the $a_1$\=/the cycle of $\alpha$ and to the $b_1$\=/cycle of $\beta$. It is not hard to see\todo{Maybe explain why?}, once again by inspecting the proof, that the same can be said for \cref{monodromy:th:product-reduction-large-v-odd}. As a consequence, the statement of \cref{monodromy:th:product-reduction-odd} can be enhanced by adding the following line: \emph{$\alpha$ and $\beta$ can be chosen in such a way that $1$ belongs to the $a_1$\=/the cycle of $\alpha$ and to the $b_1$\=/cycle of $\beta$, provided that $b_1\ge 2$}. We will need this improvement for the upcoming proof.
\end{remark}

\begin{proposition}\label{monodromy:th:product-reduction-large-v-even}
Let $\pi,\rho\in\Partitions{d}$ be partitions of $d$. Assume that $v(\pi)+v(\rho)\ge d$ and $v(\pi)+v(\rho)\equiv d\pmod{2}$. Then there exist permutations $\alpha,\beta\in\symgroup[d]$ with $[\alpha]=\pi$ and $[\beta]=\rho$ such that $\angled{\alpha,\beta}$ acts transitively on $\{1,\ldots,d\}$ and
\[
[\alpha\beta]=\begin{dcases*}
[d/2,d/2]&if $\pi=\rho=[2,\ldots,2]$,\\
[1,d-1]&otherwise.
\end{dcases*}
\]
\end{proposition}
\begin{proof}
Assume first that $\pi=\rho=[2,\ldots,d]$. We can choose
\begin{align*}
\alpha=\cycle{2,3}\cycle{4,5}\cdots\cycle{d,1},&&\beta=\cycle{1,2}\cycle{3,4}\cdots\cycle{d-1,d}.
\end{align*}
The action of $\angled{\alpha,\beta}$ is obviously transitive, and
\[
\alpha\beta=\cycle{1,3,\ldots,d-1}\cycle{2,4,\ldots,2}.
\]
Otherwise, since $v(\pi)+v(\rho)\ge d$, at least one of $\pi$ and $\rho$ has an entry which is greater than $2$; without loss of generality, we can assume it is $\rho$. Write $\pi=[a_1,\ldots,a_r]$, $\rho=[b_1,\ldots,b_s]$ with $a_1\ge 2$ (since $v(\pi)\ge 1$) and $b_1\ge 3$. Fix
\[
\beta=\cycle{1,\ldots,d_1}\cycle{d_1+1,\ldots,d_2}\cdots\cycle{d_{s-1}+1,\ldots,d_s},
\]
where $b_1=d_1\ge 3$, $b_i=d_i-d_{i-1}$ for $2\le i\le s$ (in particular, $d_s=d$). Consider the partitions
\begin{align*}
\pi'=[a_1-1,\ldots,a_r],&&\rho'=[b_1-1,\ldots,b_s].
\end{align*}
Since $\sum\pi'=\sum\rho'=d-1$ and $v(\pi')+v(\rho')=v(\pi)+v(\rho)-2$, by \cref{monodromy:th:product-reduction-odd} we can find permutations $\alpha',\beta'\in\symgroup(\{2,\ldots,d\})$ with $[\alpha']=\pi'$ and $[\beta']=\rho'$ such that $\alpha'\beta'$ is a $(d-1)$\=/cycle. Up to conjugation, we can assume that
\[
\beta'=\cycle{2,\ldots,d_1}\cycle{d_1+1,\ldots,d_2}\cdots\cycle{d_{s-1}+1,\ldots,d_s}
\]
or, in other words, $\beta=\cycle{1,2}\beta'$; moreover, as explained in \cref{monodromy:rm:product-reduction-odd-prescribed-cycles}, we can choose $\alpha'$ in such a way that its $(a_1-1)$\=/cycle contains $2$. By setting $\alpha=\alpha'\cycle{1,2}$, we immediately get that $\alpha\beta=\alpha'\beta'$ is a $(d-1)$\=/cycle fixing $1$. Finally, the action of $\angled{\alpha,\beta}$ is transitive since $\alpha$ does not fix $1$.
\end{proof}

\begin{corollary}\label{monodromy:th:product-reduction-large-v}
Let $\pi,\rho\in\Partitions{d}$ be partitions of $d$. Assume that $v(\pi)+v(\rho)\ge d-1$. Then there exist permutations $\alpha,\beta\in\symgroup[d]$ with $[\alpha]=\pi$, $[\beta]=\rho$ such that $\angled{\alpha,\beta}$ acts transitively on $\{1,\ldots,d\}$ and
\[
[\alpha\beta]\in\{[d],[1,d-1],[d/2,d/2]\}.
\]
\end{corollary}
\begin{proof}
The conclusion follows immediately from \cref{monodromy:th:product-reduction-large-v-odd} or \cref{monodromy:th:product-reduction-large-v-even} depending on the parity of $v(\pi)+v(\rho)+d$.
\end{proof}

\section{Projective plane}

\begin{theorem}
Let $\DD=\datum{\tSigma,\RP[2]}{d}{\pi_1,\ldots,\pi_n}$ be a candidate datum. If $\tSigma$ is non-orientable, then $\DD$ is realizable.
\end{theorem}
\begin{proof}
First of all, since $\DD$ is a candidate datum, the \RH{} formula implies that
\[
v(\pi_1)+\ldots+v(\pi_n)=d\chi(\RP[2])-\chi(\tSigma)\ge d-1
\]
(recall that $\tSigma$ is non-orientable, so $\chi(\tSigma)\le 1$). Moreover, the total branching $v(\pi_1)+\ldots+v(\pi_n)$ is even. In order to apply \cref{monodromy:th:monodromy-realizability-non-orientable}, we will now inductively define representatives $\alpha_i\in\symgroup[d]$ with $[\alpha_i]=\pi_i$, satisfying the following invariant: for every $0\le i\le n$, either
\[
v(\alpha_1\cdots\alpha_i)=v(\pi_1)+\ldots+v(\pi_i)
\]
or
\[
\text{$[\alpha_1\cdots\alpha_i]\in\{[d],[1,d-1],[d/2,d/2]\}$ and $\angled{\alpha_1,\ldots,\alpha_i}$ acts transitively.}
\]
Assume we have already defined $\alpha_1,\ldots,\alpha_{i-1}$; we want to suitably choose $\alpha_i$. Let $\alpha=\alpha_1\cdots\alpha_{i-1}$; there are two cases.
\begin{itemize}
\item If $v(\alpha)+v(\pi_i)<d$, by \cref{monodromy:th:product-reduction-small-v} we can find $\alpha_i\in\symgroup[d]$ with $[\alpha_i]=\pi_i$ such that $v(\alpha\alpha_i)=v(\alpha)+v(\alpha_i)$. The invariant is still satisfied: if $v(\alpha)=v(\pi_1)+\ldots+v(\pi_{i-1})$ then obviously $v(\alpha\alpha_i)=v(\pi_1)+\ldots+v(\pi_i)$. If instead $[\alpha]\in\{[d],[1,d-1],[d/2,d/2]\}$, then either $\alpha_i$ is the identity, or $\alpha\alpha_i$ is a $d$\=/cycle; either way, $[\alpha\alpha_i]\in\{[d],[1,d-1],[d/2,d/2]\}$. Note that if the action of $\angled{\alpha_1,\ldots,\alpha_{i-1}}$ is transitive, then the action of $\angled{\alpha_1,\ldots,\alpha_i}$ is transitive as well.
\item If $v(\alpha)+v(\alpha_i)\ge d$, \cref{monodromy:th:product-reduction-large-v} gives a permutation $\alpha_i\in\symgroup[d]$ with $[\alpha_i]=\pi_i$ such that $[\alpha\alpha_i]\in\{[d],[1,d-1],[d/2,d/2]\}$ and the action of $\angled{\alpha,\alpha_i}$ is transitive. The invariant is obviously satisfied.
\end{itemize}
By induction, we can find $\alpha_1,\ldots,\alpha_n\in\symgroup[d]$ with $[\alpha_i]=\pi_i$ such that
\[
[\alpha_1\cdots\alpha_n]\in\{[d],[1,d-1],[d/2,d/2]\}
\]
and $\angled{\alpha_1,\ldots,\alpha_n}$ acts transitively on $\{1,\ldots,d\}$ (note that $v(\alpha_1\cdots\alpha_n)=v(\alpha_1)+\ldots+v(\alpha_n)$ also implies that $[\alpha_1\cdots\alpha_n]=[d]$). Note that\todo{Show that $v(\alpha\beta)\equiv v(\alpha)+v(\beta)\pmod{2}$.}
\[
v(\alpha_1\cdots\alpha_n)\equiv v(\alpha_1)+\ldots+v(\alpha_n)\equiv 0\pmod{2}.
\]
We now prove that $\alpha=\alpha_1\cdots\alpha_n$ is a square.
\begin{itemize}
\item If $[\alpha]=[d]$, then $d$ is odd, so $\alpha$ is the square of $\alpha^{(d+1)/2}$.
\item If $[\alpha]=[1,d-2]$, then $d$ is even, so $\alpha$ is the square of $\alpha^{d/2}$.
\item If $[\alpha]=[d/2,d/2]$, then $d$ is even, and it is easy to see that $\alpha$ is the square of a $d$-cycle.
\end{itemize}
By \cref{monodromy:th:monodromy-realizability-non-orientable}, this implies that there exists a realizable candidate datum $\DD'=\datum{\tSigma',\RP[2]}{d}{\pi_1,\ldots,\pi_n}$. To see that $\tSigma'$ is non-orientable (and, therefore, equal to $\tSigma$, as shown in \cref{monodromy:rm:sigma-tilde-unique-non-orientable}), simply note that ??.
\end{proof}

\section{Reduction technique on the sphere}

\todo{Assume $n\ge 3$.}

\begin{proposition}\label{monodromy:th:sphere-[d]}
Let $\DD=\datum{\surf{g},\sphere{}}{d}{\pi_1,\ldots,\pi_{n-1},[d]}$ be a candidate datum. Then $\DD$ is realizable.
\end{proposition}
\begin{proof}
We proceed by induction on $n$, starting with the base case $n=3$. If $\DD=\datum{\surf{g},\sphere{}}{d}{\pi_1,\pi_2,[d]}$ is a candidate datum, by the \RH{} formula we have that
\[
v(\pi_1)+v(\pi_2)=2d-2+2g-(d-1)\ge d-1.
\]
Moreover, the total branching number $v(\pi_1)+v(\pi_2)+d-1$ is even. \Cref{monodromy:th:product-reduction-odd} then gives permutations $\alpha_1,\alpha_2\in\symgroup[d]$ with $[\alpha_1]=\pi_1$ and $[\alpha_2]=\pi_2$ such that $[\alpha_1\alpha_2]=[d]$. By \cref{monodromy:th:monodromy-realizability-orientable}, $\DD$ is realizable (see \cref{monodromy:rm:sigma-tilde-unique-orientable}).

We now turn to the case $n\ge 4$. Fix a candidate datum $\DD=\datum{\surf{g},\sphere{}}{d}{\pi_1,\ldots,\pi_{n-1},[d]}$; there are two cases.
\begin{itemize}
\item Assume that $v(\pi_1)+v(\pi_2)\le d-1$. By \cref{monodromy:th:product-reduction-small-v}, we can find permutations $\alpha_1,\alpha_2\in\symgroup[d]$ with $[\alpha_1]=\pi_1$ and $[\alpha_2]=\pi_2$ such that $v(\alpha_1\alpha_2)=v(\alpha_1)+v(\alpha_2)$. Consider the candidate datum $\DD'=\datum{\surf{g},\sphere{}}{d}{[\alpha_1\alpha_2],\pi_3,\ldots,\pi_{n-1},[d]}$. By induction, $\DD'$ is realizable; \cref{monodromy:th:monodromy-realizability-orientable} then gives permutations $\alpha_3,\ldots,\alpha_n\in\symgroup[d]$ with $[\alpha_i]=\pi_i$ for $3\le i\le n-1$ and $[\alpha_n]=[d]$ such that $(\alpha_1\alpha_2)\alpha_3\cdots\alpha_n=1$. It is easy to see that the permutations $\alpha_1,\alpha_2,\alpha_3,\ldots,\alpha_n$ imply the realizability of $\DD$, again by \cref{monodromy:th:monodromy-realizability-orientable} (together, as usual, with \cref{monodromy:rm:sigma-tilde-unique-orientable}).
\item Otherwise, we have that $v(\pi_1)+v(\pi_2)\ge d$. By \cref{monodromy:th:product-reduction-large-v}, we can find permutations $\alpha_1,\alpha_2$ with $[\alpha_1]=\pi_1$ and $[\alpha_2]=\pi_2$ such that $v(\alpha_1\alpha_2)\ge d-2$. Let
\begin{align*}
g'&=\frac{1}{2}(v(\alpha_1\alpha_2)+v(\pi_3)+\ldots+v(\pi_{n-1})+d-1)-d+1\\
&\ge\frac{1}{2}(d-2+1+d-1)-d+1\ge 0.
\end{align*}
It is easy to see that $\DD'=\datum{\surf{g'},\sphere{}}{d}{[\alpha_1\alpha_2],\pi_3,\ldots,\pi_n}$ is a candidate datum, so it is realizable by induction. Similarly to the previous case, this implies that $\DD$ is realizable.\qedhere
\end{itemize}
\end{proof}

\todo{Explain the reduction technique, with transitivity.}

\begin{proposition}\label{monodromy:th:sphere-[1 d-1]}
Let $\DD=\datum{\surf{g},\sphere{}}{d}{\pi_1,\ldots,\pi_{n-1},[1,d-1]}$ be a proper candidate datum. Then $\DD$ is non-realizable if and only if it satisfies one of the following.
\begin{enumerate}[(1)]
\item $\DD=\datum{\surf{n-3},\sphere{}}{4}{[2,2],\ldots,[2,2],[1,3]}$.
\item $\DD=\datum{\sphere{},\sphere{}}{2k}{[2,\ldots,2],[2,\ldots,2],[1,2k-1]}$ with $k\ge 2$.
\end{enumerate}
\end{proposition}
\begin{proof}
\todo{The non-realizability of the listed candidate data will be addressed at some point.}We proceed by induction on $n$, starting from the base case $n=3$. Let
\[
\DD=\datum{\surf{g},\sphere{}}{d}{\pi_1,\pi_2,[1,d-1]}\neq\datum{\sphere{},\sphere{}}{d}{[2,\ldots,2],[2,\ldots,2],[1,d-1]}
\]
be a candidate datum. The \RH{} formula implies that
\[
v(\pi_1)+v(\pi_2)=2d-2+2g-(d-2)\ge d.
\]
Moreover, the total branching number $v(\pi_1)+v(\pi_2)+d-2$ is even. \Cref{monodromy:th:product-reduction-large-v-even} then gives permutations $\alpha_1,\alpha_2\in\symgroup[d]$ with $[\alpha_1]=\pi_1$ and $[\alpha_2]=\pi_2$ such that $[\alpha_1\alpha_2]=[1,d-1]$ and the action of $\angled{\alpha_1,\alpha_2}$ is transitive. As usual, \cref{monodromy:th:monodromy-realizability-orientable} implies that $\DD$ is realizable.

We now turn to the case $n\ge 4$; we will employ a reduction technique. Fix a candidate datum
\[
\DD=\datum{\surf{n-3},\sphere{}}{d}{\pi_1,\ldots,\pi_{n-1},[1,d-1]}\neq\datum{\surf{n-3},\sphere{}}{4}{[2,2],\ldots,[2,2],[1,3]}.
\]
There are two cases.
\begin{itemize}
\item Assume that the inequality $v(\pi_i)+v(\pi_j)<d$ holds for a pair of indices $1\le i<j\le n-1$; up to reindexing, we can assume that $i=1$ and $j=2$. By \cref{monodromy:th:product-reduction-small-v}, we can find permutations $\alpha_1,\alpha_2\in\symgroup[d]$ with $[\alpha_1]=\pi_1$ and $[\alpha_2]=\pi_2$ such that $v(\alpha_1\alpha_2)=v(\alpha_1)+v(\alpha_2)$. Consider the candidate datum $\DD'=\datum{\surf{g},\sphere{}}{d}{[\alpha_1\alpha_2],\pi_3,\ldots,\pi_{n-1},[1,d-1]}$. By induction, $\DD'$ is realizable unless one of the following happens.
\begin{itemize}
\item $\DD'=\datum{\sphere{},\sphere{}}{2k}{[2,\ldots,2],[2,\ldots,2],[1,2k-1]}$ with $2k=d$. If this is the case, then $n=4$ and $v(\pi_1)+v(\pi_2)=v(\pi_3)=k$. This implies that
\[
k<1+k\le v(\pi_1)+v(\pi_3)<2k=d.
\]
Repeating the construction with $i=1$ and $j=3$ will yield a realizable $\DD'$.
\item $\DD'=\datum{\surf{n-3},\sphere{}}{4}{[2,2],\ldots,[2,2],[1,3]}$. This implies that $\pi_1=\pi_2=[1,1,2]$ and $\pi_3=\ldots=\pi_{n-1}=[2,2]$. Repeating the construction with $i=1$ and $j=3$ will yield a realizable $\DD'$.
\end{itemize}
Therefore we can assume that $\DD'$ is realizable. A reduction argument shows that $\DD$ is realizable as well.
\item Otherwise, the inequality $v(\pi_1)+v(\pi_j)\ge d$ holds for every $1\le i<j\le n-1$. \Cref{monodromy:th:product-reduction-large-v} then gives permutations $\alpha_1,\alpha_2\in\symgroup[d]$ with $[\alpha_1]=\pi_1$ and $[\alpha_2]=\pi_2$ such that $v(\alpha_1\alpha_2)\ge d-2$. If $d=4$ we can assume that $\pi_1\neq[2,2]$, so that  we can choose $[\alpha_1\alpha_2]\neq[2,2]$ (see \cref{monodromy:th:product-reduction-large-v-even}). Let
\begin{align*}
g'&=\frac{1}{2}(v(\alpha_1\alpha_2)+v(\pi_3)+\ldots+v(\pi_{n-1})+d-2)-d+1\\
&\ge \frac{1}{2}(d-2+1+d-2)-d+1\ge-\frac{1}{2}.
\end{align*}
Actually, since the total branching number of $\DD$ is even, $g'$ must be an integer, therefore $g'\ge 0$. It is easy to see that $\DD'=\datum{\surf{g'},\sphere{}}{d}{[\alpha_1\alpha_2],\pi_3,\ldots,\pi_{n-1},[1,d-1]}$ is a candidate datum. Moreover, $\DD'$ is realizable by induction: the case $d=4$ was addressed above, and $\DD'=\datum{\sphere{},\sphere{}}{d}{[2,\ldots,2],[2,\ldots,2],[1,d-2]}$ is impossible  for $d\ge 6$, since
\[
v(\alpha_1\alpha_2)\ge d-2>\frac{d}{2}=v([2,\ldots,2]).
\]
The usual reduction argument implies that $\DD$ is realizable as well.\qedhere
\end{itemize}
\end{proof}


\begin{proposition}\label{monodromy:th:sphere-large-g}
Let $\DD=\datum{\surf{g},\sphere{}}{d}{\pi_1,\ldots,\pi_n}$ be a candidate datum. Assume that $d\neq 4$ and that $2g\ge d-1$. Then $\DD$ is realizable.
\end{proposition}
\begin{proof}
Note that the the condition $2g\ge d-1$ is equivalent to
\[
v(\pi_1)+\ldots+v(\pi_n)\ge 3d-3
\]
by the \RH{} formula. Moreover, the cases where $d=2$ are trivial; therefore, assume that $d\ge 3$.

We proceed by induction, starting from the base case $n=3$. If $n=3$, the inequality $v(\pi_1)+v(\pi_2)+v(\pi_3)\ge 3d-3$ implies that $\pi_1=\pi_2=\pi_3=[d]$; by \cref{monodromy:th:sphere-[d]}, $\DD$ is realizable.

We now turn to the case $n\ge 4$; we will employ a reduction technique. Fix a candidate datum $\DD=\datum{\surf{g},\sphere{}}{d}{\pi_1,\ldots,\pi_n}$. We can assume that $\pi_i\neq[d]$ for every $1\le i\le n$, otherwise $\DD$ is immediately realizable by \cref{monodromy:th:sphere-[d]}. There are two cases.
\begin{itemize}
\item Assume that the inequality $v(\pi_i)+v(\pi_j)<d$ holds for a pair of indices $1\le i<j\le n$. The standard reduction argument shows that $\DD$ is realizable in this case.
\item Otherwise, we have $v(\pi_i)+v(\pi_j)\ge d$ for every $1\le i<j\le n$. We consider two sub-cases.
\begin{itemize}
\item Assume first that there is a partition, say $\pi_1$, which is different from $[2,\ldots,2]$. By \cref{monodromy:th:product-reduction-odd,monodromy:th:product-reduction-large-v-even}, we can find permutations $\alpha_1,\alpha_2\in\symgroup[d]$ with $[\alpha_1]=\pi_1$ and $[\alpha_2]=\pi_2$ such that $[\alpha_1\alpha_2]\in\{[d],[1,d-1]\}$. Let
\begin{align*}
g'&=\frac{1}{2}(v(\alpha_1\alpha_2)+v(\pi_3)+\ldots+v(\pi_n))-d+1\\
&\ge \frac{1}{2}(d-2+d)-d+1\ge 0.
\end{align*}
Consider the candidate datum $\DD'=\datum{\surf{g'},\sphere{}}{d}{[\alpha_1\alpha_2],\pi_3,\ldots,\pi_n}$. By \cref{monodromy:th:sphere-[d],monodromy:th:sphere-[1 d-1]}, $\DD'$ is realizable unless $\DD'=\datum{\sphere{},\sphere{}}{d}{[2,\ldots,2],[2,\ldots,2],[1,d-1]}$. If this were the case, we would have $n=4$ and
\[
v(\pi_1)+v(\pi_2)+v(\pi_3)+v(\pi_4)\le d-2+d-2+\frac{d}{2}+\frac{d}{2}=3d-4,
\]
which contradicts the hypothesis. Therefore $\DD'$ is realizable; by the usual reduction argument, $\DD$ is realizable as well.
\item Finally, consider the case where $\pi_1=\ldots=\pi_n=[2,\ldots,2]$. In this situation $d=2k$ is even and
\[
v(\pi_1)+\ldots+v(\pi_n)=nk>6k-3
\]
(the inequality is strict since the total branching number is even, while $6k-3$ is odd). Since $k\ge 3$, this immediately implies that $n\ge 6$. Applying \cref{monodromy:th:product-reduction-large-v-even} once and then \cref{monodromy:th:product-reduction-odd} or \cref{monodromy:th:product-reduction-large-v-even} again yields permutations $\alpha_1,\alpha_2,\alpha_3\in\symgroup[d]$ with $[\alpha_1]=\pi_1$, $[\alpha_2]=\pi_2$ and $[\alpha_3]=\pi_3$ such that $[\alpha_1\alpha_2\alpha_3]\in\{[d],[1,2k-1]\}$. Let
\begin{align*}
g'&=\frac{1}{2}(v(\alpha_1\alpha_2\alpha_3)+v(\pi_4)+\ldots+v(\pi_n))-2k+1\\
&\ge\frac{1}{2}(2k-2+(n-3)k)-2k+1=\frac{n-5}{2}\cdot k>0.
\end{align*}
It is easy to see that $\DD'=\datum{\surf{g'},\sphere{}}{2k}{[\alpha_1\alpha_2\alpha_3],\pi_4,\ldots,\pi_n}$ is a candidate datum. By \cref{monodromy:th:sphere-[d],monodromy:th:sphere-[1 d-1]}, $\DD'$ is realizable; a reduction argument shows that  $\DD$ is realizable as well.\qedhere
\end{itemize} 
\end{itemize}
\end{proof}

\begin{corollary}
Let $d$ be a positive integer with $d\neq 4$. Then there exist at most finitely many non-realizable proper candidate data of the form $\datum{\surf{g},\sphere{}}{d}{\pi_1,\ldots,\pi_n}$.
\end{corollary}
\begin{proof}
By \cref{monodromy:th:sphere-large-g}, every proper candidate datum with $n\ge 3d-3$ is realizable, and there are a finite number of proper candidate data with $n<3d-3$.
\end{proof}

\begin{proposition}\label{monodromy:th:sphere-d-equals-4}
Let $\DD=\datum{\surf{g},\sphere{}}{4}{\pi_1,\ldots,\pi_n}$ be a proper candidate datum. Then $\DD$ is realizable if and only if
\[
\DD\neq\datum{\surf{n-3},\sphere{}}{4}{[2,2],\ldots,[2,2],[1,3]}.
\]
\end{proposition}
\begin{proof}
\todo{Non-realizability will be addressed at some point.}We proceed by induction, starting from the base case $n=3$. Note that if any partition is either $[4]$ or $[1,3]$, we immediately conclude by \cref{monodromy:th:sphere-[d],monodromy:th:sphere-[1 d-1]}. By the \RH{} formula, we have the inequality $v(\pi_1)+v(\pi_2)+v(\pi_3)\ge 6$; therefore, the only candidate datum left to consider is $\datum{\sphere{},\sphere{}}{4}{[2,2],[2,2],[2,2]}$. This datum is realizable, for instance, by choosing
\begin{align*}
\alpha_1=\cycle{1,2}\cycle{3,4},&&\alpha_2=\cycle{1,3}\cycle{2,4},&&\alpha_3=\cycle{1,4}\cycle{2,3}.
\end{align*}

We now turn to the case $n\ge 4$; once again, we can assume that all the partitions are either $[1,1,2]$ or $[2,2]$. There are three cases.
\begin{itemize}
\item If all the partitions are equal to $[2,2]$, then we can apply \cref{monodromy:th:product-reduction-large-v-even} to combine two partitions into a single $[2,2]$, and conclude by a reduction argument.
\item If all the partitions are equal to $[1,1,2]$, then by the \RH{} formula we have $n\ge 6$. We can apply \cref{monodromy:th:product-reduction-small-v} twice to combine three partitions into a single $[d]$, and conclude by reduction.
\item Otherwise, there is at least one $[2,2]$ and one $[1,1,2]$; using \cref{monodromy:th:product-reduction-small-v}, we can combine them into a single $[d]$ and conclude by reduction.\qedhere
\end{itemize}
\end{proof}


\section{Prime-degree conjecture}
\begin{proposition}
Let $d$ be a positive integer. Assume that every candidate datum $\datum{\surf{g},\sphere{}}{d}{\pi_1,\pi_2,\pi_3}$ is realizable. Then every candidate datum $\datum{\surf{g},\sphere{}}{d}{\pi_1,\ldots,\pi_n}$ with $n\ge 3$ is realizable.
\end{proposition}
\begin{proof}
We proceed by induction on $n\ge 4$. Fix a candidate datum $\DD=\datum{\surf{g},\sphere{}}{d}{\pi_1,\ldots,\pi_{n}}$; there are two cases.
\begin{itemize}
\item If there are two indices $1\le i<j\le n$ such that $v(\pi_i)+v(\pi_j)\le d-1$, then a routine reduction argument shows that $\DD$ is realizable.
\item Otherwise, $v(\pi_i)+v(\pi_j)\ge d$ for every $1\le i<j\le n$. By \cref{monodromy:th:product-reduction-large-v}, we can find permutations $\alpha_1,\alpha_2$ with $[\alpha_1]=\pi_1$ and $[\alpha_2]=\pi_2$ such that $v(\alpha_1\alpha_2)\ge d-2$. Let
\begin{align*}
g'&=\frac{1}{2}(v(\alpha_1\alpha_2)+v(\pi_3)+\ldots+v(\pi_n))-d+1\\
&\ge\frac{1}{2}(d-2+d)-d+1\ge 0.
\end{align*}
It is easy to see that $\DD'=\datum{\surf{g'},\sphere{}}{d}{[\alpha_1\alpha_2],\pi_3,\ldots,\pi_n}$ is a candidate datum, so it is realizable by induction. A reduction argument shows that $\DD$ is realizable as well.\qedhere
\end{itemize}
\end{proof}