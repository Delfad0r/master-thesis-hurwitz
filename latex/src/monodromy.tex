\chapter{Realizablility by monodromy}



\section{Non-positive Euler characteristic}

The monodromy approach developed in the previous chapter will allow us to completely solve the Hurwitz existence problem for a large class of candidate data. For the majority of this chapter, we will closely follow \cite{edmonds}, with a slight rewording of the exposition. Perhaps, the most notable deviation will occur in \cref{monodromy:sc:combinatorial-moves,monodromy:sc:results-sphere}, where we will introduce \emph{combinatorial moves} and apply them extensively in our proofs. It should be noted, however, that combinatorial moves can make some arguments easier to follow and less repetitive, but they do not provide any conceptual innovations; the core ideas of the proofs are the same as those presented in aforementioned article.

With the following elementary result about symmetric groups, we will already be able to fully address the cases where $\chi(\Sigma)\le 0$.

\begin{lemma}\label{monodromy:th:product-of-two-cycles}
Let $\alpha\in\symgroup[d]$ be a permutation. Set $r=d-v(\alpha)$, and let $t\ge 0$ be an integer such that $2t\le v(\alpha)$. Then $\alpha$ can be written as the product of a $(r+2t)$\=/cycle and a $d$\=/cycle.
\end{lemma}
\begin{proof}
Without loss of generality, assume that
\[
\alpha=\cycle{1,\ldots,d_1}\cycle{d_1+1,\ldots,d_2}\cdots\cycle{d_{r-1}+1,\ldots,d_r},
\]
where $d_r=d$. Fix
\begin{align*}
\beta_0=\cycle{1,b_1,b_2,\ldots,b_{2t}},&&\beta_1=\cycle{1,d_1+1,d_2+1,\ldots,d_{r-1}+1},
\end{align*}
where $b_1<b_2<\ldots<b_{2t}$ are elements of $\{1,\ldots,d\}\setminus\{d_1+1,\ldots,d_{r-1}+1\}$. Set
\[
\beta=\beta_0\beta_1=\cycle{1,d_1+1,d_2+1,\ldots,d_{r-1}+1,b_1,b_2,\ldots,b_{2t}}.
\]
An easy computation shows that
\begin{align*}
\beta\alpha&=\beta_0\beta_1\alpha\\
&=\cycle{1,b_1,b_2,\ldots,b_{2t}}\cycle{1,2,\ldots,d}\\
&=\cycle{1,\ldots,b_1-1,b_2,\ldots,b_3-1,\ldots,b_4,\ldots,b_{2t-1}-1,b_{2t},\ldots,d,b_1,\ldots,b_2-1,b_3,\ldots,b_{2t}-1}.
\end{align*}
Writing $\alpha=\beta^{-1}(\beta\alpha)$ gives the desired decomposition.
\end{proof}

\begin{corollary}\label{monodromy:th:even-permutation-commutator-or-squares}
Let $\alpha\in\altgroup[d]$ be an even permutation. Then $\alpha$ can be written as:
\begin{enumroman}
\item a commutator $[\beta,\gamma]$, where $\gamma$ is a $d$\=/cycle;
\item a product of two squares $\delta^2\epsilon^2$, where $\delta\epsilon$ is a $d$\=/cycle.
\end{enumroman}
\end{corollary}
\begin{proof}
Since $\alpha$ is an even permutation, its branching number $v(\alpha)$ is even. By \cref{monodromy:th:product-of-two-cycles}, there exist two $d$\=/cycles $\tau,\sigma\in\symgroup[d]$ such that $\alpha=\tau\sigma$.
\begin{enumroman}
\item Since $\tau$ and $\sigma^{-1}$ are conjugated, there exists a permutation $\beta\in\symgroup[d]$ such that $\tau=\beta\sigma^{-1}\beta^{-1}$. Setting $\gamma=\sigma^{-1}$, we immediately get that
\[
\alpha=\tau\sigma=\beta\sigma^{-1}\beta^{-1}\sigma=\beta\gamma\beta^{-1}\gamma^{-1}=[\beta,\gamma].
\]
\item Since $\tau$ and $\sigma$ are conjugated, there exists a permutation $\delta\in\symgroup[d]$ such that $\tau=\delta\sigma\delta^{-1}$. Setting $\epsilon=\delta^{-1}\sigma$, we have that
\[
\alpha=\tau\sigma=\delta\sigma\delta^{-1}\sigma=\delta^2(\delta^{-1}\sigma)^2=\delta^2\epsilon^2.\qedhere
\]
\end{enumroman}
\end{proof}

\begin{theorem}\label{monodromy:th:realizability-nonpositive-chi}
Let $\DD=\datum{\tSigma,\Sigma}{d}{\pi_1,\ldots,\pi_n}$ be a candidate datum. If $\chi(\Sigma)\le 0$, then $\DD$ is realizable.
\end{theorem}
\begin{proof}
There are three cases.
\begin{sideline}{Case 1.}
Let us first assume that $\Sigma$ is orientable; this means that $\Sigma=\surf{g}$ is the connected sum of $g\ge 1$ tori, and that $\tSigma$ is orientable as well. Choose permutations $\alpha_1,\ldots,\alpha_n\in\symgroup[d]$ with $[\alpha_i]=\pi_i$ for each $1\le i\le n$. Let $\alpha=\alpha_1\cdots\alpha_n$. Since $\DD$ is a candidate datum, we have that
\[
v(\alpha)\equiv v(\alpha_1)+\ldots+v(\alpha_n)\equiv 0\pmod{2}.
\]
By \cref{monodromy:th:even-permutation-commutator-or-squares}, we can find permutations $\beta_1,\gamma_1\in\symgroup[d]$ such that $\alpha=[\gamma_1,\beta_1]$ and $\beta_1$ is a $d$\=/cycle. Set $\beta_2=\ldots=\beta_g=\gamma_2=\ldots=\gamma_g=\id\in\symgroup[d]$. All the conditions of \cref{hurwitz:th:monodromy-realizability-orientable} are satisfied; since $\tSigma$ is orientable, this implies that $\DD$ is realizable (see \cref{hurwitz:rm:sigma-tilde-unique-orientable}).
\end{sideline}
\begin{sideline}{Case 2.}
Assume now that $\Sigma$ and $\tSigma$ are both non-orientable; this means that $\Sigma=\nosurf{g}$ is the connected sum of $g\ge 2$ projective planes. Choose permutations $\alpha_1,\ldots,\alpha_n\in\symgroup[d]$ with $[\alpha_i]=\pi_i$ for each $1\le i\le n$. Let $\alpha=\alpha_1\cdots\alpha_n$. Similarly to what we did for the previous case, we can find $\beta_1,\beta_2\in\symgroup[d]$ such that $\alpha=\beta_2^{-2}\beta_1^{-2}$ and $\beta_1\beta_2$ is a $d$\=/cycle. Note that, since $\beta_1\beta_2$ is a $d$\=/cycle, there exists an integer $m\ge 0$ such that $\beta_1(\beta_1\beta_2)^m$ has a fixed point. By setting $\beta_3=\ldots=\beta_g=\id\in\symgroup[d]$, \cref{hurwitz:th:monodromy-realizability-non-orientable} (together with \cref{hurwitz:rm:sigma-tilde-unique-non-orientable}) implies the realizability of $\DD$.
\end{sideline}
\begin{sideline}{Case 3.}
Finally, consider the case where $\Sigma$ is non-orientable and $\tSigma$ is orientable. Since $\DD$ is a candidate datum, $d$ is even and there exist partitions $\pi'_1,\ldots,\pi'_n,\pi''_1,\ldots,\pi''_n\in\Partitions{d/2}$ with $\pi_i=\pi'_1\cup\pi''_i$ for each $1\le i\le n$. Let $\hSigma$ be the double orientable covering of $\Sigma$; by Case 1, the candidate datum
\[
\DD'=\datum{\tSigma,\hSigma}{d/2}{\pi'_1,\ldots,\pi'_n,\pi''_1,\ldots,\pi''_n}
\]
is realizable. By \cref{hurwitz:th:monodromy-realizability-double-covering}, $\DD$ is realizable as well.\qedhere
\end{sideline}
\end{proof}

\section{Products in symmetric groups}

The monodromy approach can prove to be quite fruitful also in the cases where $\chi(\Sigma)>0$ (namely, when $\Sigma$ is either the sphere $\sphere$ or the projective plane $\RP[2]$). In order to deal with this instances of the Hurwitz existence problem, however, we will need a few more technical results about products of permutations.

\begin{lemma}\label{monodromy:th:same-number-of-cycles}
Let $X,Y$ be finite sets; denote by $h$ the cardinality of $Y$, and by $k$ the cardinality of $X\cap Y$. Let $\alpha\in\symgroup(X)$, $\beta\in\symgroup(Y)$, $\gamma\in\symgroup(X\cap Y)$. Assume that $\beta=\cycle{b_1,\ldots,b_h}$ is a $h$\=/cycle, and that $\gamma$ is a $k$\=/cycle of the form $\gamma=\cycle{b_{i_1},\ldots,b_{i_k}}$ with $1\le i_1\le\ldots\le i_k\le h$. Then $\alpha\in\symgroup(X)$ and $\alpha\gamma^{-1}\beta\in\symgroup(X\cup Y)$ have the same number of cycles.
\end{lemma}
\begin{proof}
Write $\gamma=\cycle{u_1,\ldots,u_k}$, where $u_j=b_{i_j}$. Without loss of generality, assume that $i_1=1$. Then we can write
\[
\beta=\cycle{u_1,\ldots,w_1,u_2,\ldots,w_2,u_3,\ldots,w_{k-1},u_k,\ldots,w_k},
\]
possibly with $u_j=w_j$ for some values of $j$. We immediately get that
\[
\gamma^{-1}\beta=\cycle{u_1,\ldots,w_1}\cycle{u_2,\ldots,w_2}\cdots\cycle{u_k,\ldots,w_k}.
\]
If we denote by $A_1,\ldots,A_r\subs X$ the orbits of $\alpha$, it is then easy to see that the orbits of $\alpha\gamma^{-1}\beta$ are $A_1',\ldots,A_r'\subs X\cup Y$, where
\[
A_j'=A_j\cup\bigcup_{u_l\in A_j}\{u_l,\ldots,w_l\}.\qedhere
\]
\end{proof}

\begin{comment}
\begin{lemma}
\todo{Maybe useless?}Let $\alpha,\beta\in\symgroup[d]$ be permutations. Then $v(\alpha\beta)\le v(\angled{\alpha,\beta})\le v(\alpha)+v(\beta)$. Moreover if $v(\angled{\alpha,\beta})=v(\alpha)+v(\beta)$ then $v(\alpha\beta)=v(\alpha)+v(\beta)$.
\end{lemma}
\begin{proof}
Let $q=v(\angled{\alpha,\beta})$, and let $\gamma=\alpha\beta$. Since $\gamma\in\angled{\alpha,\beta}$, the inequality $v(\gamma)\le q$ trivially holds. We have that $\alpha\beta\gamma^{-1}=1$; by \cref{?}, this implies that the combinatorial datum $\datum{\tSigma,S^2}{d}{[\alpha],[\beta],[\gamma]}$ is realizable for some orientable surface $\tSigma$. The \RH{} formula for this datum is
\[
v(\alpha)+v(\beta)+v(\gamma)=2d-\chi(\tSigma).
\]
Given that $\chi(\tSigma)\le 2\le 2(d-q)$, we find that
\[
v(\alpha)+v(\beta)=(v(\alpha)+v(\beta)+v(\gamma))-v(\gamma)\stackrel{(*)}{\ge} 2q-q=q\ge v(\gamma),
\]
which proves the first part of the lemma. Moreover, if $v(\alpha)+v(\beta)=q$, inequality $(*)$ must be an equality, hence $v(\gamma)=q$.
\end{proof}
\end{comment}

\begin{proposition}\label{monodromy:th:product-reduction-small-v}
Let $\pi,\rho\in\Partitions{d}$ be partitions of $d$. Assume that $v(\pi)+v(\rho)\le d-1$. Then there exist permutations $\alpha,\beta\in\symgroup[d]$ with $[\alpha]=\pi$ and $[\beta]=\rho$ such that $v(\alpha\beta)=v(\alpha)+v(\beta)$.
\end{proposition}
\begin{proof}
First of all, note that the conclusion is trivial whenever $v(\pi)=0$ or $v(\rho)=0$. This already solves the cases $d=1$ and $d=2$. We now proceed by induction on $d\ge 3$, assuming that $v(\pi)>0$ and $v(\rho)>0$. Write $\pi=[a_1,\ldots,a_r]$, $\rho=[b_1,\ldots,b_s]$; without loss of generality, assume that $b_1>1$. Fix
\[
\beta=\cycle{1,\ldots,d_1}\cycle{d_1+1,\ldots,d_2}\cdots\cycle{d_{s-1}+1,\ldots,d_s},
\]
where $b_1=d_1\ge 2$, $b_i=d_i-d_{i-1}$ for $2\le i\le s$ (in particular, $d_s=d$). Note that
\[
d-1\ge v(\pi)+v(\rho)=(a_1-1)+\ldots+(a_r-1)+d-s\ge a_1-1+d-s,
\]
hence $a_1\le s$. Fix $\alpha_1=\cycle{1,d_1+1,\ldots,d_{a_1-1}+1}$, and let $A=\{1,d_1+1,\ldots,d_{a_1-1}+1\}$ be the support of $\alpha_1$. Define $Q=\{1,\ldots,d_{a_1}\}\setminus A$; note that $Q$ is non-empty, since $d_1+1\ge 3$ implies that $2\in Q_1$. Consider the partitions
\begin{align*}
\pi'=[a_2,a_3,\ldots,a_r],&&\rho'=[\card{Q},b_{a_1+1},\ldots,b_s].
\end{align*}
We have that
\begin{align*}
\sum\pi'=\sum\rho'=d-a_1,&&v(\pi')=v(\pi)-a_1+1,&&v(\rho')=v(\rho)-1.
\end{align*}
Since $d-a_1<d$ and $v(\pi')+v(\rho')=v(\pi)+v(\rho)-a_1<d-a_1$, by induction we find $\alpha',\beta'\in\symgroup(\{1,\ldots,d\}\setminus A)$ with $[\alpha']=\pi'$ and $[\beta']=\rho'$ such that $v(\alpha'\beta')=v(\alpha')+v(\beta')$. Up to conjugation, we may assume that
\[
\beta'=\beta_1\cycle{d_{a_1}+1,\ldots,d_{a_1+1}}\cycle{d_{a_1+1}+1,\ldots,d_{a_1+2}}\cdots\cycle{d_{s-1}+1,\ldots,d_s},
\]
where $\beta_1$ is the $\card{Q}$\=/cycle whose entries are the elements of $Q$ in increasing order. An easy computation shows that
\[
\alpha_1\beta=\cycle{1,\ldots,d_{a_1}}\cycle{d_{a_1}+1,\ldots,d_{a_1+1}}\cdots\cycle{d_{s-1}+1,\ldots,d_s}.
\]
Therefore, setting $\alpha=\alpha'\alpha_1$, we have that
\begin{align*}
\alpha\beta&=\alpha_1\alpha'\beta\\
&=\alpha'\cycle{1,\ldots,d_{a_1}}\cycle{d_{a_1}+1,\ldots,d_{a_1+1}}\cdots\cycle{d_{s-1}+1,\ldots,d_s}\\
&=\alpha'\beta'\beta_1^{-1}\cycle{1,\ldots,d_{a_1}}
\end{align*}
By \cref{monodromy:th:same-number-of-cycles}, this implies that $\alpha\beta$ has the same number of cycles as $\alpha'\beta'$, so that
\begin{align*}
v(\alpha\beta)&=a_1+v(\alpha'\beta')\\
&=a_1+v(\alpha')+v(\beta')\\
&=a_1+(v(\pi)-a_1+1)+(v(\rho)-1)\\
&=v(\pi)+v(\rho).
\end{align*}
Since $[\alpha]=\pi$ and $[\beta]=\rho$, the conclusion follows.
\end{proof}

\begin{proposition}\label{monodromy:th:product-reduction-odd}
Let $\pi,\rho\in\Partitions{d}$ be partitions of $d$. Assume that $v(\pi)+v(\rho)\ge d-1$ and $v(\pi)+v(\rho)\equiv d-1\pmod{2}$. Then there exist permutations $\alpha,\beta\in\symgroup[d]$ with $[\alpha]=\pi$ and $[\beta]=\rho$ such that $\alpha\beta$ is a $d$\=/cycle.
\end{proposition}
\begin{proof}
If $v(\pi)+v(\rho)=d-1$, the conclusion follows immediately from \cref{monodromy:th:product-reduction-small-v}; therefore, we can assume that $v(\pi)+v(\rho)\ge d$.

Write $\pi=[a_1,\ldots,a_r]$. Since $v(\rho)\le d-1$ and $v(\pi)+v(\rho)\ge d$, there exists a largest integer $0\le i\le r$ such that $(a_1-1)+\ldots(a_i-1)+v(\rho)\le d-1$. Define
\[
z=d-v(\rho)-(a_1-1)-\ldots-(a_i-1).
\]
Consider the partition $\pi'=[a_1,\ldots,a_i,z,1,\ldots,1]\in\Partitions{d}$. Since by construction $v(\pi')+v(\rho)=d-1$, thanks to \cref{monodromy:th:product-reduction-small-v} we can find permutations $\alpha',\beta\in\symgroup[d]$ with $[\alpha']=\pi'$ and $[\beta]=\rho$ such that $v(\alpha'\beta)=d-1$; in other words, $\alpha'\beta$ is a $d$\=/cycle. Consider now the partition $\pi''=[a_{i+1}-z+1,a_{i+2},\ldots,a_r]$, whose branching number is $v(\pi'')=v(\pi)+v(\rho)-d+1$. Let $n=\sum\pi''$; fix an element $u_1$ of the $z$\=/cycle of $\alpha'$, and let $u_2,\ldots,u_n$ be the fixed points of $\alpha'$ corresponding to the last ones of $\pi'$ (it is easy to see that there are exactly $n-1$ such ones). Since $v(\pi'')$ is even, \cref{monodromy:th:product-of-two-cycles} gives permutations $\alpha'',\gamma\in\symgroup(\{u_1,\ldots,u_n\})$ such that $[\alpha'']=\pi''$ and $\gamma$ and $\alpha''\gamma$ are $n$\=/cycles. Up to conjugation, we can assume that $\gamma=\cycle{u_1,\ldots,u_n}$. Moreover, it is not restrictive to assume that $u_1,\ldots,u_n$ appear in this order in the $d$\=/cycle $\alpha'\beta$. Therefore, setting $\alpha=\alpha''\alpha'$, we have that
\[
\alpha\beta=(\alpha''\gamma)\gamma^{-1}(\alpha'\beta);
\]
by \cref{monodromy:th:same-number-of-cycles}, this implies that $\alpha\beta$ has the same number of cycles as $\alpha''\gamma\in\symgroup(\{u_1,\ldots,u_n\})$, that is $1$. Since $[\alpha]=\pi$ and $[\beta]=\rho$, the proof is complete.
\end{proof}

\begin{remark}\label{monodromy:rm:product-reduction-odd-prescribed-cycles}
Write $\pi=[a_1,\ldots,a_r]$, $\rho=[b_1,\ldots,b_s]$; assume that $b_1\ge 2$. By directly examining the proof of \cref{monodromy:th:product-reduction-odd}, we see that the proposed construction yields permutations $\alpha,\beta\in\symgroup[d]$ such that $1$ belongs to both the $a_1$\=/cycle of $\alpha$ and to the $b_1$\=/cycle of $\beta$. As a consequence, the statement of that proposition can be enhanced by adding the following line: \emph{$\alpha$ and $\beta$ can be chosen in such a way that $1$ belongs to the $a_1$\=/cycle of $\alpha$ and to the $b_1$\=/cycle of $\beta$, provided that $b_1\ge 2$}. We will need this improvement for the upcoming proof.
\end{remark}

\begin{proposition}\label{monodromy:th:product-reduction-large-v-even}
Let $\pi,\rho\in\Partitions{d}$ be partitions of $d$. Assume that $v(\pi)+v(\rho)\ge d$ and $v(\pi)+v(\rho)\equiv d\pmod{2}$. Then there exist permutations $\alpha,\beta\in\symgroup[d]$ with $[\alpha]=\pi$ and $[\beta]=\rho$ such that $\angled{\alpha,\beta}$ acts transitively on $\{1,\ldots,d\}$ and
\[
[\alpha\beta]=\begin{dcases*}
[d/2,d/2]&if $\pi=\rho=[2,\ldots,2]$,\\
[1,d-1]&otherwise.
\end{dcases*}
\]
\end{proposition}
\begin{proof}
Assume first that $\pi=\rho=[2,\ldots,d]$. We can choose
\begin{align*}
\alpha=\cycle{2,3}\cycle{4,5}\cdots\cycle{d,1},&&\beta=\cycle{1,2}\cycle{3,4}\cdots\cycle{d-1,d}.
\end{align*}
The action of $\angled{\alpha,\beta}$ is obviously transitive, and
\[
\alpha\beta=\cycle{1,3,\ldots,d-1}\cycle{2,4,\ldots,2}.
\]
Otherwise, since $v(\pi)+v(\rho)\ge d$, at least one of $\pi$ and $\rho$ has an entry which is greater than $2$; without loss of generality, we can assume it is $\rho$. Write $\pi=[a_1,\ldots,a_r]$, $\rho=[b_1,\ldots,b_s]$ with $a_1\ge 2$ (since $v(\pi)\ge 1$) and $b_1\ge 3$. Fix
\[
\beta=\cycle{1,\ldots,d_1}\cycle{d_1+1,\ldots,d_2}\cdots\cycle{d_{s-1}+1,\ldots,d_s},
\]
where $b_1=d_1\ge 3$, $b_i=d_i-d_{i-1}$ for $2\le i\le s$ (in particular, $d_s=d$). Consider the partitions
\begin{align*}
\pi'=[a_1-1,\ldots,a_r],&&\rho'=[b_1-1,\ldots,b_s].
\end{align*}
Since $\sum\pi'=\sum\rho'=d-1$ and $v(\pi')+v(\rho')=v(\pi)+v(\rho)-2$, by \cref{monodromy:th:product-reduction-odd} we can find permutations $\alpha',\beta'\in\symgroup(\{2,\ldots,d\})$ with $[\alpha']=\pi'$ and $[\beta']=\rho'$ such that $\alpha'\beta'$ is a $(d-1)$\=/cycle. Up to conjugation, we can assume that
\[
\beta'=\cycle{2,\ldots,d_1}\cycle{d_1+1,\ldots,d_2}\cdots\cycle{d_{s-1}+1,\ldots,d_s}
\]
or, in other words, $\beta=\cycle{1,2}\beta'$; moreover, as explained in \cref{monodromy:rm:product-reduction-odd-prescribed-cycles}, we can choose $\alpha'$ in such a way that its $(a_1-1)$\=/cycle contains $2$. By setting $\alpha=\alpha'\cycle{1,2}$, we immediately get that $\alpha\beta=\alpha'\beta'$ is a $(d-1)$\=/cycle fixing $1$. Finally, the action of $\angled{\alpha,\beta}$ is transitive since $\alpha$ does not fix $1$.
\end{proof}

The following result sums up the statements of \cref{monodromy:th:product-reduction-odd,monodromy:th:product-reduction-large-v-even}, at the cost of some generality. However, for many of the upcoming applications, it will be more than enough.

\begin{corollary}\label{monodromy:th:product-reduction-large-v}
Let $\pi,\rho\in\Partitions{d}$ be partitions of $d$. Assume that $v(\pi)+v(\rho)\ge d-1$. Then there exist permutations $\alpha,\beta\in\symgroup[d]$ with $[\alpha]=\pi$ and $[\beta]=\rho$ such that $\angled{\alpha,\beta}$ acts transitively on $\{1,\ldots,d\}$ and
\[
[\alpha\beta]\in\{[d],[1,d-1],[d/2,d/2]\}.
\]
\end{corollary}
\begin{proof}
The conclusion follows immediately from \cref{monodromy:th:product-reduction-odd} or \cref{monodromy:th:product-reduction-large-v-even} depending on the parity of $v(\pi)+v(\rho)+d$.
\end{proof}

\section{Projective plane}

As a first application of the results established in the previous section, we fully solve the Hurwitz existence problem in the cases where $\Sigma=\RP[2]$ and $\tSigma$ is non-orientable.

\begin{theorem}\label{monodromy:th:realizability-projective-plane}
Let $\DD=\datum{\tSigma,\RP[2]}{d}{\pi_1,\ldots,\pi_n}$ be a candidate datum. If $\tSigma$ is non-orientable, then $\DD$ is realizable.
\end{theorem}
\begin{proof}
First of all, since $\DD$ is a candidate datum, the \RH{} formula implies that
\[
v(\pi_1)+\ldots+v(\pi_n)=d\chi(\RP[2])-\chi(\tSigma)\ge d-1
\]
(recall that $\tSigma$ is non-orientable, so $\chi(\tSigma)\le 1$). Moreover, the total branching $v(\pi_1)+\ldots+v(\pi_n)$ is even. In order to apply \cref{hurwitz:th:monodromy-realizability-non-orientable}, we will now inductively define permutations $\alpha_i\in\symgroup[d]$ with $[\alpha_i]=\pi_i$, satisfying the following invariant: for every $0\le i\le n$, either
\[
v(\alpha_1\cdots\alpha_i)=v(\pi_1)+\ldots+v(\pi_i)
\]
or
\[
\text{$[\alpha_1\cdots\alpha_i]\in\{[d],[1,d-1],[d/2,d/2]\}$ and $\angled{\alpha_1,\ldots,\alpha_i}$ acts transitively.}
\]
Assume we have already defined $\alpha_1,\ldots,\alpha_{i-1}$; we want to suitably choose $\alpha_i$. Let $\alpha=\alpha_1\cdots\alpha_{i-1}$; there are two cases.
\begin{itemize}
\item If $v(\alpha)+v(\pi_i)<d$, by \cref{monodromy:th:product-reduction-small-v} we can find $\alpha_i\in\symgroup[d]$ with $[\alpha_i]=\pi_i$ such that $v(\alpha\alpha_i)=v(\alpha)+v(\alpha_i)$. The invariant is still satisfied: if $v(\alpha)=v(\pi_1)+\ldots+v(\pi_{i-1})$ then obviously $v(\alpha\alpha_i)=v(\pi_1)+\ldots+v(\pi_i)$. If instead $[\alpha]\in\{[d],[1,d-1],[d/2,d/2]\}$, then either $\alpha_i$ is the identity, or $\alpha\alpha_i$ is a $d$\=/cycle; either way, $[\alpha\alpha_i]\in\{[d],[1,d-1],[d/2,d/2]\}$. Note that if the action of $\angled{\alpha_1,\ldots,\alpha_{i-1}}$ is transitive, then the action of $\angled{\alpha_1,\ldots,\alpha_i}$ is transitive as well.
\item If $v(\alpha)+v(\alpha_i)\ge d$, \cref{monodromy:th:product-reduction-large-v} gives a permutation $\alpha_i\in\symgroup[d]$ with $[\alpha_i]=\pi_i$ such that $[\alpha\alpha_i]\in\{[d],[1,d-1],[d/2,d/2]\}$ and the action of $\angled{\alpha,\alpha_i}$ is transitive. The invariant is obviously satisfied.
\end{itemize}
By induction, we can find $\alpha_1,\ldots,\alpha_n\in\symgroup[d]$ with $[\alpha_i]=\pi_i$ such that
\[
[\alpha_1\cdots\alpha_n]\in\{[d],[1,d-1],[d/2,d/2]\}
\]
and $\angled{\alpha_1,\ldots,\alpha_n}$ acts transitively on $\{1,\ldots,d\}$ (note that $v(\alpha_1\cdots\alpha_n)=v(\alpha_1)+\ldots+v(\alpha_n)$ also implies that $[\alpha_1\cdots\alpha_n]=[d]$). By \cref{hurwitz:th:branching-number-permutations-product} we have that
\[
v(\alpha_1\cdots\alpha_n)\equiv v(\alpha_1)+\ldots+v(\alpha_n)\equiv 0\pmod{2}.
\]
We now prove that $\alpha=\alpha_1\cdots\alpha_n$ is a square.
\begin{itemize}
\item If $[\alpha]=[d]$, then $d$ is odd, so $\alpha$ is the square of $\alpha^{(d+1)/2}$.
\item If $[\alpha]=[1,d-1]$, then $d$ is even, so $\alpha$ is the square of $\alpha^{d/2}$.
\item If $[\alpha]=[d/2,d/2]$, then $d$ is even, and it is easy to see that $\alpha$ is the square of a $d$-cycle.
\end{itemize}
In any case, we obtain a permutations $\beta_1\in\symgroup[d]$ such that $\alpha=\beta_1^{-2}$
By \cref{hurwitz:th:monodromy-realizability-non-orientable}, this implies that there exists a realizable candidate datum $\DD'=\datum{\tSigma',\RP[2]}{d}{\pi_1,\ldots,\pi_n}$. To see that $\tSigma'$ is non-orientable (and, therefore, equal to $\tSigma$, as shown in \cref{hurwitz:rm:sigma-tilde-unique-non-orientable}), simply note that there exists a permutation $\gamma\in\angled{\alpha_1,\ldots,\alpha_n}$ such that $\gamma\beta_1$ has a fixed point, since $\angled{\alpha_1,\ldots,\alpha_n}$ acts transitively on $\{1,\ldots,d\}$.
\end{proof}

\section{Combinatorial moves}\label{monodromy:sc:combinatorial-moves}

With \cref{monodromy:th:realizability-nonpositive-chi,monodromy:th:realizability-projective-plane}, we have shown that every candidate datum $\DD=\datum{\tSigma,\Sigma}{d}{\pi_1,\ldots,\pi_n}$ is realizable whenever
\begin{itemize}
\item $\chi(\Sigma)\le 0$, or
\item $\Sigma=\RP[2]$ and $\tSigma$ is non-orientable.
\end{itemize}
Moreover, \cref{hurwitz:th:monodromy-realizability-double-covering} shows that the cases where $\Sigma=\RP[2]$ and $\tSigma$ is orientable can be reduced to the analysis of realizable candidate data on the sphere. As a consequence, it stands to reason that, from now on, we will only be concerned with candidate data where $\Sigma=\sphere{}$; in fact, a classification of the realizable data of this kind would lead to a full solution of the Hurwitz existence problem\todo{To be fair, even after solving the existence problem on the sphere, compiling the full list of exceptional data with $\Sigma=\RP[2]$ and $\tSigma$ orientable could still be non-trivial.}. However, we will soon find out that realizability on the sphere can be a very delicate matter: there are several seemingly unrelated families of exceptional data, and even to this day we are still far from a complete understanding of the problem.

Since, from now on, we will only deal with branched coverings on the sphere, we adopt the following convention: whenever we have a combinatorial datum $\DD=\datum{\surf{g},\sphere{}}{d}{\pi_1,\ldots\pi_n}$, we can omit the $\sphere{}$ and simply write $\DD=\datum{\surf{g}}{d}{\pi_1,\ldots\pi_n}$.

For convenience, we will now briefly discuss once and for all the cases where $n\le 2$.
\begin{itemize}
\item If $n=1$, \cref{hurwitz:th:monodromy-realizability-orientable} immediately implies that every candidate datum $\DD=\datum{\surf{g}}{d}{\pi_1}$ is not realizable (recall that we are always assuming $\pi_1\neq[1,\ldots,1]$).
\item If $n=2$, the only realizable candidate data are of the form $\DD=\datum{\sphere{}}{d}{[d],[d]}$, again by \cref{hurwitz:th:monodromy-realizability-orientable}.
\end{itemize}

Finally, we turn to the topic promised by the title of this section. \emph{Combinatorial moves} are a conceptually trivial tool, but they offer a convenient framework for presenting proofs based on a reduction technique. The definition is extremely simple.

\begin{definition}
Let $\DD,\DD'$ be combinatorial data. We say that there is a \emph{combinatorial move} between them, and we write $\DD\cmove\DD'$, if the realizability of $\DD'$ implies that $\DD$ is realizable as well.
\end{definition}

In this thesis we will encounter two kinds of combinatorial moves. The first one, which we introduce in this section, focuses on reducing the number of partitions of a combinatorial datum by means of products in the symmetric group. In \cref{dessins:sc:combinatorial-moves}, we will describe another kind of combinatorial moves, which relies instead on topological constructions to reduce the genus of the surface $\tSigma$.

\begin{combinatorialmovea}\label{combinatorial-move:a:small-v}
Let $\DD=\datum{\surf{g}}{d}{\pi_1,\pi_2,\ldots,\pi_n}$ be a candidate datum. Assume that $v(\pi_1)+v(\pi_2)\le d-1$. Then there exists a partition $\pi_1'\in\Partitions{d}$ such that $v(\pi_1')=v(\pi_1)+v(\pi_2)$, $\DD'=\datum{\surf{g}}{d}{\pi_1',\pi_3,\ldots,\pi_n}$ is a candidate datum and $\DD\cmove\DD'$.
\end{combinatorialmovea}
\begin{proof}
By \cref{monodromy:th:product-reduction-small-v}, we can find permutations $\alpha_1,\alpha_2\in\symgroup[d]$ with $[\alpha_1]=\pi_1$ and $[\alpha_2]=\pi_2$ such that $v(\alpha_1\alpha_2)=v(\pi_1)+v(\pi_2)$. Let $\pi_1'=[\alpha_1\alpha_2]$; it is easy to check that $\DD'=\datum{\surf{g}}{d}{\pi_1',\pi_3,\ldots,\pi_n}$ is a candidate datum. Assume that $\DD'$ is realizable; by \cref{hurwitz:th:monodromy-realizability-orientable}, this implies the existence of permutations $\alpha_1',\alpha_3,\ldots,\alpha_n\in\symgroup[d]$ matching $\pi_1',\pi_3,\ldots,\pi_n$ respectively, such that $\alpha_1'\alpha_3\cdots\alpha_n=1$ and $\angled{\alpha_1',\alpha_3,\ldots,\alpha_n}$ acts transitively on $\{1,\ldots,d\}$. Since $[\alpha_1']=[\alpha_1\alpha_2]$, up to conjugation we may assume that $\alpha_1'=\alpha_1\alpha_2$. It is clear that the permutations $\alpha_1,\alpha_2,\alpha_3,\ldots,\alpha_n$ imply the realizability of $\DD$, again by \cref{hurwitz:th:monodromy-realizability-orientable} (see also \cref{hurwitz:rm:sigma-tilde-unique-orientable}).
\end{proof}

\begin{remark}\label{monodromy:rm:combinatorial-move:a:small-v}
The proof of \cref{combinatorial-move:a:small-v} shows that we have some freedom in selecting the partition $\pi_1'$: given two permutations $\alpha_1,\alpha_2\in\symgroup[d]$ matching, respectively, $\pi_1$ and $\pi_2$, we have the combinatorial move
\[
\DD\cmove\datum{\surf{g}}{d}{[\alpha_1\alpha_2],\pi_3,\ldots,\pi_n},
\]
provided that $v(\alpha_1\alpha_2)=v(\alpha_1)+v(\alpha_2)$.
\end{remark}

\begin{combinatorialmovea}\label{combinatorial-move:a:large-v}
Let $\DD=\datum{\surf{g}}{d}{\pi_1,\pi_2,\ldots,\pi_n}$ be a candidate datum. Assume that:
\begin{assumptions}
\item $v(\pi_1)+v(\pi_2)\ge d-1$;
\item $v(\pi_3)+\ldots+v(\pi_n)\ge d-1$.
\end{assumptions}
Then there exist an orientable surface $\surf{g'}$ and a partition $\pi_1'\in\Partitions{d}$ such that $\DD'=\datum{\surf{g'}}{d}{\pi_1',\pi_3,\ldots,\pi_n}$ is a candidate datum and $\DD\cmove\DD'$. Moreover, if $\pi_1=\pi_2=[2,\ldots,2]$ then $\pi_1'$ can be chosen equal to $[d/2,d/2]$, otherwise it can be chosen in such a way that $\pi_1'\in\{[d],[1,d-1]\}$.
\end{combinatorialmovea}
\begin{proof}
By \cref{monodromy:th:product-reduction-odd} or \cref{monodromy:th:product-reduction-large-v-even}, depending on the parity of $v(\pi_1)+v(\pi_2)+d$, we can find permutations $\alpha_1,\alpha_2\in\symgroup[d]$ with $[\alpha_1]=\pi_1$ and $[\alpha_2]=\pi_2$ such that $[\alpha_1\alpha_2]\in\{[d],[1,d-1],[d/2,d/2]\}$. More specifically, if $\pi_1=\pi_2=[2,\ldots,2]$ we can assume that $[\alpha_1\alpha_2]=[d/2,d/2]$, otherwise we can choose $[\alpha_1\alpha_2]\in\{[d],[1,d-1]\}$. Let $\pi_1'=[\alpha_1\alpha_2]$; if we want $\DD'=\datum{\surf{g'}}{d}{\pi_1',\pi_3,\ldots,\pi_n}$ to be a candidate datum, the \RH{} formula requires that
\[
g'=\frac{1}{2}(v(\pi_1')+v(\pi_3)+\ldots+v(\pi_n))-d+1.
\]
But $v(\pi_1')\ge d-2$ and $v(\pi_3)+\ldots+v(\pi_n)\ge d-1$, so
\[
v(\pi_1')+v(\pi_3)+\ldots+v(\pi_n)\ge 2d-3.
\]
Moreover, $v(\pi_1')$ and $v(\pi_1)+v(\pi_2)$ have the same parity, so $v(\pi_1')+v(\pi_3)+\ldots+v(\pi_n)$ is even and actually
\[
v(\pi_1')+v(\pi_3)+\ldots+v(\pi_n)\ge 2d-2.
\]
Therefore $g'$ is a non-negative integer, and $\DD'$ is a candidate datum. The very same argument used in the proof of \cref{combinatorial-move:a:small-v} shows that $\DD\cmove\DD'$.
\end{proof}

\begin{comment}
\begin{remark}
We can extract some additional information from the statement of \cref{combinatorial-move:a:large-v}. First of all, since both $\DD$ and $\DD'$ are candidate data, their total branching number must be even, so $v(\pi_1')\equiv v(\pi_1)+v(\pi_2)\pmod{2}$. Moreover, $g'$ can be explicitly computed by means of the \RH{} formula:
\[
g'=\frac{1}{2}(v(\pi_1')+v(\pi_2)+\ldots+v(\pi_n))-d+1.
\]
\end{remark}
\end{comment}

\section{Some results on the sphere}\label{monodromy:sc:results-sphere}

In this section, we will make extensive use of the combinatorial moves we have just described to solve the Hurwitz existence problem for some specific families of candidate data. As we have already discussed the realizability of candidate data with $n\le 2$, from now on we will always assume that $n\ge 3$. We start by addressing the cases where one partition (without loss of generality, the last one) has length $1$.

\begin{proposition}\label{monodromy:th:sphere-[d]}
Let $\DD=\datum{\surf{g}}{d}{\pi_1,\ldots,\pi_{n-1},[d]}$ be a candidate datum. Then $\DD$ is realizable.
\end{proposition}
\begin{proof}
We proceed by induction on $n$, starting with the base case $n=3$. If $\DD=\datum{\surf{g},\sphere{}}{d}{\pi_1,\pi_2,[d]}$ is a candidate datum, by the \RH{} formula we have that
\[
v(\pi_1)+v(\pi_2)=2d-2+2g-(d-1)\ge d-1.
\]
Moreover, the total branching number $v(\pi_1)+v(\pi_2)+d-1$ is even. \Cref{monodromy:th:product-reduction-odd} then gives permutations $\alpha_1,\alpha_2\in\symgroup[d]$ with $[\alpha_1]=\pi_1$ and $[\alpha_2]=\pi_2$ such that $[\alpha_1\alpha_2]=[d]$. By \cref{hurwitz:th:monodromy-realizability-orientable}, $\DD$ is realizable (see \cref{hurwitz:rm:sigma-tilde-unique-orientable}).

We now turn to the case $n\ge 4$. Fix a candidate datum $\DD=\datum{\surf{g},\sphere{}}{d}{\pi_1,\ldots,\pi_{n-1},[d]}$; there are two cases.
\begin{sideline}{Case 1.}
Assume that $v(\pi_1)+v(\pi_2)\le d-1$. \Cref{combinatorial-move:a:small-v} gives
\[
\DD\cmove\DD'=\datum{\surf{g}}{d}{\pi_1',\pi_3,\ldots,\pi_{n-1},[d]}
\]
for a suitable candidate datum $\DD'$. By induction, $\DD'$ is realizable, therefore the same holds for $\DD$.
\end{sideline}
\begin{sideline}{Case 2.}
Otherwise, we have that $v(\pi_1)+v(\pi_2)\ge d$. Note that $v([d])=d-1$, so \cref{combinatorial-move:a:large-v} gives
\[
\DD\cmove\DD'=\datum{\surf{g}}{d}{\pi_1',\pi_3,\ldots,\pi_{n-1},[d]}
\]
for a suitable candidate datum $\DD'$. Similarly to the previous case, this implies that $\DD$ is realizable.\qedhere
\end{sideline}
\end{proof}

We now turn to candidate data containing the partition $[1,d-1]$. This is the first step towards the goal of this thesis, which will be further developed in \cref{short-partition:ch}: the full classification of exceptional data containing a partition of length $2$.

\begin{proposition}\label{monodromy:th:sphere-[1 d-1]}
Let $\DD=\datum{\surf{g}}{d}{\pi_1,\ldots,\pi_{n-1},[1,d-1]}$ be a candidate datum. Then $\DD$ is exceptional if and only if it satisfies one of the following.
\begin{enumarabic}
\item $\DD=\datum{\sphere{}}{2k}{[2,\ldots,2],[2,\ldots,2],[1,2k-1]}$ with $k\ge 2$.
\item $\DD=\datum{\surf{n-3}}{4}{[2,2],\ldots,[2,2],[1,3]}$.
\end{enumarabic}
\end{proposition}
\begin{proof}
The exceptionality of the listed candidate data is addressed in \cref{short-partition:th:exceptional-sphere,short-partition:th:exceptional-d-4} respectively. In order to prove that every other candidate datum is realizable, we proceed by induction on $n$, starting from the base case $n=3$. Let
\[
\DD=\datum{\surf{g}}{d}{\pi_1,\pi_2,[1,d-1]}\neq\datum{\sphere{}}{d}{[2,\ldots,2],[2,\ldots,2],[1,d-1]}
\]
be a candidate datum. The \RH{} formula implies that
\[
v(\pi_1)+v(\pi_2)=2d-2+2g-(d-2)\ge d.
\]
Moreover, the total branching number $v(\pi_1)+v(\pi_2)+d-2$ is even. \Cref{monodromy:th:product-reduction-large-v-even} then gives permutations $\alpha_1,\alpha_2\in\symgroup[d]$ with $[\alpha_1]=\pi_1$ and $[\alpha_2]=\pi_2$ such that $[\alpha_1\alpha_2]=[1,d-1]$ and the action of $\angled{\alpha_1,\alpha_2}$ is transitive. As usual, \cref{hurwitz:th:monodromy-realizability-orientable} implies that $\DD$ is realizable.

We now turn to the case $n\ge 4$; we will employ a reduction technique, similar to the proof of \cref{monodromy:th:sphere-[d]}. Fix a candidate datum
\[
\DD=\datum{\surf{n-3}}{d}{\pi_1,\ldots,\pi_{n-1},[1,d-1]}\neq\datum{\surf{n-3}}{4}{[2,2],\ldots,[2,2],[1,3]}.
\]
There are two cases.
\begin{sideline}{Case 1.}
Assume that the inequality $v(\pi_i)+v(\pi_j)\le d-1$ holds for a pair of indices $1\le i<j\le n-1$; up to reindexing, we can assume that $i=1$ and $j=2$. \Cref{combinatorial-move:a:small-v} gives
\[
\DD\cmove\DD'=\datum{\surf{g}}{d}{\pi_1',\pi_3,\ldots,\pi_{n-1},[1,d-1]}
\]
for a suitable candidate datum $\DD'$, where $v(\pi_1')=v(\pi_1)+v(\pi_2)$. By induction, $\DD'$ is realizable unless one of the following happens.
\begin{itemize}
\item $\DD'=\datum{\sphere{}}{2k}{[2,\ldots,2],[2,\ldots,2],[1,2k-1]}$ with $2k=d$. If this is the case, then $n=4$, $g=0$ and $v(\pi_1')=v(\pi_3)=k$. This implies that
\[
k<1+k\le v(\pi_1)+v(\pi_3)<2k=d.
\]
Repeating the construction with $i=1$ and $j=3$ will yield a realizable $\DD'$.
\item $\DD'=\datum{\surf{n-3}}{4}{[2,2],\ldots,[2,2],[1,3]}$. This implies that $\pi_1=\pi_2=[1,1,2]$ and $\pi_3=\ldots=\pi_{n-1}=[2,2]$. Repeating the construction with $i=1$ and $j=3$ will yield a realizable $\DD'$.
\end{itemize}
Therefore we can assume that $\DD'$ is realizable. Since $\DD\cmove\DD'$, we have that $\DD$ is realizable as well.
\end{sideline}
\begin{sideline}{Case 2.}
Otherwise, the inequality $v(\pi_1)+v(\pi_j)\ge d$ holds for every $1\le i<j\le n-1$. In particular, $v(\pi_1)+v(\pi_2)\ge d$ and $v(\pi_3)+v([1,d-1])\ge d-1$. \Cref{combinatorial-move:a:large-v} gives
\[
\DD\cmove\DD'=\datum{\surf{g'}}{d}{\pi_1',\pi_3,\ldots,\pi_{n-1},[1,d-1]}
\]
for a suitable candidate datum $\DD'$ with $v(\pi_1')\ge d-2$. Note that, if $d=4$, we can assume that $\pi_1\neq[2,2]$; as remarked in the statement of \cref{combinatorial-move:a:large-v}, this allows us to choose $\pi_1'\neq[2,2]$. It is then easy to see that $\DD'$ is realizable by induction: the case $d=4$ was addressed just now, while $\DD'=\datum{\sphere{}}{d}{[2,\ldots,2],[2,\ldots,2],[1,d-1]}$ is impossible for $d\ge 6$ since
\[
v(\pi_1')\ge d-2>\frac{d}{2}=v([2,\ldots,2]).
\]
Since $\DD\cmove\DD'$, this implies that $\DD$ is realizable as well.\qedhere
\end{sideline}
\end{proof}

The next result we are going to prove does not have immediate applications to the study of candidate data with a short partition, but the theoretical consequences it provides are definitely interesting.

\begin{proposition}\label{monodromy:th:sphere-large-g}
Let $\DD=\datum{\surf{g}}{d}{\pi_1,\ldots,\pi_n}$ be a candidate datum. Assume that $d\neq 4$ and that $2g\ge d-1$. Then $\DD$ is realizable.
\end{proposition}
\begin{proof}
Note that the the condition $2g\ge d-1$ is equivalent to
\[
v(\pi_1)+\ldots+v(\pi_n)\ge 3d-3
\]
by the \RH{} formula. Moreover, the cases where $d=2$ are trivial because of \cref{monodromy:th:sphere-[d]}; therefore, assume that $d\ge 3$.

We proceed by induction, starting from the base case $n=3$. If $n=3$, the inequality $v(\pi_1)+v(\pi_2)+v(\pi_3)\ge 3d-3$ implies that $\pi_1=\pi_2=\pi_3=[d]$; by \cref{monodromy:th:sphere-[d]}, $\DD$ is then realizable.

We now turn to the case $n\ge 4$; we will once again employ a reduction technique. Fix a candidate datum $\DD=\datum{\surf{g}}{d}{\pi_1,\ldots,\pi_n}$. We can assume that $\pi_i\neq[d]$ for every $1\le i\le n$, otherwise $\DD$ is immediately realizable by \cref{monodromy:th:sphere-[d]}. There are two cases.
\begin{sideline}{Case 1.}
Assume that the inequality $v(\pi_i)+v(\pi_j)\le d-1$ holds for a pair of indices $1\le i<j\le n$. It is now routine to employ \cref{combinatorial-move:a:small-v} to show that $\DD$ is realizable by induction.
\end{sideline}
\begin{sideline}{Case 2.}
Otherwise, we have $v(\pi_i)+v(\pi_j)\ge d$ for every $1\le i<j\le n$. We consider two sub-cases.
\begin{itemize}
\item Assume first that there is a partition, say $\pi_1$, which is different from $[2,\ldots,2]$. \Cref{combinatorial-move:a:small-v} gives
\[
\DD\cmove\DD'=\datum{\surf{g'}}{d}{\pi_1',\pi_3,\ldots,\pi_n}
\]
for a suitable candidate datum $\DD'$ with $\pi_1'\in\{[d],[1,d-1]\}$. By \cref{monodromy:th:sphere-[d],monodromy:th:sphere-[1 d-1]}, $\DD'$ is realizable unless $\DD'=\datum{\sphere{}}{d}{[2,\ldots,2],[2,\ldots,2],[1,d-1]}$. If this were the case, we would have $n=4$ and
\[
v(\pi_1)+v(\pi_2)+v(\pi_3)+v(\pi_4)\le d-2+d-2+\frac{d}{2}+\frac{d}{2}=3d-4,
\]
which contradicts the hypothesis. Therefore $\DD'$ is realizable, and the same holds for $\DD$.
\item Finally, consider the case where $\pi_1=\ldots=\pi_n=[2,\ldots,2]$. In this situation $d=2k\ge 6$ and
\[
v(\pi_1)+\ldots+v(\pi_n)=nk>6k-3
\]
(the inequality is strict since the total branching number is even, while $6k-3$ is odd). Since $k\ge 3$, this immediately implies that $n\ge 6$. We have that $v(\pi_1)+v(\pi_2)=2k$ and $v(\pi_3)+\ldots+v(\pi_n)=(n-2)k>2k$, so we can apply \cref{combinatorial-move:a:large-v} to get
\[
\DD\cmove\DD'=\datum{\surf{g'}}{2k}{[k,k],\pi_3,\ldots,\pi_n}
\]
for a suitable candidate datum $\DD'$. Now, $v([k,k])+v(\pi_3)=2k-2+k\ge 2k$ and $v(\pi_4)+\ldots+v(\pi_n)=(n-3)k>2k$, so we can apply \cref{combinatorial-move:a:large-v} once again and find that
\[
\DD'\cmove\DD''=\datum{\surf{g''}}{2k}{\pi_1'',\pi_4,\ldots,\pi_n}
\]
for a suitable candidate datum $\DD''$, where $\pi_1''\in\{[2k],[1,2k-1]\}$. Since $n-2\ge 4$ and $2k\ge 6$, \cref{monodromy:th:sphere-[d],monodromy:th:sphere-[1 d-1]} imply that $\DD''$ is realizable. As a consequence, $\DD$ is realizable as well.\sdlendhere\qedhere
\end{itemize} 
\end{sideline}
\end{proof}

\begin{corollary}
Let $d$ be a positive integer with $d\neq 4$. Then there exist at most finitely many exceptional candidate data of the form $\datum{\surf{g}}{d}{\pi_1,\ldots,\pi_n}$.
\end{corollary}
\begin{proof}
By \cref{monodromy:th:sphere-large-g}, every candidate datum with $n\ge 3d-3$ is realizable, and there are only a finite number of candidate data with $n<3d-3$.
\end{proof}

Instead, there exist infinitely many exceptional candidate data of degree $4$.

\begin{proposition}\label{monodromy:th:sphere-d-equals-4}
Let $\DD=\datum{\surf{g}}{4}{\pi_1,\ldots,\pi_n}$ be a candidate datum. Then $\DD$ is realizable if and only if
\[
\DD\neq\datum{\surf{n-3}}{4}{[2,2],\ldots,[2,2],[1,3]}.
\]
\end{proposition}
\begin{proof}
The exceptionality of the mentioned candidate datum is addressed in \cref{short-partition:th:exceptional-d-4}. In order to show that every other candidate datum with $d=4$ is realizable, we proceed by induction, starting from the base case $n=3$. Note that if any of the partitions is either $[4]$ or $[1,3]$, we immediately conclude by \cref{monodromy:th:sphere-[d],monodromy:th:sphere-[1 d-1]}. By the \RH{} formula, we have the inequality $v(\pi_1)+v(\pi_2)+v(\pi_3)\ge 6$; therefore, the only candidate datum left to consider is $\datum{\sphere{}}{4}{[2,2],[2,2],[2,2]}$. This datum is realizable, for instance, by choosing
\begin{align*}
\alpha_1=\cycle{1,2}\cycle{3,4},&&\alpha_2=\cycle{1,3}\cycle{2,4},&&\alpha_3=\cycle{1,4}\cycle{2,3},
\end{align*}
and applying \cref{hurwitz:th:monodromy-realizability-orientable}.

We now turn to the case $n\ge 4$; once again, we can assume that all the partitions are either $[1,1,2]$ or $[2,2]$. There are three cases.
\begin{sideline}{Case 1.}
If all the partitions are equal to $[2,2]$, then we can apply \cref{combinatorial-move:a:large-v} to combine two partitions into a single $[2,2]$, and conclude by a reduction argument.
\end{sideline}
\begin{sideline}{Case 2.}
If all the partitions are equal to $[1,1,2]$, then by the \RH{} formula we have $n\ge 6$. We can apply \cref{combinatorial-move:a:small-v} twice to combine three partitions into a single $[4]$, and conclude by reduction.
\end{sideline}
\begin{sideline}{Case 3.}
Otherwise, there is at least one $[2,2]$ and one $[1,1,2]$; using \cref{combinatorial-move:a:small-v}, we can combine them into a single $[4]$ and conclude by reduction.\qedhere
\end{sideline}
\end{proof}


\section{Prime-degree conjecture}

This section is devoted to a fascinating and surprising conjecture related to the Hurwitz existence problem. After studying many examples of exceptional data, a very unexpected pattern begins to emerge: it looks like no such datum can be found when the degree is a prime number. In other words, one comes to suspect the following.

\begin{prime-degree-conjecture*}\label{prime-degree-conjecture}
Let $p$ be a prime number. Then every candidate datum $\datum{\surf{g}}{p}{\pi_1,\ldots,\pi_n}$ with $n\ge 3$ is realizable.
\end{prime-degree-conjecture*}

It is not immediately clear why prime numbers should be even remotely related to the realizability of candidate data. In fact, to the best of the author's knowledge, there is no number-theoretic argument in favor (or against) this conjecture; all the evidence gathered so far supporting the prime-degree conjecture is ``experimental''.
\begin{itemize}
\item As shown in \cref{short-partition:th:exceptional-composite}, for every composite number $d\ge 4$ there is an exceptional candidate datum of degree $d$.
\item In general, all the exceptional data found so far have a composite degree; for all the families of candidate data whose realizability is completely understood (including those presented in the previous section), the prime-degree conjecture holds.
\item The first efficient algorithmic approach to the Hurwitz existence problem was presented in \cite{zheng}. Every candidate datum with $n=3$ and $d\le 20$\todo{Or $d\le 22$, according to some sources} was analyzed by a computer, and no exceptional data with prime degree were found; therefore, the prime-degree conjecture was verified for numbers up to $19$ (see \cref{monodromy:th:reduction-to-n-3} below). In \cref{computational-results:ch}, with a much heavier computational effort, we extend this result to prime numbers up to $29$.
\end{itemize}

The following proposition is an essential tool for the computational approach to the existence problem, since it reduces the verification of the prime-degree conjecture to the case of $n=3$ partitions.

\begin{proposition}\label{monodromy:th:reduction-to-n-3}
Let $d$ be a positive integer. Assume that every candidate datum $\datum{\surf{g}}{d}{\pi_1,\pi_2,\pi_3}$ is realizable. Then every candidate datum $\datum{\surf{g}}{d}{\pi_1,\ldots,\pi_n}$ with $n\ge 3$ is realizable.
\end{proposition}
\begin{proof}
We proceed by induction on $n\ge 4$, using the standard reduction technique. Fix a candidate datum $\DD=\datum{\surf{g}}{d}{\pi_1,\ldots,\pi_{n}}$; there are two cases.
\begin{sideline}{Case 1.}
If there are two indices $1\le i<j\le n$ such that $v(\pi_i)+v(\pi_j)\le d-1$, then a routine application of \cref{combinatorial-move:a:small-v} shows that $\DD$ is realizable.
\end{sideline}
\begin{sideline}{Case 2.}
Otherwise, $v(\pi_i)+v(\pi_j)\ge d$ for every $1\le i<j\le n$. In particular, $v(\pi_1)+v(\pi_2)\ge d$ and $v(\pi_3)+\ldots+v(\pi_n)\ge d$. By applying \cref{combinatorial-move:a:large-v}, it is easy to see that $\DD$ is realizable.\qedhere
\end{sideline}
\end{proof}